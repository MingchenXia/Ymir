
\documentclass{amsbook} 
%\usepackage{xr}
\usepackage{xr-hyper}
\usepackage[unicode]{hyperref}


\usepackage[T1]{fontenc}
\usepackage[utf8]{inputenc}
\usepackage{lmodern}
\usepackage{amssymb,tikz-cd}
%\usepackage{natbib}
\usepackage[english]{babel}

\usepackage[nameinlink,capitalize]{cleveref}
\usepackage[style=alphabetic,maxnames=99,maxalphanames=5, isbn=false, giveninits=true, doi=false]{biblatex}
\usepackage{lipsum, physics}
\usepackage{ifthen}
\usepackage{microtype}
\usepackage{booktabs}
\usetikzlibrary{calc}
\usepackage{emptypage}
\usepackage{setspace}
\usepackage[margin=0.75cm, font={small,stretch=0.80}]{caption}
\usepackage{subcaption}
\usepackage{url}
\usepackage{bookmark}
\usepackage{graphicx}
\usepackage{dsfont}
\usepackage{enumitem}
\usepackage{mathtools}
\usepackage{csquotes}
\usepackage{silence}
\usepackage{mathrsfs}
\usepackage{bigints}

\WarningFilter{biblatex}{Patching footnotes failed}


\ProcessOptions\relax

\emergencystretch=1em

\hypersetup{
colorlinks=true,
linktoc=all
}

\setcounter{tocdepth}{1}


\hyphenation{archi-medean  Archi-medean Tru-ding-er}

%\captionsetup[table]{position=bottom}   %% or below
\renewcommand{\thefootnote}{\fnsymbol{footnote}}
%\DeclareMathAlphabet{\mathcal}{OMS}{cmsy}{m}{n}
\renewbibmacro{in:}{}

\DeclareFieldFormat[article]{citetitle}{#1}
\DeclareFieldFormat[article]{title}{#1}
\DeclareFieldFormat[inbook]{citetitle}{#1}
\DeclareFieldFormat[inbook]{title}{#1}
\DeclareFieldFormat[incollection]{citetitle}{#1}
\DeclareFieldFormat[incollection]{title}{#1}
\DeclareFieldFormat[inproceedings]{citetitle}{#1}
\DeclareFieldFormat[inproceedings]{title}{#1}
\DeclareFieldFormat[phdthesis]{citetitle}{#1}
\DeclareFieldFormat[phdthesis]{title}{#1}
\DeclareFieldFormat[misc]{citetitle}{#1}
\DeclareFieldFormat[misc]{title}{#1}
\DeclareFieldFormat[book]{citetitle}{#1}
\DeclareFieldFormat[book]{title}{#1} 


%% Define various environments.

\theoremstyle{definition}
\newtheorem{theorem}{Theorem}[section]
\newtheorem{thm}[theorem]{Theorem}
\newtheorem{proposition}[theorem]{Proposition}
\newtheorem{corollary}[theorem]{Corollary}
\newtheorem{lemma}[theorem]{Lemma}
\newtheorem{conjecture}[theorem]{Conjecture}
\newtheorem{question}[theorem]{Question}
\newtheorem{example}[theorem]{Example}
\newtheorem{definition}[theorem]{Definition}
\newtheorem{condition}[theorem]{Condition}

\theoremstyle{remark}
\newtheorem{remark}[theorem]{Remark}
\numberwithin{equation}{section}

%\renewcommand{\thesection}{\thechapter.\arabic{section}}
%\renewcommand{\thetheorem}{\thesection.\arabic{theorem}}
%\renewcommand{\thedefinition}{\thesection.\arabic{definition}}
%\renewcommand{\theremark}{\thesection.\arabic{remark}}


%% Define new operators

\DeclareMathOperator{\nd}{nd}
\DeclareMathOperator{\ord}{ord}
\DeclareMathOperator{\Hom}{Hom}
\DeclareMathOperator{\PreSh}{PreSh}
\DeclareMathOperator{\Gr}{Gr}
\DeclareMathOperator{\Homint}{\mathcal{H}\mathrm{om}}
\DeclareMathOperator{\Torint}{\mathcal{T}\mathrm{or}}
\DeclareMathOperator{\Div}{div}
\DeclareMathOperator{\DSP}{DSP}
\DeclareMathOperator{\Diff}{Diff}
\DeclareMathOperator{\MA}{MA}
\DeclareMathOperator{\NA}{NA}
\DeclareMathOperator{\AN}{an}
\DeclareMathOperator{\Rep}{Rep}
\DeclareMathOperator{\Rest}{Res}
\DeclareMathOperator{\DF}{DF}
\DeclareMathOperator{\VCart}{VCart}
\DeclareMathOperator{\PL}{PL}
\DeclareMathOperator{\Bl}{Bl}
\DeclareMathOperator{\Td}{Td}
\DeclareMathOperator{\Fitt}{Fitt}
\DeclareMathOperator{\Ric}{Ric}
\DeclareMathOperator{\coeff}{coeff}
\DeclareMathOperator{\Aut}{Aut}
\DeclareMathOperator{\Capa}{Cap}
\DeclareMathOperator{\loc}{loc}
\DeclareMathOperator{\vol}{vol}
\DeclareMathOperator{\Val}{Val}
\DeclareMathOperator{\ST}{ST}
\DeclareMathOperator{\Amp}{Amp}
\DeclareMathOperator{\Herm}{Herm}
\DeclareMathOperator{\trop}{trop}
\DeclareMathOperator{\Trop}{Trop}
\DeclareMathOperator{\Cano}{Can}
\DeclareMathOperator{\PS}{PS}
\DeclareMathOperator{\Var}{Var}
\DeclareMathOperator{\Psef}{Psef}
\DeclareMathOperator{\Jac}{Jac}
\DeclareMathOperator{\Char}{char}
\DeclareMathOperator{\Red}{red}
\DeclareMathOperator{\Spf}{Spf}
\DeclareMathOperator{\Span}{Span}
\DeclareMathOperator{\Der}{Der}
%\DeclareMathOperator{\Mod}{mod}
\DeclareMathOperator{\Hilb}{Hilb}
\DeclareMathOperator{\triv}{triv}
\DeclareMathOperator{\Frac}{Frac}
\DeclareMathOperator{\diam}{diam}
\DeclareMathOperator{\Spec}{Spec}
\DeclareMathOperator{\Spm}{Spm}
\DeclareMathOperator{\Specrel}{\underline{Sp}}
\DeclareMathOperator{\Sp}{Sp}
\DeclareMathOperator{\reg}{reg}
\DeclareMathOperator{\sing}{sing}
\DeclareMathOperator{\Star}{Star}
\DeclareMathOperator{\relint}{relint}
\DeclareMathOperator{\Cvx}{Cvx}
\DeclareMathOperator{\Int}{Int}
\DeclareMathOperator{\Supp}{Supp}
\DeclareMathOperator{\FS}{FS}
\DeclareMathOperator{\RZ}{RZ}
\DeclareMathOperator{\Redu}{red}
\DeclareMathOperator{\lct}{lct}
\DeclareMathOperator{\Proj}{Proj}
\DeclareMathOperator{\Sing}{Sing}
\DeclareMathOperator{\Conv}{Conv}
\DeclareMathOperator{\Max}{Max}
\DeclareMathOperator{\Tor}{Tor}
\DeclareMathOperator{\Gal}{Gal}
\DeclareMathOperator{\Frob}{Frob}
\DeclareMathOperator{\coker}{coker}
\DeclareMathOperator{\Sym}{Sym}
\DeclareMathOperator{\CSp}{CSp}
\DeclareMathOperator{\Img}{Im}


\newcommand{\alg}{\mathrm{alg}}
\newcommand{\Sh}{\mathrm{Sh}}
\newcommand{\fin}{\mathrm{fin}}
\newcommand{\BPF}{\mathrm{BPF}}
\newcommand{\dBPF}{\mathrm{dBPF}}
\newcommand{\divf}{\mathrm{Div}^f}
\newcommand{\nef}{\mathrm{nef}}
\newcommand{\Bir}{\mathrm{Bir}}
\newcommand{\hO}{\hat{\mathcal{O}}}
\newcommand{\bDiv}{\mathrm{Div}^{\mathrm{b}}}
\newcommand{\un}{\mathrm{un}}
\newcommand{\sep}{\mathrm{sep}}
\newcommand{\diag}{\mathrm{diag}}
\newcommand{\Pic}{\mathrm{Pic}}
\newcommand{\GL}{\mathrm{GL}}
\newcommand{\SL}{\mathrm{SL}}
\newcommand{\LS}{\mathrm{LS}}
\newcommand{\GLS}{\mathrm{GLS}}
\newcommand{\GLSi}{\mathrm{GLS}_{\cap}}
\newcommand{\PGLS}{\mathrm{PGLS}}
\newcommand{\Loc}[1][S]{_{\{{#1}\}}}
\newcommand{\cl}{\mathrm{cl}}
\newcommand{\otL}{\hat{\otimes}^{\mathbb{L}}}
\newcommand{\ddpp}{\mathrm{d}'\mathrm{d}''}
\newcommand{\TC}{\mathcal{TC}}
\newcommand{\ddPP}{\mathrm{d}'_{\mathrm{P}}\mathrm{d}''_{\mathrm{P}}}
\newcommand{\PSs}{\mathcal{PS}}
\newcommand{\Gm}{\mathbb{G}_{\mathrm{m}}}
\newcommand{\End}{\mathrm{End}}
\newcommand{\Aff}[1][X]{\mathcal{M}\left(\mathcal{#1}\right)}
\newcommand{\XG}[1][X]{{#1}_{\mathrm{G}}}
\newcommand{\convC}{\xrightarrow{C}}
\newcommand{\Vect}{\mathrm{Vect}}
\newcommand{\abso}[1]{\lvert#1\rvert}
\newcommand{\Mdl}{\mathrm{Model}}
\newcommand{\cn}{\stackrel{\sim}{\longrightarrow}}
\newcommand{\sbc}{\mathbf{s}}
\newcommand{\CH}{\mathrm{CH}}
\newcommand{\GR}{\mathrm{GR}}
\newcommand{\dc}{\mathrm{d}^{\mathrm{c}}}
\newcommand{\Nef}{\mathrm{Nef}}
\newcommand{\Adj}{\mathrm{Adj}}
\newcommand{\DHm}{\mathrm{DH}}
\newcommand{\An}{\mathrm{an}}
\newcommand{\Rec}{\mathrm{Rec}}
\newcommand{\dP}{\mathrm{d}_{\mathrm{P}}}
\newcommand{\ddp}{\mathrm{d}_{\mathrm{P}}'\mathrm{d}_{\mathrm{P}}''}
\newcommand{\ddc}{\mathrm{dd}^{\mathrm{c}}}
\newcommand{\ddL}{\mathrm{d}'\mathrm{d}''}
\newcommand{\PSH}{\mathrm{PSH}}
\newcommand{\CPSH}{\mathrm{CPSH}}
\newcommand{\PSP}{\mathrm{PSP}}
\newcommand{\WPSH}{\mathrm{WPSH}}
\newcommand{\Ent}{\mathrm{Ent}}
\newcommand{\NS}{\mathrm{NS}}
\newcommand{\QPSH}{\mathrm{QPSH}}
\newcommand{\proet}{\mathrm{pro-ét}}
\newcommand{\XL}{(\mathcal{X},\mathcal{L})}
\newcommand{\ii}{\mathrm{i}}
\newcommand{\Cpt}{\mathrm{Cpt}}
\newcommand{\bp}{\bar{\partial}}
\newcommand{\ddt}{\frac{\mathrm{d}}{\mathrm{d}t}}
\newcommand{\dds}{\frac{\mathrm{d}}{\mathrm{d}s}}
\newcommand{\Ep}{\mathcal{E}^p(X,\theta;[\phi])}
\newcommand{\Ei}{\mathcal{E}^{\infty}(X,\theta;[\phi])}
\newcommand{\infs}{\operatorname*{inf\vphantom{p}}}
\newcommand{\sups}{\operatorname*{sup*}}
\newcommand{\colim}{\operatorname*{colim}}
\newcommand{\ddtz}[1][0]{\left.\ddt\right|_{t={#1}}}
\newcommand{\tube}[1][Y]{]{#1}[}
\newcommand{\ddsz}[1][0]{\left.\ddt\right|_{s={#1}}}
\newcommand{\floor}[1]{\left \lfloor{#1}\right \rfloor }
\newcommand{\dec}[1]{\left \{{#1}\right \} }
\newcommand{\ceil}[1]{\left \lceil{#1}\right \rceil }
\newcommand{\Projrel}{\mathcal{P}\mathrm{roj}}
\newcommand{\Weil}{\mathrm{Weil}}
\newcommand{\Cart}{\mathrm{Cart}}
\newcommand{\bWeil}{\mathrm{b}\mathrm{Weil}}
\newcommand{\bCart}{\mathrm{b}\mathrm{Cart}}
\newcommand{\Cond}{\mathrm{Cond}}
\newcommand{\IC}{\mathrm{IC}}
\newcommand{\IH}{\mathrm{IH}}
\newcommand{\cris}{\mathrm{cris}}
\newcommand{\Zar}{\mathrm{Zar}}
\newcommand{\HvbCat}{\overline{\mathcal{V}\mathrm{ect}}}
\newcommand{\BanModCat}{\mathcal{B}\mathrm{an}\mathcal{M}\mathrm{od}}
\newcommand{\DesCat}{\mathcal{D}\mathrm{es}}
\newcommand{\RingCat}{\mathcal{R}\mathrm{ing}}
\newcommand{\SchCat}{\mathcal{S}\mathrm{ch}}
\newcommand{\AbCat}{\mathcal{A}\mathrm{b}}
\newcommand{\RSCat}{\mathcal{R}\mathrm{S}}
\newcommand{\LRSCat}{\mathcal{L}\mathrm{RS}}
\newcommand{\CLRSCat}{\mathbb{C}\text{-}\LRSCat}
\newcommand{\CRSCat}{\mathbb{C}\text{-}\RSCat}
\newcommand{\CLA}{\mathbb{C}\text{-}\mathcal{L}\mathrm{A}}
\newcommand{\CASCat}{\mathbb{C}\text{-}\mathcal{A}\mathrm{n}}
\newcommand{\LiuCat}{\mathcal{L}\mathrm{iu}}
\newcommand{\BanCat}{\mathcal{B}\mathrm{an}}
\newcommand{\BanAlgCat}{\mathcal{B}\mathrm{an}\mathcal{A}\mathrm{lg}}
\newcommand{\AnaCat}{\mathcal{A}\mathrm{n}}
\newcommand{\LiuAlgCat}{\mathcal{L}\mathrm{iu}\mathcal{A}\mathrm{lg}}
\newcommand{\AlgCat}{\mathcal{A}\mathrm{lg}}
\newcommand{\SetCat}{\mathcal{S}\mathrm{et}}
\newcommand{\ModCat}{\mathcal{M}\mathrm{od}}
\newcommand{\TopCat}{\mathcal{T}\mathrm{op}}
\newcommand{\CohCat}{\mathcal{C}\mathrm{oh}}
\newcommand{\SolCat}{\mathcal{S}\mathrm{olid}}
\newcommand{\AffCat}{\mathcal{A}\mathrm{ff}}
\newcommand{\AffAlgCat}{\mathcal{A}\mathrm{ff}\mathcal{A}\mathrm{lg}}
\newcommand{\QcohLiuAlgCat}{\mathcal{L}\mathrm{iu}\mathcal{A}\mathrm{lg}^{\mathrm{QCoh}}}
\newcommand{\LiuMorCat}{\mathcal{L}\mathrm{iu}}
\newcommand{\Isom}{\mathcal{I}\mathrm{som}}
\newcommand{\Cris}{\mathcal{C}\mathrm{ris}}
\newcommand{\Pro}{\mathrm{Pro}-}
\newcommand{\Fin}{\mathcal{F}\mathrm{in}}
\newcommand{\norms}[1]{\left\|#1\right\|}
\newcommand{\HPDDiff}{\mathbf{D}\mathrm{iff}}
\newcommand{\Menn}[2]{\begin{bmatrix}#1\\#2\end{bmatrix}}
\newcommand{\Fins}{\widehat{\Vect}^F}
\newcommand\blfootnote[1]{%
  \begingroup
  \renewcommand\thefootnote{}\footnote{#1}%
  \addtocounter{footnote}{-1}%
  \endgroup
}

\externaldocument[Introduction-]{Introduction}
%One variable complex analysis
%Several variables complex analysis
\externaldocument[Topology-]{Topology-Bornology}
\externaldocument[Banach-]{Banach-Rings}
\externaldocument[Commutative-]{Commutative-Algebra}
\externaldocument[Local-]{Local-Algebras}
\externaldocument[Complex-]{Complex-Analytic-Spaces}
%Properties of space
\externaldocument[Morphisms-]{Morphisms}
%Differential calculus
%GAGA
%Hilbert scheme complex analytic version

%Complex differential geometry

\externaldocument[Affinoid-]{Affinoid-Algebras}
\externaldocument[Berkovich-]{Berkovich-Analytic-Spaces}


\bibliography{Ymir}

\endinput
\title{Affinoid algebras}
\begin{document}
\maketitle
\tableofcontents




\section{Introduction}\label{sec-introduction}
Our references  for this chapter include \cite{BGR}, \cite{Berk12}.
\section{Tate algebras}
Let $(k,|\bullet|)$ be a complete non-Archimedean valued-field. 

\begin{definition}
    Let $n\in \mathbb{N}$ and $r=(r_1,\ldots,r_n)\in \mathbb{R}^n_{>0}$. We set 
    \[
        \begin{split}
        k\{r^{-1}T\}=& k\{r_1^{-1}T_1,\ldots,r_n T_n^{-1}\} \\
        :=&\left\{f=\sum_{\alpha\in \mathbb{N}^n} a_{\alpha}T^{\alpha}\in k[[T_1,\ldots,T_n]]:a_{\alpha}\in k, |a_{\alpha}|r^{\alpha}\to 0\text{ as }|\alpha|\to\infty \right\}.
        \end{split}
    \]
    For any $f=\sum_{\alpha\in \mathbb{N}^n} a_{\alpha}T^{\alpha}\in k\{r^{-1}T\}$, we set
    \[
        \|f\|_r=\max_{\alpha}|a_{\alpha}|r^{\alpha}.  
    \]
    We call $(k\{r^{-1}T\},\|\bullet\|_r)$ the \emph{Tate algebra} in $n$-variables with radii $r$. The norm $\|\bullet\|_r$ is called the \emph{Gauss norm}.

    We omit $r$ from the notation if $r=(1,\ldots,1)$.
\end{definition}
This is a special case of \cref{Banach-ex-strictconvseriesradius}  in the chapter Banach Rings.
\begin{proposition}
    Let $n\in \mathbb{N}$ and $r=(r_1,\ldots,r_n)\in \mathbb{R}^n_{>0}$.  Then the Tate algebra $(k\{r^{-1}T\},\|\bullet\|_r)$ is a Banach $k$-algebra and $\|\bullet\|_r$ is a valuation.
\end{proposition}

\begin{proof}
    This is a special case of \cref{Banach-prop-strictconvseriesradiusBanach} in the chapter Banach Rings.
\end{proof}

\begin{remark}
One should think of $k\{r^{-1}T\}$ as analogues of $\mathbb{C}\langle r^{-1}T\rangle$ in the theory of complex analytic spaces.  We could have studied complex analytic spaces directly from the Banach rings $\mathbb{C}\langle r^{-1}T\rangle$, as we will do in the rigid world. But in the complex world, the miracle is that we have \emph{a priori} a good theory of functions on all open subsets of the unit polydisk, so things are greatly simplified. The unit polydisk is a ringed space for free.

As we will see, constructing a good function theory, or more precisely, enhancing the unit disk to a ringed site is the main difficulty in the theory of rigid spaces. And Tate's innovation comes in at this point.
\end{remark}

\begin{example}
    Assume that the valuation on $k$ is trivial. 

    Let $n\in \mathbb{N}$ and $r\in \mathbb{R}^n_{>0}$. 
    Then $k\{r^{-1}T\}\cong k[T_1,\ldots,T_n]$ if $r_i\geq 1$ for all $i$ and $k\{r^{-1}T\}\cong k[[T_1,\ldots,T_n]]$ otherwise.
\end{example}






\section{Affinoid algebras}
Let $(k,|\bullet|)$ be a complete non-Archimedean valued-field. 

\begin{definition}
    A Banach $k$-algebra $A$ is \emph{$k$-affinoid} (resp. \emph{strictly $k$-affinoid}) if there are $n\in \mathbb{N}$, $r\in \mathbb{R}^n_{>0}$ and an admissible epimorphism $k\{r^{-1}T\}\rightarrow A$ (resp. an admissible epimorphism $k\{T\}\rightarrow A$).

    An affinoid $k$-algebra is a $K$-affinoid algebra for some complete non-Archimedean field extension $K/k$.
\end{definition}
For the notion of admissible morphisms, we refer to \cref{Banach-def-admissiblemorphism} in the chapter Banach rings.



\begin{example}\label{ex-Kraffinoid}
    Let $r\in \mathbb{R}_{>0}$. We let $K_r$ denote the subring of $k[[T]]$ consisting of $f=\sum_{i=-\infty}^{\infty} a_i T^i$ satisfying $|a_i| r^i\to 0$ for $i\to \infty$ and $i\to -\infty$. We define a norm $\|\bullet\|_r$ on $K_r$ as follows:
    \[
          \|f\|_{r}:=\max_{i\in \mathbb{Z}}|a_i| r^i.
    \]
    We will show in \cref{prop-Krisaffinoid} that $K_r$ is $k$-affinoid.
\end{example}
\begin{proposition}\label{prop-Krisaffinoid}
    Let $r\in \mathbb{R}_{>0}$, then $(K_r,\|\bullet\|_r)$ defined in \cref{ex-Kraffinoid} is a $k$-affinoid algebra. Moreover, $\|\bullet\|_r$ is a valuation.
\end{proposition}
\begin{proof}
    Observe that we have an admissible epimorphism
    \[
        \iota:k\{r^{-1}T_1,rT_2\}\rightarrow K_r,\quad T_1\mapsto T, T_2\mapsto T^{-1}.  
    \]
    As we do not have the universal property at our disposal yet, let us verify by hand that this defines a ring homomorphism: consider a series
    \[
        f=\sum_{(i,j)\in \mathbb{N}^2} a_{i,j}T_1^i T_2^j\in  k\{r^{-1}T_1,rT_2\}, 
    \]
    namely, 
    \begin{equation}\label{eq-aijrijto0}
        |a_{i,j}|r^{i-j}\to 0  
    \end{equation}
    as $i+j\to \infty$. Observe that for each $k\in \mathbb{Z}$, the series 
    \[
        c_k:=  \sum_{i-j=k,i,j\in \mathbb{N}} a_{i,j}
    \]
    is convergent.

    Then by definition, the image $\iota(f)$ is given by
    \[
        \sum_{k=-\infty}^{\infty} c_k T^{k}.
    \]
    We need to verify that $\iota(f)\in K_r$. That is 
    \[
        |c_k|r^k\to 0  
    \]
    as $k\to \pm\infty$. When $k\geq 0$, we have $|c_k|\leq |a_{k0}|$ by definition of $c_k$. So $|c_k|r^k\to 0$ as $k\to\infty$ by \eqref{eq-aijrijto0}. The case $k\to -\infty$ is similar.

    We conclude that we have a well-defined map of sets $\iota$. It is straightforward to verify that $\iota$ is a ring homomorphism. Next we show that $\iota$ is surjective. Take $g=\sum_{i=-\infty}^{\infty} c_i T^i\in K_r$. We want to show that $g$ lies in the image of $\iota$. As $\iota$ is a ring homomorphism, it suffices to treat two cases separately: $g=\sum_{i=0}^{\infty} c_i T^i$ and $g=\sum_{i=-\infty}^0 c_iT^i$. We handle the first case only, as the second case is similar. In this case, it suffices to consider $f=\sum_{i=0}^{\infty} c_i T_1^i\in k\{r^{-1}T_1,rT_2\}$. It is immediate that $\iota(f)=g$.


    Next we show that $\iota$ is admissible. We first identify the kernel of $\iota$. We claim that the kenrel is the ideal $I$ generated by $T_1T_2-1$. It is obvious that $I\subseteq \ker \iota$. Conversely, consider an element
    \[
        f=\sum_{(i,j)\in \mathbb{N}^2} a_{i,j}T_1^i T_2^j\in  k\{r^{-1}T_1,rT_2\}  
    \]
    lying in the kenrel of $\iota$. Observe that 
    \[
        f=\sum_{k=-\infty}^{\infty}f_k, \quad f_k=\sum_{(i,j)\in \mathbb{N}^2, i-j=k} a_{i,j}T_1^i T_2^j.
    \]
    If $f\in \ker\iota$, then so is each $f_k$ by our construction. 
    
    We first show that each $f_k$ lies in the ideal generated by $T_1T_2-1$. The condition that $f_k\in \ker\iota$ means
    \[
        \sum_{(i,j)\in \mathbb{N}^2, i-j=k} a_{i,j}=0.  
    \]
    It is elementary to find $b_{i,j}\in k$ for $i,j\in \mathbb{N}$, $i-j=k$ such that
    \[
        a_{i,j}=b_{i-1,j-1} -b_{i,j}.
    \]
    Then 
    \[
      f_k=(T_1T_2-1)\sum_{i,j\in \mathbb{N},i-j=k}b_{i,j}T_1^i T_2^j.
    \]
    Observe that we can make sure that $|b_{i,j}|\leq \max\{|a_{i',j'}|:i-j=i'-j'\}$. In particular, the sum of $\sum_{i,j\in \mathbb{N},i-j=k}b_{i,j}T_1^i T_2^j$ for various $k$ converges to some $g\in k\{r^{-1}T_1,rT_2\}$ and hence $f_k=(T_1T_2-1)g$.
    Therefore, we have proved that $\ker \iota$ is generated by $T_1T_2-1$.

    It remains to show that $\iota$ is admissible. In fact, we will prove a stronger result: $\iota$ induces an isometric isomorphism
    \[
        k\{r^{-1}T_1,rT_2\}/I\rightarrow K_r. 
    \]
    To see this, take $f=\sum_{k=-\infty}^{\infty}c_k T^k\in K_r$ and we need to show that 
    \[
        \|f\|_r=\inf\{\|g\|_{(r,r^{-1})}:\iota(g)=f\}.
    \]
    Observe that if we set $g=\sum_{k=0}^{\infty} c_k T_1^k+\sum_{k=1}^{\infty}c_{-k}T_2^k$, then $\iota(g)=f$ and $\|g\|_{(r,r^{-1})}=\|f\|$. So it suffices to show that for any $h=\sum_{(i,j)\in \mathbb{N}^2} d_{i,j}T_1^iT_2^j\in k\{r^{-1}T_1,rT_2\}$, we have
    \begin{equation}\label{eq-fnormequalquotientnorm}
        \|f\|_r\leq \|g+h(T_1T_2-1)\|_{r,r^{-1}}.  
    \end{equation}
    We compute
    \[
        g+h(T_1T_2-1)=\sum_{k=1}^{\infty}(c_k-d_{k,0})T_1^k+\sum_{k=1}^{\infty}(c_{-k}-d_{0,k})T_2^k+(c_0-d_0)+\sum_{i,j\geq 1} (d_{i-1,j-1}-d_{i,j})T_1^iT_2^j.
    \]
    So 
    \[
        \|g+h(T_1T_2-1)\|_{r,r^{-1}}=\max\left\{\max_{k\geq 0} C_{1,k},\max_{k\geq 1} C_{2,k} \right\},  
    \]
    where
    \[
        C_{1,k}=\max\left\{ |c_k-d_{k,0}|, \left|\sum_{i-j=k,i,j\geq 1} d_{i-1,j-1}-d_{i,j}\right| \right\} 
    \]
    for $k\geq 0$
    and
    \[
        C_{2,k}=\max\left\{ |c_{-k}-d_{0,k}|, \left|\sum_{i-j=-k,i,j\geq 1} d_{i-1,j-1}-d_{i,j}\right| \right\}
    \] 
    for $k\geq 1$.
    It follows from the strong triangle inequality that $|c_k|\leq C_{1,k}$ for $k\geq 0$ and $c_{-k}\leq C_{2,k}$ for $k\geq 1$. So \eqref{eq-fnormequalquotientnorm} follows.
\end{proof}

\begin{proposition}
    Let $r\in \mathbb{R}_{>0}\setminus \sqrt{|k^{\times}|}$, then $\|\bullet\|_r$ defined in \cref{ex-Kraffinoid} is a valuation on $K_r$.
\end{proposition}
\begin{proof}
    Take $f,g\in K_r$, we need to show that
    \[
      \|fg\|_r\geq \|f\|_r\|g\|_r.  
    \]
    Let us expand
    \[
      f=\sum_{i=-\infty}^{\infty} a_i T^i,\quad   g=\sum_{i=-\infty}^{\infty} b_i T^i.
    \]
    Take $i$ and $j$ so that
    \begin{equation}\label{eq-aibjnorm}
      |a_i|r^i=\|f\|_r,\quad |b_j|r^j=\|g\|_r.  
    \end{equation}
    By our assumption on $r$, $i,j$ are unique.
    Then
    \[
      \|fg\|_r=\max_{k\in \mathbb{Z}}\{ |c_k|r^k\},
    \]
    where 
    \[
      c_k:=\sum_{u,v\in \mathbb{Z},u+v=k}a_ub_v.  
    \]
    It suffices to show that
    \begin{equation}\label{eq-ckrkequalfrgr}
        |c_k|r^k=\|f\|_r\|g\|_r.
    \end{equation}
    for $k=i+j$. Of course, we may assume that $a_i\neq 0$ and $b_j\neq 0$ as otherwise there is nothing to prove.
    For $u,v\in \mathbb{Z}$, $u+v=i+j$ while $(u,v)\neq (i,j)$, we may assume that $u\neq i$. Then $|a_u|r^u < |a_i|r^i$ and $|b_v|r^v\leq |b_j|r^j$. So $|a_ub_v|< |a_ib_j|$ and we conclude \eqref{eq-ckrkequalfrgr}.
\end{proof}
\begin{remark}
    The argument of \cref{Banach-prop-strictconvseriesradiusBanach} in the chapter Banch Rings does not work here if $r\in \sqrt{|k^{\times}|}$, as in general one can not take minimal $i,j$ so that \eqref{eq-aibjnorm} is satisfied.
\end{remark}

\begin{proposition}\label{prop-Krvaluation}
    Assume that $r\in \mathbb{R}_{>0}\setminus \sqrt{|k^{\times}|}$. Then $K_r$ is a valuation field and $\|\bullet\|_r$ is non-trivial.
\end{proposition}
\begin{proof}
    We first show that $\Sp K_r$ consists of a single point: $\|\bullet\|_r$. Assume that $|\bullet|\in \Sp K_r$. As $\|\bullet\|_r$ is a valuation, we find 
    \begin{equation}\label{eq-compseminormnormr}
      |\bullet|\leq \|\bullet\|_r.  
    \end{equation}
    In particular, $|\bullet|$ restricted to $k$ is the given valuation on $k$. It suffices to show that $|T|=r$. This follows from \eqref{eq-compseminormnormr} applied to $T$ and $T^{-1}$.

    It follows that $K_r$ does not have any non-zero proper closed ideals: if $I$ is such an ideal, $K_r/I$ is a Banach $k$-algebra. By \cref{Banach-prop-Berkospecnonempty} in the chapter Banach rings, $\Sp K_r$ is non-empty. So $K_r$ has to admit bounded semi-valuation with non-trivial kernel.

    In particular, by \cref{Banach-cor:maximalidealclosedinBanachring} in the chapter Banach rings, the only maximal ideal of $K_r$ is $0$. It follows that $K_r$ is a field.

    The valuation $\|\bullet\|_r$ is non-trivial as $\|T\|_r=r$.
\end{proof}

\begin{definition}
    Let $n\in \mathbb{N}$ and $r=(r_1,\ldots,r_n)\in \mathbb{R}^n_{>0}$. Assume that $r_1,\ldots,r_n$ are linearly independent in the $\mathbb{Q}$-linear space $\mathbb{R}_{>0}/\sqrt{|k^{\times}|}$. We define 
    \[
        K_r=K_{r_1}\hat{\otimes}_k \cdots \hat{\otimes}_k K_{r_n}.  
    \]
\end{definition}
By an interated application of \cref{prop-Krvaluation}, $K_r$ is a complete valuation field.



\section{Properties of affinoid algebras}


\printbibliography
\end{document}