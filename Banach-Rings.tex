
\documentclass{amsbook} 
%\usepackage{xr}
\usepackage{xr-hyper}
\usepackage[unicode]{hyperref}


\usepackage[T1]{fontenc}
\usepackage[utf8]{inputenc}
\usepackage{lmodern}
\usepackage{amssymb,tikz-cd}
%\usepackage{natbib}
\usepackage[english]{babel}

\usepackage[nameinlink,capitalize]{cleveref}
\usepackage[style=alphabetic,maxnames=99,maxalphanames=5, isbn=false, giveninits=true, doi=false]{biblatex}
\usepackage{lipsum, physics}
\usepackage{ifthen}
\usepackage{microtype}
\usepackage{booktabs}
\usetikzlibrary{calc}
\usepackage{emptypage}
\usepackage{setspace}
\usepackage[margin=0.75cm, font={small,stretch=0.80}]{caption}
\usepackage{subcaption}
\usepackage{url}
\usepackage{bookmark}
\usepackage{graphicx}
\usepackage{dsfont}
\usepackage{enumitem}
\usepackage{mathtools}
\usepackage{csquotes}
\usepackage{silence}
\usepackage{mathrsfs}
\usepackage{bigints}

\WarningFilter{biblatex}{Patching footnotes failed}


\ProcessOptions\relax

\emergencystretch=1em

\hypersetup{
colorlinks=true,
linktoc=all
}

\setcounter{tocdepth}{1}


\hyphenation{archi-medean  Archi-medean Tru-ding-er}

%\captionsetup[table]{position=bottom}   %% or below
\renewcommand{\thefootnote}{\fnsymbol{footnote}}
%\DeclareMathAlphabet{\mathcal}{OMS}{cmsy}{m}{n}
\renewbibmacro{in:}{}

\DeclareFieldFormat[article]{citetitle}{#1}
\DeclareFieldFormat[article]{title}{#1}
\DeclareFieldFormat[inbook]{citetitle}{#1}
\DeclareFieldFormat[inbook]{title}{#1}
\DeclareFieldFormat[incollection]{citetitle}{#1}
\DeclareFieldFormat[incollection]{title}{#1}
\DeclareFieldFormat[inproceedings]{citetitle}{#1}
\DeclareFieldFormat[inproceedings]{title}{#1}
\DeclareFieldFormat[phdthesis]{citetitle}{#1}
\DeclareFieldFormat[phdthesis]{title}{#1}
\DeclareFieldFormat[misc]{citetitle}{#1}
\DeclareFieldFormat[misc]{title}{#1}
\DeclareFieldFormat[book]{citetitle}{#1}
\DeclareFieldFormat[book]{title}{#1} 


%% Define various environments.

\theoremstyle{definition}
\newtheorem{theorem}{Theorem}[section]
\newtheorem{thm}[theorem]{Theorem}
\newtheorem{proposition}[theorem]{Proposition}
\newtheorem{corollary}[theorem]{Corollary}
\newtheorem{lemma}[theorem]{Lemma}
\newtheorem{conjecture}[theorem]{Conjecture}
\newtheorem{question}[theorem]{Question}
\newtheorem{example}[theorem]{Example}
\newtheorem{definition}[theorem]{Definition}
\newtheorem{condition}[theorem]{Condition}

\theoremstyle{remark}
\newtheorem{remark}[theorem]{Remark}
\numberwithin{equation}{section}

%\renewcommand{\thesection}{\thechapter.\arabic{section}}
%\renewcommand{\thetheorem}{\thesection.\arabic{theorem}}
%\renewcommand{\thedefinition}{\thesection.\arabic{definition}}
%\renewcommand{\theremark}{\thesection.\arabic{remark}}


%% Define new operators

\DeclareMathOperator{\nd}{nd}
\DeclareMathOperator{\ord}{ord}
\DeclareMathOperator{\Hom}{Hom}
\DeclareMathOperator{\PreSh}{PreSh}
\DeclareMathOperator{\Gr}{Gr}
\DeclareMathOperator{\Homint}{\mathcal{H}\mathrm{om}}
\DeclareMathOperator{\Torint}{\mathcal{T}\mathrm{or}}
\DeclareMathOperator{\Div}{div}
\DeclareMathOperator{\DSP}{DSP}
\DeclareMathOperator{\Diff}{Diff}
\DeclareMathOperator{\MA}{MA}
\DeclareMathOperator{\NA}{NA}
\DeclareMathOperator{\AN}{an}
\DeclareMathOperator{\Rep}{Rep}
\DeclareMathOperator{\Rest}{Res}
\DeclareMathOperator{\DF}{DF}
\DeclareMathOperator{\VCart}{VCart}
\DeclareMathOperator{\PL}{PL}
\DeclareMathOperator{\Bl}{Bl}
\DeclareMathOperator{\Td}{Td}
\DeclareMathOperator{\Fitt}{Fitt}
\DeclareMathOperator{\Ric}{Ric}
\DeclareMathOperator{\coeff}{coeff}
\DeclareMathOperator{\Aut}{Aut}
\DeclareMathOperator{\Capa}{Cap}
\DeclareMathOperator{\loc}{loc}
\DeclareMathOperator{\vol}{vol}
\DeclareMathOperator{\Val}{Val}
\DeclareMathOperator{\ST}{ST}
\DeclareMathOperator{\Amp}{Amp}
\DeclareMathOperator{\Herm}{Herm}
\DeclareMathOperator{\trop}{trop}
\DeclareMathOperator{\Trop}{Trop}
\DeclareMathOperator{\Cano}{Can}
\DeclareMathOperator{\PS}{PS}
\DeclareMathOperator{\Var}{Var}
\DeclareMathOperator{\Psef}{Psef}
\DeclareMathOperator{\Jac}{Jac}
\DeclareMathOperator{\Char}{char}
\DeclareMathOperator{\Red}{red}
\DeclareMathOperator{\Spf}{Spf}
\DeclareMathOperator{\Span}{Span}
\DeclareMathOperator{\Der}{Der}
%\DeclareMathOperator{\Mod}{mod}
\DeclareMathOperator{\Hilb}{Hilb}
\DeclareMathOperator{\triv}{triv}
\DeclareMathOperator{\Frac}{Frac}
\DeclareMathOperator{\diam}{diam}
\DeclareMathOperator{\Spec}{Spec}
\DeclareMathOperator{\Spm}{Spm}
\DeclareMathOperator{\Specrel}{\underline{Sp}}
\DeclareMathOperator{\Sp}{Sp}
\DeclareMathOperator{\reg}{reg}
\DeclareMathOperator{\sing}{sing}
\DeclareMathOperator{\Star}{Star}
\DeclareMathOperator{\relint}{relint}
\DeclareMathOperator{\Cvx}{Cvx}
\DeclareMathOperator{\Int}{Int}
\DeclareMathOperator{\Supp}{Supp}
\DeclareMathOperator{\FS}{FS}
\DeclareMathOperator{\RZ}{RZ}
\DeclareMathOperator{\Redu}{red}
\DeclareMathOperator{\lct}{lct}
\DeclareMathOperator{\Proj}{Proj}
\DeclareMathOperator{\Sing}{Sing}
\DeclareMathOperator{\Conv}{Conv}
\DeclareMathOperator{\Max}{Max}
\DeclareMathOperator{\Tor}{Tor}
\DeclareMathOperator{\Gal}{Gal}
\DeclareMathOperator{\Frob}{Frob}
\DeclareMathOperator{\coker}{coker}
\DeclareMathOperator{\Sym}{Sym}
\DeclareMathOperator{\CSp}{CSp}
\DeclareMathOperator{\Img}{Im}


\newcommand{\alg}{\mathrm{alg}}
\newcommand{\Sh}{\mathrm{Sh}}
\newcommand{\fin}{\mathrm{fin}}
\newcommand{\BPF}{\mathrm{BPF}}
\newcommand{\dBPF}{\mathrm{dBPF}}
\newcommand{\divf}{\mathrm{Div}^f}
\newcommand{\nef}{\mathrm{nef}}
\newcommand{\Bir}{\mathrm{Bir}}
\newcommand{\hO}{\hat{\mathcal{O}}}
\newcommand{\bDiv}{\mathrm{Div}^{\mathrm{b}}}
\newcommand{\un}{\mathrm{un}}
\newcommand{\sep}{\mathrm{sep}}
\newcommand{\diag}{\mathrm{diag}}
\newcommand{\Pic}{\mathrm{Pic}}
\newcommand{\GL}{\mathrm{GL}}
\newcommand{\SL}{\mathrm{SL}}
\newcommand{\LS}{\mathrm{LS}}
\newcommand{\GLS}{\mathrm{GLS}}
\newcommand{\GLSi}{\mathrm{GLS}_{\cap}}
\newcommand{\PGLS}{\mathrm{PGLS}}
\newcommand{\Loc}[1][S]{_{\{{#1}\}}}
\newcommand{\cl}{\mathrm{cl}}
\newcommand{\otL}{\hat{\otimes}^{\mathbb{L}}}
\newcommand{\ddpp}{\mathrm{d}'\mathrm{d}''}
\newcommand{\TC}{\mathcal{TC}}
\newcommand{\ddPP}{\mathrm{d}'_{\mathrm{P}}\mathrm{d}''_{\mathrm{P}}}
\newcommand{\PSs}{\mathcal{PS}}
\newcommand{\Gm}{\mathbb{G}_{\mathrm{m}}}
\newcommand{\End}{\mathrm{End}}
\newcommand{\Aff}[1][X]{\mathcal{M}\left(\mathcal{#1}\right)}
\newcommand{\XG}[1][X]{{#1}_{\mathrm{G}}}
\newcommand{\convC}{\xrightarrow{C}}
\newcommand{\Vect}{\mathrm{Vect}}
\newcommand{\abso}[1]{\lvert#1\rvert}
\newcommand{\Mdl}{\mathrm{Model}}
\newcommand{\cn}{\stackrel{\sim}{\longrightarrow}}
\newcommand{\sbc}{\mathbf{s}}
\newcommand{\CH}{\mathrm{CH}}
\newcommand{\GR}{\mathrm{GR}}
\newcommand{\dc}{\mathrm{d}^{\mathrm{c}}}
\newcommand{\Nef}{\mathrm{Nef}}
\newcommand{\Adj}{\mathrm{Adj}}
\newcommand{\DHm}{\mathrm{DH}}
\newcommand{\An}{\mathrm{an}}
\newcommand{\Rec}{\mathrm{Rec}}
\newcommand{\dP}{\mathrm{d}_{\mathrm{P}}}
\newcommand{\ddp}{\mathrm{d}_{\mathrm{P}}'\mathrm{d}_{\mathrm{P}}''}
\newcommand{\ddc}{\mathrm{dd}^{\mathrm{c}}}
\newcommand{\ddL}{\mathrm{d}'\mathrm{d}''}
\newcommand{\PSH}{\mathrm{PSH}}
\newcommand{\CPSH}{\mathrm{CPSH}}
\newcommand{\PSP}{\mathrm{PSP}}
\newcommand{\WPSH}{\mathrm{WPSH}}
\newcommand{\Ent}{\mathrm{Ent}}
\newcommand{\NS}{\mathrm{NS}}
\newcommand{\QPSH}{\mathrm{QPSH}}
\newcommand{\proet}{\mathrm{pro-ét}}
\newcommand{\XL}{(\mathcal{X},\mathcal{L})}
\newcommand{\ii}{\mathrm{i}}
\newcommand{\Cpt}{\mathrm{Cpt}}
\newcommand{\bp}{\bar{\partial}}
\newcommand{\ddt}{\frac{\mathrm{d}}{\mathrm{d}t}}
\newcommand{\dds}{\frac{\mathrm{d}}{\mathrm{d}s}}
\newcommand{\Ep}{\mathcal{E}^p(X,\theta;[\phi])}
\newcommand{\Ei}{\mathcal{E}^{\infty}(X,\theta;[\phi])}
\newcommand{\infs}{\operatorname*{inf\vphantom{p}}}
\newcommand{\sups}{\operatorname*{sup*}}
\newcommand{\colim}{\operatorname*{colim}}
\newcommand{\ddtz}[1][0]{\left.\ddt\right|_{t={#1}}}
\newcommand{\tube}[1][Y]{]{#1}[}
\newcommand{\ddsz}[1][0]{\left.\ddt\right|_{s={#1}}}
\newcommand{\floor}[1]{\left \lfloor{#1}\right \rfloor }
\newcommand{\dec}[1]{\left \{{#1}\right \} }
\newcommand{\ceil}[1]{\left \lceil{#1}\right \rceil }
\newcommand{\Projrel}{\mathcal{P}\mathrm{roj}}
\newcommand{\Weil}{\mathrm{Weil}}
\newcommand{\Cart}{\mathrm{Cart}}
\newcommand{\bWeil}{\mathrm{b}\mathrm{Weil}}
\newcommand{\bCart}{\mathrm{b}\mathrm{Cart}}
\newcommand{\Cond}{\mathrm{Cond}}
\newcommand{\IC}{\mathrm{IC}}
\newcommand{\IH}{\mathrm{IH}}
\newcommand{\cris}{\mathrm{cris}}
\newcommand{\Zar}{\mathrm{Zar}}
\newcommand{\HvbCat}{\overline{\mathcal{V}\mathrm{ect}}}
\newcommand{\BanModCat}{\mathcal{B}\mathrm{an}\mathcal{M}\mathrm{od}}
\newcommand{\DesCat}{\mathcal{D}\mathrm{es}}
\newcommand{\RingCat}{\mathcal{R}\mathrm{ing}}
\newcommand{\SchCat}{\mathcal{S}\mathrm{ch}}
\newcommand{\AbCat}{\mathcal{A}\mathrm{b}}
\newcommand{\RSCat}{\mathcal{R}\mathrm{S}}
\newcommand{\LRSCat}{\mathcal{L}\mathrm{RS}}
\newcommand{\CLRSCat}{\mathbb{C}\text{-}\LRSCat}
\newcommand{\CRSCat}{\mathbb{C}\text{-}\RSCat}
\newcommand{\CLA}{\mathbb{C}\text{-}\mathcal{L}\mathrm{A}}
\newcommand{\CASCat}{\mathbb{C}\text{-}\mathcal{A}\mathrm{n}}
\newcommand{\LiuCat}{\mathcal{L}\mathrm{iu}}
\newcommand{\BanCat}{\mathcal{B}\mathrm{an}}
\newcommand{\BanAlgCat}{\mathcal{B}\mathrm{an}\mathcal{A}\mathrm{lg}}
\newcommand{\AnaCat}{\mathcal{A}\mathrm{n}}
\newcommand{\LiuAlgCat}{\mathcal{L}\mathrm{iu}\mathcal{A}\mathrm{lg}}
\newcommand{\AlgCat}{\mathcal{A}\mathrm{lg}}
\newcommand{\SetCat}{\mathcal{S}\mathrm{et}}
\newcommand{\ModCat}{\mathcal{M}\mathrm{od}}
\newcommand{\TopCat}{\mathcal{T}\mathrm{op}}
\newcommand{\CohCat}{\mathcal{C}\mathrm{oh}}
\newcommand{\SolCat}{\mathcal{S}\mathrm{olid}}
\newcommand{\AffCat}{\mathcal{A}\mathrm{ff}}
\newcommand{\AffAlgCat}{\mathcal{A}\mathrm{ff}\mathcal{A}\mathrm{lg}}
\newcommand{\QcohLiuAlgCat}{\mathcal{L}\mathrm{iu}\mathcal{A}\mathrm{lg}^{\mathrm{QCoh}}}
\newcommand{\LiuMorCat}{\mathcal{L}\mathrm{iu}}
\newcommand{\Isom}{\mathcal{I}\mathrm{som}}
\newcommand{\Cris}{\mathcal{C}\mathrm{ris}}
\newcommand{\Pro}{\mathrm{Pro}-}
\newcommand{\Fin}{\mathcal{F}\mathrm{in}}
\newcommand{\norms}[1]{\left\|#1\right\|}
\newcommand{\HPDDiff}{\mathbf{D}\mathrm{iff}}
\newcommand{\Menn}[2]{\begin{bmatrix}#1\\#2\end{bmatrix}}
\newcommand{\Fins}{\widehat{\Vect}^F}
\newcommand\blfootnote[1]{%
  \begingroup
  \renewcommand\thefootnote{}\footnote{#1}%
  \addtocounter{footnote}{-1}%
  \endgroup
}

\externaldocument[Introduction-]{Introduction}
%One variable complex analysis
%Several variables complex analysis
\externaldocument[Topology-]{Topology-Bornology}
\externaldocument[Banach-]{Banach-Rings}
\externaldocument[Commutative-]{Commutative-Algebra}
\externaldocument[Local-]{Local-Algebras}
\externaldocument[Complex-]{Complex-Analytic-Spaces}
%Properties of space
\externaldocument[Morphisms-]{Morphisms}
%Differential calculus
%GAGA
%Hilbert scheme complex analytic version

%Complex differential geometry

\externaldocument[Affinoid-]{Affinoid-Algebras}
\externaldocument[Berkovich-]{Berkovich-Analytic-Spaces}


\bibliography{Ymir}

\endinput
\title{Banach rings}
\begin{document}
\maketitle
\tableofcontents



\section{Introduction}\label{sec-introduction}

This section conerns the theory of Banach algebras. Our references are \cite{Berk12} and \cite{BGR}.

In this chapter, all rings are assumed to be commutative.

\section{Semi-normed Abelian groups}

\begin{definition}
    Let $A$ be an Abelian group. A \emph{semi-norm} on $A$ is a function $\|\bullet\|:A\rightarrow [0,\infty]$ satisfying
    \begin{enumerate}
        \item $\|0\|=0$;
        \item $\|f-g\|\leq \|f\|+\|g\|$ for all $f,g\in A$.
    \end{enumerate}
    A semi-norm $\|\bullet\|$ on $A$ is a \emph{norm} if moreover the following conditions is satisfied:
    \begin{enumerate}
        \item[(0)] if $\|f\|=0$ for some $f\in A$, then $f=0$.
    \end{enumerate}
    We write
    \[
        \ker \|\bullet\|=\{a\in A:\|a\|=0\}.
    \]

    A semi-norm $\|\bullet\|$ on $A$ is \emph{non-Archimedean} or \emph{ultra-metric} if Condition~(2) can be replaced by
    \begin{enumerate}[resume]
        \item[(2')] $\|f-g\|\leq \max\{\|f\|,\|g\|\}$ for all $f,g\in A$.
    \end{enumerate}
\end{definition}
\begin{definition}
    A \emph{semi-normed Abelian group} (resp. \emph{normed Abelian group}) is a pair $(A,\|\bullet\|)$ consisting of an Abelian group $A$ and a semi-norm (resp. norm) $\|\bullet\|$ on $A$. When $\|\bullet\|$ is clear from the context, we also say $A$ is a semi-normed Abelian group (resp. normed Abelian group).
\end{definition}


\begin{definition}\label{def-quotsubsemnorm}
    Let $(A,\|\bullet\|_A)$ be a semi-normed Abelian group and $B\subseteq A$ be a subgroup. Then we define the \emph{quotient semi-norm} $\|\bullet\|_{A/B}$ on $A/B$ as follows:
    \[
      \|a+B\|_{A/B}:=\inf\{\|a+b\|_A:b\in B\}  
    \]
    for all $a+B\in A/B$.

    We define the \emph{subgroup semi-norm} on $B$ as follows:
    \[
        \|b\|_B=\|b\|_A
    \]
    for all $b\in B$.
\end{definition}

\begin{definition}
    Let $A$ be an Abelian group and $\|\bullet\|$, $\|\bullet\|'$ be two seminorms on $A$. We say $\|\bullet\|$ and $\|\bullet\|'$ are \emph{equivalent} if there is a constant $C>0$ such that
    \[
      C^{-1}\|f\|\leq \|f\|'\leq C\|f\|   
    \]
    for all $f\in A$.
\end{definition}

\begin{definition}\label{def-admissiblemorphism}
    Let $(A,\|\bullet\|_A)$, $(B,\|\bullet\|_B)$ be semi-normed Abelian groups. A homomorphism $\varphi:A\rightarrow B$ is said to be 
    \begin{enumerate}
        \item \emph{bounded} if there is a constant $C>0$ such that $\|\varphi(f)\|_B\leq C \|f\|_{A}$ for any $f\in A$;
        \item \emph{admissible} if the quotient semi-norm on $A/\ker \varphi$ is equivalent to the subspace semi-norm on $\Img \varphi$.
    \end{enumerate}
\end{definition}
Observe that an admissible homomorphism is always bounded.

Next we study the topology defined by a semi-norm.
\begin{lemma}\label{lma-pmetricinducedbyseminorm}
    Let $(A,\|\bullet\|)$ be a semi-normed Abelian group. Define
    \[
      d(a,b)=\|a-b\|  
    \]
    for $a,b\in A$. Then $\|\bullet\|$ is a pseudo-metric on $A$. This psuedo-metric is a metric if and only if $\|\bullet\|$ is a norm.

    Let $\hat{A}$ be the metric completion of $A$, then there is a norm $\|\bullet\|$ on $\hat{A}$ inducing its metric. Moreover, the natural homomorphism $A\rightarrow \hat{A}$ is an isometric homomorphism with dense image.
\end{lemma}
\begin{proof}
    This is clear from the definitions.
\end{proof}
We always endow $A$ with the topology induced by the psuedo-metric $d$.

\begin{proposition}\label{prop-bddimplycont}
    Let $f:A\rightarrow B$ be a homomorphism between semi-normed Abelian groups. Assume that $f$ is bounded, then it is continuous.
\end{proposition}
The converse is not true.
\begin{proof}
    Clear from the definition.
\end{proof}

\begin{proposition}\label{prop-epsilondenseimpliesdense}
    Let $(A,\|\bullet\|)$ be a normed Abelian group and $B$ be a subgroup of $A$. Assume that there is $\epsilon\in (0,1)$ such that for each $a\in A$, there is $b\in B$ such that
    \[
      \|a+b\|\leq \epsilon \|a\|.  
    \]
    Then $B$ is dense in $A$.
\end{proposition}
\begin{proof}
    Assume to the contrary that there exists $a\in A$ so that
    \[
      c:=\inf_{b\in B}\|a-b\|>0.  
    \]
    Choose $b_1\in B$ so that
    \[
      \|a+b_1\|<\epsilon^{-1}c.
    \]
    By our hypothesis, there is $b_2\in B$ such that
    \[
      \|a+b_1+b_2\|\leq \epsilon\|a+b_1\|<c.  
    \]
    This is a contradiction.
\end{proof}

\begin{definition}\label{def-completionAbgroups}
    Let $(A,\|\bullet\|)$ be a semi-normed Abelian group. The normed Abelian group $(\hat{A},\|\bullet\|)$ constructed in \cref{lma-pmetricinducedbyseminorm} is called the \emph{completion} of $(A,\|\bullet\|)$.
\end{definition}



\section{Semi-normed rings}

\begin{definition}\label{def-seminormring}
    Let $A$ be a ring. A \emph{semi-norm} $\|\bullet\|$ on $A$ is a semi-norm $\|\bullet\|$ on the underlying additive group satisfying the following extra properties: 
    \begin{enumerate}
        \item[(3)]  $\|1\|=1$;
        \item[(4)] for any $f,g\in A$, $\|fg\|\leq \|f\|\cdot \|g\|$.
    \end{enumerate}
    
    A semi-norm $\|\bullet\|$ on $A$ is called \emph{power-multiplicative} if $\|f\|^n=\|f^n\|$ for all $f\in A$ and $n\in \mathbb{N}$.

    A semi-norm $\|\bullet\|$ on $A$ is called \emph{multiplicative} if $\|fg\|=\|f\|\|g\|$ for all $f,g\in A$.

\end{definition}

\begin{definition}
    A \emph{semi-normed ring} (resp. \emph{normed ring}) is a pair $(A,\|\bullet\|)$ consisting of a ring $A$ and a semi-norm (resp. norm) $\|\bullet\|$ on $A$. When $\|\bullet\|$ is clear from the context, we also say $A$ is a semi-normed  ring (resp. normed ring).
\end{definition}

\begin{definition}
    Let $(A,\|\bullet\|)$ be a semi-normed ring. An element $a\in A$ is \emph{multiplicative} if $a\not\in \ker\|\bullet\|$ and for any $x\in A$,
    \[
      \|ax\|=\|a\|\cdot\|x\|.  
    \]
\end{definition}

\begin{definition}\label{def-topnilpowerbdd}
    Let $(A,\|\bullet\|)$ be a normed ring. An element $a\in A$ is \emph{power-bounded} if $\{|a^n|:n\in \mathbb{N}\}$ is bounded in $\mathbb{R}$. The set of power-bounded elements in $A$ is denoted by $\mathring{A}$.

    An element $a\in A$ is called \emph{topologically nilpotent} if $a^n\to 0$ as $n\to\infty$. The set of topologically nilpotent elements in $A$ is denoted by $\check{A}$.


\end{definition}

\begin{proposition}\label{prop-topnilring}
    Let $(A,\|\bullet\|)$ be a non-Archimedean normed ring. Then $\mathring{A}$ is a subring of $A$ and $\check{A}$ is an ideal in $\mathring{A}$. Moreover, $\mathring{A}$, $\check{A}$ are open and closed in $A$.
\end{proposition}
\begin{proof}
    Choose $a,b\in \mathring{A}$, by definition, there is a constant $C>0$ so that for any $n\in \mathbb{N}$,
    \[
      \|a^n\|\leq C,\quad \|b^n\|\leq C.  
    \]
    It follows that
    \[
      \|(ab)^n\|\leq \|a^n\|\cdot \|b^n\|\leq C^2  
    \]
    and
    \[
      \|(a-b)^n\|\leq  \max_{i=0,\ldots,n}\|a^ib^{n-i}\|\leq C^2. 
    \]
    So $\mathring{A}$ is a subring.

    Next we show that $\check{A}$ is an ideal in $\mathring{A}$.
    On the other hand, take $c\in \check{A}$, then
    \[
        \|(ac)^n\|\leq \|a^n\|\cdot \|c^n\|\leq C\|c^n\|  
    \]
    But $\|c^n\|\to 0$ as $n\to \infty$, hence $ac\in \check{A}$. 
    
    On the other hand, consider $c,d\in \check{A}$, we need to show $c-d\in \check{A}$. Choose $C>0$ so that
    \[
        \|a^n\|\leq C,\quad \|b^n\|\leq C  
    \]
    for all $n\in \mathbb{N}$. Fix $\epsilon>0$, then there is $m\in \mathbb{N}$ so that for any $k\geq m$,
    \[
        \|a^k\|\leq \epsilon C^{-1},\quad   \|b^k\|\leq \epsilon C^{-1}.
    \]
    In particular, for $k\geq 2m$, we have
    \[
        \|(a-b)^k\|\leq \max_{i=0,\ldots,k} \|a^i\|\cdot\|b^{k-i}\|\leq \epsilon.
    \]
    It follows that $a-b\in \check{A}$. 
    This proves that $\check{A}$ is an ideal in $\mathring{A}$.

    In order to see $\check{A}$ is open and closed in $A$, observe that it is a subgroup of $A$, so it suffices to show that $\check{A}$ is open in $A$. It suffices to show that
    \[
        \{a\in A:\|a\|<1\}\subseteq  \check{A}. 
    \]
    But this is obvious, if $\|a\|<1$, then $\|a^n\|\leq \|a\|^n$ for all $n\in \mathbb{N}$, it follows that $a^n\to 0$ as $n\to\infty$, namely, $a\in \check{A}$.

    As $\check{A}$ is a subgroup of $\mathring{A}$, it follows that $\mathring{A}$ is both open and closed.
\end{proof}

\begin{definition}\label{def-reduction}
    Let $(A,\|\bullet\|)$ be a non-Archimedean normed ring. We define the \emph{reduction} of $A$ as $\tilde{A}=\mathring{A}/\check{A}$. The map $\mathring{A}\rightarrow \tilde{A}$ is called the \emph{reduction map}. We usually denote the reduction map by $a\mapsto \tilde{a}$.
\end{definition}
This definition makes sense thanks to \cref{prop-topnilring}.

\begin{definition}
    Let $A$ be a ring. A \emph{semi-valuation} on $A$ is a multiplicative semi-norm on $A$. A semi-valuation on $A$ is a \emph{valuation} on $A$ if its underlying semi-norm of Abelian groups is a norm. 
\end{definition}

\begin{definition}
    A \emph{semi-valued ring} (resp. \emph{valued ring}) is a pair $(A,\|\bullet\|)$ consisting of a ring $A$ and a semi-valuation (resp. valuation) $\|\bullet\|$ on $A$. When $\|\bullet\|$ is clear from the context, we also say $A$ is a semi-valued  ring (resp. valued ring).

    A semi-valued ring (resp. valued ring) $(A,\|\bullet\|)$ is called a \emph{semi-valued field} (resp. \emph{valued field}) if $A$ is a field.
\end{definition}

\iffalse
\begin{lemma}\label{lma-gausslemmav1}
    Let $(A,\|\bullet\|)$ be a normed ring. Assume that 
    \begin{enumerate}
        \item For any $a\in A$, $a\neq 0$, there is a multiplicative element $m\in A$ and $s\in \mathbb{Z}_{>0}$ such that $\|ma^s\|=\|m\|\cdot \|a\|^s=1$;
        \item $\tilde{A}$ is an integral domain.
    \end{enumerate}
    Then $\|\bullet\|$ is a valuation.
\end{lemma}
\begin{proof}
    Take $a_1,a_2\in A$ such that $|a_1a_2|<|a_1|\cdot |a_2|$. By (1), we can
    choose multiplicative elements $m_1,m_2\in A$ and $s_1,s_2\in \mathbb{Z}_{>0}$ so that
    \[
        \|m_1a_1^{s_1}\|=\|m_1\|\cdot\|a_1\|^{s_1}=1,\quad \|m_2a_2^{s_2}\|=\|m_2\|\cdot\|a_2\|^{s_2}=1.  
    \]
    In particular, $m_ia_i^{s_i}\in \mathring{A}\setminus \check{A}$ for $i=1,2$. We may assume that $s_2\geq s_1$. As $m_1,m_2$ are multiplicative, we find
    \[
        \begin{split}
            \|(m_1a_1^{s_1})(m_2a_2^{s_2})\|=\|m_1\|\cdot\|m_2\|\cdot \|a_1^{s_1}a_2^{s_2}\|\leq \|m_1\|\cdot\|m_2\|\cdot \|a_1a_2\|^{s_1}\|a_2\|^{s_2-s_1} \\
            < \|m_1\|\cdot\|m_2\|\cdot (\|a_1\| \cdot\|a_2\|)^{s_1}\|a_2\|^{s_2-s_1}=1.
        \end{split}
    \]
    This contradicts (2).
\end{proof}
\fi

\section{Banach rings}

\begin{definition}
    A \emph{Banach ring} is a normed ring that is complete with respect to the metric defined in \cref{lma-pmetricinducedbyseminorm}.
\end{definition}

\begin{definition}\label{def-completionring}
    Let $A$ be a semi-normed ring. There is an obvious ring structure on the completion $\hat{A}$ of $A$ defined in \cref{def-completionAbgroups}. We call the resulting Banach ring the \emph{completion} of $A$.
\end{definition}

\begin{proposition}\label{prop-inversesmallelementBanachring}
    Let $(A,\|\bullet\|)$ be a Banach ring and $f\in A$. Assume that $\|f\|<1$, then $1-f$ is invertible.
\end{proposition}
\begin{proof}
    Define 
    \[
        g=\sum_{i=0}^{\infty}f^i.
    \]
    From our assumption, the series converges and $g\in A$. It is elementary to check that $g$ is the inverse of $1-f$.
\end{proof}
In the non-Archimedean case, we have a stronger result:

\begin{proposition}\label{prop-inversesmallelementBanachring2}
    Let $(A,\|\bullet\|)$ be a non-Archimedean Banach ring and $f\in \check{A}$. Then $1-f$ is invertible. Moreover, $(1-f)^{-1}$ can be written as $1+z$ for some $z\in \check{A}$.
\end{proposition}
\begin{proof}
    Define 
    \[
        g=\sum_{i=0}^{\infty}f^i.
    \]
    From our assumption, the series converges and $g\in A$. It is elementary to check that $g$ is the inverse of $1-f$. Moreover, in view of \cref{prop-topnilring} as for any $i\geq 1$, $f^i\in \check{A}$, the same holds for their sum, we conclude the final assertion.
\end{proof}

\begin{corollary}\label{cor-Banachringinvopen}
    Let $(A,\|\bullet\|)$ be a Banach ring. Then the set of invertible elements in $A$ is open.
\end{corollary}
\begin{proof}
    Let $x\in A$ be an invertible element. It suffices to show that for any $y\in A$, $|y|<1/(\|x^{-1}\|)$, $y+x$ is invertible. For this purpose, it suffices to show that $1+x^{-1}y$ is invertible. But this follows from \cref{prop-inversesmallelementBanachring}.
\end{proof}

\begin{corollary}\label{cor:maximalidealclosedinBanachring}
    Let $A$ be a Banach ring and $\mathfrak{m}$ be a maximal ideal in $A$. Then $\mathfrak{m}$ is closed.
\end{corollary}
\begin{proof}
    The closure $\bar{\mathfrak{m}}$ is obviously an ideal in $A$. We need to show that $\mathfrak{m}\neq A$. Namely, $1$ is not in the closure of $\mathfrak{m}$. But clearly, $\mathfrak{m}$ is contained in the set of non-invertible elements, the latter being closed by \cref{cor-Banachringinvopen}. So we conclude.
\end{proof}

\begin{lemma}\label{lma-unitmoduloreduction}
    Let $A$ be a non-Archimedean Banach ring. An element $a\in \mathring{A}$ is a unit in $\mathring{A}$ if and only if $\tilde{a}$ is a unit in $\tilde{A}$.
\end{lemma}
\begin{proof}
    The direct implication is trivial. Conversely, assume that $a\in \mathring{A}$ and there is an element $b\in \mathring{A}$ such that
    \[
        \tilde{a}\tilde{b}=1.  
    \]
    Then $1-ab\in \check{A}$. It follows from \cref{prop-inversesmallelementBanachring2} that $ab$ is a unit in $\mathring{A}$ and hence $a$ is a unit in $\mathring{A}$.
\end{proof}

\begin{definition}
    Let $(A,\|\bullet\|)$ be a Banach ring. We define the \emph{spectral radius} $\rho=\rho_A:A\rightarrow [0,\infty)$ as follows:
    \[
        \rho(f)=\inf_{n\geq 1} \|f^n\|^{1/n},\quad f\in A.
    \]
\end{definition}

\begin{lemma}
    Let $(A,\|\bullet\|)$ be a Banach ring. Then for any $f\in A$, we have 
    \[
        \rho(f)=  \lim_{n\to \infty} \|f^n\|^{1/n}.
    \]
\end{lemma}
\begin{proof}
    This follows from the multiplicative version of Fekete's lemma.
\end{proof}





\begin{example}
    The ring $\mathbb{C}$ with its usual norm $|\bullet|$ is a Banach ring. In fact, $(\mathbb{C},|\bullet|)$ is a complete valued field.
\end{example}

\begin{example}\label{ex-convseriesradius}
    For any Banach ring $(A,\|\bullet\|)$, any $n\in \mathbb{N}$ and any $r=(r_1,\ldots,r_n)\in \mathbb{R}^n_{>0}$, we define $A\langle r^{-1}z \rangle =A\langle r_1^{-1}z_1,\ldots, r_n^{-1}z_n\rangle$ as the subring of $A[[z_1,\ldots,z_n]]$ consisting of formal power series
    \[
      f=\sum_{\alpha\in \mathbb{N}^{n}} a_{\alpha}z^{\alpha},\quad a_{\alpha}\in A  
    \]
    such that
    \[
        \|f\|_r:=   \sum_{\alpha\in \mathbb{N}^{n}} \|a_{\alpha}\|r^{\alpha}<\infty.
    \]
    We will verify in \cref{prop-convseriesradiusBanach} that $(A\langle r^{-1}z \rangle ,\|\bullet\|_r)$ is a Banach ring.

    When $r=(1,\ldots,1)$, we omit $r^{-1}$ from our notations.
\end{example}

\begin{proposition}\label{prop-convseriesradiusBanach}
    In the setting of \cref{ex-convseriesradius}, $(A\langle r^{-1}z \rangle  ,\|\bullet\|_r)$ is a Banach ring.
\end{proposition}
\begin{proof}
    By induction, we may assume that $n=1$.

    It is obvious that $\|\bullet\|_r$ is a norm on the undelrying Abelian group. To see that $\|\bullet\|_r$ is a norm on the ring $A\langle r^{-1}z \rangle $, we need to verify the condition in \cref{def-seminormring}. Condition~(3) in \cref{def-seminormring} is obvious. 
    Let us consider Condition~(4). 
    Let 
    \[
        f=\sum_{i=0}^{\infty} a_i z^i,\quad g=\sum_{j=0}^{\infty} b_j z^j
    \]
    be two elements in $A\langle r^{-1}z \rangle $. Then
    \[
        fg=\sum_{k=0}^{\infty} \left(\sum_{i+j=k}a_i b_j\right) z^k. 
    \]
    We compute
    \[
        \|fg\|_r=  \sum_{k=0}^{\infty} \left\|\sum_{i+j=k}a_i b_j\right\| r^k\leq \sum_{k=0}^{\infty} \left(\sum_{i+j=k}\|a_i\|\cdot \|b_j\|\right) r^k=\|f\|_r\cdot \|g\|_r. 
    \]
    It remains to verify that $A\langle r^{-1}z \rangle$ is complete. 


    For this purpose, take a Cauchy sequence 
    \[
        f^b=\sum_{i=0}^{\infty}a^b_i z^i\in A\langle r^{-1}z \rangle
    \]
    for $b\in \mathbb{N}$. Then for each $i$, the coefficients $(a^b_i)_b$ is a Cauchy sequence in $A$. Let $a_i$ be the limit of $a^b_i$ as $b\to\infty$ and set 
    \[
        f=\sum_{i=0}^{\infty}a_i z^i\in A[[z]].
    \]
    We need to show that $f\in A\langle r^{-1}z \rangle$ and $f^b\to f$. 
    
    Fix a constant $\epsilon>0$. There is $m=m(\epsilon)>0$ such that for all $j\geq m$ and all $k\geq 0$, we have
    \[
        \sum_{i=0}^{\infty}\|a^{j+k}_i-a^{j}_i\|r^{i}<\epsilon/2.
    \]
    In particular, for any $s>0$, we have
    \[
        \sum_{i=0}^{s}\|a_i-a^{j}_i\|r^{i}\leq \sum_{i=0}^{s}\|a_i-a^{j+k}_i\|r^{i}+\sum_{i=0}^{s}\|a^j_i-a^{j+k}_i\|r^{i}\leq \sum_{i=0}^{s}\|a_i-a^{j+k}_i\|r^{i}+\epsilon/2.
    \]
    When $k$ is large enough, we can guarantee that
    \[
        \sum_{i=0}^{s}\|a_i-a^{j+k}_i\|r^{i}<\epsilon/2.  
    \]
    So
    \[
        \sum_{i=0}^{s}\|a_i-a^{j}_i\|r^{i}\leq \epsilon.  
    \]
    Let $s\to\infty$, we find
    \[
        \|f-f^j\|_r\leq \sum_{i=0}^{\infty}\|a_i-a^{j}_i\|r^{i}\leq \epsilon.  
    \]
    In particular, $\|f\|_r<\infty$ and $f^j\to f$ as $j\to\infty$.
\end{proof}


\begin{example}\label{ex-strictconvseriesradius}
    For any non-Archimedean Banach ring $(A,\|\bullet\|)$, any $n\in \mathbb{N}$ and any $r=(r_1,\ldots,r_n)\in \mathbb{R}^n_{>0}$, we define $A\{ r^{-1}T \} =A\{ r_1^{-1}T_1,\ldots, r_n^{-1}T_n \}$ as the subring of $A[[T_1,\ldots,T_n]]$ consisting of formal power series
    \[
      f=\sum_{\alpha\in \mathbb{N}^{n}} a_{\alpha}T^{\alpha},\quad a_{\alpha}\in A  
    \]
    such that $\|a_{\alpha}\|r^{\alpha}\to 0$ as $|\alpha|\to\infty$. We set
    \[
        \|f\|_r:=   \max_{\alpha\in \mathbb{N}^{n}} \|a_{\alpha}\|r^{\alpha}.
    \]
    We will verify in \cref{prop-strictconvseriesradiusBanach} that $(A\langle r^{-1}T \rangle ,\|\bullet\|_r)$ is a Banach ring.

    The semi-norm $\|\bullet\|_r$ is called the \emph{Gauss norm}. 
\end{example}

\begin{proposition}\label{prop-strictconvseriesradiusBanach}
    In the setting of \cref{ex-strictconvseriesradius}, $(A\{ r^{-1}T \}  ,\|\bullet\|_r)$ is a Banach ring.

    Moreover, if the norm $\|\bullet\|$ on $A$ is a valuation, so is $\|\bullet\|_r$.
\end{proposition}
The second part is usually known as the \emph{Gauss lemma}.
\begin{proof}
    By induction on $n$, we may assume that $n=1$.

    The proof of the fact that $\|\bullet\|_r$ is a norm is similar to that of \cref{prop-convseriesradiusBanach}. We leave the details to the readers.

    Next we argue that $(A\{ r^{-1}T \}  ,\|\bullet\|_r)$  is complete. 
    Take a Cauchy sequence 
    \[
        f^b=\sum_{i=0}^{\infty}a^b_i T^i\in A\{ r^{-1}T \}
    \]
    for $b\in \mathbb{N}$. As 
    \[
        \|a^b_i-a^{b'}_i\|r^i\leq \|f^b-f^{b'}\|_r
    \]
    for any $i,b,b'\geq 0$, it follows that for any $i\geq 0$, $\{a^b_i\}_b$ is a Cauchy sequence. Let $a_i\in A$ be its limit and set 
    \[
        f=\sum_{i=0}^{\infty}a_i T^i\in A[[T]].
    \]
    We need to show that $f\in A\{ r^{-1}T \}$ and $f^b\to f$.
    
    Fix $\epsilon>0$. We can find $m=m(\epsilon)>0$ such that for all $j\geq m$ and all $k\geq 0$, 
    \[
      \|f^j-f^{j+k}\|_r\leq \epsilon.  
    \]
    It follows that $\|a^j_i-a^{j+k}_i\|r^i\leq \epsilon$ for all $i\geq 0$. Let $k\to\infty$, we find
    \[
        \|a^j_i-a_i\|r^i\leq \epsilon
    \]
    for all $i\geq 0$. Fix $j\geq 0$, take $i$ large enough so that $|a^j_i|r^i<\epsilon$. Then $\|a_i\|r^i\leq \epsilon$. So we find $f\in A\{r^{-1}T\}$. On the other hand,
    \[
         \|f-f^j\|_r=\max_{i} \|a^j_i-a_i\|r^i\leq \epsilon.  
    \]
    This proves that $f^j\to f$.

    Now assume that $\|\bullet\|$ is a valuation, we verify that $\|\bullet\|_r$ is also a valuation. Again, we may assume that $n=1$. Take two elements $f,g\in A\{r^{-1}T\}$:
    \[
        f=\sum_{i=0}^{\infty}a_i T^i,\quad g=\sum_{j=0}^{\infty}b_j T^j.
    \]
    As we have already shown $|fg|_r\leq |f|_r|g|_r$, it suffices to check the reverse inequality. For this purpose, choose the minimal indices $i$, $j$ so that
    \[
        \|f\|_r=\|a_i\|r^i,\quad \|g\|_r=\|b_j\|r^j.  
    \]
    Write
    \[
        fg=\sum_{k=0}^{\infty} \left(\sum_{p+q=k}a_pb_q  \right)T^k.
    \]
    Then we claim that
    \[
        \left\|\sum_{p+q=k}a_pb_q\right\| r^k =  \|f\|_r \|g\|_r
    \] 
    when $k=i+j$.
    This implies the desired inequality.  Of course, we may assume that $a_i\neq 0$ and $b_j\neq 0$ as otherwise there is nothing to prove.
    To verify our claim, it suffices to observe that for $(p,q)\neq (i,j)$, $r+s=i+j$, say $p<i$ and $q>j$, we have
    \[
        \|a_p b_q\|r^k=\|a_p\|r^p\cdot \|b_q\|r^q<\|a_i\|r^i\cdot \|b_j\|r^j.
    \]
    So 
    \[
        \|a_p b_q\|< \|a_i b_j\|.
    \]
    Since the valuation on $A$ is non-Archimedean, it follows that 
    \[
        \|\sum_{p+q=k}a_pb_q\|=\|a_ib_j\|.  
    \]
    Our claim follows.
\end{proof}

\begin{proposition}\label{prop-Tateunivprop}
    Let $A$, $B$ be a non-Archimedean Banach ring and $f:A\rightarrow B$ be a continuous homomorphism. Then for any $b\in \mathring{B}$, there is a unique continuous homomorphism $F:A\{T\}\rightarrow B$ extending $f$ and sending $T$ to $b$. 
\end{proposition}
\begin{proof}
    From the continuity and the fact that $A[T]$ is dense in $A\{T\}$, $F$ is clearly unique. To prove the existence, we define $F$ directly: consider $g=\sum_{i=0}^{\infty}a_i T^i\in A\{T\}$, we define
    \[
        F(g):=\sum_{i=0}^{\infty} f(a_i)f^i.      
    \]
    As $f_i\in \mathring{A}$ and $a_i\to 0$, the right-hand side is well-defined. It is straightforward to check that $F$ is a continuous homomorphism.
\end{proof}

\begin{proposition}\label{prop-strictlyconvcirccheck}
    For any non-Archimedean Banach ring $(A,\|\bullet\|)$, we have 
    \[
        (A\{T\})^{\circ}=\mathring{A}\{T\},\quad (A\{T\})^{\check{}}=\check{A}\{T\}.  
    \]
\end{proposition}
For the definitions of $\mathring{\bullet}$ and $\check{\bullet}$, we refer to \cref{def-topnilpowerbdd}.
\begin{proof}
    We first show that
    \[
        \mathring{A}\{T\}\subseteq (A\{T\})^{\circ}.
    \]
    Let $f\in \mathring{A}\{T\}$. We expand $f$ as
    \[
        f=\sum_{i=0}^{\infty}a_i T^i,\quad a_i\in  \mathring{A}. 
    \] 
    Then for each $i,j\in \mathbb{N}$, $\|a_iT^i\|_1^j=\|a_i\|^j$. So for each $i\in \mathbb{N}$, $a_iT^i\in (A\{T\})^{\circ}$. By \cref{prop-topnilring}, it follows that $f\in (A\{T\})^{\circ}$.

    Next we prove the reverse inclusion. Take $f\in (A\{T\})^{\circ}$, suppose by contrary that $f\not\in \mathring{A}\{T\}$. Expand $f$ as
    \[
        f=\sum_{i=0}^{\infty}a_i T^i,\quad a_i\in  A. 
    \] 
    We can take a minimal $m\in \mathbb{N}$ so that $a_m\not\in \mathring{A}$. Then $\sum_{i=0}^{m-1}a_iT^i\in \mathring{A}\{T\}\subseteq (A\{T\})^{\circ}$ by what we have proved. It follows that 
    \[
        g:=f-\sum_{i=0}^{m-1}a_iT^i=\sum_{i=m}^{\infty}a_iT^i\in (A\{T\})^{\circ}.
    \]
    Then it follows that
    \[
        \|g^j\|\geq \|a_m^j\|  
    \] 
    for any $j\in \mathbb{N}$. It follows that $a_m\in \mathring{A}$, which is a contradiction.

    Next we show that 
    \[
        \check{A}\{T\}\subseteq (A\{T\})^{\check{}}.
    \]
    Let $f\in \check{A}\{T\}$. We expand $f$ as
    \[
        f=\sum_{i=0}^{\infty}a_i T^i,\quad a_i\in  \check{A}. 
    \] 
    Then for each $i,j\in \mathbb{N}$, $\|a_iT^i\|_1^j=\|a_i\|^j$. So for each $i\in \mathbb{N}$, $a_iT^i\in (A\{T\})^{\check{}}$. By \cref{prop-topnilring}, it follows that $f\in (A\{T\})^{\check{}}$.

    Conversely, take $f\in (A\{T\})^{\check{}}$, suppose by contrary that $f\not\in \check{A}\{T\}$. Expand $f$ as
    \[
        f=\sum_{i=0}^{\infty}a_i T^i,\quad a_i\in  A. 
    \] 
    We can take a minimal $m\in \mathbb{N}$ so that $a_m\not\in \check{A}$. Then $\sum_{i=0}^{m-1}a_iT^i\in \check{A}\{T\}\subseteq (A\{T\})^{\check{}}$ by what we have proved. It follows that 
    \[
        g:=f-\sum_{i=0}^{m-1}a_iT^i=\sum_{i=m}^{\infty}a_iT^i\in (A\{T\})^{\check{}}.
    \]
    Then it follows that
    \[
        \|g^j\|\geq \|a_m^j\|  
    \] 
    for any $j\in \mathbb{N}$. It follows that $a_m\in \check{A}$, which is a contradiction.
\end{proof}

\begin{corollary}\label{cor-reductionstrictlyconv}
    For any non-Archimedean Banach ring $(A,\|\bullet\|)$, we have a canonical isomorphism
    \[
        \widetilde{A\{T\}}\cong \tilde{A}[T].
    \]
    The natural map ${A\{T\}}^{\circ}\rightarrow \widetilde{A\{T\}}$ corresponds to a homomorphism $\mathring{A}\{T\}\rightarrow \tilde{A}[T]$ extending the homomorphism $\mathring{A}\rightarrow \tilde{A}$ and sending $T$ to $T$.
\end{corollary}
\begin{proof}
    Let $f=\sum_{i=0}^{\infty}a_i T^i\in {A\{T\}}^{\circ}$. 
    Then $a_i\in \mathring{A}$ by \cref{prop-strictlyconvcirccheck}. But $\|a_i\|\to 0$ as $i\to\infty$, so $a_i\in \check{A}$ for almost all $i$. It follows that the image of $f$ in $\widetilde{A\{T\}}$ is the same as the image of an element from $\mathring{A}[T]$. On the other hand, for each $f\in \tilde{A}[T]$, we can expand $f=a_NT^N+\cdots+a_1T^1+a_0$ with $a_N\in \tilde{A}$. Lift each $a_i$ to $b_i\in \mathring{A}$. Then the image of $b_NT^N+\cdots+b_1T^1+b_0$ under the reduction corresponds to $f$. The assertions follow.
\end{proof}

\begin{corollary}\label{cor-unitsstrictlyconv}
    Let $(A,\|\bullet\|)$ be a non-Archimedean Banach ring. An element $f=\sum_{i=0}^{\infty}a_i T^i\in \mathring{A}\{T\}$ is a unit in $\mathring{A}\{T\}$ if and only if $a_0$ is a unit in  $\mathring{A}$ and $a_i\in \check{A}$ for all $i>0$.
\end{corollary}
\begin{proof}
By \cref{prop-strictconvseriesradiusBanach}, we know that $A\{T\}$ is complete.
By \cref{lma-unitmoduloreduction} and \cref{prop-strictlyconvcirccheck}, $f$ is a unit in $\mathring{A}\{T\}$ if and only if $\sum_{i=0}^{\infty} \tilde{a}_i T^i$  is a unit in $\tilde{A}[T]$. By \cref{lma-unitmoduloreduction} again, $a_0$ is a unit in $A$ if and only if $\tilde{a_0}$ is a unit in $\tilde{A}$. So we are reduced to argue that units in $\tilde{A}[T]$ are exactly units in $\tilde{A}$. This follows from the general fact about units in polynomial rings over a reduced ring.
\end{proof}

\textcolor{red}{The lemma needs to be places elsewhere.}
\begin{lemma}
    Let $R$ be a commutative ring. A polynomial $a_0+a_1X+\cdots+a_nX^n\in R[X]$ is a unit if and only if $a_0$ is a unit in $R$ and $a_1,\ldots,a_n$ are nilpotents.
\end{lemma}

\section{Semi-normed modules}

\begin{definition}
    Let $(A,\|\bullet\|_A)$ be a normed ring. A \emph{semi-normed $A$-module} (resp. \emph{normed $A$-module}) is a pair $(M,\|\bullet\|_M)$ consisting of a $A$-module $M$ and a semi-norm (resp. norm) on the underlying Abelian group of $M$ such that there is a constant $C>0$ such that
    \[
      \|fm\|_M \leq C \|f\|_A \|m\|_M
    \]
    for all $f\in A$ and $m\in M$. When $\|\bullet\|_M$ is clear from the context, we say $M$ is a semi-normed $A$-module (resp. normed $A$-module).

    An $A$-module homomorphism $\varphi:M\rightarrow N$ between two semi-normed $A$-modules $M$ and $N$ is \emph{bounded} if the homomorphism of the underlying semi-normed Abelian groups is bounded in the sense of \cref{def-admissiblemorphism}.

    A \emph{Banach $\mathcal{A}$-module} is a normed $A$-module which is complete with respect to the metric \cref{lma-pmetricinducedbyseminorm}.

    We denote by $\BanCat_A$ the category of Banach $A$-modules with bounded $A$-module homomorphisms as morphisms.
\end{definition}

\begin{definition}
    Let $A$ be a semi-normed ring and $M$ be a semi-normed $A$-module. 
    There is an obvious $\hat{A}$-module structure on the completion $\hat{M}$ of $A$ defined in \cref{def-completionAbgroups}. We call the resulting Banach module the \emph{completion} of $M$.
\end{definition}

\begin{definition}\label{def-tensorproduct}
    Let $A$ be a non-Archimedean semi-normed ring. Consider semi-normed $A$-modules $(M,\|\bullet\|_M)$ and $(N,\|\bullet\|_N)$. We define the \emph{tensor product} of  $(M,\|\bullet\|_M)$ and $(N,\|\bullet\|_N)$ as the semi-normed $A$-module $(M\otimes N,\|\bullet\|_{M\otimes N})$, where
    \[
        \|x\|_{M\otimes N}=\inf \max_i (\|m_i\|_M\cdot \|n_i\|_N),
    \]
    where the infimum is taken over all decompositions $x=\sum_i m_i\otimes n_i$.
\end{definition}

\begin{definition}
    Let $A$ be a non-Archimedean Banach ring. Consider semi-normed $A$-modules $M$ and $M$, we define the \emph{complete tensor product} of $M$ and $N$ as the metric completion $M\hat{\otimes}_A N$ of the tensor product of $M$ and $N$ defined in \cref{def-tensorproduct}.
\end{definition}




\begin{thm}
    Let $(A,\|\bullet\|_A)$ be a normed ring. Then $\BanCat_A$ is a quasi-Abelian category.
\end{thm}
\begin{proof}
    We first observe that $\BanCat_A$ is preadditive, as for any $M,N\in \BanCat_A$, $\Hom_{\BanCat_A}(M,N)$ can be given the group structure inherited from the Abelian group $\Hom_A(M,N)$. It is obvious that $\BanCat_A$ is preadditive.

    Next we show that finite biproducts exist in $\BanCat_A$. Given $(M,\|\bullet\|_{M}),(N,\|\bullet\|_{N})\in \BanCat_A$, we set 
    \begin{equation}\label{eq-normedmodulesum}
        (M,\|\bullet\|_{M})\oplus  (N,\|\bullet\|_{N}):=(M\oplus N,\|\bullet\|_{M\oplus N}),
    \end{equation}
    where $\|(m,n)\|_{M\oplus N}:=\|m\|_M+\|n\|_N$ for $m\in M$ and $n\in N$. It is easy to verify that this gives the biproduct in $\BanCat_A$. 

    We have shown that $\BanCat_A$ is an additive category.

    Next given a morphism $\varphi:(M,\|\bullet\|_M)\rightarrow (N,\|\bullet\|_N)$ in $\BanCat_A$, we construct its kernel $(\ker \varphi, \|\bullet\|_{\ker\varphi})$ as the kernel of the underlying homomorphism of $A$-modules of $\varphi$ endowed with the subgroup semi-norm induced from $\|\bullet\|_M$ as in \cref{def-quotsubsemnorm}. It is easy to verify that $(\ker \varphi, \|\bullet\|_{\ker\varphi})$ is the kernel of $\varphi$ in  $\BanCat_A$. 

    We can similarly construct the cokernels. To be more precise, let $\varphi:(M,\|\bullet\|_M)\rightarrow (N,\|\bullet\|_N)$ be a morphism in $\BanCat_A$, then the $\coker \varphi=\{N/\overline{\varphi(M)}\}$ with quotient norm.

    We have shown that $\BanCat_A$ is a pre-Abelian category. 
    
    Observe that given a morphism  $\varphi:(M,\|\bullet\|_M)\rightarrow (N,\|\bullet\|_N)$ in $\BanCat_A$, its image is given by $\Img \varphi=\overline{\varphi(M)}$ with the subspace norm induced from $N$; its coimage is $M/\ker f$ with the residue norm. The morphism $\varphi$ is admissible if the natural map
    \[
        M/\ker f\rightarrow \overline{\varphi(M)}
    \]
    is an isomorphism in $\BanCat_A$.

    It remains to show that pull-backs preserve admissible epimorphisms and pushouts preserve admissible monomorphisms. We first handle the case of admissible epimorphisms. Consider a Cartesian square in $\BanCat_A$:
    \[
        \begin{tikzcd}
            M \arrow[r, "p"] \arrow[d, "q"] \arrow[rd, "\square", phantom] & U \arrow[d, "f"] \\
            V \arrow[r, "g"]                                               & W               
        \end{tikzcd}  
    \]
    with $g$ being an admissible epimorphism. We need to show that $p$ is also an admissible epimorphism, namely $U\cong M/\ker p$.

    We define $\alpha:U\oplus V\rightarrow W$, $\alpha=(f,-g)$, then there is a natural isomorphism $j:M\rightarrow \ker \alpha$. Let us write $i:\ker \alpha\rightarrow U\oplus V$ the natural morphism. Then
    \[
        q=\pi_V\circ i \circ j,\quad p=\pi_U\circ i \circ j,  
    \]
    where $\pi_U:U\oplus V\rightarrow U,\pi_V:U\oplus V\rightarrow V$ are the natural morphisms. We may assume that $M=\ker \alpha$ and $j$ is the identity. Then it is obvious that $p$ is surjective on the underlying sets. In order to compute the quotient norm on  $M/\ker p$, we need a more explicit description of $\ker p\subseteq \ker \alpha$. We know that 
    \[
        \ker \alpha=\{(u,v)\in U\oplus V: f(u)=g(v)\}  
    \]
    with the subspace norm induced from the product norm on $U\oplus V$ defined in \eqref{eq-normedmodulesum}. Then
    \[
        \ker p=\{(u,v)\in U\oplus V:u=0, g(v)=0\}.    
    \]
    It follows that for $(u,v)\in \ker \alpha$,
    \[
        \inf_{(u',v')\in \ker p} \|(u,v)+(u',v')\|_{U\oplus V}=\inf_{v'\in \ker g} (\|v+v'\|_V)+ \|x\|_U,
    \]
    where $\|\bullet\|_U$ and $\|\bullet\|_V$ denote the norms on $U$ and $V$ respectively. By our assumption that $g$ is an admissible epimorphism, there is a constant $C>0$ so that 
    \[
        \inf_{v'\in \ker g} (\|v+v'\|_V)\leq C\|g(v)\|_W
    \]
    for any $v\in V$.
    As $f$ is bounded, we can also find a constant $C'>0$ so that for any $(u,v)\in \ker \alpha$, 
    \[
        \|g(v)\|_W=\|f(u)\|_W\leq C'\|u\|_U.
    \]
    It follows that $p$ is admissible epimorphism.

    It remains to check that the pushforwards preserve admissible monomorphisms. Consider a co-Cartesian diagram
    \[
        \begin{tikzcd}
            W \arrow[r, "g"] \arrow[d, "f"]  & U \arrow[d, "q"] \\
            V \arrow[r, "p"]                                               & M               
        \end{tikzcd}  
    \]
    with $g$ being an admissible monomorphism. We need to show that $p$ is an adimissible monomorphism. This boils down to the following: $p$ is injective with closed image and the norms on $p(V)$ obtained in the obvious ways are equivalent.
    As in the case of pull-backs, we may let $\alpha:W\rightarrow U\oplus V$ be the  morphism $(g,-f)$ and assume that $M=\coker \alpha$. It is then easy to see that $p$ is injective. The proof that the two norms on $p(V)$ are equivalent is parallel to the argument in the pull-back case and we omit it.

    It remains to verify that $p(V)$ is closed in $W$. Consider the admissibly coexact sequence in $\BanCat_A$:
    \[
      W \xrightarrow{\alpha}U\oplus V\xrightarrow{\pi} M\rightarrow 0. 
    \]
    It is also admissibly coexact in the category of semi-normed $A$-modules. \textcolor{red}{Include details later.}
    Let $x_n\in V$ be a sequence so that $p(x_n)\to y\in M$. We may write $y=\pi(u,v)$ for some $(u,v)\in U\oplus V$. Then
    \[
        \pi(-u,x_n-v)\to 0  
    \]
    as $n\to \infty$. From the strict coexact sequence, we can find a sequence $w_n\in W$ so that
    \[
        (-u-g(w_n),x_n-v+f(w_n))\to 0  
    \]
    as $n\to \infty$. Then $g(w_n)\to -u$ in $U$ and hence there is $w\in W$ so that $w_n\to w\in W$ and $g(w)=-u$. But then $x_n\to x$ and $p(x)=y$.
\end{proof}


\begin{definition}
    Let $(A,\|\bullet\|_A)$ be a normed ring. A \emph{Banach $A$-algebra} is a pair $(B,\|\bullet\|_B)$ such that $(B,\|\bullet\|_B)$ is a Banach $A$-module and $(B,\|\bullet\|_B)$ is a Banach ring.
\end{definition}

\section{Berkovich spectra}


\begin{definition}
    Let $(A,\|\bullet\|_A)$ be a Banach ring. A semi-norm $|\bullet|$ on $A$ is \emph{bounded} if there is a constant $C>0$ such that for any $f\in A$, $|f|\leq C\|f\|_A$.

    We write $\Sp A$ for the set of bounded semi-valuations on $A$. We call $\Sp A$ the \emph{Berkovich spectrum} of $A$.
\end{definition}
Later on, we will endow $\Sp A$ with more structures. In the literature, it is more common to denote $\Sp A$ by $\mathcal{M}(A)$.
\begin{proposition}\label{prop-Berkospecnonempty}
    Let $(A,\|\bullet\|)$ be a Banach ring. Then $\Sp A$ is empty if and only if $A=0$.
\end{proposition}

\begin{proof}
    If $A=0$, $\Sp A$ is clearly empty. Conversely, suppose that $\Sp A$ is empty. Assume that $A\neq 0$. For any maximal ideal $\mathfrak{m}$, by \cref{cor:maximalidealclosedinBanachring}, $A/\mathfrak{m}$ is a Banach ring and $\Sp A/\mathfrak{m}$ is a subset of $\Sp A$. So we may assume that $A$ is a field. Let $S$ be the set of bounded semi-norms on $A$. Then $S$ is non-empty as $\|\bullet\|\in S$. By Zorn's lemma, we can take a minimal element $|\bullet|\in S$. Up to replacing $A$ by the completion with respect to $|\bullet|$, we may assume that $|\bullet |$ is a norm on $A$. As $A$ is a field, we may further assume that $|\bullet|=\|\bullet\|$.
    
    We claim that $\|\bullet\|$ is multiplicative. As $A$ is a field, it suffices to show that $\|f^{-1}\|=\|f\|^{-1}$ for any non-zero $f\in A$. We may assume that $\|f\|^{-1}<\|f^{-1}\|$.
    
    Let $r$ be a positive real number. Let $\varphi:A\rightarrow A\{r^{-1}T\}/(T-f)$ be the natural map. The map is injective as $A$ is a field. We endow $A\{r^{-1}T\}/(T-f)$ with the quotient semi-norm induced by $\|\bullet\|_r$. We still denote this semi-norm by $\|\bullet\|_r$.

    We claim that $f-T$ is not invertible in $A\{r^{-1}T\}$ for the choice $r=\|f^{-1}\|^{-1}$. From this, it follows that 
    \[
        \|\varphi(f)\|_r  =\|T\|_r\leq r <\|f\|.
    \]
    The last step is our assumption. This contradicts our choice of $\|\bullet\|$.
    
    In order to prove the claim, we need to show that $\|\bullet\|$ is power multiplicative first. Assuming this, it is obvious that 
    \[
        \sum_{i=0}^{\infty}|f^{-i}|r^i=\sum_{i=0}^{\infty}|f^{-1}|^{i}|f^{-1}|^{-i}
    \]
    diverges.
    
    It remains to show that $\|\bullet\|$ is power multiplicative.
    Suppose that is $f\in A$ so that $\|f^n\|<\|f\|^n$ for some $n>1$. We claim that $f-T$ is not invertible in $A\{r^{-1}T\}$  for the choice $r=\|f^{n}\|^{1/n}$. From this,
    \[
      \|\varphi(f)\|_r=\|T\|_r\leq r<\|f\|.  
    \]
    This contradicts our choice of $\|\bullet\|$. The claim amounts to the divergence of
    \[
        \sum_{i=0}^{\infty}\|f^{-i}\|r^i.
    \]
    For a general $i\geq 0$, we write $i=pn+q$ for $p,q\in \mathbb{N}$ and $q\leq n-1$. Then $\|f^i\|\leq \|f^n\|^p \|f^q\|$. So 
    \[
        \|f^{-i}\|r^i\geq \|f^i\|^{-1} \|f^n\|^{p+n^{-1}q}\geq \|f^n\|^{n^{-1}q}\|f^q\|^{-1}.  
    \]
    It therefore follows that $|f^{-i}|r^i$ admits a positive lower bound, and we conclude.
\end{proof}

\section{Open mapping theorem}
Let $(k,|\bullet|)$ be a complete non-trivially valued field. All results in this section fail when $k$ is trivially valued.

\begin{proposition}\label{prop-bddequivcont}
    Let $A$ be a normed $k$-algebra and $f:(M,\|\bullet\|_M)\rightarrow (N,\|\bullet\|_N)$ be an $A$-homomorphism of normed $A$-modules. Then $f$ is bounded if and only if $f$ is continuous. 
\end{proposition}
\begin{proof}
    The direct implication follows from \cref{prop-bddimplycont}. Assume that $f$ is continuous. We may assume that $A=k$. 

    Assume that $f$ is not bounded. Fix $a\in k$ with $|a|\in (0,1)$. This is possible as $k$ is non-trivially valued.
    Then we can find a sequence $m_i\in M$ such that $\|f(m_i)\|_N>|a|^{-i}\|m_i\|_M$. Up to replace $m_i$ by a scalar multiple, we may assume that $\|m_i\|_M\in [1,|a|^{-1})$: if $\|m_i\|_M\geq 1$, choose $n\in \mathbb{N}$ such that $|a|^{-n}\leq \|m_i\|_M<|a|^{-n-1}$, then replace $m_i$ with $a^nm_i$. The case $|x|<1$ is similar. 
    Then $\|f(a^im_i)\|_N>\|m_i\|_M\geq 1$ while $\|a^im_i\|_M<|a|^n |a|^{-1}\to 0$. This is a contradiction.
\end{proof}

\begin{thm}[Open mapping theorem]\label{thm-openmapping}
    Let $(V,\|\bullet\|_V), (W,\|\bullet\|_W)$ be $k$-Banach spaces and $L:V\rightarrow W$ be a bounded and surjective $k$-homomorphism. Then $L$ is open.   
\end{thm}
\begin{proof}
    We write $V_0=\{v\in V:\|v\|_V<1\}$. Similarly define $W_0$.

    \textbf{Step~1}. We claim that there is a constant $C>0$ such that for all $w'\in W$, there is $v'\in V$ such that 
    \[
        \|v'\|_V\leq  C \|w'\|_W,\quad \|w'-L(v')\|_W<1/2.
    \]

    As $k$ is non-trivially valued, we can take $c\in k$ with $|c|\in (0,1)$, so 
    \[
        V=\bigcup_{n\in \mathbb{N}} c^n V_0.  
    \]
    As $L$ is surjective, we have
    \[
        W=\bigcup_{n\in \mathbb{N}} c^n L(V_0).   
    \]
    By Baire's category theorem, we may assume that $\overline{L(V_0)}$ has non-empty interior. Take $w\in W$ and $r>0$ so that
    \[
        \{w'\in W: \|w-w'\|_W<r\}\subseteq    \overline{L(V_0)}.
    \]
    Take $d\in W_0$ and $c'\in k^{\times}$ so that $|c'|<r$, then $w+c'd\in \overline{L(V_0)}$. It follows that
    \[
        c'd\in   \overline{L(V_0)}+\overline{L(V_0)}\subseteq \overline{L(V_0)+L(V_0)}=\overline{L(V_0)}.
    \]
    So
    \[
        W_0\subseteq  \overline{L(c'^{-1}V_0)}. 
    \]
    It suffices to take $C=|c'^{-1}|$.

    \textbf{Step~2}.
    Now given $w\in W_0$, we want to show that $w\in L(\{v\in V:\|v\|_V<C\})$. 
    This will finish the argument: as $k$ is non-trivially valued, this implies that $L(V_0)$ contains an open neighbourhood of $0$. 
    
    From Step~1, we can construct $v_1\in V$ with $\|v_1\|_V< C$ and $\|w-L(v_1)\|_W<1/2$. Repeat this process, we can $v_n\in V$ inductively so that
    \[
        \|v_n\|_V<  2^{1-n} C,\quad \|w-L(v_1+\cdots+v_n)\|_W<2^{-n}.
    \]
    We set $v=\sum_{i=1}^{\infty} v_i$. Then $v\in V$ and $Av=w$ by continuity. Moreover,
    \[
        \|v\|_V\leq \max_n \|v_n\|_V<C.   
    \]
\end{proof}

\begin{corollary}\label{cor-completionfinitecomplete}
    Let $A$ be a $k$-Banach algebra and $M$ be a normed $A$-module. Assume that $\hat{M}$ is a finite $A$-module, then $M$ is complete.
\end{corollary}
\begin{proof}
    Take $x_1,\ldots,x_n\in \hat{M}$ so that $\pi:A^n\rightarrow \hat{M}$ sending $(a_1,\ldots,a_n)$ to $\sum_{i=1}^n a_ix_i$ is surjective. By open mapping theorem \cref{thm-openmapping}, $\sum_{i=1}^n \check{A}x_i$ is a neighbourhood of $0$ in $\hat{M}$. So
    \[
        x_j\in M+  \sum_{i=1}^n \check{A} x_i.
    \]
    It follows from (a version of) Nakayama's lemma that $M=\hat{M}$.
\end{proof}

\begin{corollary}\label{cor-idealclosed}
    Let $A$ be a $k$-Banach algebra and $M$ be a Noetherian Banach $A$-module. Let $N$ be a submodule of $M$. Then $N$ is closed in $M$.

    In particular, if $A$ is Noetherian, then all ideals of $A$ are closed.
\end{corollary}
\begin{proof}
    As $M$ is noetherian, $\bar{N}$ is a finite $A$-module. In particular, $N$ is complete by \cref{cor-completionfinitecomplete}. Hence, $N$ is closed in $M$.
\end{proof}

\begin{corollary}\label{cor-bddisadmi}
    A bounded homomorphism of $k$-Banach algebras $f:A\rightarrow B$ is admissible.
\end{corollary}
\begin{proof}
    We may assume that $f$ is surjective by replacing $B$ by the image of $f$. Similarly, by replacing $A$ by $A/\ker f$, we may assume that $f$ is bijective. It follows from \cref{thm-openmapping} that $f$ is a homeomorphism. The inverse of $f$ is therefore continuous, and hence bounded by \cref{prop-bddequivcont}.
\end{proof}

\section{Bornology}
\textcolor{red}{This section may be placed elsewhere.}
\begin{definition}
    Let $X$ be a set. A \emph{bornology} on $X$ is a collection $\mathcal{B}$ of subsets of $X$ such that
    \begin{enumerate}
        \item For any $x\in X$, there is $B\in \mathcal{B}$ such that $x\in \mathcal{B}$;
        \item For any $B\in \mathcal{B}$ and any subset $A\subseteq B$, $A\in \mathcal{B}$;
        \item $\mathcal{B}$ is stable under finite union.
    \end{enumerate}

    The pair $(X,\mathcal{B})$ is called a \emph{bornological set}. The elements of $\mathcal{B}$ are called the \emph{bounded subsets} of $(X,\mathcal{B})$. When $\mathcal{B}$ is obvious from the context, we omit it from the notations.
    
    A morphism between bornological sets $(X,\mathcal{B}_X)$ and $(Y,\mathcal{B}_Y)$ is a map of sets $f:X\rightarrow Y$ such that for any $A\in \mathcal{B}_X$, $f(A)\in \mathcal{B}_Y$. Such a map is called a \emph{bounded map}.
\end{definition}
\begin{definition}
    Let $(X,\mathcal{B})$ be a bornological set. A \emph{basis} for $\mathcal{B}$ is a subset $\mathcal{A}\subseteq \mathcal{B}$ such that for any $B\in \mathcal{B}$, there are $A_1,\ldots,A_n\in \mathcal{A}$ such that $B\subseteq A_1\cup\cdots\cup A_n$.
\end{definition}

\printbibliography
\end{document}