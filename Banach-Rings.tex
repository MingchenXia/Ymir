
\documentclass{amsbook} 



%\usepackage{xr}
\usepackage{xr-hyper}
\usepackage[unicode]{hyperref}


\usepackage[T1]{fontenc}
\usepackage[utf8]{inputenc}
\usepackage{lmodern}
\usepackage{amssymb,tikz-cd}
%\usepackage{natbib}
\usepackage[english]{babel}
\usepackage{nameref}

\usepackage[nameinlink,capitalize]{cleveref}
\usepackage[style=alphabetic,maxnames=99,maxalphanames=5, isbn=false, giveninits=true, doi=false]{biblatex}
\usepackage{lipsum, physics}
\usepackage{ifthen}
\usepackage{microtype}
\usepackage{booktabs}
\usetikzlibrary{calc}
\usepackage{emptypage}
\usepackage{setspace}
\usepackage[margin=0.75cm, font={small,stretch=0.80}]{caption}
\usepackage{subcaption}
\usepackage{url}
\usepackage{bookmark}
\usepackage{graphicx}
\usepackage{dsfont}
\usepackage{enumitem}
\usepackage{mathtools}
\usepackage{csquotes}
\usepackage{silence}
\usepackage{mathrsfs}
\usepackage{bigints}

\WarningFilter{biblatex}{Patching footnotes failed}


\ProcessOptions\relax

\emergencystretch=1em

\hypersetup{
colorlinks=true,
linktoc=all
}

\setcounter{tocdepth}{1}


\hyphenation{archi-medean  Archi-medean Tru-ding-er}

%\captionsetup[table]{position=bottom}   %% or below
\renewcommand{\thefootnote}{\fnsymbol{footnote}}
%\DeclareMathAlphabet{\mathcal}{OMS}{cmsy}{m}{n}
\renewbibmacro{in:}{}

\DeclareFieldFormat[article]{citetitle}{#1}
\DeclareFieldFormat[article]{title}{#1}
\DeclareFieldFormat[inbook]{citetitle}{#1}
\DeclareFieldFormat[inbook]{title}{#1}
\DeclareFieldFormat[incollection]{citetitle}{#1}
\DeclareFieldFormat[incollection]{title}{#1}
\DeclareFieldFormat[inproceedings]{citetitle}{#1}
\DeclareFieldFormat[inproceedings]{title}{#1}
\DeclareFieldFormat[phdthesis]{citetitle}{#1}
\DeclareFieldFormat[phdthesis]{title}{#1}
\DeclareFieldFormat[misc]{citetitle}{#1}
\DeclareFieldFormat[misc]{title}{#1}
\DeclareFieldFormat[book]{citetitle}{#1}
\DeclareFieldFormat[book]{title}{#1} 


%% Define various environments.

\theoremstyle{definition}
\newtheorem{theorem}{Theorem}[section]
\newtheorem{thm}[theorem]{Theorem}
\newtheorem{proposition}[theorem]{Proposition}
\newtheorem{corollary}[theorem]{Corollary}
\newtheorem{lemma}[theorem]{Lemma}
\newtheorem{conjecture}[theorem]{Conjecture}
\newtheorem{question}[theorem]{Question}
\newtheorem{example}[theorem]{Example}
\newtheorem{definition}[theorem]{Definition}
\newtheorem{condition}[theorem]{Condition}

\theoremstyle{remark}
\newtheorem{remark}[theorem]{Remark}
\numberwithin{equation}{section}

%\renewcommand{\thesection}{\thechapter.\arabic{section}}
%\renewcommand{\thetheorem}{\thesection.\arabic{theorem}}
%\renewcommand{\thedefinition}{\thesection.\arabic{definition}}
%\renewcommand{\theremark}{\thesection.\arabic{remark}}


%% Define new operators

\DeclareMathOperator{\rad}{rad}
\DeclareMathOperator{\nd}{nd}
\DeclareMathOperator{\ord}{ord}
\DeclareMathOperator{\Hom}{Hom}
\DeclareMathOperator{\PreSh}{PreSh}
\DeclareMathOperator{\Gr}{Gr}
\DeclareMathOperator{\Homint}{\mathcal{H}\mathrm{om}}
\DeclareMathOperator{\Torint}{\mathcal{T}\mathrm{or}}
\DeclareMathOperator{\Div}{div}
\DeclareMathOperator{\DSP}{DSP}
\DeclareMathOperator{\Diff}{Diff}
\DeclareMathOperator{\MA}{MA}
\DeclareMathOperator{\NA}{NA}
\DeclareMathOperator{\AN}{an}
\DeclareMathOperator{\Rep}{Rep}
\DeclareMathOperator{\Rest}{Res}
\DeclareMathOperator{\DF}{DF}
\DeclareMathOperator{\VCart}{VCart}
\DeclareMathOperator{\PL}{PL}
\DeclareMathOperator{\Bl}{Bl}
\DeclareMathOperator{\Td}{Td}
\DeclareMathOperator{\Fitt}{Fitt}
\DeclareMathOperator{\Ric}{Ric}
\DeclareMathOperator{\coeff}{coeff}
\DeclareMathOperator{\Aut}{Aut}
\DeclareMathOperator{\Capa}{Cap}
\DeclareMathOperator{\loc}{loc}
\DeclareMathOperator{\vol}{vol}
\DeclareMathOperator{\Val}{Val}
\DeclareMathOperator{\ST}{ST}
\DeclareMathOperator{\het}{ht}
\DeclareMathOperator{\Amp}{Amp}
\DeclareMathOperator{\Herm}{Herm}
\DeclareMathOperator{\trop}{trop}
\DeclareMathOperator{\Trop}{Trop}
\DeclareMathOperator{\Cano}{Can}
\DeclareMathOperator{\PS}{PS}
\DeclareMathOperator{\codim}{codim}
\DeclareMathOperator{\Var}{Var}
\DeclareMathOperator{\Psef}{Psef}
\DeclareMathOperator{\Jac}{Jac}
\DeclareMathOperator{\Char}{char}
\DeclareMathOperator{\Red}{red}
\DeclareMathOperator{\Spf}{Spf}
\DeclareMathOperator{\Span}{Span}
\DeclareMathOperator{\Der}{Der}
%\DeclareMathOperator{\Mod}{mod}
\DeclareMathOperator{\Hilb}{Hilb}
\DeclareMathOperator{\triv}{triv}
\DeclareMathOperator{\Frac}{Frac}
\DeclareMathOperator{\diam}{diam}
\DeclareMathOperator{\Spec}{Spec}
\DeclareMathOperator{\Spm}{Spm}
\DeclareMathOperator{\Specrel}{\underline{Sp}}
\DeclareMathOperator{\Sp}{Sp}
\DeclareMathOperator{\reg}{reg}
\DeclareMathOperator{\sing}{sing}
\DeclareMathOperator{\Star}{Star}
\DeclareMathOperator{\relint}{relint}
\DeclareMathOperator{\Cvx}{Cvx}
\DeclareMathOperator{\Int}{Int}
\DeclareMathOperator{\dep}{dep}
\DeclareMathOperator{\pd}{pd}
\DeclareMathOperator{\codep}{codep}
\DeclareMathOperator{\Supp}{Supp}
\DeclareMathOperator{\FS}{FS}
\DeclareMathOperator{\RZ}{RZ}
\DeclareMathOperator{\Ext}{Ext}
\DeclareMathOperator{\Redu}{red}
\DeclareMathOperator{\lct}{lct}
\DeclareMathOperator{\Proj}{Proj}
\DeclareMathOperator{\Sing}{Sing}
\DeclareMathOperator{\Conv}{Conv}
\DeclareMathOperator{\Max}{Max}
\DeclareMathOperator{\Tor}{Tor}
\DeclareMathOperator{\Gal}{Gal}
\DeclareMathOperator{\Frob}{Frob}
\DeclareMathOperator{\coker}{coker}
\DeclareMathOperator{\Sym}{Sym}
\DeclareMathOperator{\CSp}{CSp}
\DeclareMathOperator{\Cov}{Cov}
\DeclareMathOperator{\Img}{Im}


\newcommand{\alg}{\mathrm{alg}}
\newcommand{\Sh}{\mathrm{Sh}}
\newcommand{\fin}{\mathrm{fin}}
\newcommand{\BPF}{\mathrm{BPF}}
\newcommand{\dBPF}{\mathrm{dBPF}}
\newcommand{\divf}{\mathrm{Div}^f}
\newcommand{\nef}{\mathrm{nef}}
\newcommand{\Bir}{\mathrm{Bir}}
\newcommand{\hO}{\hat{\mathcal{O}}}
\newcommand{\bDiv}{\mathrm{Div}^{\mathrm{b}}}
\newcommand{\un}{\mathrm{un}}
\newcommand{\sep}{\mathrm{sep}}
\newcommand{\diag}{\mathrm{diag}}
\newcommand{\Pic}{\mathrm{Pic}}
\newcommand{\GL}{\mathrm{GL}}
\newcommand{\SL}{\mathrm{SL}}
\newcommand{\LS}{\mathrm{LS}}
\newcommand{\GLS}{\mathrm{GLS}}
\newcommand{\GLSi}{\mathrm{GLS}_{\cap}}
\newcommand{\PGLS}{\mathrm{PGLS}}
\newcommand{\Loc}[1][S]{_{\{{#1}\}}}
\newcommand{\cl}{\mathrm{cl}}
\newcommand{\otL}{\hat{\otimes}^{\mathbb{L}}}
\newcommand{\ddpp}{\mathrm{d}'\mathrm{d}''}
\newcommand{\TC}{\mathcal{TC}}
\newcommand{\ddPP}{\mathrm{d}'_{\mathrm{P}}\mathrm{d}''_{\mathrm{P}}}
\newcommand{\PSs}{\mathcal{PS}}
\newcommand{\Gm}{\mathbb{G}_{\mathrm{m}}}
\newcommand{\End}{\mathrm{End}}
\newcommand{\Aff}[1][X]{\mathcal{M}\left(\mathcal{#1}\right)}
\newcommand{\XG}[1][X]{{#1}_{\mathrm{G}}}
\newcommand{\convC}{\xrightarrow{C}}
\newcommand{\Vect}{\mathrm{Vect}}
\newcommand{\abso}[1]{\lvert#1\rvert}
\newcommand{\Mdl}{\mathrm{Model}}
\newcommand{\cn}{\stackrel{\sim}{\longrightarrow}}
\newcommand{\sbc}{\mathbf{s}}
\newcommand{\CH}{\mathrm{CH}}
\newcommand{\GR}{\mathrm{GR}}
\newcommand{\bir}{\mathrm{bir}}
\newcommand{\dc}{\mathrm{d}^{\mathrm{c}}}
\newcommand{\Nef}{\mathrm{Nef}}
\newcommand{\Adj}{\mathrm{Adj}}
\newcommand{\DHm}{\mathrm{DH}}
\newcommand{\An}{\mathrm{an}}
\newcommand{\Rec}{\mathrm{Rec}}
\newcommand{\dP}{\mathrm{d}_{\mathrm{P}}}
\newcommand{\ddp}{\mathrm{d}_{\mathrm{P}}'\mathrm{d}_{\mathrm{P}}''}
\newcommand{\ddc}{\mathrm{dd}^{\mathrm{c}}}
\newcommand{\ddL}{\mathrm{d}'\mathrm{d}''}
\newcommand{\PSH}{\mathrm{PSH}}
\newcommand{\CPSH}{\mathrm{CPSH}}
\newcommand{\PSP}{\mathrm{PSP}}
\newcommand{\WPSH}{\mathrm{WPSH}}
\newcommand{\Ent}{\mathrm{Ent}}
\newcommand{\NS}{\mathrm{NS}}
\newcommand{\QPSH}{\mathrm{QPSH}}
\newcommand{\proet}{\mathrm{pro-ét}}
\newcommand{\XL}{(\mathcal{X},\mathcal{L})}
\newcommand{\ii}{\mathrm{i}}
\newcommand{\Ann}{\mathrm{Ann}}
\newcommand{\ExtFun}{\mathcal{E}\mathrm{xt}}
\newcommand{\Cpt}{\mathrm{Cpt}}
\newcommand{\bp}{\bar{\partial}}
\newcommand{\ddt}{\frac{\mathrm{d}}{\mathrm{d}t}}
\newcommand{\dds}{\frac{\mathrm{d}}{\mathrm{d}s}}
\newcommand{\Ep}{\mathcal{E}^p(X,\theta;[\phi])}
\newcommand{\Ei}{\mathcal{E}^{\infty}(X,\theta;[\phi])}
\newcommand{\infs}{\operatorname*{inf\vphantom{p}}}
\newcommand{\sups}{\operatorname*{sup*}}
\newcommand{\colim}{\operatorname*{colim}}
\newcommand{\ddtz}[1][0]{\left.\ddt\right|_{t={#1}}}
\newcommand{\tube}[1][Y]{]{#1}[}
\newcommand{\ddsz}[1][0]{\left.\ddt\right|_{s={#1}}}
\newcommand{\floor}[1]{\left \lfloor{#1}\right \rfloor }
\newcommand{\dec}[1]{\left \{{#1}\right \} }
\newcommand{\ceil}[1]{\left \lceil{#1}\right \rceil }
\newcommand{\Projrel}{\mathcal{P}\mathrm{roj}}
\newcommand{\Weil}{\mathrm{Weil}}
\newcommand{\Cart}{\mathrm{Cart}}
\newcommand{\bWeil}{\mathrm{b}\mathrm{Weil}}
\newcommand{\bCart}{\mathrm{b}\mathrm{Cart}}
\newcommand{\Cond}{\mathrm{Cond}}
\newcommand{\IC}{\mathrm{IC}}
\newcommand{\IH}{\mathrm{IH}}
\newcommand{\Eq}{\mathrm{Eq}}
\newcommand{\cris}{\mathrm{cris}}
\newcommand{\Zar}{\mathrm{Zar}}
\newcommand{\HvbCat}{\overline{\mathcal{V}\mathrm{ect}}}
\newcommand{\BanModCat}{\mathcal{B}\mathrm{an}\mathcal{M}\mathrm{od}}
\newcommand{\DesCat}{\mathcal{D}\mathrm{es}}
\newcommand{\RingCat}{\mathcal{R}\mathrm{ing}}
\newcommand{\SchCat}{\mathcal{S}\mathrm{ch}}
\newcommand{\AbCat}{\mathcal{A}\mathrm{b}}
\newcommand{\RSCat}{\mathcal{R}\mathrm{S}}
\newcommand{\LRSCat}{\mathcal{L}\mathrm{RS}}
\newcommand{\CLRSCat}{\mathbb{C}\text{-}\LRSCat}
\newcommand{\CRSCat}{\mathbb{C}\text{-}\RSCat}
\newcommand{\CLA}{\mathbb{C}\text{-}\mathcal{L}\mathrm{A}}
\newcommand{\CASCat}{\mathbb{C}\text{-}\mathcal{A}\mathrm{n}}
\newcommand{\LiuCat}{\mathcal{L}\mathrm{iu}}
\newcommand{\BanCat}{\mathcal{B}\mathrm{an}}
\newcommand{\BanAlgCat}{\mathcal{B}\mathrm{an}\mathcal{A}\mathrm{lg}}
\newcommand{\AnaCat}{\mathcal{A}\mathrm{n}}
\newcommand{\LiuAlgCat}{\mathcal{L}\mathrm{iu}\mathcal{A}\mathrm{lg}}
\newcommand{\AlgCat}{\mathcal{A}\mathrm{lg}}
\newcommand{\SetCat}{\mathcal{S}\mathrm{et}}
\newcommand{\ModCat}{\mathcal{M}\mathrm{od}}
\newcommand{\GerCat}{\mathcal{G}\mathrm{er}}
\newcommand{\AnaGerCat}{\mathbb{C}\text{-}\GerCat}
\newcommand{\TopCat}{\mathcal{T}\mathrm{op}}
\newcommand{\CohCat}{\mathcal{C}\mathrm{oh}}
\newcommand{\SolCat}{\mathcal{S}\mathrm{olid}}
\newcommand{\AffCat}{\mathcal{A}\mathrm{ff}}
\newcommand{\AffAlgCat}{\mathcal{A}\mathrm{ff}\mathcal{A}\mathrm{lg}}
\newcommand{\QcohLiuAlgCat}{\mathcal{L}\mathrm{iu}\mathcal{A}\mathrm{lg}^{\mathrm{QCoh}}}
\newcommand{\LiuMorCat}{\mathcal{L}\mathrm{iu}}
\newcommand{\Isom}{\mathcal{I}\mathrm{som}}
\newcommand{\Cris}{\mathcal{C}\mathrm{ris}}
\newcommand{\Pro}{\mathrm{Pro}-}
\newcommand{\Fin}{\mathcal{F}\mathrm{in}}
\newcommand{\norms}[1]{\left\|#1\right\|}
\newcommand{\HPDDiff}{\mathbf{D}\mathrm{iff}}
\newcommand{\Menn}[2]{\begin{bmatrix}#1\\#2\end{bmatrix}}
\newcommand{\Fins}{\widehat{\Vect}^F}
\newcommand\blfootnote[1]{%
  \begingroup
  \renewcommand\thefootnote{}\footnote{#1}%
  \addtocounter{footnote}{-1}%
  \endgroup
}


\makeatletter
\newcommand*{\addFileDependency}[1]{% argument=file name and extension
  \typeout{(#1)}
  \@addtofilelist{#1}
  \IfFileExists{#1}{}{\typeout{No file #1.}}
}
\makeatother



\newcommand*{\myexternaldocument}[2]{%
\externaldocument[#1]{#2}%
\addFileDependency{#2.tex}%
\addFileDependency{#2.aux}%
%\addFileDependency{#2.pdf}%
}


%\iffalse

\myexternaldocument{Introduction-}{Introduction}
\myexternaldocument{Topology-}{Topology-Bornology}
\myexternaldocument{Banach-}{Banach-Rings}
\myexternaldocument{Commutative-}{Commutative-Algebra}



\myexternaldocument{Local-}{Local-Algebras}
\myexternaldocument{Complex-}{Complex-Analytic-Spaces}
\myexternaldocument{ConstructionComplex-}{Constructions-Complex-Spaces}
\myexternaldocument{PropertyComplex-}{Properties-Complex-Spaces}
\myexternaldocument{GPropertyComplex-}{Global-Properties-Complex-Spaces}
\myexternaldocument{Analytic-}{Analytic-Sets}
\myexternaldocument{Morphisms-}{Morphisms-Complex-Spaces}

\myexternaldocument{Affinoid-}{Affinoid-Algebras}
\myexternaldocument{Berkovich-}{Berkovich-Analytic-Spaces}
\myexternaldocument{BerkProperty-}{Properties-Berkovich-Spaces}
%\fi


\bibliography{Ymir}

\endinput
\title{Banach algebras}
\begin{document}
\maketitle
\tableofcontents



\section{Introduction}\label{sec-introduction}

This section conerns the theory of Banach algebras. Our references are \cite{Berk12} and \cite{BGR}.

In this chapter, all rings are assumed to be commutative.

\section{Semi-normed Abelian groups}

\begin{definition}
    Let $A$ be an Abelian group. A \emph{semi-norm} on $A$ is a function $\|\bullet\|:A\rightarrow [0,\infty]$ satisfying
    \begin{enumerate}
        \item $\|0\|=0$;
        \item $\|f-g\|\leq \|f\|+\|g\|$ for all $f,g\in A$.
    \end{enumerate}
    A semi-norm $\|\bullet\|$ on $A$ is a \emph{norm} if moreover the following conditions is satisfied:
    \begin{enumerate}
        \item[(0)] if $\|f\|=0$ for some $f\in A$, then $f=0$.
    \end{enumerate}

    A semi-norm $\|\bullet\|$ on $A$ is \emph{non-Archimedean} or \emph{ultra-metric} if Condition~(2) can be replaced by
    \begin{enumerate}[resume]
        \item[(2')] $\|f-g\|\leq \max\{\|f\|,\|g\|\}$ for all $f,g\in A$.
    \end{enumerate}
\end{definition}
\begin{definition}
    A \emph{semi-normed Abelian group} (resp. \emph{normed Abelian group}) is a pair $(A,\|\bullet\|)$ consisting of an Abelian group $A$ and a semi-norm (resp. norm) $\|\bullet\|$ on $A$. When $\|\bullet\|$ is clear from the context, we also say $A$ is a semi-normed Abelian group (resp. normed Abelian group).
\end{definition}


\begin{definition}
    Let $(A,\|\bullet\|_A)$ be a semi-normed Abelian group and $B\subseteq A$ be a subgroup. Then we define the \emph{quotient semi-norm} $\|\bullet\|_{A/B}$ on $A/B$ as follows:
    \[
      \|a+B\|_{A/B}:=\inf\{\|a+b\|_A:b\in B\}  
    \]
    for all $a+B\in A/B$.

    We define the \emph{subgroup semi-norm} on $B$ as follows:
    \[
        \|b\|_B=\|b\|_A
    \]
    for all $b\in B$.
\end{definition}

\begin{definition}
    Let $A$ be an Abelian group and $\|\bullet\|$, $\|\bullet\|'$ be two seminorms on $A$. We say $\|\bullet\|$ and $\|\bullet\|'$ are \emph{equivalent} if there is a constant $C>0$ such that
    \[
      C^{-1}\|f\|\leq \|f\|'\leq C\|f\|   
    \]
    for all $f\in A$.
\end{definition}

\begin{definition}

    Let $(A,\|\bullet\|_A)$, $(B,\|\bullet\|_B)$ be semi-normed Abelian groups. A homomorphism $\varphi:A\rightarrow B$ is said to be 
    \begin{enumerate}
        \item \emph{bounded} if there is a constant $C>0$ such that $\|\varphi(f)\|_B\leq C \|f\|_{A}$ for any $f\in A$;
        \item \emph{admissible} if the quotient semi-norm on $A/\ker \varphi$ is equivalent to the subspace semi-norm on $\Img \varphi$.
    \end{enumerate}
\end{definition}
Observe that an admissible homomorphism is always bounded.

Next we study the topology defined by a semi-norm.
\begin{lemma}\label{lma-pmetricinducedbyseminorm}
    Let $(A,\|\bullet\|)$ be a semi-normed Abelian group. Define
    \[
      d(a,b)=\|a-b\|  
    \]
    for $a,b\in A$. Then $\|\bullet\|$ is a pseudo-metric on $A$. This psuedo-metric is a metric if and only if $\|\bullet\|$ is a norm.
\end{lemma}
\begin{proof}
    This is clear from the definitions.
\end{proof}
We always endow $A$ with the topology induced by the psuedo-metric $d$.


\section{Semi-normed rings}

\begin{definition}\label{def-seminormring}
    Let $A$ be a ring. A \emph{semi-norm} $\|\bullet\|$ on $A$ is a semi-norm $\|\bullet\|$ on the underlying additive group satisfying the following extra properties: 
    \begin{enumerate}
        \item[(3)]  $\|1\|=1$;
        \item[(4)] for any $f,g\in A$, $\|fg\|\leq \|f\|\|g\|$.
    \end{enumerate}
    
    A semi-norm $\|\bullet\|$ on $A$ is called \emph{power-multiplicative} if $\|f\|^n=\|f^n\|$ for all $f\in A$ and $n\in \mathbb{N}$.

    A semi-norm $\|\bullet\|$ on $A$ is called \emph{multiplicative} if $\|fg\|=\|f\|\|g\|$ for all $f,g\in A$.

\end{definition}

\begin{definition}
    A \emph{semi-normed ring} (resp. \emph{normed ring}) is a pair $(A,\|\bullet\|)$ consisting of a ring $A$ and a semi-norm (resp. norm) $\|\bullet\|$ on $A$. When $\|\bullet\|$ is clear from the context, we also say $A$ is a semi-normed  ring (resp. normed ring).
\end{definition}




\begin{definition}
    Let $A$ be a ring. A \emph{semi-valuation} on $A$ is a multiplicative semi-norm on $A$. A semi-valuation on $A$ is a \emph{valuation} on $A$ if its underlying semi-norm of Abelian groups is a norm. 
\end{definition}

\begin{definition}
    A \emph{semi-valued ring} (resp. \emph{valued ring}) is a pair $(A,\|\bullet\|)$ consisting of a ring $A$ and a semi-valuation (resp. valuation) $\|\bullet\|$ on $A$. When $\|\bullet\|$ is clear from the context, we also say $A$ is a semi-valued  ring (resp. valued ring).

    A semi-valued ring (resp. valued ring) $(A,\|\bullet\|)$ is called a \emph{semi-valued field} (resp. \emph{valued field}) if $A$ is a field.
\end{definition}



\section{Banach rings}

\begin{definition}
    A \emph{Banach ring} is a normed ring that is complete with respect to the metric defined in \cref{lma-pmetricinducedbyseminorm}.
\end{definition}

\begin{proposition}\label{prop-inversesmallelementBanachring}
    Let $(A,\|\bullet\|)$ be a Banach ring and $f\in A$. Assume that $\|f\|<1$, then $1-f$ is invertible.
\end{proposition}
\begin{proof}
    Define 
    \[
        g=\sum_{i=0}^{\infty}f^i.
    \]
    From our assumption, the series converges and $g\in A$. It is elementary to check that $g$ is the inverse of $1-f$.
\end{proof}

\begin{example}
    The ring $\mathbb{C}$ with its usual norm $|\bullet|$ is a Banach ring. In fact, $(\mathbb{C},|\bullet|)$ is a complete valued field.
\end{example}

\begin{example}\label{ex-convseriesradius}
    For any Banach ring $(A,\|\bullet\|)$, any $n\in \mathbb{N}$ and any $r=(r_1,\ldots,r_n)\in \mathbb{R}^n_{>0}$, we define $A\langle r^{-1}z \rangle =A\langle r_1^{-1}z_1,\ldots, r_n^{-1}z_n\rangle_{r_1,\ldots,r_n}$ as the subring of $A[[z_1,\ldots,z_n]]$ consisting of formal power series
    \[
      f=\sum_{\alpha\in \mathbb{N}^{n}} a_{\alpha}z^{\alpha},\quad a_{\alpha}\in A  
    \]
    such that
    \[
        \|f\|_r:=   \sum_{\alpha\in \mathbb{N}^{n}} \|a_{\alpha}\|r^{\alpha}<\infty.
    \]
    We will verify in \cref{prop-convseriesradiusBanach} that $(A\langle r^{-1}z \rangle ,\|\bullet\|_r)$ is a Banach ring.

    When $r=(1,\ldots,1)$, we omit $r^{-1}$ from our notations.
\end{example}

\begin{proposition}\label{prop-convseriesradiusBanach}
    In the setting of \cref{ex-convseriesradius}, $(A\langle r^{-1}z \rangle  ,\|\bullet\|_r)$ is a Banach ring.
\end{proposition}
\begin{proof}
    By induction, we may assume that $n=1$.

    It is obvious that $\|\bullet\|_r$ is a norm on the undelrying Abelian group. To see that $\|\bullet\|_r$ is a norm on the ring $A\langle r^{-1}z \rangle $, we need to verify the condition in \cref{def-seminormring}. Condition~(3) in \cref{def-seminormring} is obvious. 
    Let us consider Condition~(4). 
    Let 
    \[
        f=\sum_{i=0}^{\infty} a_i z^i,\quad g=\sum_{j=0}^{\infty} b_j z^j
    \]
    be two elements in $A\langle r^{-1}z \rangle $. Then
    \[
        fg=\sum_{k=0}^{\infty} \left(\sum_{i+j=k}a_i b_j\right) z^k. 
    \]
    We compute
    \[
        \|fg\|_r=  \sum_{k=0}^{\infty} \left\|\sum_{i+j=k}a_i b_j\right\| r^k\leq \sum_{k=0}^{\infty} \left(\sum_{i+j=k}\|a_i\|\cdot \|b_j\|\right) r^k=\|f\|_r\cdot \|g\|_r. 
    \]
    It remains to verify that $A\langle r^{-1}z \rangle$ is complete. 


    For this purpose, take a Cauchy sequence 
    \[
        f^b=\sum_{i=0}^{\infty}a^b_i z^i\in A\langle r^{-1}z \rangle
    \]
    for $b\in \mathbb{N}$. Then for each $i$, the coefficients $(a^b_i)_b$ is a Cauchy sequence in $A$. Let $a_i$ be the limit of $a^b_i$ as $b\to\infty$ and set 
    \[
        f=\sum_{i=0}^{\infty}a_i z^i\in A[[z]].
    \]
    We need to show that $f\in A\langle r^{-1}z \rangle$ and $f^b\to f$. 
    
    Fix a constant $\epsilon>0$. There is $m=m(\epsilon)>0$ such that for all $j\geq m$ and all $k\geq 0$, we have
    \[
        \sum_{i=0}^{\infty}\|a^{j+k}_i-a^{j}_i\|r^{i}<\epsilon/2.
    \]
    In particular, for any $s>0$, we have
    \[
        \sum_{i=0}^{s}\|a_i-a^{j}_i\|r^{i}\leq \sum_{i=0}^{s}\|a_i-a^{j+k}_i\|r^{i}+\sum_{i=0}^{s}\|a^j_i-a^{j+k}_i\|r^{i}\leq \sum_{i=0}^{s}\|a_i-a^{j+k}_i\|r^{i}+\epsilon/2.
    \]
    When $k$ is large enough, we can guarantee that
    \[
        \sum_{i=0}^{s}\|a_i-a^{j+k}_i\|r^{i}<\epsilon/2.  
    \]
    So
    \[
        \sum_{i=0}^{s}\|a_i-a^{j}_i\|r^{i}\leq \epsilon.  
    \]
    Let $s\to\infty$, we find
    \[
        \|f-f^j\|_r\leq \sum_{i=0}^{\infty}\|a_i-a^{j}_i\|r^{i}\leq \epsilon.  
    \]
    In particular, $\|f\|_r<\infty$ and $f^j\to f$ as $j\to\infty$.
\end{proof}

\section{Semi-normed modules}

\begin{definition}
    Let $(A,\|\bullet\|_A)$ be a normed ring. A \emph{semi-normed $A$-module} (resp. \emph{normed $A$-module}) is a pair $(M,\|\bullet\|_M)$ consisting of a $A$-module $M$ and a semi-norm (resp. norm) on the underlying Abelian group of $M$ such that there is a constant $C>0$ such that
    \[
      \|fm\|_M \leq C \|f\|_A \|m\|_M
    \]
    for all $f\in A$ and $m\in M$. When $\|\bullet\|_M$ is clear from the context, we say $M$ is a semi-normed $A$-module (resp. normed $A$-module).

    A \emph{Banach $\mathcal{A}$-module} is a normed $A$-module which is complete with respect to the metric \cref{lma-pmetricinducedbyseminorm}.
\end{definition}



\printbibliography
\end{document}