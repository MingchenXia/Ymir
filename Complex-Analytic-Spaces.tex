
\documentclass{amsbook} 
%\usepackage{xr}
\usepackage{xr-hyper}
\usepackage[unicode]{hyperref}


\usepackage[T1]{fontenc}
\usepackage[utf8]{inputenc}
\usepackage{lmodern}
\usepackage{amssymb,tikz-cd}
%\usepackage{natbib}
\usepackage[english]{babel}

\usepackage[nameinlink,capitalize]{cleveref}
\usepackage[style=alphabetic,maxnames=99,maxalphanames=5, isbn=false, giveninits=true, doi=false]{biblatex}
\usepackage{lipsum, physics}
\usepackage{ifthen}
\usepackage{microtype}
\usepackage{booktabs}
\usetikzlibrary{calc}
\usepackage{emptypage}
\usepackage{setspace}
\usepackage[margin=0.75cm, font={small,stretch=0.80}]{caption}
\usepackage{subcaption}
\usepackage{url}
\usepackage{bookmark}
\usepackage{graphicx}
\usepackage{dsfont}
\usepackage{enumitem}
\usepackage{mathtools}
\usepackage{csquotes}
\usepackage{silence}
\usepackage{mathrsfs}
\usepackage{bigints}

\WarningFilter{biblatex}{Patching footnotes failed}


\ProcessOptions\relax

\emergencystretch=1em

\hypersetup{
colorlinks=true,
linktoc=all
}

\setcounter{tocdepth}{1}


\hyphenation{archi-medean  Archi-medean Tru-ding-er}

%\captionsetup[table]{position=bottom}   %% or below
\renewcommand{\thefootnote}{\fnsymbol{footnote}}
%\DeclareMathAlphabet{\mathcal}{OMS}{cmsy}{m}{n}
\renewbibmacro{in:}{}

\DeclareFieldFormat[article]{citetitle}{#1}
\DeclareFieldFormat[article]{title}{#1}
\DeclareFieldFormat[inbook]{citetitle}{#1}
\DeclareFieldFormat[inbook]{title}{#1}
\DeclareFieldFormat[incollection]{citetitle}{#1}
\DeclareFieldFormat[incollection]{title}{#1}
\DeclareFieldFormat[inproceedings]{citetitle}{#1}
\DeclareFieldFormat[inproceedings]{title}{#1}
\DeclareFieldFormat[phdthesis]{citetitle}{#1}
\DeclareFieldFormat[phdthesis]{title}{#1}
\DeclareFieldFormat[misc]{citetitle}{#1}
\DeclareFieldFormat[misc]{title}{#1}
\DeclareFieldFormat[book]{citetitle}{#1}
\DeclareFieldFormat[book]{title}{#1} 


%% Define various environments.

\theoremstyle{definition}
\newtheorem{theorem}{Theorem}[section]
\newtheorem{thm}[theorem]{Theorem}
\newtheorem{proposition}[theorem]{Proposition}
\newtheorem{corollary}[theorem]{Corollary}
\newtheorem{lemma}[theorem]{Lemma}
\newtheorem{conjecture}[theorem]{Conjecture}
\newtheorem{question}[theorem]{Question}
\newtheorem{example}[theorem]{Example}
\newtheorem{definition}[theorem]{Definition}
\newtheorem{condition}[theorem]{Condition}

\theoremstyle{remark}
\newtheorem{remark}[theorem]{Remark}
\numberwithin{equation}{section}

%\renewcommand{\thesection}{\thechapter.\arabic{section}}
%\renewcommand{\thetheorem}{\thesection.\arabic{theorem}}
%\renewcommand{\thedefinition}{\thesection.\arabic{definition}}
%\renewcommand{\theremark}{\thesection.\arabic{remark}}


%% Define new operators

\DeclareMathOperator{\nd}{nd}
\DeclareMathOperator{\ord}{ord}
\DeclareMathOperator{\Hom}{Hom}
\DeclareMathOperator{\PreSh}{PreSh}
\DeclareMathOperator{\Gr}{Gr}
\DeclareMathOperator{\Homint}{\mathcal{H}\mathrm{om}}
\DeclareMathOperator{\Torint}{\mathcal{T}\mathrm{or}}
\DeclareMathOperator{\Div}{div}
\DeclareMathOperator{\DSP}{DSP}
\DeclareMathOperator{\Diff}{Diff}
\DeclareMathOperator{\MA}{MA}
\DeclareMathOperator{\NA}{NA}
\DeclareMathOperator{\AN}{an}
\DeclareMathOperator{\Rep}{Rep}
\DeclareMathOperator{\Rest}{Res}
\DeclareMathOperator{\DF}{DF}
\DeclareMathOperator{\VCart}{VCart}
\DeclareMathOperator{\PL}{PL}
\DeclareMathOperator{\Bl}{Bl}
\DeclareMathOperator{\Td}{Td}
\DeclareMathOperator{\Fitt}{Fitt}
\DeclareMathOperator{\Ric}{Ric}
\DeclareMathOperator{\coeff}{coeff}
\DeclareMathOperator{\Aut}{Aut}
\DeclareMathOperator{\Capa}{Cap}
\DeclareMathOperator{\loc}{loc}
\DeclareMathOperator{\vol}{vol}
\DeclareMathOperator{\Val}{Val}
\DeclareMathOperator{\ST}{ST}
\DeclareMathOperator{\Amp}{Amp}
\DeclareMathOperator{\Herm}{Herm}
\DeclareMathOperator{\trop}{trop}
\DeclareMathOperator{\Trop}{Trop}
\DeclareMathOperator{\Cano}{Can}
\DeclareMathOperator{\PS}{PS}
\DeclareMathOperator{\Var}{Var}
\DeclareMathOperator{\Psef}{Psef}
\DeclareMathOperator{\Jac}{Jac}
\DeclareMathOperator{\Char}{char}
\DeclareMathOperator{\Red}{red}
\DeclareMathOperator{\Spf}{Spf}
\DeclareMathOperator{\Span}{Span}
\DeclareMathOperator{\Der}{Der}
%\DeclareMathOperator{\Mod}{mod}
\DeclareMathOperator{\Hilb}{Hilb}
\DeclareMathOperator{\triv}{triv}
\DeclareMathOperator{\Frac}{Frac}
\DeclareMathOperator{\diam}{diam}
\DeclareMathOperator{\Spec}{Spec}
\DeclareMathOperator{\Spm}{Spm}
\DeclareMathOperator{\Specrel}{\underline{Sp}}
\DeclareMathOperator{\Sp}{Sp}
\DeclareMathOperator{\reg}{reg}
\DeclareMathOperator{\sing}{sing}
\DeclareMathOperator{\Star}{Star}
\DeclareMathOperator{\relint}{relint}
\DeclareMathOperator{\Cvx}{Cvx}
\DeclareMathOperator{\Int}{Int}
\DeclareMathOperator{\Supp}{Supp}
\DeclareMathOperator{\FS}{FS}
\DeclareMathOperator{\RZ}{RZ}
\DeclareMathOperator{\Redu}{red}
\DeclareMathOperator{\lct}{lct}
\DeclareMathOperator{\Proj}{Proj}
\DeclareMathOperator{\Sing}{Sing}
\DeclareMathOperator{\Conv}{Conv}
\DeclareMathOperator{\Max}{Max}
\DeclareMathOperator{\Tor}{Tor}
\DeclareMathOperator{\Gal}{Gal}
\DeclareMathOperator{\Frob}{Frob}
\DeclareMathOperator{\coker}{coker}
\DeclareMathOperator{\Sym}{Sym}
\DeclareMathOperator{\CSp}{CSp}
\DeclareMathOperator{\Img}{Im}


\newcommand{\alg}{\mathrm{alg}}
\newcommand{\Sh}{\mathrm{Sh}}
\newcommand{\fin}{\mathrm{fin}}
\newcommand{\BPF}{\mathrm{BPF}}
\newcommand{\dBPF}{\mathrm{dBPF}}
\newcommand{\divf}{\mathrm{Div}^f}
\newcommand{\nef}{\mathrm{nef}}
\newcommand{\Bir}{\mathrm{Bir}}
\newcommand{\hO}{\hat{\mathcal{O}}}
\newcommand{\bDiv}{\mathrm{Div}^{\mathrm{b}}}
\newcommand{\un}{\mathrm{un}}
\newcommand{\sep}{\mathrm{sep}}
\newcommand{\diag}{\mathrm{diag}}
\newcommand{\Pic}{\mathrm{Pic}}
\newcommand{\GL}{\mathrm{GL}}
\newcommand{\SL}{\mathrm{SL}}
\newcommand{\LS}{\mathrm{LS}}
\newcommand{\GLS}{\mathrm{GLS}}
\newcommand{\GLSi}{\mathrm{GLS}_{\cap}}
\newcommand{\PGLS}{\mathrm{PGLS}}
\newcommand{\Loc}[1][S]{_{\{{#1}\}}}
\newcommand{\cl}{\mathrm{cl}}
\newcommand{\otL}{\hat{\otimes}^{\mathbb{L}}}
\newcommand{\ddpp}{\mathrm{d}'\mathrm{d}''}
\newcommand{\TC}{\mathcal{TC}}
\newcommand{\ddPP}{\mathrm{d}'_{\mathrm{P}}\mathrm{d}''_{\mathrm{P}}}
\newcommand{\PSs}{\mathcal{PS}}
\newcommand{\Gm}{\mathbb{G}_{\mathrm{m}}}
\newcommand{\End}{\mathrm{End}}
\newcommand{\Aff}[1][X]{\mathcal{M}\left(\mathcal{#1}\right)}
\newcommand{\XG}[1][X]{{#1}_{\mathrm{G}}}
\newcommand{\convC}{\xrightarrow{C}}
\newcommand{\Vect}{\mathrm{Vect}}
\newcommand{\abso}[1]{\lvert#1\rvert}
\newcommand{\Mdl}{\mathrm{Model}}
\newcommand{\cn}{\stackrel{\sim}{\longrightarrow}}
\newcommand{\sbc}{\mathbf{s}}
\newcommand{\CH}{\mathrm{CH}}
\newcommand{\GR}{\mathrm{GR}}
\newcommand{\dc}{\mathrm{d}^{\mathrm{c}}}
\newcommand{\Nef}{\mathrm{Nef}}
\newcommand{\Adj}{\mathrm{Adj}}
\newcommand{\DHm}{\mathrm{DH}}
\newcommand{\An}{\mathrm{an}}
\newcommand{\Rec}{\mathrm{Rec}}
\newcommand{\dP}{\mathrm{d}_{\mathrm{P}}}
\newcommand{\ddp}{\mathrm{d}_{\mathrm{P}}'\mathrm{d}_{\mathrm{P}}''}
\newcommand{\ddc}{\mathrm{dd}^{\mathrm{c}}}
\newcommand{\ddL}{\mathrm{d}'\mathrm{d}''}
\newcommand{\PSH}{\mathrm{PSH}}
\newcommand{\CPSH}{\mathrm{CPSH}}
\newcommand{\PSP}{\mathrm{PSP}}
\newcommand{\WPSH}{\mathrm{WPSH}}
\newcommand{\Ent}{\mathrm{Ent}}
\newcommand{\NS}{\mathrm{NS}}
\newcommand{\QPSH}{\mathrm{QPSH}}
\newcommand{\proet}{\mathrm{pro-ét}}
\newcommand{\XL}{(\mathcal{X},\mathcal{L})}
\newcommand{\ii}{\mathrm{i}}
\newcommand{\Cpt}{\mathrm{Cpt}}
\newcommand{\bp}{\bar{\partial}}
\newcommand{\ddt}{\frac{\mathrm{d}}{\mathrm{d}t}}
\newcommand{\dds}{\frac{\mathrm{d}}{\mathrm{d}s}}
\newcommand{\Ep}{\mathcal{E}^p(X,\theta;[\phi])}
\newcommand{\Ei}{\mathcal{E}^{\infty}(X,\theta;[\phi])}
\newcommand{\infs}{\operatorname*{inf\vphantom{p}}}
\newcommand{\sups}{\operatorname*{sup*}}
\newcommand{\colim}{\operatorname*{colim}}
\newcommand{\ddtz}[1][0]{\left.\ddt\right|_{t={#1}}}
\newcommand{\tube}[1][Y]{]{#1}[}
\newcommand{\ddsz}[1][0]{\left.\ddt\right|_{s={#1}}}
\newcommand{\floor}[1]{\left \lfloor{#1}\right \rfloor }
\newcommand{\dec}[1]{\left \{{#1}\right \} }
\newcommand{\ceil}[1]{\left \lceil{#1}\right \rceil }
\newcommand{\Projrel}{\mathcal{P}\mathrm{roj}}
\newcommand{\Weil}{\mathrm{Weil}}
\newcommand{\Cart}{\mathrm{Cart}}
\newcommand{\bWeil}{\mathrm{b}\mathrm{Weil}}
\newcommand{\bCart}{\mathrm{b}\mathrm{Cart}}
\newcommand{\Cond}{\mathrm{Cond}}
\newcommand{\IC}{\mathrm{IC}}
\newcommand{\IH}{\mathrm{IH}}
\newcommand{\cris}{\mathrm{cris}}
\newcommand{\Zar}{\mathrm{Zar}}
\newcommand{\HvbCat}{\overline{\mathcal{V}\mathrm{ect}}}
\newcommand{\BanModCat}{\mathcal{B}\mathrm{an}\mathcal{M}\mathrm{od}}
\newcommand{\DesCat}{\mathcal{D}\mathrm{es}}
\newcommand{\RingCat}{\mathcal{R}\mathrm{ing}}
\newcommand{\SchCat}{\mathcal{S}\mathrm{ch}}
\newcommand{\AbCat}{\mathcal{A}\mathrm{b}}
\newcommand{\RSCat}{\mathcal{R}\mathrm{S}}
\newcommand{\LRSCat}{\mathcal{L}\mathrm{RS}}
\newcommand{\CLRSCat}{\mathbb{C}\text{-}\LRSCat}
\newcommand{\CRSCat}{\mathbb{C}\text{-}\RSCat}
\newcommand{\CLA}{\mathbb{C}\text{-}\mathcal{L}\mathrm{A}}
\newcommand{\CASCat}{\mathbb{C}\text{-}\mathcal{A}\mathrm{n}}
\newcommand{\LiuCat}{\mathcal{L}\mathrm{iu}}
\newcommand{\BanCat}{\mathcal{B}\mathrm{an}}
\newcommand{\BanAlgCat}{\mathcal{B}\mathrm{an}\mathcal{A}\mathrm{lg}}
\newcommand{\AnaCat}{\mathcal{A}\mathrm{n}}
\newcommand{\LiuAlgCat}{\mathcal{L}\mathrm{iu}\mathcal{A}\mathrm{lg}}
\newcommand{\AlgCat}{\mathcal{A}\mathrm{lg}}
\newcommand{\SetCat}{\mathcal{S}\mathrm{et}}
\newcommand{\ModCat}{\mathcal{M}\mathrm{od}}
\newcommand{\TopCat}{\mathcal{T}\mathrm{op}}
\newcommand{\CohCat}{\mathcal{C}\mathrm{oh}}
\newcommand{\SolCat}{\mathcal{S}\mathrm{olid}}
\newcommand{\AffCat}{\mathcal{A}\mathrm{ff}}
\newcommand{\AffAlgCat}{\mathcal{A}\mathrm{ff}\mathcal{A}\mathrm{lg}}
\newcommand{\QcohLiuAlgCat}{\mathcal{L}\mathrm{iu}\mathcal{A}\mathrm{lg}^{\mathrm{QCoh}}}
\newcommand{\LiuMorCat}{\mathcal{L}\mathrm{iu}}
\newcommand{\Isom}{\mathcal{I}\mathrm{som}}
\newcommand{\Cris}{\mathcal{C}\mathrm{ris}}
\newcommand{\Pro}{\mathrm{Pro}-}
\newcommand{\Fin}{\mathcal{F}\mathrm{in}}
\newcommand{\norms}[1]{\left\|#1\right\|}
\newcommand{\HPDDiff}{\mathbf{D}\mathrm{iff}}
\newcommand{\Menn}[2]{\begin{bmatrix}#1\\#2\end{bmatrix}}
\newcommand{\Fins}{\widehat{\Vect}^F}
\newcommand\blfootnote[1]{%
  \begingroup
  \renewcommand\thefootnote{}\footnote{#1}%
  \addtocounter{footnote}{-1}%
  \endgroup
}

\externaldocument[Introduction-]{Introduction}
%One variable complex analysis
%Several variables complex analysis
\externaldocument[Topology-]{Topology-Bornology}
\externaldocument[Banach-]{Banach-Rings}
\externaldocument[Commutative-]{Commutative-Algebra}
\externaldocument[Local-]{Local-Algebras}
\externaldocument[Complex-]{Complex-Analytic-Spaces}
%Properties of space
\externaldocument[Morphisms-]{Morphisms}
%Differential calculus
%GAGA
%Hilbert scheme complex analytic version

%Complex differential geometry

\externaldocument[Affinoid-]{Affinoid-Algebras}
\externaldocument[Berkovich-]{Berkovich-Analytic-Spaces}


\bibliography{Ymir}

\endinput



%\usepackage[color]{showkeys}
%\definecolor{refkey}{rgb}{0,0,1}

\title{Ymir}

\begin{document}







\maketitle


\tableofcontents

\chapter*{The notion of complex analytic spaces}\label{chap-complex}

\section{Introduction}\label{sec-introduction-Complexanalyticspaces}
We introduce the notion of complex analytic spaces in this section.

\section{\texorpdfstring{$\mathbb{C}$}{C}-ringed space}

\begin{definition}
    A \emph{$\mathbb{C}$-ringed space} is a pair $(X,\mathcal{O}_X)$ consisting of a topological space $X$ and a sheaf $\mathcal{O}_X$ of $\mathbb{C}$-algebras on $X$.

    A \emph{morphism of $\mathbb{C}$-ringed spaces $f:(Y,\mathcal{O}_Y)\rightarrow (X,\mathcal{O}_X)$} is a pair consisting of a continuous map $f:Y\rightarrow X$ and a morphism of sheaves of $\mathbb{C}$-algebras $f^{\#}:f^{-1}\mathcal{O}_X\rightarrow \mathcal{O}_Y$.

    Given two morphisms of $\mathbb{C}$-ringed spaces $f:(Y,\mathcal{O}_Y)\rightarrow (X,\mathcal{O}_X)$ and $g:(Z,\mathcal{O}_Z)\rightarrow (Y,\mathcal{O}_Y)$, their \emph{composition} is the morphism $f\circ g:(Z,\mathcal{O}_Z)\rightarrow (X,\mathcal{O}_X)$ consisting of the continuous map $f\circ g:Z\rightarrow X$ and a morphism of sheaves $(f\circ g)^{\#}=g^{\#}\circ g^{-1}f^{\#}:(f\circ g)^{-1}\mathcal{O}_X\cn g^{-1}f^{-1}\mathcal{O}_X\rightarrow \mathcal{O}_Z$.

    When there is no risk of confusion, we say $X$ is a $\mathbb{C}$-ringed space. In this case, we write $|X|$ for the topological space underlying $X$.
\end{definition}
It is straightforward to verify that $\mathbb{C}$-ringed spaces form a category, which we denote by $\CRSCat$.
Similarly, we denote by $\RSCat$ the category of ringed spaces defined in \cite[\href{https://stacks.math.columbia.edu/tag/0090}{Tag 0090}]{stacks-project}.

In fact, by definition a $\mathbb{C}$-ringed space is nothing but a morphism in the category of ringed spaces $X\rightarrow \mathbb{C}^0$, where $\mathbb{C}^0$ is a single point $*$ endowed with the sheaf of rings $\mathcal{O}_{\mathbb{C}^0}$ with $\mathcal{O}_{\mathbb{C}^0}(*)=\mathbb{C}$. In terms of slice categories, we have a canonical equivalence of categories 
\[
    \CRSCat\approx\RSCat/\mathbb{C}^0. 
\]
From this identification, most of the basic results above $\CRSCat$ follows, which we will use freely.

There is an obvious faithful forget functor $\CRSCat\rightarrow \RSCat$.


\begin{definition}
    A \emph{locally $\mathbb{C}$-ringed space} is a $\mathbb{C}$-ringed space $(X,\mathcal{O}_X)$ which when regarded as a ringed space is a locally ringed space.

    A \emph{morphism} between two locally $\mathbb{C}$-ringed spaces is a morphism between the underlying $\mathbb{C}$-ringed spaces which is a morphism of locally ringed spaces at the same time.

    The category of locally $\mathbb{C}$-ringed spaces is denoted by $\CLRSCat$.
\end{definition}
We refer to \cite[\href{https://stacks.math.columbia.edu/tag/01HA}{Tag 01HA}]{stacks-project} for the notion of locally ringed spaces. Similar to the case of $\mathbb{C}$-ringed space, we have a canonical equivalence of categories
\[
    \CLRSCat\approx\LRSCat/\mathbb{C}^0. 
\]

\begin{example}\label{ex-Cnringed}
    Let $n\in \mathbb{N}$, we define a sheaf of $\mathbb{C}$-algebras $\mathcal{O}_{\mathbb{C}^n}$ on $\mathbb{C}^n$ as follows: for any open subset $U\subseteq \mathbb{C}^n$, $\mathcal{O}_{\mathbb{C}^n}(U)$ is the $\mathbb{C}$-algebra of holomorphic functions on $U$. It is easy to see that $\mathcal{O}_{\mathbb{C}^n}$ is a sheaf and $(\mathbb{C}^n,\mathcal{O}_{\mathbb{C}^n})$ is a $\mathbb{C}$-ringed space. Moreover, it is easy to show that $(\mathbb{C}^n,\mathcal{O}_{\mathbb{C}^n})$ is a locally $\mathbb{C}$-ringed space.
\end{example}

\begin{proposition}\label{prop-Cnlocalring}
    Let $n\in \mathbb{N}$, $w\in \mathbb{C}^n$, then there is a natural isomorphism $\mathcal{O}_{\mathbb{C}^n,w}\cong \mathbb{C}\{ z_1,\ldots,z_n\}$.
\end{proposition}
The ring on the right-hand side is defined  in 
\cref{Local-def-ringconvpowerseries} in the Complex Analytic Local Algebras.
\begin{proof}
    This is a well-known result from classical complex analysis. \textcolor{red}{Include details later}.
\end{proof}


\section{Complex model spaces and complex analytic spaces}



\begin{definition}\label{def-sheafondomain}
Given any domain $D$ in $\mathbb{C}^n$, we can define a sheaf of $\mathbb{C}$-algebras $\mathcal{O}_D$ on $D$ as the restriction of $\mathcal{O}_{\mathbb{C}^n}$ defined in \cref{ex-Cnringed} to $D$. Observe that $(D,\mathcal{O}_D)$ is a locally $\mathbb{C}$-ringed space. 
\end{definition}


\begin{definition}\label{def-complexmodelspace}
    A \emph{complex model space} is a $\mathbb{C}$-ringed space $(X,\mathcal{O}_X)$ such that there exist
    \begin{enumerate}
        \item a domain $D$ in $\mathbb{C}^n$ for some $n\in \mathbb{N}$ and
        \item an ideal sheaf $\mathcal{I}$ in $\mathcal{O}_D$ of finite type
    \end{enumerate}
    such that thre is an isomorphism
    \[
        (X,\mathcal{O}_X)\cong (\Supp \mathcal{O}_D/\mathcal{I},i^{-1}(\mathcal{O}_D/\mathcal{I}))
    \]
    in the category of $\CRSCat$, where $i:\Supp \mathcal{O}_D/\mathcal{I}\rightarrow D$ is the inclusion map. Here $\mathcal{O}_D$ is the sheaf of $\mathbb{C}$-algebras defined in \cref{def-sheafondomain}.

    Clearly, $(X,\mathcal{O}_X)$ is a locally $\mathbb{C}$-ringed space.
\end{definition}
Observe that $X$ is always a Hausdorff space.

\begin{definition}\label{def-complexanalyticspace}
    A \emph{complex analytic space} is a locally $\mathbb{C}$-ringed space $(X,\mathcal{O}_X)$ such that 
    \begin{enumerate}
        \item $X$ is a Hausdorff space.
        \item For any $x\in X$, there is an open neighbourhood $U\subseteq X$ of $x$ such that $(U,\mathcal{O}_U:=\mathcal{O}_X|_U)$ is isomorphic to a complex model space in the sense of \cref{def-complexmodelspace} in the category $\CLRSCat$.
    \end{enumerate}
    When there is no risk of confusion, we also omit $\mathcal{O}_X$ from the notation say $X$ is a complex analytic space.

    A morphism between complex analytic spaces is a morphism of the underlying locally $\mathbb{C}$-ringed spaces. Such a morphism is also known as a \emph{holomorphic map}.

    The category of complex analytic spaces is denoted as $\CASCat$.
\end{definition}
\begin{remark}
    It seems that all authors on this subject requires that complex analytic spaces be Hausdorff, which may seem unnatural from the eyes of an algebro-geometrist. Morally, Hausdorffness corresponds to separatedness in the scheme world. However, non-Hausdorff analytic spaces do not seem to play a major role, in contrast to non-separated schemes, so we stick to the current definition.
\end{remark}
\begin{remark}
    Most of the authors require extra conditions in the definition of a complex analytic space: $\sigma$-compactness, paracompactness, having countable basis etc. We will not put these constraints in the definition, instead, we choose to include them into the assumptions of the theorems.
\end{remark}

\begin{proposition}
    Let $X$ be a complex analytic space, $x\in X$. Then $\mathcal{O}_{X,x}$ is a complex analytic local algebra.
\end{proposition}
Recall that complex analytic local algebras are defined in \cref{Local-def-complexanalylocaalg} in the Complex Analytic Local Algebras.

\begin{proof}
    The problem is local, so we may assume that $X$ is a complex model space. In this case, the result follows easily from \cref{prop-Cnlocalring}.
\end{proof}

\section{Weierstrass map}

\section{Oka's coherence theorem}

\textcolor{red}{This lemma needs to be placed elsewhere. Proof at CAS p58 needs to be included}
\begin{lemma}\label{lma-formalcoherencecriterion}
    Let $X$ be a topological space and $\mathcal{A}$ be a Hausdorff sheaf of rings on $X$ (in the sense that the espace étalé of $\mathcal{A}$ is Hausdorff) such that all stalks of $\mathcal{A}$ are integral domains. Then $\mathcal{A}$ is coherent if and only if for any open set $V\subseteq X$ and any section $s\in \mathcal{A}(X)$, $\mathcal{A}_V/s\mathcal{A}_V$ is coherent at every $x\in V$ where $s_x\neq 0$.
\end{lemma}

\begin{lemma}[Oka]\label{lma-Okacoh}
    For any $n\in \mathbb{N}$, $\mathcal{O}_{\mathcal{C}^n}$ is coherent.
\end{lemma}
\begin{proof}
    As a preparation, observe that $\mathcal{O}_{\mathbb{C}^n}$ is a Hausdorff sheaf.
    
    For any two germs $s_i\in \mathcal{O}_{\mathbb{C}^n,a_i}$ ($i=1,2$), we need to construct disjoint open neighbourhoods $U_i$ in the espace étalé of $\mathcal{O}_{\mathbb{C}^n}$ of $s_i$. If $a_1\neq a_2$, the assertion is clear. So assume that $a_1=a_2=0$. We extend $s_i$ to $f_i\in \mathcal{O}_{\mathbb{C}^n}(U)$ for a connected open neighbourhood $U\subseteq \mathbb{C}^n$ of $0$. Then $\{f_x:x\in U\}$ and $\{g_x:x\in U\}$ are disjoint: if for some $z\in U$, $f_z=g_z$, then the same holds in a neighbourhood of $z$ and so $f=g$ on $U$ by Identitätssatz. \textcolor{red}{Include the proof}

    We will prove the coherence of $\mathcal{O}_{\mathcal{C}^n}$ by induction on $n$. The case $n=0$ is trivial. Assume that $n>0$ and the theorem has been proved for all smaller $n$. We will apply \cref{lma-formalcoherencecriterion}. Take an open set $U\subseteq \mathbb{C}^n$ and $g\in \mathcal{O}_{\mathbb{C}^n}(U)$. We need to show that $\mathcal{O}_U/g\mathcal{O}_U$ is coherent at all $x\in U$ with $g_x\neq 0$.

    Fix such a point $x$, which may be assumed to be $0$. We may assume that $g(0)=0$ as otherwise, the stalk of $\mathcal{O}_U/g\mathcal{O}_U$ at $0$ is trivial. By perturbing the coordinates, we may guarantee that $g_0(0,w)$ is not identically $0$ for $w\in \mathbb{C}$. By Weierstrass preparation theorem \cref{Commutative-thm-Weierstrassprep}, there is a Weierstrass polynomial $\omega_0\in \mathcal{O}_{\mathbb{C}^{n-1},0}[w]$ such that $g_0\mathcal{O}_{\mathbb{C}^n,0}=\omega_0\mathcal{O}_{\mathbb{C}^n,0}$. Lift $\omega_0$ to $\omega\in \mathcal{O}_{\mathbb{C}^{n-1}}(B)$ for some neighbourhood $B\subseteq \mathbb{C}^{n-1}$ of $0$. In order to show the coherence of $\mathcal{O}_U/g\mathcal{O}_U$ near $0$, it suffices to show that $\mathcal{O}_{B\times \mathbb{C}}/\omega \mathcal{O}_{B\times \mathbb{C}}$ near $0$. Let $A=Z(\omega)\subseteq B\times \mathbb{C}$ be the closed subspace defined by the coherent sheaf generated by $\omega$, then it suffices to show that $\mathcal{O}_{A}$ is coherent near $0$. Now we have the finite Weierstrass morphism $A\rightarrow B$\textcolor{red}{Include}, it suffices to prove the coherence of $\mathcal{O}_B$, which follows from inductive hypothesis.
\end{proof}
As a corollary, we have the important Oka's coherence theorem.
\begin{thm}\label{thm-Okacoh}
    Let $X$ be a complex analytic space, then $\mathcal{O}_X$ is coherent.
\end{thm}
\begin{proof}
    The problem is local on $X$, so we may assume that $X$ is a complex model space, say there is a closed immersion into a domain $D$ in $\mathbb{C}^n$ defined by an ideal of finite type $\mathcal{I}$. By \cref{lma-Okacoh}, $\mathcal{O}_D$ is coherent and hence $\mathcal{I}$ is coherent. It follows that $\mathcal{O}_D/\mathcal{I}$ is coherent and hence $\mathcal{O}_X$ is coherent.
\end{proof}



\section{Rückert Nullstellensatz}

Let $X$ be a complex analytic space. It is a sheaf of $\mathbb{C}$-algebras.
For any sheaf of local $\mathbb{C}$-algebras $\mathcal{A}$ on $X$, any open set $U\subseteq X$ and any $s\in \mathcal{A}_X(U)$. We want to construct a function $[s]:U\rightarrow \mathbb{C}$.

Take $x\in U$, there is a canonical splitting
\begin{equation}\label{eq:Axdecomp}
  \mathcal{A}_{x}\cong \mathbb{C}\oplus \mathfrak{m}_{x}, 
\end{equation}
where $\mathfrak{m}_x$ is the maximal ideal of $\mathcal{A}_{x}$. Then we define $[s](x)$ as the image of $s_x$ in the $\mathbb{C}$-factor in \eqref{eq:Axdecomp}.

\begin{definition}\label{def-valuesectionatapoint}
    Let $X,\mathcal{A},U,x,s$ be as above. The value $[s](x)\in \mathbb{C}$ is called the \emph{value} of $s$ at $x$. We sometimes denote it by $s(x)$ as well.
\end{definition}

\begin{lemma}
    Let $X$ be a complex analytic space. We denote by $\mathcal{C}_X$ the sheaf of continuous functions on $X$. 
    The association $s\mapsto [s]$ in \cref{def-valuesectionatapoint} defines a homomorphism of sheaves of $\mathbb{C}$-algebras $\mathcal{O}_X\rightarrow \mathcal{C}_X$. 
\end{lemma}
When there is no risk of confusion, we also write $s$ instead of $[s]$.
\begin{proof}
    We need to show that for any open set $U\subseteq X$ and any $s\in \mathcal{O}_X(U)$, $[s]$ is a continuous function on $U$. 
    
    We may clearly assume that $U=X$. The problem is local on $X$, so we may assume that $X$ is a complex model space in the sense of \cref{def-complexmodelspace} defined by a coherent ideal $\mathcal{I}$ in a domain $D$ in $\mathbb{C}^n$. By further localizing, we may assume that $s$ can be lifted to a section $f\in\mathcal{O}_D(D)$. Then $[s]=f|_X$ by definition. So the assertion follows from the fact that a holomorphic function on a domain is continuous.
\end{proof}

\begin{thm}[Rückert Nullstellensatz]
    Let $X$ be a complex analytic space and $\mathcal{F}$ be a coherent sheaf of $\mathcal{O}_X$-modules. Let $f\in \mathcal{O}_X(X)$ be a function that vanishes on $\Supp \mathcal{F}$. Then for any $x\in X$, there is an open neighbourhood $U\subseteq X$ of $x$ and $m\in \mathbb{Z}_{>0}$ such that $f^m\mathcal{F}|_U=0$.
\end{thm}
\begin{proof}
    We may assume that $x\in \Supp \mathcal{F}$ as otherwise there is nothing to prove. In particular, $f(x)=0$.

    \textbf{Step~1}. We first reduce the problem to a relatively simple situation.

    The problem is local on $X$, so we may assume that there is a domain $D$ containing $0$ in $\mathbb{C}^n$ and a closed immersion $\iota:X\rightarrow D$ sending $x$ to $0$. Consider the closed immersion $g:V\rightarrow D\times \mathbb{C}$ induced by $\iota$ and $f$. Assume that this theorem has been proved for $w$, $B\times \mathbb{C}$, $g_*\mathcal{F}$ in place of $f$, $X$, $\mathcal{F}$ respectively, then we would find an integer $m\in \mathbb{Z}_{>0}$ such that $w^m(g_*\mathcal{F})_0=0$. In particular, $f^m\mathcal{F}_x=0$. As $\mathcal{F}$ is coherent, there is an open neighbourhood $U\subseteq X$ of $x$ such that $f^t\mathcal{F}|_U=0$.

    \textbf{Step~2}. We are reduced to prove the following special case: let $D$ be a domain in $\mathbb{C}^n$ containing $0$, $\mathcal{F}$ is a coherent sheaf on $D$ whose support is contained in $\{(z,w)\in \mathbb{C}^{n-1}\times \mathbb{C}:(z,w)\in D,w=0\}$. Then there is $m\in \mathbb{Z}_{>0}$ such that $w^m\mathcal{F}_0=0$.

    Let $\mathcal{G}$ be the annihilator sheaf of $\mathcal{F}$: 
    \[
       \mathcal{G}:=\ker \left(\mathcal{O}_D\rightarrow \Homint_{\mathcal{O}_D}(\mathcal{F},\mathcal{F})\right),
    \]
    where the map $\mathcal{O}_D\rightarrow \Homint_{\mathcal{O}_D}(\mathcal{F},\mathcal{F})$ sends a local section $f$ of $\mathcal{O}_D$ to the endohomomorphism of multiplying by $f$ of $\mathcal{F}$.
    Then $\mathcal{G}$ is a coherent sheaf by Oka's coherence theorem \cref{thm-Okacoh}. So it has closed supports. But by our assumtion, the support of $\mathcal{G}$ contains all $w\neq 0$, so $\Supp \mathcal{G}=D$.

    Let $f\in \mathcal{G}_0$ be a non-zero element. Write
    \[
      f=\sum_{j=b}^{\infty}a_jw^j,\quad a_j\in \mathcal{O}_{\mathbb{C}^{n-1},0},a_b\neq 0
    \]
    for some $b\in \mathbb{N}$.
    We may assume that $b=0$ by replacing $f$ and $\mathcal{F}$ with $w^{-b}f$ and  $w^b\mathcal{F}$ respectively. We want to show that $w^m\mathcal{F}_0=0$ for some positive integer $m$. 
    
    When $a_0$ is a unit, namely when $a_0(0)\neq 0$, then $f$ is a unit, so $\mathcal{F}_0=0$. We make an induction on $n$. The case $n=1$ is trivial, as $a_0$ is always a unit.
    So we may assume that $a_0(0)=0$ and $n>1$. By perturbing the coordinates in $\mathbb{C}^{n-1}$, we may assume that $a_0$ is not identically zero in the variable $z_1$. 
    
    Shrinking $D$, we may assume that $f$ can be lifted to a holomorphic function $g\in \mathcal{O}_D(D)$ with $g\mathcal{F}=0$. By our assumption on $a_0$, we may assume that $Z(g)\cap \{(z_1,0,\ldots,0)\in D\}=\{0\}$. Hence, $D\cap \Supp \mathcal{F}$, which is a subset of $Z(g)$ also intersects the $z_1$-axis only at the origin.

    By \textcolor{red}{To be included}, we can find a product domain $B\times W\subseteq D$ with $B\subseteq \mathbb{C}$ and $W\subseteq \mathbb{C}^{n-1}$ containing $0$ such that the projection $h:(B\times W)\cap \Supp \mathcal{F}\rightarrow B$ is finite and $\mathcal{F}':=h_*(\mathcal{F}|_{B\times W})$ is a coherent sheaf of $\mathcal{O}_B$-modules. Observe that $\Supp \mathcal{F}'\subseteq \{(z_2,\ldots,z_{n-1},w)\in B:w=0\}$, we can apply the induction hypothesis to obtain $m\in \mathbb{Z}_{>0}$ such that $w^{m}\mathcal{F}'_0=0$. It follows that $w^{m}\mathcal{F}_0=0$.
\end{proof}

\section{Finite limits in the category of complex analytic spaces}

The goal of this section is to show that the category of complex analytic spaces admits finite limits.

As the category $\CASCat$ admits a final object, namely $\mathbb{C}^0$, the existence of finite limits is the same as the existence of fiber products by general abstract nonsense \cite[\href{https://stacks.math.columbia.edu/tag/002O}{Tag 002O}]{stacks-project}.

We begin by considering direct products, namely fiber products over $\mathbb{C}^0$. 

\begin{lemma}\label{lma-CmCnproductexist}
Let $m,n\in \mathbb{N}$. Then 
\[
    \mathbb{C}^m\times \mathbb{C}^n\cong \mathbb{C}^{m+n}. 
\]    
Here $\times$ denotes the product in $\CASCat$.
\end{lemma}
\begin{proof}
    By Yoneda lemma \cite[\href{https://stacks.math.columbia.edu/tag/001P}{Tag 001P}]{stacks-project}, it suffices to establish
    \[
        h_{\mathbb{C}^m\times \mathbb{C}^n}\cong h_{\mathbb{C}^{m+n}},
    \]
    where $h_{\bullet}$ denotes the functor of points \cite[\href{https://stacks.math.columbia.edu/tag/001O}{Tag 001O}]{stacks-project}. Take $T\in \CASCat$, then there are isomorphisms
    \[
        h_{\mathbb{C}^m\times \mathbb{C}^n}(T)\cn h_{\mathbb{C}^m}(T)\times h_{\mathbb{C}^m}(T)\cn (\mathcal{O}_{T}(T))^{m+n}\cn   h_{\mathbb{C}^{m+n}}(T),
    \]
    which are all functorial in $T$. We conclude.
\end{proof}

\begin{lemma}\label{lma-existfiberproductimmersion}
    Let $f:X\rightarrow Y$ be a morphism in $\CASCat$. Let $i:Z\rightarrow Y$ be a closed (resp. an open) immersion. Then the fiber product $X\times_Y Z$ exists. Moreover, $X\times_Y Z\rightarrow X$ is a closed (resp. an open) immersion and there is a natural identification $|X\times_Y Z|\cong |X|\times_{|Y|}|Z|$.
\end{lemma}
We can draw a Cartesian diagram
\[
    \begin{tikzcd}
        & X\times_Y Z \arrow[r]\arrow[d]\arrow[rd,"\square",phantom] & X \arrow[d,"f"]\\
        & Z  \arrow[r,"i"] & Y
    \end{tikzcd}
\]
\begin{proof}
    When $i$ is an open immersion, it suffices to take $X\times_Y Z$ as the open subspace of $X$ defined by $f^{-1}(i(Z))$.

    Let us consider the case where $i$ is a closed immersion defined by a coherent ideal sheaf $\mathcal{I}$. It is a general result that $X\times_Y Z$ in the category $\LRSCat$ exists \cite[\href{https://stacks.math.columbia.edu/tag/01HQ}{Tag 01HQ}]{stacks-project}. Let us show that $X\times_Y Z$ is a closed complex analytic subspace of $X$ and conclude. To do so, recall that $X\times_Y Z$ is by construction a closed subspace of $X$ defined by $\mathcal{J}:=\Img \left(f^*\mathcal{I}\rightarrow f^*\mathcal{O}_Y=\mathcal{O}_X\right)$. It suffices to show that $\mathcal{J}$ is of finite type. By this is clear as $\mathcal{I}$ is of finite type.

    The identification of the underlying topological space is obvious.
\end{proof}

\begin{lemma}\label{lma-existproductimplyexistimm}
    Let $X$, $Y$ be complex analytic spaces. Consider open (resp. closed) immersions $X'\rightarrow X$ and $Y'\rightarrow Y$. If $X\times Y$ exists, then so is $X'\times Y'$ and the natural morphism $X'\times Y'\rightarrow X\times Y$ is an open (resp. a closed) immersion.
\end{lemma}
\begin{proof}
    We form the following large Cartesian diagram
    \[
        \begin{tikzcd}
            Z \arrow[r] \arrow[d] \arrow[rd, "\square", phantom]   & X'' \arrow[r] \arrow[d] \arrow[rd, "\square", phantom]       & X' \arrow[d] \\
            Y'' \arrow[d] \arrow[r] \arrow[rd, "\square", phantom] & X\times Y \arrow[d] \arrow[r] \arrow[rd, "\square", phantom] & X \arrow[d]  \\
            Y' \arrow[r]                                           & Y \arrow[r]                                                  & \mathbb{C}^0
        \end{tikzcd}
    \]
    The existences of all but the lower right square are guaranteed by \cref{lma-existfiberproductimmersion}. More precisely, we first define the upper right square and the lower left square by \cref{lma-existfiberproductimmersion}. It follows from \cref{lma-existfiberproductimmersion} that $X''\rightarrow X\times Y$ is an open (resp. a closed) immersion. So we can apply \cref{lma-existfiberproductimmersion} again to construct the upper left square.

    It follows from general abstract nonsense that the big square is also Cartesian. Moreover, by \cref{lma-existfiberproductimmersion} again, $Z\rightarrow Y''$ and $Y''\rightarrow X\times Y$ are both open (resp. closed) immersions. It follows that $Z\rightarrow X\times Y$ is also an open (resp. a closed) immersion.
\end{proof}

\begin{corollary}\label{cor-existproductmodel}
    Let $X$, $Y$ be complex model spaces. Then $X\times Y$ exists.
\end{corollary}
\begin{proof}
    By \cref{lma-existproductimplyexistimm}, we may assume that $X$ and $Y$ are both domains in some $\mathbb{C}^m$ and $\mathbb{C}^n$ respectively. Then applying \cref{lma-existproductimplyexistimm} again, we reduce to the case where $X=\mathbb{C}^m$ and $Y=\mathbb{C}^n$. This case is handled in \cref{lma-CmCnproductexist}.
\end{proof}

\begin{corollary}\label{cor-finitedirectproductexist}
    Let $X$, $Y$ be complex analytic spaces. Then $X\times Y$ exists in $\CASCat$. Moreover, there is a natural identification $|X\times Y|\cong |X|\times |Y|$.
\end{corollary}
\begin{proof}
    Let 
    \[
        X=\bigcup_{i\in I_X} X_i,\quad   X=\bigcup_{j\in I_Y} Y_j
    \]
    be open coverings of $X$ by complex model spaces. Let $K=I_X\times I_Y$. For each $k=(i,j)\in K$, we let $Z_k=X_i\times Y_j$, whose existence is guaranteed by \cref{cor-existproductmodel}. Take another $k'=(i',j')\in K$, then 
    \[
        Z_{kk'}:=Z_k\cap Z_{k'}=(X_i\times X_{i'})\cap (Y_{j}\times Y_{j'})
    \]
    is an open subspace of $Z_k$. It is clear that $Z_{kk'}$ forms a glueing data. From the general result \cite[\href{https://stacks.math.columbia.edu/tag/01JB}{Tag 01JB}]{stacks-project}, we can glue $Z_k$'s into a locally ringed space $Z$. From the construction, $|Z|=|X|\times |Y|$ in the category of topological spaces, so $|Z|$ is Hausdorff. On the other hand, from the construction, locally $Z$ is isomorphic to some $Z_k$, so $Z$ is a complex analytic space. As $Z$ is clearly the product in the category of locally $\mathbb{C}$-ringed spaces, we conclude that $Z=X\times Y$ in  $\CASCat$.
\end{proof}

\begin{corollary}
    The category $\CASCat$ admits all finite limits. Moreover, finite limits commute with the forgetful functor $\CASCat\rightarrow \TopCat$.
\end{corollary}
\begin{proof}
    By \cite[\href{https://stacks.math.columbia.edu/tag/002O}{Tag 002O}]{stacks-project}, \cref{cor-finitedirectproductexist} and the existence of a final object in $\CASCat$ (namely, $\mathbb{C}^0$), it suffices to show the existence of fiber products. In other words, suppose that we are given three complex analytic spaces $Z,X,Y$ and morphisms $X\rightarrow Z$ and $Y\rightarrow Z$ in $\CASCat$, we need to prove the existence of $X\times_Z Y$. From the general abstract nonsense, we can define $X\times_Z Y=(X\times Z)_{Y\times Y,\Delta_Y}Y$:
    \[ 
    \begin{tikzcd}
        & X\times_Y Z \arrow[r]\arrow[d]\arrow[rd,"\square",phantom] & X\times Z \arrow[d]\\
        & Y  \arrow[r,"\Delta_Y"] & Y\times Y
    \end{tikzcd},
    \]
    where $\Delta_Y:Y\rightarrow Y\times Y$ is the diagonal morphism, which is a closed immersion, the existence of $X\times Z$ is guaranteed by \cref{cor-finitedirectproductexist} and the existence of the fiber product is guaranteed by \cref{lma-existfiberproductimmersion}.

    In order to verify that finite limits commute with the forgetful functor $\CASCat\rightarrow \TopCat$, it suffices to consider fiber products. By \cref{lma-existfiberproductimmersion}, we reduced to the case of finite products. In this case, the result is proved in \cref{cor-finitedirectproductexist}.
\end{proof}
\begin{remark}
    It is important to remember that the forgetful functor $\CASCat\rightarrow \CLRSCat$ does \emph{not} commute with finite limits, in contrast to the case of schemes \cite[\href{https://stacks.math.columbia.edu/tag/01JN}{Tag 01JN}]{stacks-project}. While the forgetful functor from the category of schemes $\SchCat$ to $\TopCat$ does not commute with finite limits.
    
    These facts indicate that there are essential differences between the theory of analytic spaces and the theory of schemes.
\end{remark}

Next we study the local rings of fiber products.
\begin{theorem}
    Let $Y$ be an object in $\CASCat$ and $X_1,X_2\in \CASCat_{/Y}$. Let $(x_1,x_2)$ be a point of $X_1\times_Y X_2$, namely, $x_i\in X_i$ for $i=1,2$ and the images of $x_1$ and $x_2$ in $Y$ coincide, say $y\in Y$. 
    Then there is a caonical isomorphism
    \[
        \mathcal{O}_{X_1\times_Y X_2, (x_1,x_2)}\cong \mathcal{O}_{X_1,x_1}\overline{\otimes}_{\mathcal{O}_{Y,y}}  \mathcal{O}_{X_2,x_2}.     
    \]
\end{theorem}
The analytic tensor product here is defined \cref{Local-def-analytictensor} in the Complex Analytic Local Algebras. We have shown its existence in \cref{Local-thm-relanalytictensorexist} in the same chapter.

\begin{proof}
    Comparing the constructions of both sides, we see that it suffices to prove the theorem in two special cases: when $Y=\mathbb{C}^0$ and when $X_2\rightarrow Y$ is a closed immersion. 

    We first consider the case where $Y=\mathbb{C}^0$. As our problem is local, we may assume that $X_1$ and $X_2$ are both complex model spaces. From the constructions, we easily reduce to the case where $X_1$ and $X_2$ are both domains in $\mathbb{C}^m$ and $\mathbb{C}^n$ respectively. In this case, the result is proved in \cref{Local-lma-Cmexistencetensor} in the Complex Analytic Local Algebras and \cref{prop-Cnlocalring}.
    
    Next we handle the case where $X_2\rightarrow Y$ is a closed immersion. This case is immediately clear from the constructions of both sides.
\end{proof}
 
\section{Complex analytic topos}

\printbibliography
\end{document}