
\documentclass{amsbook} 
%\usepackage{xr}
\usepackage{xr-hyper}
\usepackage[unicode]{hyperref}


\usepackage[T1]{fontenc}
\usepackage[utf8]{inputenc}
\usepackage{lmodern}
\usepackage{amssymb,tikz-cd}
%\usepackage{natbib}
\usepackage[english]{babel}

\usepackage[nameinlink,capitalize]{cleveref}
\usepackage[style=alphabetic,maxnames=99,maxalphanames=5, isbn=false, giveninits=true, doi=false]{biblatex}
\usepackage{lipsum, physics}
\usepackage{ifthen}
\usepackage{microtype}
\usepackage{booktabs}
\usetikzlibrary{calc}
\usepackage{emptypage}
\usepackage{setspace}
\usepackage[margin=0.75cm, font={small,stretch=0.80}]{caption}
\usepackage{subcaption}
\usepackage{url}
\usepackage{bookmark}
\usepackage{graphicx}
\usepackage{dsfont}
\usepackage{enumitem}
\usepackage{mathtools}
\usepackage{csquotes}
\usepackage{silence}
\usepackage{mathrsfs}
\usepackage{bigints}

\WarningFilter{biblatex}{Patching footnotes failed}


\ProcessOptions\relax

\emergencystretch=1em

\hypersetup{
colorlinks=true,
linktoc=all
}

\setcounter{tocdepth}{1}


\hyphenation{archi-medean  Archi-medean Tru-ding-er}

%\captionsetup[table]{position=bottom}   %% or below
\renewcommand{\thefootnote}{\fnsymbol{footnote}}
%\DeclareMathAlphabet{\mathcal}{OMS}{cmsy}{m}{n}
\renewbibmacro{in:}{}

\DeclareFieldFormat[article]{citetitle}{#1}
\DeclareFieldFormat[article]{title}{#1}
\DeclareFieldFormat[inbook]{citetitle}{#1}
\DeclareFieldFormat[inbook]{title}{#1}
\DeclareFieldFormat[incollection]{citetitle}{#1}
\DeclareFieldFormat[incollection]{title}{#1}
\DeclareFieldFormat[inproceedings]{citetitle}{#1}
\DeclareFieldFormat[inproceedings]{title}{#1}
\DeclareFieldFormat[phdthesis]{citetitle}{#1}
\DeclareFieldFormat[phdthesis]{title}{#1}
\DeclareFieldFormat[misc]{citetitle}{#1}
\DeclareFieldFormat[misc]{title}{#1}
\DeclareFieldFormat[book]{citetitle}{#1}
\DeclareFieldFormat[book]{title}{#1} 


%% Define various environments.

\theoremstyle{definition}
\newtheorem{theorem}{Theorem}[section]
\newtheorem{thm}[theorem]{Theorem}
\newtheorem{proposition}[theorem]{Proposition}
\newtheorem{corollary}[theorem]{Corollary}
\newtheorem{lemma}[theorem]{Lemma}
\newtheorem{conjecture}[theorem]{Conjecture}
\newtheorem{question}[theorem]{Question}
\newtheorem{example}[theorem]{Example}
\newtheorem{definition}[theorem]{Definition}
\newtheorem{condition}[theorem]{Condition}

\theoremstyle{remark}
\newtheorem{remark}[theorem]{Remark}
\numberwithin{equation}{section}

%\renewcommand{\thesection}{\thechapter.\arabic{section}}
%\renewcommand{\thetheorem}{\thesection.\arabic{theorem}}
%\renewcommand{\thedefinition}{\thesection.\arabic{definition}}
%\renewcommand{\theremark}{\thesection.\arabic{remark}}


%% Define new operators

\DeclareMathOperator{\nd}{nd}
\DeclareMathOperator{\ord}{ord}
\DeclareMathOperator{\Hom}{Hom}
\DeclareMathOperator{\PreSh}{PreSh}
\DeclareMathOperator{\Gr}{Gr}
\DeclareMathOperator{\Homint}{\mathcal{H}\mathrm{om}}
\DeclareMathOperator{\Torint}{\mathcal{T}\mathrm{or}}
\DeclareMathOperator{\Div}{div}
\DeclareMathOperator{\DSP}{DSP}
\DeclareMathOperator{\Diff}{Diff}
\DeclareMathOperator{\MA}{MA}
\DeclareMathOperator{\NA}{NA}
\DeclareMathOperator{\AN}{an}
\DeclareMathOperator{\Rep}{Rep}
\DeclareMathOperator{\Rest}{Res}
\DeclareMathOperator{\DF}{DF}
\DeclareMathOperator{\VCart}{VCart}
\DeclareMathOperator{\PL}{PL}
\DeclareMathOperator{\Bl}{Bl}
\DeclareMathOperator{\Td}{Td}
\DeclareMathOperator{\Fitt}{Fitt}
\DeclareMathOperator{\Ric}{Ric}
\DeclareMathOperator{\coeff}{coeff}
\DeclareMathOperator{\Aut}{Aut}
\DeclareMathOperator{\Capa}{Cap}
\DeclareMathOperator{\loc}{loc}
\DeclareMathOperator{\vol}{vol}
\DeclareMathOperator{\Val}{Val}
\DeclareMathOperator{\ST}{ST}
\DeclareMathOperator{\Amp}{Amp}
\DeclareMathOperator{\Herm}{Herm}
\DeclareMathOperator{\trop}{trop}
\DeclareMathOperator{\Trop}{Trop}
\DeclareMathOperator{\Cano}{Can}
\DeclareMathOperator{\PS}{PS}
\DeclareMathOperator{\Var}{Var}
\DeclareMathOperator{\Psef}{Psef}
\DeclareMathOperator{\Jac}{Jac}
\DeclareMathOperator{\Char}{char}
\DeclareMathOperator{\Red}{red}
\DeclareMathOperator{\Spf}{Spf}
\DeclareMathOperator{\Span}{Span}
\DeclareMathOperator{\Der}{Der}
%\DeclareMathOperator{\Mod}{mod}
\DeclareMathOperator{\Hilb}{Hilb}
\DeclareMathOperator{\triv}{triv}
\DeclareMathOperator{\Frac}{Frac}
\DeclareMathOperator{\diam}{diam}
\DeclareMathOperator{\Spec}{Spec}
\DeclareMathOperator{\Spm}{Spm}
\DeclareMathOperator{\Specrel}{\underline{Sp}}
\DeclareMathOperator{\Sp}{Sp}
\DeclareMathOperator{\reg}{reg}
\DeclareMathOperator{\sing}{sing}
\DeclareMathOperator{\Star}{Star}
\DeclareMathOperator{\relint}{relint}
\DeclareMathOperator{\Cvx}{Cvx}
\DeclareMathOperator{\Int}{Int}
\DeclareMathOperator{\Supp}{Supp}
\DeclareMathOperator{\FS}{FS}
\DeclareMathOperator{\RZ}{RZ}
\DeclareMathOperator{\Redu}{red}
\DeclareMathOperator{\lct}{lct}
\DeclareMathOperator{\Proj}{Proj}
\DeclareMathOperator{\Sing}{Sing}
\DeclareMathOperator{\Conv}{Conv}
\DeclareMathOperator{\Max}{Max}
\DeclareMathOperator{\Tor}{Tor}
\DeclareMathOperator{\Gal}{Gal}
\DeclareMathOperator{\Frob}{Frob}
\DeclareMathOperator{\coker}{coker}
\DeclareMathOperator{\Sym}{Sym}
\DeclareMathOperator{\CSp}{CSp}
\DeclareMathOperator{\Img}{Im}


\newcommand{\alg}{\mathrm{alg}}
\newcommand{\Sh}{\mathrm{Sh}}
\newcommand{\fin}{\mathrm{fin}}
\newcommand{\BPF}{\mathrm{BPF}}
\newcommand{\dBPF}{\mathrm{dBPF}}
\newcommand{\divf}{\mathrm{Div}^f}
\newcommand{\nef}{\mathrm{nef}}
\newcommand{\Bir}{\mathrm{Bir}}
\newcommand{\hO}{\hat{\mathcal{O}}}
\newcommand{\bDiv}{\mathrm{Div}^{\mathrm{b}}}
\newcommand{\un}{\mathrm{un}}
\newcommand{\sep}{\mathrm{sep}}
\newcommand{\diag}{\mathrm{diag}}
\newcommand{\Pic}{\mathrm{Pic}}
\newcommand{\GL}{\mathrm{GL}}
\newcommand{\SL}{\mathrm{SL}}
\newcommand{\LS}{\mathrm{LS}}
\newcommand{\GLS}{\mathrm{GLS}}
\newcommand{\GLSi}{\mathrm{GLS}_{\cap}}
\newcommand{\PGLS}{\mathrm{PGLS}}
\newcommand{\Loc}[1][S]{_{\{{#1}\}}}
\newcommand{\cl}{\mathrm{cl}}
\newcommand{\otL}{\hat{\otimes}^{\mathbb{L}}}
\newcommand{\ddpp}{\mathrm{d}'\mathrm{d}''}
\newcommand{\TC}{\mathcal{TC}}
\newcommand{\ddPP}{\mathrm{d}'_{\mathrm{P}}\mathrm{d}''_{\mathrm{P}}}
\newcommand{\PSs}{\mathcal{PS}}
\newcommand{\Gm}{\mathbb{G}_{\mathrm{m}}}
\newcommand{\End}{\mathrm{End}}
\newcommand{\Aff}[1][X]{\mathcal{M}\left(\mathcal{#1}\right)}
\newcommand{\XG}[1][X]{{#1}_{\mathrm{G}}}
\newcommand{\convC}{\xrightarrow{C}}
\newcommand{\Vect}{\mathrm{Vect}}
\newcommand{\abso}[1]{\lvert#1\rvert}
\newcommand{\Mdl}{\mathrm{Model}}
\newcommand{\cn}{\stackrel{\sim}{\longrightarrow}}
\newcommand{\sbc}{\mathbf{s}}
\newcommand{\CH}{\mathrm{CH}}
\newcommand{\GR}{\mathrm{GR}}
\newcommand{\dc}{\mathrm{d}^{\mathrm{c}}}
\newcommand{\Nef}{\mathrm{Nef}}
\newcommand{\Adj}{\mathrm{Adj}}
\newcommand{\DHm}{\mathrm{DH}}
\newcommand{\An}{\mathrm{an}}
\newcommand{\Rec}{\mathrm{Rec}}
\newcommand{\dP}{\mathrm{d}_{\mathrm{P}}}
\newcommand{\ddp}{\mathrm{d}_{\mathrm{P}}'\mathrm{d}_{\mathrm{P}}''}
\newcommand{\ddc}{\mathrm{dd}^{\mathrm{c}}}
\newcommand{\ddL}{\mathrm{d}'\mathrm{d}''}
\newcommand{\PSH}{\mathrm{PSH}}
\newcommand{\CPSH}{\mathrm{CPSH}}
\newcommand{\PSP}{\mathrm{PSP}}
\newcommand{\WPSH}{\mathrm{WPSH}}
\newcommand{\Ent}{\mathrm{Ent}}
\newcommand{\NS}{\mathrm{NS}}
\newcommand{\QPSH}{\mathrm{QPSH}}
\newcommand{\proet}{\mathrm{pro-ét}}
\newcommand{\XL}{(\mathcal{X},\mathcal{L})}
\newcommand{\ii}{\mathrm{i}}
\newcommand{\Cpt}{\mathrm{Cpt}}
\newcommand{\bp}{\bar{\partial}}
\newcommand{\ddt}{\frac{\mathrm{d}}{\mathrm{d}t}}
\newcommand{\dds}{\frac{\mathrm{d}}{\mathrm{d}s}}
\newcommand{\Ep}{\mathcal{E}^p(X,\theta;[\phi])}
\newcommand{\Ei}{\mathcal{E}^{\infty}(X,\theta;[\phi])}
\newcommand{\infs}{\operatorname*{inf\vphantom{p}}}
\newcommand{\sups}{\operatorname*{sup*}}
\newcommand{\colim}{\operatorname*{colim}}
\newcommand{\ddtz}[1][0]{\left.\ddt\right|_{t={#1}}}
\newcommand{\tube}[1][Y]{]{#1}[}
\newcommand{\ddsz}[1][0]{\left.\ddt\right|_{s={#1}}}
\newcommand{\floor}[1]{\left \lfloor{#1}\right \rfloor }
\newcommand{\dec}[1]{\left \{{#1}\right \} }
\newcommand{\ceil}[1]{\left \lceil{#1}\right \rceil }
\newcommand{\Projrel}{\mathcal{P}\mathrm{roj}}
\newcommand{\Weil}{\mathrm{Weil}}
\newcommand{\Cart}{\mathrm{Cart}}
\newcommand{\bWeil}{\mathrm{b}\mathrm{Weil}}
\newcommand{\bCart}{\mathrm{b}\mathrm{Cart}}
\newcommand{\Cond}{\mathrm{Cond}}
\newcommand{\IC}{\mathrm{IC}}
\newcommand{\IH}{\mathrm{IH}}
\newcommand{\cris}{\mathrm{cris}}
\newcommand{\Zar}{\mathrm{Zar}}
\newcommand{\HvbCat}{\overline{\mathcal{V}\mathrm{ect}}}
\newcommand{\BanModCat}{\mathcal{B}\mathrm{an}\mathcal{M}\mathrm{od}}
\newcommand{\DesCat}{\mathcal{D}\mathrm{es}}
\newcommand{\RingCat}{\mathcal{R}\mathrm{ing}}
\newcommand{\SchCat}{\mathcal{S}\mathrm{ch}}
\newcommand{\AbCat}{\mathcal{A}\mathrm{b}}
\newcommand{\RSCat}{\mathcal{R}\mathrm{S}}
\newcommand{\LRSCat}{\mathcal{L}\mathrm{RS}}
\newcommand{\CLRSCat}{\mathbb{C}\text{-}\LRSCat}
\newcommand{\CRSCat}{\mathbb{C}\text{-}\RSCat}
\newcommand{\CLA}{\mathbb{C}\text{-}\mathcal{L}\mathrm{A}}
\newcommand{\CASCat}{\mathbb{C}\text{-}\mathcal{A}\mathrm{n}}
\newcommand{\LiuCat}{\mathcal{L}\mathrm{iu}}
\newcommand{\BanCat}{\mathcal{B}\mathrm{an}}
\newcommand{\BanAlgCat}{\mathcal{B}\mathrm{an}\mathcal{A}\mathrm{lg}}
\newcommand{\AnaCat}{\mathcal{A}\mathrm{n}}
\newcommand{\LiuAlgCat}{\mathcal{L}\mathrm{iu}\mathcal{A}\mathrm{lg}}
\newcommand{\AlgCat}{\mathcal{A}\mathrm{lg}}
\newcommand{\SetCat}{\mathcal{S}\mathrm{et}}
\newcommand{\ModCat}{\mathcal{M}\mathrm{od}}
\newcommand{\TopCat}{\mathcal{T}\mathrm{op}}
\newcommand{\CohCat}{\mathcal{C}\mathrm{oh}}
\newcommand{\SolCat}{\mathcal{S}\mathrm{olid}}
\newcommand{\AffCat}{\mathcal{A}\mathrm{ff}}
\newcommand{\AffAlgCat}{\mathcal{A}\mathrm{ff}\mathcal{A}\mathrm{lg}}
\newcommand{\QcohLiuAlgCat}{\mathcal{L}\mathrm{iu}\mathcal{A}\mathrm{lg}^{\mathrm{QCoh}}}
\newcommand{\LiuMorCat}{\mathcal{L}\mathrm{iu}}
\newcommand{\Isom}{\mathcal{I}\mathrm{som}}
\newcommand{\Cris}{\mathcal{C}\mathrm{ris}}
\newcommand{\Pro}{\mathrm{Pro}-}
\newcommand{\Fin}{\mathcal{F}\mathrm{in}}
\newcommand{\norms}[1]{\left\|#1\right\|}
\newcommand{\HPDDiff}{\mathbf{D}\mathrm{iff}}
\newcommand{\Menn}[2]{\begin{bmatrix}#1\\#2\end{bmatrix}}
\newcommand{\Fins}{\widehat{\Vect}^F}
\newcommand\blfootnote[1]{%
  \begingroup
  \renewcommand\thefootnote{}\footnote{#1}%
  \addtocounter{footnote}{-1}%
  \endgroup
}

\externaldocument[Introduction-]{Introduction}
%One variable complex analysis
%Several variables complex analysis
\externaldocument[Topology-]{Topology-Bornology}
\externaldocument[Banach-]{Banach-Rings}
\externaldocument[Commutative-]{Commutative-Algebra}
\externaldocument[Local-]{Local-Algebras}
\externaldocument[Complex-]{Complex-Analytic-Spaces}
%Properties of space
\externaldocument[Morphisms-]{Morphisms}
%Differential calculus
%GAGA
%Hilbert scheme complex analytic version

%Complex differential geometry

\externaldocument[Affinoid-]{Affinoid-Algebras}
\externaldocument[Berkovich-]{Berkovich-Analytic-Spaces}


\bibliography{Ymir}

\endinput
\title{Ymir}
\begin{document}
\maketitle
\tableofcontents

\chapter*{Analytic sets}\label{chap-AS}

\section{Introduction}\label{sec-introduction-AS}

\section{Remmert--Stein theorem}


\begin{lemma}\label{lma-generichypplaneintdimupp}
    Let $n\in \mathbb{N}$ and $U$ be a relatively compact open neighbourhood of $0$ in $\mathbb{C}^n$. Let $k\in \{0,1,\ldots,n-1\}$. We write $L^k$ for the intersection of $z_1=\cdots=z_{n-k}=0$ with $U$, where $z_1,\ldots,z_n$ are the coordinates on $\mathbb{C}^n$. Let $A$ be an analytic set in $U\setminus L^k$ of dimension $\leq k$. Then for $i=0,\ldots,k$, we can find a linear subspace $L'$ of $\mathbb{C}^n$ of dimension $n-k+i$ such that 
    \[
        \dim L'\cap A\leq i,\quad \dim L'\cap L^k\leq i.  
    \]
 \end{lemma}
 \begin{proof}
    We make an induction on $n$. When $n=0,1$, there is nothing to prove. Let $n>1$. If $i=k$, we just take $L'=\mathbb{C}^n$. 
    Assume $0\leq i<k$.

    Let $M_1,\ldots,M_N$ be the irreducible components of $A$. We may assume that no components are single points.
    Take a non-zero base point $p_j\in M_j$ for $j=1,\ldots,N$. Let $H$ be an $(n-1)$-dimensional linear subspace of $\mathbb{C}^n$ which does not contain $L^k$ or any of the points $p_1,\ldots,p_N$. Without loss of generality, we may guarantee that $H$ is given by $z_n=0$.

    Let $k_j$ denote the dimension of $M_j$ for $j=1,\ldots,N$. Let $M_j'=M_j\cap H$ for $j=1,\ldots,N$. Observe that the dimension of $M_j'$ is either $k_j$ or $k_j-1$ for $j=1,\ldots,N$. Let 
    \[
        M':=\bigcup_{i=1}^N M_i'.  
    \]
    Then $\dim M'\leq k-1$.
    By the inductive hypothesis, we can find a linear subspace $L'$ of $\mathbb{C}^n$ of dimension $n-k+1$ with the desired properties.
 \end{proof}

\begin{lemma}\label{lma-intersectionpundprod}
    Let $k\leq n$ be two elements in $\mathbb{N}$ and $D=\Delta^{k}\times \Delta^{n-k}$ be the product of two unit polycylinders. Write $L$ for $\Delta^k\times \{0\}$. Consider a non-empty analytic subset $M$ of $D\setminus L$ of dimension $k$ everywhere. Assume that $M$ does not intersect a neighbourhood of $\Delta^k\times\{y\in \mathbb{C}^{n-k}: \|y\|_{L^{\infty}}=1\}$. Then for any $\epsilon>0$, $M$ meets the polycylinder $\{(x,y)\in D:\|x\|_{L^{\infty}}<\epsilon, |y|_{L^{\infty}}\in (0,1)\}$.
\end{lemma}
\begin{proof}
    \textbf{Step~1}. We observe that for each $a\in \Delta^k$, the intersection
    \[
        \{(x,y)\in D:x=a\}\cap M  
    \]
    is discrete. In fact, by our assumption, the absolute values the coordinate functions of $\Delta^{n-k}$ obtain their maxima on each irreducible component of the intersection. By \cref{Morphisms-cor-takeallvalues}  in \nameref{Morphisms-chap-morphismscomplex}, these coordinates are all constant.

    \textbf{Step~2}. Let $(x^1,y^1)\in M$. Then $y^1\neq 0$ by assumption. We may assume that $x^1\neq 0$ as otherwise there is nothing to prove. Let us write $x^1=(x^1_1,\ldots,x^1_k)$, $y^1=(y^1_1,\ldots,y^1_{n-k})$ with $x^1_1\neq 0$ and $y^1_1\neq 0$.

    Let $b=(x^1_2,\ldots,x^1_k)$. Let $N$ be the intersection of $M$ with $\Delta\times \{b\}\times \Delta^{n-k}$. Then $N$ is non-empty and has dimension $1$ everywhere. In fact, by Krulls Hauptidealsatz, the dimension of $N$ at each point is at least $1$. By Step~1, the dimension is at most $1$.

    We argue that we can take $|z_1|$ on $M$ as small as wel wish. Suppose otherwise,
    \[
        \sup_{z\in M}|z_1|>0.
    \]
    Tkae $q\in \mathbb{Z}_{>0}$ with
    \[
        |x_1^1|^q<|y_1^1|.    
    \]
    Consider the function $f:N\rightarrow \mathbb{C}$ sending $(x,y)$ to $y_1/x_1^q$. Then $f$ is a morphism of complex analytic spaces and is bounded, say
    \[
        \sup_{(x,y)\in  N}|f(x,y)|=C_0.  
    \]
    Then $C_0>1$ by our choice of $q$. But at the boundary of $D$, $|z_1|=1$, so we find that $|f(x,y)|$ obtains its maximum on each irreducible component of $N$. So in particular, $|z_1|$ obtains its infimum on each irreducible component of $N$. This contradicts the fact that $N$ has dimension $1$ everywhere.

    We can now assume that $|x_1^1|<\epsilon$.
    Now we can replace $M$ by $\{x_1^1\}\times \Delta^{k-1}\times \Delta^{n-k}$ and reduce the value of $k$ by $1$. By induction, we conclude.
\end{proof}

\begin{lemma}[Fundamental lemma]\label{lma-fundlma}
    Let $X$ be a complex manifold and $F$ be a nowhere dense analytic set of dimension $\leq k$, where $k\in \mathbb{N}$. Let $E$ be an analytic set in $X\setminus F$ such that for any $x\in E$,
    \[
        \dim_x E=k.  
    \]
    Then
    \[
        \left\{x\in F: \bar{E}\text{ is analytic at }x \right\}  
    \]
    is clopen in $X$.
\end{lemma}
\begin{proof}
    The given set is clearly open. It suffices to show that it is closed.

    Let $p\in F$ be a point in the closure of the given set. We need to show that $\bar{E}$ is analytic at $p$. The problem is local on $X$, we may assume that $X$ is a complex model space. Then it is immediate that we can reduce to the case where $X$ is a domain in $\mathbb{C}^n$ for some $n\in \mathbb{N}$. By enlarging $F$, we may assume that $F$ is defined by $y=0$, where $x,y$ denote the first $k$ and the last $n-k$ coordinates on $X\subseteq \mathbb{C}^n$. Finally, we may assume that $p=0$.

    By \cref{lma-generichypplaneintdimupp}, we can take a linear subspace $L$ of $\mathbb{C}^n$ which meets $F$ and $E$ only at discrete points. We may arrange that $L$ is defined by the condition $x=0$.

    Take $\epsilon,\delta>0$ so that 
    \begin{enumerate}
        \item 
        \[
            S:=\left\{(x,y)\in \mathbb{C}^k\times \mathbb{C}^{n-k}: \|x\|_{L^{\infty}}<\epsilon,\|y\|_{L^{\infty}}<\delta \right\}\subseteq D;
        \]
        \item
        \[
            \left\{(x,y)\in \mathbb{C}^k\times \mathbb{C}^{n-k}: \|x\|_{L^{\infty}}<\epsilon,\|y\|_{L^{\infty}}=\delta \right\}\cap E=\emptyset.
        \]
    \end{enumerate}
    Observe that for all $a\in \mathbb{C}^k$, $\|a\|_{L^{\infty}}<\epsilon$, the intersection
    \[
        (\{a\}\times \mathbb{C}^{n-k})\cap E \cap \Int S 
    \]
    is discrete. In fact, the intersection is an analytic set in $S\setminus F$ and the absolute values of $y_1,\ldots,y_{n-k}$ take their maxima on each irreducible components by (2). So they are in fact constant.

    By our assumption, there are points at which $\bar{E}$ is analytic on $Z:=\{|x|<\epsilon,y=0\}$. Let $B_0$ be a connected component of the set of such points. We can equivalently view $B_0$ as an open subset of $\{|x|<\epsilon\}$. Then for any $a\in B_0$, the set
    \[
        F_a:=\{(x,y)\in \mathbb{C}^n:x=a\}\cap \bar{E}\cap \Int S  
    \]
    is discrete. Let $(x^1,y^1)$ be a point in this set, then $\bar{E}$ is equidimensional of dimension $k$ at this point. Each irreducible component $K_j$ at $(x^1,y^1)$ is a ramified covering of order $m_j$. We define the order $m(x^1,y^1)$ as this sum. 

    For each $a\in B_0$, we define $s(a)$ as the sum of multiplicities of points of $F_a$.  Then $s(a)$ is locally constant on $B_0$ and by (2), $s(a)$ is actually constant. Let $s$ be this common value.

    Assume that $\bar{E}$ is not analytic at $0$.  Then $B_0$ meets $|x|=\epsilon$, say at $x'$. Let $s'$ be the number of intersection points of $\{x=x'\}\cap E$ counting mulitiplicity. 

    Observe that $s'\leq s$, as otherwise, there will be more than $s$ points of $E$ over points of $B_0$ close to $x'$. But $s'\neq s$ as otherwise, we contradict \cref{lma-intersectionpundprod}.

    So $s'<s$. If $x\in B_0$ converges to $x'$, then at lest one of the $s$ points of $\bar{E}$ over $x$ converges to $(x',0)$ and the coordinates $y_1,\ldots,y_{n-k}$ of this point converge to $0$. The same holds for all boundary points of $B_0$ in $\{\|x\|_{L^{\infty}}<\epsilon\}$.

    We introduce $n-k$ unknowns $X_1,\ldots,X_{n-k}$ and set
    \[
        z=\sum_{j=1}^{n-k}y_j X_j.  
    \]
    If $(x,y^i)$ ($i=1,\ldots,s$) denotes the $s$-points of $\bar{E}$ lying over $x\in B_0$, then we set
    \[
        z^i:=  \sum_{j=1}^{n-k}y^i_j X_j
    \]
    for $i=1,\ldots,s$. Then $z^1\cdots z^s$ is a homogeneous polynomial of degree $s$. The coefficients are holomorphic on $B_0$ by Riemann extension theorem. As $B_0$ is not contained in $\bar{E}$, the coefficients are not all $0$.

    If $x\in B_0$ converges to a boundary point of $B_0$ in $\{\|x\|_{L^{\infty}}<\epsilon\}$, then all coefficients converge to $0$. 

    By \cref{Morphisms-prop-Rado} in  \nameref{Morphisms-chap-morphismscomplex}, we conclude that the boundary points of $B_0$ in the interior of $\{\|x\|_{L^{\infty}}<\epsilon\}$ lie in an analytic subset of codimension $1$. 

    Let $Q(z)=(z-z^1)\cdots (z-z^s)$. Hen $Q$ is a homogeneous polynomial of degree $s$ with respect to the $u_j$'s. The coefficients are holomorphic on $x\in B_0$ and are polynomials in the $y_i$'s. The vanishing of the coordinates defines exactly the part of $\bar{E}$ over $B_0$ in the interior of $S$. But the coefficients are bounded at the boundary, so they extend to holomorphic functions everywhere and in particular on $\{\|x\|_{L^{\infty}}<\epsilon\}$. The vanishing of the coefficients define an analytic set $E'$ in $B_0\times \{\|y\|_{L^{\infty}}<\delta\}$. Each point of $E'$ belongs to the part of $\bar{E}$ lying over $B_0$. So $\bar{E}$ is analytic at each point of $\{\{(x,0):\|x\|_{L^{\infty}}<\epsilon\}<\epsilon\}$. In particular, $B_0=\{\{\|x\|_{L^{\infty}}<\epsilon\}<\epsilon\}$. This is a contradiction.   
\end{proof}

\begin{thm}\label{thm-RS1}
    Let $X$ be a complex manifold and $F$ be a nowhere-dense analytic set in $X$ of dimension $\leq k\in \mathbb{N}$. Let $E$ be an analytic set in $X\setminus F$ all of whose irreducible components are of dimension $\geq k$ on each point. Consider a point $x\in F$ with $\dim_x F<k$. Then $\bar{E}$ is analytic at $x$.
\end{thm}
\begin{proof}
    Let $r=\dim_x F$.
    The problem is local. By \cref{PropertyComplex-thm-equidimlocusopen} in \nameref{PropertyComplex-chap-propcomplex},  we may assume that $F$ is of dimension $\leq r$ everywhere. We need to show that $\bar{E}$ is an analytic set in $X$.
     By induction on $r$, we may clearly assume that $F$ is a complex manifold of equidimension $r$ with respect to the reduced indudced structure.

    Again, as the problem is local, we may reduce to the case where $X$ is a complex model space and then to the case where $X$ is an open neighbourhood of $0$ $\mathbb{C}^n$ for some $n\in \mathbb{N}$. Let $p\in F$, we want to show that $\bar{E}$ is analytic at $p$. We may then assume that $p=0$. We can then rearrange $F$ so that $F$ is a linear subspace of dimension $r_0$. We can take a closed subspace $V$ of $X$ such that $V\setminus F$ intersects $E$ at an analytic subset of dimension $<k$. Let $E_1=E\setminus V$. Then 
    \[
        \overline{E_1}=\bar{E}.  
    \]
    As $\overline{E_1}$ is analytic at all points in $V\setminus F$, it follows from \cref{lma-fundlma} that $\overline{E_1}$ is analytic on all points of $V$. So $\bar{E}$ is analytic at points in $F$.

\end{proof}

\begin{thm}[Remmert--Stein]\label{thm-RS2}
    Let $X$ be a complex analytic space and $F$ be a nowhere-dense analytic set in $X$ of dimension $\leq k\in \mathbb{N}$. Let $E$ be an analytic set in $X\setminus F$ all of whose irreducible components are of dimension $\geq k$ on each point. Then 
    \[
        \left\{x\in F: \bar{E}\text{ is not analytic at }x \right\}  
    \]
    is an analytic set of dimension $k$ at each point.
\end{thm}
\begin{proof}
    The problem is local on $X$, so we may assume that $X$ is a complex model space. Then we reduce immediately to the case where $X$ is a domain in $\mathbb{C}^n$ for some $n\in \mathbb{N}$. In particular, we may assume that $X$ is a complex manifold.

    Let $F'$ be set of regular points of $F$ of dimension $k$ and $F_0'\subseteq F$ be the set of points where $\bar{E}$ is analytic. Then $F_0'$ is the union of some connected components of $F_0$ by \cref{lma-fundlma}.

    Let $F_1'$ be the union of the other connected components of $F'$. Observe that $G:=\overline{F_1'}$ is an analytic subset of $F$. Observe that $\bar{E}$ is analytic at no points of $G$. It suffices to show that $\bar{E}$ is analytic at all points of $F\setminus G$.

    Let $p\in F\setminus G$. We show that $\bar{E}$ is 
    If $\dim_p F<k$, we just apply \cref{thm-RS1}. So we may assume that $\dim_p F=k$. By our choice, $p\in \overline{F_0'}$. In a neighbourhood of $p$, the subset of $F$ consisting of points where $\bar{E}$ is not analytic is contained in $\overline{F_0'}\setminus F_0'$, which is an analytic set of dimension $<k$. We conclude again by \cref{thm-RS1}.
\end{proof}


\begin{corollary}\label{cor-fiberdimuscanatop}
    Let $f:X\rightarrow Y$ be a morphism of complex analytic spaces and $n\in \mathbb{N}$. Assume that $X$ is a complex manifold.
    Then 
    \[
        \left\{ x\in X: \dim_x f^{-1}(x)\geq n \right\}  
    \]
    is closed.
\end{corollary}
\begin{proof}
    Let $x\in X$, $\dim_x f^{-1}(x)=n$. We need to show that the fiber dimension in a neighbourhood of $x$ is at most $n$. 

    The problem is local, so we may assume that $Y$ is Hausdorff. Suppose our assertion is false, then we can find a sequence $x_i\in X$ convering to $x$ such that $\dim_{x_i}f^{-1}(x_i)>d$ for all $i\in \mathbb{Z}_{>0}$. Let $E_i$ be the irreducible component of $f^{-1}(x_i)$ containing $x_i$ such that $\dim_{x_i} E_i=\dim_{x_i}f^{-1}(x_i)$ for $i\in \mathbb{Z}_{>0}$.

    We may assume that $E_i$'s have the same dimension $d>n$ and $x_i$ and $x$ are all different. Let $M$ be the union of the $E_i$'s, then $M$ is an analytic set in $X\setminus f^{-1}(x)$. By \cref{thm-RS2}, $\bar{M}$ is analytic near $x$. This is absurd.
\end{proof}


\begin{corollary}[Remmert]\label{cor-fiberdimusczar}
    Let $f:X\rightarrow Y$ be a morphism of complex analytic spaces and $n\in \mathbb{N}$. 
    Then 
    \[
        \left\{ x\in X: \dim_x f^{-1}(x)\geq n \right\}  
    \]
    is an analytic set in $X$.
\end{corollary}
This result is not stated in the correct way in Remmert's paper. In most of Remmert's papers, the notion of codimension is misused.
\begin{proof}
    By \cref{cor-fiberdimuscanatop}, the given set is closed. It suffices to show that it is analytic along each point on $X$. In particular, we may assume that $X$ is connected.

    \textbf{Step~1}. We reduce to the case where $Y$ is a complex manifold.
    
    The problem is local on $Y$, so we may assume that  $Y$ is a complex model space. Then clearly, we can assume that $Y$ is a domain in $\mathbb{C}^n$ for some $n\in \mathbb{N}$. In particular, $Y$ is a complex manifold.


    \textbf{Step~2}. We first handle the case where $X$ is a complex manifold and the rank of $\Omega_{X/Y}$ is constant.

    In this case, we simply observe that $\dim_x f^{-1}(x)=\rank_x \Omega_{X/Y}$ and our assertion is obvious.

    \textbf{Step~3}. The problem is local on $X$, so we may assume that $\dim X<\infty$.

    Let 
    \[
        B=\left\{x\in X^{\reg}: \rank_x \Omega_{X/Y}>\tau \right\},  
    \]
    where 
    \[
        \tau:=  \min_{x'\in X}\rank_{x'} \Omega_{X/Y}.
    \]
    Then $B$ is an analytic set in $X^{\reg}$ by Step~2. The closure $\bar{B}$ is an analytic set in $X$, as this can be characterized by the condition that $\rank_x \Omega_{X/Y}>\tau$.
    Moreover, $\dim \bar{B}<N$. 
    
    We may assume that $n>\tau$, as there is nothing to prove otherwise. In particular,
    \[
        \left\{ x\in X: \dim_x f^{-1}(x)\geq n \right\}\subseteq \bar{B}\cup X^{\Sing}.  
    \]
    We write $D=\bar{B}\cup X^{\Sing}$ and endow it with the reduced induced structure.

    We make induction on $N:=\dim X$. The problem is trivial when $N=0$. Assume that $N\geq 1$. Then
    \[
        \left\{ x\in D_0: \dim_x f^{-1}(x)\geq n \right\}  
    \]
    is an analytic set in $D$ for each connected component $D_0$ of $D$.

    We observe that
    \[
        \left\{ x\in X: \dim_x f^{-1}(x)\geq n \right\}=\bigcup_{D_0} \left\{ x\in D_0: \dim_x f^{-1}(x)\geq n \right\},
    \]
    where $D_0$ runs over all connected components of $D$ and $N-s_0$ is the dimension of $D_0$. From this it follows that $\left\{ x\in X: \dim_x f^{-1}(x)\geq n \right\}$ is analytic, as the formula union on the right-hand side is locally finite.
\end{proof}

\begin{corollary}\label{cor-dimforopenspread}
    Let $f:X\rightarrow Y$ be a morphism of complex analytic spaces and $x\in X$. Assume that $X$ is equidimensional at $x$, $Y$ is unibranch at $f(x)$ and
    \[
        \dim_x X-\dim_x f^{-1}(f(x))= \dim_{f(x)}Y.
    \] 
    Then there is an open neighbourhood $U$ of $x$ in $X$ such that $U\rightarrow Y$ induced by $f$ is open.
\end{corollary}
\begin{proof}
    The problem is local on $X$. By \cref{PropertyComplex-thm-equidimlocusopen} in \nameref{PropertyComplex-chap-propcomplex}, up to shrinking $X$, we may assume that $X$ is equidimensional of dimension $\dim_x X$. By \cref{Morphisms-cor-dimformula}  in \nameref{Morphisms-chap-morphismscomplex}, 
    \[
        \dim_x X-\dim_z f^{-1}(f(z))\leq \dim_{f(z)}Y
    \] 
    for all $z\in X$. But as $\dim_z f^{-1}(f(z))$ is upper semi-continuous by \cref{cor-fiberdimusczar}, the set where equality holds is open. Our assertion follows from \cref{Morphisms-cor-dimformula}  in \nameref{Morphisms-chap-morphismscomplex}.
\end{proof}

\begin{corollary}\label{cor-morphismtoCopen}
    Let $X$ be a complex analytic space, $x\in X$ and $f\in \mathcal{O}_X(X)$. Assume that $f(x)=0$. Consider the following conditions:
    \begin{enumerate}
        \item $f:X\rightarrow \mathbb{C}$ is open in a neighbourhood of $x$;
        \item $f_x$ is a non-zero divisor in modulo each minimal prime of $\mathcal{O}_{X,x}$;
        \item $f:X\rightarrow \mathbb{C}$ is open at $x$.
    \end{enumerate}
    Then (1) implies (2) implies (3). If moreover $X$ is equidimensional at $x$, then (1) is equivalent to (2).
\end{corollary}
\begin{proof}
    (1) $\implies$ (2): If $f_x$ is a zero-divisor modulo some minimal prime of $\mathcal{O}_{X,x}$, then $f$ is identically $0$ on some irreducible component up to shrinking $X$. So $f$ cannot be open in a neighbourhood of $x$.

    (2) $\implies$ (3): By Krulls Hauptidealsatz, 
    \[
        \dim_x W(f)=\dim_x X-1.  
    \]
    By \cref{Morphisms-cor-dimformula}  in \nameref{Morphisms-chap-morphismscomplex}, $f$ is open at $x$. 

    (2) $\implies$ (1):
    If $X$ is equidimensional at $x$, then by Krulls Hauptidealsatz, 
    \[
        \dim_x W(f)=\dim_x X-1.  
    \]
    We conclude by \cref{cor-dimforopenspread}.
\end{proof}
\begin{corollary}\label{cor-pullbackactiveactive}
    Let $f:X\rightarrow Y$ be an open morphism of complex analytic spaces and $x\in X$. Assume that $Y$ is equidimensional at $f(x)$, then for any $g\in \mathfrak{m}_{Y,f(x)}$ which is a non-zero divisor modulo each minimal prime, then $f_x^{\#}(g)\in \mathfrak{m}_{X,x}$ is also a non-zero divisor modulo each minimal prime.
\end{corollary}
\begin{proof}
    The problem is local on $X$ and $Y$. Up to shrinking $X$ and $Y$, we may assume that $g$ and $f$ spreads to morphisms $Y\rightarrow \mathbb{C}$ and $X\rightarrow \mathbb{C}$ such that we have a commutative diagram
    \[
        \begin{tikzcd}
            X \arrow[r, "f"] \arrow[rd] & Y \arrow[d] \\
                                        & \mathbb{C} 
        \end{tikzcd}.  
    \]
    The morphism $Y\rightarrow \mathbb{C}$ is open by \cref{cor-morphismtoCopen}. It follows that $\alpha$ is also open. We conclude again by \cref{cor-morphismtoCopen}.
\end{proof}

\begin{corollary}\label{cor-openanddimformulaequiv}
    Let $f:X\rightarrow Y$ be a morphism of complex analytic spaces. Assume that $Y$ is equidimensional. Consider the following conditions:
    \begin{enumerate}
        \item $f$ is open;
        \item For any $x\in X$,
        \[
            \dim_x X-\dim_x f^{-1}(f(x))= \dim_{f(x)}Y.
        \] 
    \end{enumerate}
    Then (1) implies (2). If moreover, $Y$ is unibranch, then (1) and (2) are equivalent.
\end{corollary}
\begin{proof}
    (2) $\implies$ (1): Suppose that $Y$ is unibranch.
    This is a consequence of \cref{Morphisms-cor-dimformulaimpliesopen}  in \nameref{Morphisms-chap-morphismscomplex}.

    (1) $\implies$ (2):
    We may assume that $Y$ is connected and $X,Y$ are reduced. Fix $x\in X$ and write $y=f(x)$.
    We make an induction on $n=\dim Y$. When $n=0$, the assertion is trivial. Take $g\in \mathfrak{m}_{Y,y}$ which is a non-zero divisor modulo each minimal prime in $\mathcal{O}_{Y,y}$. By \cref{cor-pullbackactiveactive}, $h:=f_x^{\#}(g)\in \mathfrak{m}_{X,x}$ is also a non-zero divisor modulo each minimal prime.
    Let $X'$ and $Y'$ be the closed analytic spaces of $X$ and $Y$ defined by $h$ and $g$ respectively. Up to shrinking $X$ and $Y$, we may assume that there is a commutative square
    \[
        \begin{tikzcd}
            X' \arrow[r] \arrow[d, "f'"] & X \arrow[d, "f"] \\
            Y' \arrow[r]                 & Y               
        \end{tikzcd}.  
    \]
    By inductive hypothesis, 
    \[
        \dim_x X'=\dim_x X'_y+\dim_y Y'.  
    \]
    We conclude using Krulls Hauptidealsatz.
\end{proof}

\begin{corollary}
    Let $f:X\rightarrow Y$ be a flat morphism of complex analytic spaces. Then $f$ is open.
\end{corollary}
\begin{proof}
    \textbf{Step~1}.
    If $Y$ is unibranch, then we conclude using \cref{cor-openanddimformulaequiv} and \cref{Morphisms-prop-flatmorphismdimformula}  in \nameref{Morphisms-chap-morphismscomplex}. 
    
    \textbf{Step~2}.
    In general, we may assume that $Y$ is reduced. Let $\bar{Y}$ be the normalization of $Y$. Then $Y$ has the quotient topology with respect to $\bar{Y}\rightarrow Y$. So it suffices to show that the base change $X\times_Y \bar{Y}\rightarrow \bar{Y}$ is open. But we know that the latter is flat by \cref{Morphisms-prop-flatmorphismbasechange}  in \nameref{Morphisms-chap-morphismscomplex}. We conclude using Step~1.
\end{proof}



\begin{corollary}\label{cor-quasifinitelocuscoana}
    Let $f:X\rightarrow Y$ be a morphism of complex analytic spaces. Then
    \[
        \left\{x\in X: f\text{ is quasi-finite at }x\right\}  
    \] 
    is co-analytic.
\end{corollary}
\begin{proof}
    This follows immediately from \cref{cor-fiberdimusczar}.
\end{proof}

As an application of Remmert--Stein theorem, we prove Chow's theorem.

\begin{thm}
    Let $n\in \mathbb{N}$ and $X$ be a closed analytic subspace of $\mathbb{P}^n$. Then $X$ is the analytification of a closed subvariety of $\mathbb{P}^n$.
\end{thm}
\begin{proof}
    We may assume that $X$ is non-empty.
    Let $\pi:\mathbb{C}^{n+1}\setminus\{0\}\rightarrow \mathbb{P}^n$ be the projection and $Y=\pi^{-1}(X)$. Then $X$ is analytic in $\mathbb{C}^{n+1}\setminus\{0\}$. By \cref{thm-RS2}, $\bar{X}$ is an analytic set in $\mathbb{C}^{n+1}$.

    Choose an open ball $U$ in $\mathbb{C}^{n+1}$ centered at $0$ and finitely many holomorphic functions $f_1,\ldots,f_k\in \mathcal{O}_{\mathbb{C}^{n+1}}(U)$ such that $\bar{X}\cap U=W(f_1,\ldots,f_k)$. 
    Let $\mathcal{P}$ be the collection of homogeneous components of the $f_i$'s. Then
    \[
        X=\bigcap_{f\in \mathcal{P}}W(f).  
    \]
    In fact, let us denote the right-hand side by $Y$ for the moment.
    It is clear that $\bar{X}\cap U$ contains $Y\cap U$ and hence $\bar{X}\supseteq Y$. Conversely, take $x\in \bar{X}\cap U$, from the fact that $\lambda_x\in \bar{X}\cap U$ for all $\lambda\in \mathbb{C}$, $|\lambda|<1$, we find easily that all homogeneous components of the $f_i$'s vanishes at $x$. So $x\in Y$. We conclude that $\bar{X}\subseteq Y$.

    Now as $\mathbb{C}[X_0,\ldots,X_n]$ is noetherian, we may take a finite subcollection $\mathcal{P}'$ of $\mathcal{P}$ such that 
    \[
        X=\bigcap_{f\in \mathcal{P}'}W(f). 
    \]
\end{proof}

\cite{stacks-project}

\printbibliography
\end{document}