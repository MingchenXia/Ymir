
\documentclass{amsbook} 



%\usepackage{xr}
\usepackage{xr-hyper}
\usepackage[unicode]{hyperref}


\usepackage[T1]{fontenc}
\usepackage[utf8]{inputenc}
\usepackage{lmodern}
\usepackage{amssymb,tikz-cd}
%\usepackage{natbib}
\usepackage[english]{babel}
\usepackage{nameref}

\usepackage[nameinlink,capitalize]{cleveref}
\usepackage[style=alphabetic,maxnames=99,maxalphanames=5, isbn=false, giveninits=true, doi=false]{biblatex}
\usepackage{lipsum, physics}
\usepackage{ifthen}
\usepackage{microtype}
\usepackage{booktabs}
\usetikzlibrary{calc}
\usepackage{emptypage}
\usepackage{setspace}
\usepackage[margin=0.75cm, font={small,stretch=0.80}]{caption}
\usepackage{subcaption}
\usepackage{url}
\usepackage{bookmark}
\usepackage{graphicx}
\usepackage{dsfont}
\usepackage{enumitem}
\usepackage{mathtools}
\usepackage{csquotes}
\usepackage{silence}
\usepackage{mathrsfs}
\usepackage{bigints}

\WarningFilter{biblatex}{Patching footnotes failed}


\ProcessOptions\relax

\emergencystretch=1em

\hypersetup{
colorlinks=true,
linktoc=all
}

\setcounter{tocdepth}{1}


\hyphenation{archi-medean  Archi-medean Tru-ding-er}

%\captionsetup[table]{position=bottom}   %% or below
\renewcommand{\thefootnote}{\fnsymbol{footnote}}
%\DeclareMathAlphabet{\mathcal}{OMS}{cmsy}{m}{n}
\renewbibmacro{in:}{}

\DeclareFieldFormat[article]{citetitle}{#1}
\DeclareFieldFormat[article]{title}{#1}
\DeclareFieldFormat[inbook]{citetitle}{#1}
\DeclareFieldFormat[inbook]{title}{#1}
\DeclareFieldFormat[incollection]{citetitle}{#1}
\DeclareFieldFormat[incollection]{title}{#1}
\DeclareFieldFormat[inproceedings]{citetitle}{#1}
\DeclareFieldFormat[inproceedings]{title}{#1}
\DeclareFieldFormat[phdthesis]{citetitle}{#1}
\DeclareFieldFormat[phdthesis]{title}{#1}
\DeclareFieldFormat[misc]{citetitle}{#1}
\DeclareFieldFormat[misc]{title}{#1}
\DeclareFieldFormat[book]{citetitle}{#1}
\DeclareFieldFormat[book]{title}{#1} 


%% Define various environments.

\theoremstyle{definition}
\newtheorem{theorem}{Theorem}[section]
\newtheorem{thm}[theorem]{Theorem}
\newtheorem{proposition}[theorem]{Proposition}
\newtheorem{corollary}[theorem]{Corollary}
\newtheorem{lemma}[theorem]{Lemma}
\newtheorem{conjecture}[theorem]{Conjecture}
\newtheorem{question}[theorem]{Question}
\newtheorem{example}[theorem]{Example}
\newtheorem{definition}[theorem]{Definition}
\newtheorem{condition}[theorem]{Condition}

\theoremstyle{remark}
\newtheorem{remark}[theorem]{Remark}
\numberwithin{equation}{section}

%\renewcommand{\thesection}{\thechapter.\arabic{section}}
%\renewcommand{\thetheorem}{\thesection.\arabic{theorem}}
%\renewcommand{\thedefinition}{\thesection.\arabic{definition}}
%\renewcommand{\theremark}{\thesection.\arabic{remark}}


%% Define new operators

\DeclareMathOperator{\rad}{rad}
\DeclareMathOperator{\nd}{nd}
\DeclareMathOperator{\ord}{ord}
\DeclareMathOperator{\Hom}{Hom}
\DeclareMathOperator{\PreSh}{PreSh}
\DeclareMathOperator{\Gr}{Gr}
\DeclareMathOperator{\Homint}{\mathcal{H}\mathrm{om}}
\DeclareMathOperator{\Torint}{\mathcal{T}\mathrm{or}}
\DeclareMathOperator{\Div}{div}
\DeclareMathOperator{\DSP}{DSP}
\DeclareMathOperator{\Diff}{Diff}
\DeclareMathOperator{\MA}{MA}
\DeclareMathOperator{\NA}{NA}
\DeclareMathOperator{\AN}{an}
\DeclareMathOperator{\Rep}{Rep}
\DeclareMathOperator{\Rest}{Res}
\DeclareMathOperator{\DF}{DF}
\DeclareMathOperator{\VCart}{VCart}
\DeclareMathOperator{\PL}{PL}
\DeclareMathOperator{\Bl}{Bl}
\DeclareMathOperator{\Td}{Td}
\DeclareMathOperator{\Fitt}{Fitt}
\DeclareMathOperator{\Ric}{Ric}
\DeclareMathOperator{\coeff}{coeff}
\DeclareMathOperator{\Aut}{Aut}
\DeclareMathOperator{\Capa}{Cap}
\DeclareMathOperator{\loc}{loc}
\DeclareMathOperator{\vol}{vol}
\DeclareMathOperator{\Val}{Val}
\DeclareMathOperator{\ST}{ST}
\DeclareMathOperator{\het}{ht}
\DeclareMathOperator{\Amp}{Amp}
\DeclareMathOperator{\Herm}{Herm}
\DeclareMathOperator{\trop}{trop}
\DeclareMathOperator{\Trop}{Trop}
\DeclareMathOperator{\Cano}{Can}
\DeclareMathOperator{\PS}{PS}
\DeclareMathOperator{\codim}{codim}
\DeclareMathOperator{\Var}{Var}
\DeclareMathOperator{\Psef}{Psef}
\DeclareMathOperator{\Jac}{Jac}
\DeclareMathOperator{\Char}{char}
\DeclareMathOperator{\Red}{red}
\DeclareMathOperator{\Spf}{Spf}
\DeclareMathOperator{\Span}{Span}
\DeclareMathOperator{\Der}{Der}
%\DeclareMathOperator{\Mod}{mod}
\DeclareMathOperator{\Hilb}{Hilb}
\DeclareMathOperator{\triv}{triv}
\DeclareMathOperator{\Frac}{Frac}
\DeclareMathOperator{\diam}{diam}
\DeclareMathOperator{\Spec}{Spec}
\DeclareMathOperator{\Spm}{Spm}
\DeclareMathOperator{\Specrel}{\underline{Sp}}
\DeclareMathOperator{\Sp}{Sp}
\DeclareMathOperator{\reg}{reg}
\DeclareMathOperator{\sing}{sing}
\DeclareMathOperator{\Star}{Star}
\DeclareMathOperator{\relint}{relint}
\DeclareMathOperator{\Cvx}{Cvx}
\DeclareMathOperator{\Int}{Int}
\DeclareMathOperator{\dep}{dep}
\DeclareMathOperator{\pd}{pd}
\DeclareMathOperator{\codep}{codep}
\DeclareMathOperator{\Supp}{Supp}
\DeclareMathOperator{\FS}{FS}
\DeclareMathOperator{\RZ}{RZ}
\DeclareMathOperator{\Ext}{Ext}
\DeclareMathOperator{\Redu}{red}
\DeclareMathOperator{\lct}{lct}
\DeclareMathOperator{\Proj}{Proj}
\DeclareMathOperator{\Sing}{Sing}
\DeclareMathOperator{\Conv}{Conv}
\DeclareMathOperator{\Max}{Max}
\DeclareMathOperator{\Tor}{Tor}
\DeclareMathOperator{\Gal}{Gal}
\DeclareMathOperator{\Frob}{Frob}
\DeclareMathOperator{\coker}{coker}
\DeclareMathOperator{\Sym}{Sym}
\DeclareMathOperator{\CSp}{CSp}
\DeclareMathOperator{\Cov}{Cov}
\DeclareMathOperator{\Img}{Im}


\newcommand{\alg}{\mathrm{alg}}
\newcommand{\Sh}{\mathrm{Sh}}
\newcommand{\fin}{\mathrm{fin}}
\newcommand{\BPF}{\mathrm{BPF}}
\newcommand{\dBPF}{\mathrm{dBPF}}
\newcommand{\divf}{\mathrm{Div}^f}
\newcommand{\nef}{\mathrm{nef}}
\newcommand{\Bir}{\mathrm{Bir}}
\newcommand{\hO}{\hat{\mathcal{O}}}
\newcommand{\bDiv}{\mathrm{Div}^{\mathrm{b}}}
\newcommand{\un}{\mathrm{un}}
\newcommand{\sep}{\mathrm{sep}}
\newcommand{\diag}{\mathrm{diag}}
\newcommand{\Pic}{\mathrm{Pic}}
\newcommand{\GL}{\mathrm{GL}}
\newcommand{\SL}{\mathrm{SL}}
\newcommand{\LS}{\mathrm{LS}}
\newcommand{\GLS}{\mathrm{GLS}}
\newcommand{\GLSi}{\mathrm{GLS}_{\cap}}
\newcommand{\PGLS}{\mathrm{PGLS}}
\newcommand{\Loc}[1][S]{_{\{{#1}\}}}
\newcommand{\cl}{\mathrm{cl}}
\newcommand{\otL}{\hat{\otimes}^{\mathbb{L}}}
\newcommand{\ddpp}{\mathrm{d}'\mathrm{d}''}
\newcommand{\TC}{\mathcal{TC}}
\newcommand{\ddPP}{\mathrm{d}'_{\mathrm{P}}\mathrm{d}''_{\mathrm{P}}}
\newcommand{\PSs}{\mathcal{PS}}
\newcommand{\Gm}{\mathbb{G}_{\mathrm{m}}}
\newcommand{\End}{\mathrm{End}}
\newcommand{\Aff}[1][X]{\mathcal{M}\left(\mathcal{#1}\right)}
\newcommand{\XG}[1][X]{{#1}_{\mathrm{G}}}
\newcommand{\convC}{\xrightarrow{C}}
\newcommand{\Vect}{\mathrm{Vect}}
\newcommand{\abso}[1]{\lvert#1\rvert}
\newcommand{\Mdl}{\mathrm{Model}}
\newcommand{\cn}{\stackrel{\sim}{\longrightarrow}}
\newcommand{\sbc}{\mathbf{s}}
\newcommand{\CH}{\mathrm{CH}}
\newcommand{\GR}{\mathrm{GR}}
\newcommand{\bir}{\mathrm{bir}}
\newcommand{\dc}{\mathrm{d}^{\mathrm{c}}}
\newcommand{\Nef}{\mathrm{Nef}}
\newcommand{\Adj}{\mathrm{Adj}}
\newcommand{\DHm}{\mathrm{DH}}
\newcommand{\An}{\mathrm{an}}
\newcommand{\Rec}{\mathrm{Rec}}
\newcommand{\dP}{\mathrm{d}_{\mathrm{P}}}
\newcommand{\ddp}{\mathrm{d}_{\mathrm{P}}'\mathrm{d}_{\mathrm{P}}''}
\newcommand{\ddc}{\mathrm{dd}^{\mathrm{c}}}
\newcommand{\ddL}{\mathrm{d}'\mathrm{d}''}
\newcommand{\PSH}{\mathrm{PSH}}
\newcommand{\CPSH}{\mathrm{CPSH}}
\newcommand{\PSP}{\mathrm{PSP}}
\newcommand{\WPSH}{\mathrm{WPSH}}
\newcommand{\Ent}{\mathrm{Ent}}
\newcommand{\NS}{\mathrm{NS}}
\newcommand{\QPSH}{\mathrm{QPSH}}
\newcommand{\proet}{\mathrm{pro-ét}}
\newcommand{\XL}{(\mathcal{X},\mathcal{L})}
\newcommand{\ii}{\mathrm{i}}
\newcommand{\Ann}{\mathrm{Ann}}
\newcommand{\ExtFun}{\mathcal{E}\mathrm{xt}}
\newcommand{\Cpt}{\mathrm{Cpt}}
\newcommand{\bp}{\bar{\partial}}
\newcommand{\ddt}{\frac{\mathrm{d}}{\mathrm{d}t}}
\newcommand{\dds}{\frac{\mathrm{d}}{\mathrm{d}s}}
\newcommand{\Ep}{\mathcal{E}^p(X,\theta;[\phi])}
\newcommand{\Ei}{\mathcal{E}^{\infty}(X,\theta;[\phi])}
\newcommand{\infs}{\operatorname*{inf\vphantom{p}}}
\newcommand{\sups}{\operatorname*{sup*}}
\newcommand{\colim}{\operatorname*{colim}}
\newcommand{\ddtz}[1][0]{\left.\ddt\right|_{t={#1}}}
\newcommand{\tube}[1][Y]{]{#1}[}
\newcommand{\ddsz}[1][0]{\left.\ddt\right|_{s={#1}}}
\newcommand{\floor}[1]{\left \lfloor{#1}\right \rfloor }
\newcommand{\dec}[1]{\left \{{#1}\right \} }
\newcommand{\ceil}[1]{\left \lceil{#1}\right \rceil }
\newcommand{\Projrel}{\mathcal{P}\mathrm{roj}}
\newcommand{\Weil}{\mathrm{Weil}}
\newcommand{\Cart}{\mathrm{Cart}}
\newcommand{\bWeil}{\mathrm{b}\mathrm{Weil}}
\newcommand{\bCart}{\mathrm{b}\mathrm{Cart}}
\newcommand{\Cond}{\mathrm{Cond}}
\newcommand{\IC}{\mathrm{IC}}
\newcommand{\IH}{\mathrm{IH}}
\newcommand{\Eq}{\mathrm{Eq}}
\newcommand{\cris}{\mathrm{cris}}
\newcommand{\Zar}{\mathrm{Zar}}
\newcommand{\HvbCat}{\overline{\mathcal{V}\mathrm{ect}}}
\newcommand{\BanModCat}{\mathcal{B}\mathrm{an}\mathcal{M}\mathrm{od}}
\newcommand{\DesCat}{\mathcal{D}\mathrm{es}}
\newcommand{\RingCat}{\mathcal{R}\mathrm{ing}}
\newcommand{\SchCat}{\mathcal{S}\mathrm{ch}}
\newcommand{\AbCat}{\mathcal{A}\mathrm{b}}
\newcommand{\RSCat}{\mathcal{R}\mathrm{S}}
\newcommand{\LRSCat}{\mathcal{L}\mathrm{RS}}
\newcommand{\CLRSCat}{\mathbb{C}\text{-}\LRSCat}
\newcommand{\CRSCat}{\mathbb{C}\text{-}\RSCat}
\newcommand{\CLA}{\mathbb{C}\text{-}\mathcal{L}\mathrm{A}}
\newcommand{\CASCat}{\mathbb{C}\text{-}\mathcal{A}\mathrm{n}}
\newcommand{\LiuCat}{\mathcal{L}\mathrm{iu}}
\newcommand{\BanCat}{\mathcal{B}\mathrm{an}}
\newcommand{\BanAlgCat}{\mathcal{B}\mathrm{an}\mathcal{A}\mathrm{lg}}
\newcommand{\AnaCat}{\mathcal{A}\mathrm{n}}
\newcommand{\LiuAlgCat}{\mathcal{L}\mathrm{iu}\mathcal{A}\mathrm{lg}}
\newcommand{\AlgCat}{\mathcal{A}\mathrm{lg}}
\newcommand{\SetCat}{\mathcal{S}\mathrm{et}}
\newcommand{\ModCat}{\mathcal{M}\mathrm{od}}
\newcommand{\GerCat}{\mathcal{G}\mathrm{er}}
\newcommand{\AnaGerCat}{\mathbb{C}\text{-}\GerCat}
\newcommand{\TopCat}{\mathcal{T}\mathrm{op}}
\newcommand{\CohCat}{\mathcal{C}\mathrm{oh}}
\newcommand{\SolCat}{\mathcal{S}\mathrm{olid}}
\newcommand{\AffCat}{\mathcal{A}\mathrm{ff}}
\newcommand{\AffAlgCat}{\mathcal{A}\mathrm{ff}\mathcal{A}\mathrm{lg}}
\newcommand{\QcohLiuAlgCat}{\mathcal{L}\mathrm{iu}\mathcal{A}\mathrm{lg}^{\mathrm{QCoh}}}
\newcommand{\LiuMorCat}{\mathcal{L}\mathrm{iu}}
\newcommand{\Isom}{\mathcal{I}\mathrm{som}}
\newcommand{\Cris}{\mathcal{C}\mathrm{ris}}
\newcommand{\Pro}{\mathrm{Pro}-}
\newcommand{\Fin}{\mathcal{F}\mathrm{in}}
\newcommand{\norms}[1]{\left\|#1\right\|}
\newcommand{\HPDDiff}{\mathbf{D}\mathrm{iff}}
\newcommand{\Menn}[2]{\begin{bmatrix}#1\\#2\end{bmatrix}}
\newcommand{\Fins}{\widehat{\Vect}^F}
\newcommand\blfootnote[1]{%
  \begingroup
  \renewcommand\thefootnote{}\footnote{#1}%
  \addtocounter{footnote}{-1}%
  \endgroup
}


\makeatletter
\newcommand*{\addFileDependency}[1]{% argument=file name and extension
  \typeout{(#1)}
  \@addtofilelist{#1}
  \IfFileExists{#1}{}{\typeout{No file #1.}}
}
\makeatother



\newcommand*{\myexternaldocument}[2]{%
\externaldocument[#1]{#2}%
\addFileDependency{#2.tex}%
\addFileDependency{#2.aux}%
%\addFileDependency{#2.pdf}%
}


%\iffalse

\myexternaldocument{Introduction-}{Introduction}
\myexternaldocument{Topology-}{Topology-Bornology}
\myexternaldocument{Banach-}{Banach-Rings}
\myexternaldocument{Commutative-}{Commutative-Algebra}



\myexternaldocument{Local-}{Local-Algebras}
\myexternaldocument{Complex-}{Complex-Analytic-Spaces}
\myexternaldocument{ConstructionComplex-}{Constructions-Complex-Spaces}
\myexternaldocument{PropertyComplex-}{Properties-Complex-Spaces}
\myexternaldocument{GPropertyComplex-}{Global-Properties-Complex-Spaces}
\myexternaldocument{Analytic-}{Analytic-Sets}
\myexternaldocument{Morphisms-}{Morphisms-Complex-Spaces}

\myexternaldocument{Affinoid-}{Affinoid-Algebras}
\myexternaldocument{Berkovich-}{Berkovich-Analytic-Spaces}
\myexternaldocument{BerkProperty-}{Properties-Berkovich-Spaces}
%\fi


\bibliography{Ymir}

\endinput
\title{Commutative algebra}
\begin{document}
\maketitle
\tableofcontents



\section{Introduction}\label{sec-introduction}

\section{Graded commutative algebra}
Let $G$ be an Abelian group. We write the group operation of $G$ multiplicatively and denote the identity of $G$ as $1$.

\begin{definition}
    Let $A$ be an Abelian group. A \emph{$G$-grading} on $A$ is a coproduct decomposition
    \[
        A=\bigoplus_{g\in G}A_g  
    \]
    of Abelian groups such that $A_g\subseteq A$. An Abelian group with a $G$-grading is called a \emph{$G$-graded Abelian group}.

    A \emph{$G$-graded homomorphism} between $G$-graded Abelian groups $A$ and $B$ is a homogeneous of the underlying Abelian groups $f:A\rightarrow B$ such that $f(A_g)\subseteq B_g$ for any $g\in G$.

    The category of $G$-graded Abelian groups is denoted by $\AbCat^G$.
\end{definition}
A usual Abelian group $A$ can be given the \emph{trivial $G$-grading}: $A_0=A$ and $A_g=0$ for $g\in G$, $g\neq 0$. In this way, we find a fully faithful embedding 
\[
    \AbCat\rightarrow \AbCat^G.  
\]
When we regard an Abelian group as a $G$-graded Abelian group and there are no natural gradings, we always understand that we are taking the trivial $G$-grading.


\begin{definition}
    A \emph{$G$-graded ring} is a commutative ring $A$ endowed with a $G$-grading:
    \[
        A=\bigoplus_{g\in G} A_g  
    \]
    as Abelian groups and such that 
    \begin{enumerate}
        \item $A_{g}A_{h}\subseteq A_{gh}$ for any $g,h\in G$;
        \item $1\in A_1$.
    \end{enumerate}

    An element $a\in A$ is said to be \emph{homogeneous} if there is $g\in G$ such that $a\in A_g$. If $a$ is furthermore non-zero, we write $g=\rho(a)$. We set $\rho(0)=0$.

    A \emph{$G$-homomorphism} of $G$-graded rings $A$ and $B$ is a ring homomorphism $f:A\rightarrow B$ such that $f(A_g)\subseteq B_g$ for each $g\in G$.

    The category of $G$-graded rings is denoted by $\RingCat^G$.
\end{definition}

\begin{example}
    Let $A$ be a $G$-graded ring, $n\in \mathbb{N}$ and $g=(g_1,\ldots,g_n)\in G^n$. Then there is a unique $G$-grading on $A[T_1,\ldots,T_n]$ extending the grading on $A$ and such that $\rho(T_i)=g_i$ for $i=1,\ldots,n$. We will denote $A[T_1,\ldots,T_n]$ with this grading as $A[g_{1}^{-1}T_1,\ldots,g_{n}^{-1}T_n]$ or simply $A[g^{-1}T]$.
\end{example}

\begin{example}\label{ex-localizationgradedring}
    Let $A$ be a $G$-graded ring and $S$ be a multiplicative subset of $A$ consisting of homogeneous elements, then $S^{-1}A$ has a natural $G$-grading. To see this, recall the construction of $S^{-1}A$ in \cite[\href{https://stacks.math.columbia.edu/tag/00CM}{Tag 00CM}]{stacks-project}. One defines an equivalence relation on $A\times S$: $(x,s)\sim (y,t)$ if there is $u\in S$ such that $(xt-ys)u=0$. For each $g\in G$, we define $(S^{-1}A)_g$ as the image of $(x,s)$ for all $s\in S$ and $x\in A_{g\rho(s)}$. It is easy to verify that this is a well-defined $G$-grading on $S^{-1}A$.  \textcolor{red}{Add details.}
\end{example}

\begin{definition}
    Let $A$ be a $G$-graded ring. A \emph{$G$-homogeneous ideal} in $A$ is an ideal $I$ in $G$ such that if $a\in A$ can be written as 
    \[
        a=\sum_{g\in G} a_g,\quad a_g\in A_g  
    \]
    with almost all $a_g=0$, then $a_g\in I$.
\end{definition}
\begin{lemma}
    Let $f:A\rightarrow B$ be a $G$-homomorphism of $G$-graded rings. Then $\ker f$ is a $G$-homogeneous ideal in $A$.
\end{lemma}
\begin{proof}
    We need to show that
    \[
        \ker f=\sum_{g\in G} (\ker f)\cap A_g.  
    \]
    Take $x\in \ker f$, we can write $x$ as 
    \[
        \sum_{g\in G} a_g,\quad a_g\in A_g
    \]
    and almost all $a_g$'s are $0$. Then 
    \[
        f(x)=  \sum_{g\in G} f(a_g),\quad f(a_g)\in B_g.
    \]
    It follows that $f(a_g)=0$ for each $g\in G$ and hence $a_g\in (\ker f)\cap A_g$.
\end{proof}

\begin{definition}\label{def-gradquotient}
    Let $A$ be a $G$-graded ring and $I$ be a $G$-homogeneous ideal in $A$. Then we define a $G$-grading on $A/I$ as follows: for any $g\in G$
    \[
        (A/I)_g:= (A_g+I)/I.
    \]
\end{definition}
\begin{proposition}
    Let $A$ be a $G$-graded ring and $I$ be a $G$-homogeneous ideal in $A$. Then the construction in \cref{def-gradquotient} defines a grading on $A/I$. The natural map $\pi:A\rightarrow A/I$ is a $G$-homomorphism.

    For any $G$-graded ring $B$ and any $G$-homomorphism $f:A\rightarrow B$ such that $I\subseteq \ker A$, there is a unique $G$-homomorphism $f':A/I\rightarrow B$ such that $f'\circ \pi=f$.
\end{proposition} 
\begin{proof}
    We first argue that for different $g,h\in G$, $(A/I)_g\cap (A/I)_h=0$. Suppose $x\in (A/I)_g\cap (A/I)_h$, we can lift $x$ to both $y_g+i_g\in A$ and $y_h+i_h\in A$ with $y_g,y_h\in A$ and $i_g,i_h\in I$. It follows that $y_g-y_h\in I$. But $I$ is a $G$-homogeneous ideal, so it follows that $y_g,y_h\in I$ and hence $x=0$.

    Next we argue that
    \[
        A/I=\sum_{g\in G}  (A/I)_g.
    \]
    Lift an element $x\in A/I$ by $a\in A$, we represent $a$ as
    \[
        a=\sum_{g\in G} a_g,\quad a_g\in A_g 
    \]
    with almost all $a_g$'s equal to $0$. Then $x$ can be represented as
    \[
        x=\sum_{g\in G}\pi(a_g).  
    \]
    We have shown that the construction in \cref{def-gradquotient} gives a $G$-grading on $A$. It is clear from the definition that $\pi$ is a $G$-homomorphism.

    Next assume that $B$ and $f$ are given as in the proposition. Then there is a ring homomorphism $f':A/I\rightarrow B$ such that $f=f'\circ \pi$. We need to argue that $f'$ is a $G$-homomorphism. For this purpose, take $g\in G$, $x\in (A/I)_g$, we need to show that $f'(x)\in B_g$. Lift $x$ to $y+i$ with $y\in A_g$ and $i\in I$, then we know that $f'(x)=\pi(y+i)=\pi(y)\in B_g$.    
\end{proof}

\begin{definition}
    Let $A$ be a $G$-graded ring and $M$ an $A$-module which is also a $G$-graded Abelian group. We say $M$ is a \emph{$G$-graded module} if for each $g,h\in G$, we have
    \[
        A_gM_h\subseteq M_{gh}.  
    \]
    A \emph{$G$-graded homomorphism} of $G$-graded $A$-modules $M$ and $N$ is an $A$-module homomorphism $f:M\rightarrow N$ which is at the same time a homomorphism of  the underlying $G$-graded Abelian groups.

    The category of $G$-graded $A$-modules is denoted by $\ModCat^G_A$.
\end{definition}
Observe that $G$-homogeneous ideals of $A$ are $G$-graded submodules of $A$. Also observe that $\ModCat^G_{\mathbb{Z}}$ is isomorphic to $\AbCat^G$.

\begin{proposition}
    Let $A$ be a $G$-graded ring. Then $\ModCat^G_A$ is an Abelian category.
\end{proposition}
\begin{proof}
    We first show that $\ModCat^G_A$ is preadditive. Given $M,N\in \ModCat^G_A$, we can regard $\Hom_{\ModCat^G_A}(M,N)$ as a subgroup of $\Hom_A(M,N)$. It is easy to see that this gives $\ModCat^G_A$ an enrichment over $\AbCat$.

    Next we show that $\ModCat^G_A$ is additive. The zero object is clearly given by $0$ with the trivial grading. Given $M,N\in \ModCat^G_A$, we define
    \[
        (M\oplus N)_g:=M_g\oplus N_g,\quad g\in G.
    \]
    This construction makes $M\oplus N$ a $G$-graded $A$-module. It is easy to verify that $M\oplus N$ is the biproduct of $M$ and $N$.

    Next we show that $\ModCat^G_A$ is pre-Abelian. Given an arrow $f:M\rightarrow N$ in  $\ModCat^G_A$, we need to define its kernel and cokernel. We define
    \[
        (\ker f)_g:=(\ker f)\cap M_g  
    \]
    and $(\coker f)_g$ as the image of $N_g$ for any $g\in G$. It is straightforward to verify that these are kernels and cokernels.

    Finally, given a monomorphism $f:M\rightarrow N$, it is obvious that the map $f$ is injective and $f$ can be identified with the kenrel of the natural map $N/\Img f$. A dual argument shows that an epimorphism is the cokernel of some morphism as well.
\end{proof}

\begin{example}\label{ex-localizationgradedmodule}
    This is a continuition of \cref{ex-localizationgradedring}.
    Let $A$ be a $G$-graded ring and $S$ be a multiplicative subset of $A$ consisting of homogeneous elements. Consider a $G$-graded $A$-module $M$. We define a $G$-grading on $S^{-1}M$. Recall that $S^{-1}M$ can be realized as follows: one defines an equivalence relation on $M\times S$: $(x,s)\sim (y,t)$ if there is $u\in S$ such that $(xt-ys)u=0$. For each $g\in G$, we define $(S^{-1}M)_g$ as the image of $(x,s)$ for all $s\in S$ and $x\in M_{g\rho(s)}$. It is easy to verify that this is a well-defined $G$-grading on $S^{-1}M$ and $S^{-1}M$ is a $G$-graded $S^{-1}A$-module.  \textcolor{red}{Add details.}
\end{example}

\begin{example}\label{ex-twistmodule}
    Let $A$ be a $G$-graded ring and $g\in G$. We define $g^{-1}A$ as the $G$-graded $A$-module:
    \[
        (g^{-1}A)_h=A_{g^{-1}h}
    \]
    for any $h\in G$. Observe that $1\in (g^{-1}A)_g$.
\end{example}


\begin{definition}
    Let $f:A\rightarrow B$ be a $G$-graded homomorphism of $G$-graded rings. We say $f$ is \emph{finite} (resp. \emph{finitely generated}, resp. \emph{integral}) if it is finite (resp. finitely generated, resp. integral) as a usual ring map.
\end{definition}

\begin{proposition}
    Let $f:A\rightarrow B$ be a $G$-graded homomorphism of $G$-graded rings. Then 
    \begin{enumerate}
        \item $f$ is finite if and only if there are $n\in \mathbb{N}$, $g_1,\ldots,g_n\in G$ and a surjective $G$-graded homomorphism 
        \[
            \bigoplus_{i=1}^n(g_i^{-1}A)^n\rightarrow B
        \] 
        of graded $A$-modules.
        \item $f$ is finitely generated if and only if there are $n\in \mathbb{N}$,  $g_1,\ldots,g_n\in G$ and a surjective $G$-graded $A$-algebra homomorphism
        \[
            A[g_1^{-1}T_1,\ldots,g_{n}^{-1}T_n]\rightarrow B.  
        \] 
        \item $f$ is integral if and only if for any non-zero homogeneous element $b\in B$, there in $n\in \mathbb{N}$ and homogeneous elements $a_1,\ldots,a_n\in A$ such that
        \[
            b^n+f(a_1)b^{n-1}+\cdots+f(a_n)=0.  
        \]
    \end{enumerate}
\end{proposition}
\begin{proof}
    (1) The non-trivial direction is the direct implication. Assume that $f$ is finite. Take $b_1,\ldots,b_n\in B$ so that $\sum_{i=1}^nf(A)b_i=B$. Up to replacing the collection $\{b_i\}_i$ by the finite collection of non-zero homogeneous components of the $b_i$'s, we may assume that each $b_i$ is homogeneous. We define $g_i=\rho(b_i)$ and the map $\bigoplus_{i=1}^n(g_i^{-1}A)^n\rightarrow B$ sends $1$ at the $i$-th place to $b_i$. 

    (2) The non-trivial direction is the direct implication. Suppose $f$ is finitely generated, say by $b_1,\ldots,b_n$. Up to replacing the collection $\{b_i\}_i$ by the finite collection of non-zero homogeneous components of the $b_i$'s, we may assume that each $b_i$ is homogeneous. Then we define $g_i=\rho(b_i)$ for $i=1,\ldots,n$ and the $A$-algebra homomorphism $A[g_1^{-1}T_1,\ldots,g_{n}^{-1}T_n]\rightarrow B$ sends $T_i$ to $b_i$ for $i=1,\ldots,n$.

    (3) Assume that $f$ is integral, then for any non-zero homogeneous element $b\in B$, we can find $a_1,\ldots,a_n\in A$ such that 
    \[
        b^n+f(a_1)b^{n-1}+\cdots+f(a_n)=0.  
    \]
    Obviously, we can replace $a_i$ by its component in $\rho(b)^i$ for $i=1,\ldots,n$ and the equation remains true.

    The reverse direction follows from \cite[\href{https://stacks.math.columbia.edu/tag/00GO}{Tag 00GO}]{stacks-project}.
\end{proof}


\printbibliography
\end{document}