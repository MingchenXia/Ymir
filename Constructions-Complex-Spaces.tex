
\documentclass{amsbook} 



%\usepackage{xr}
\usepackage{xr-hyper}
\usepackage[unicode]{hyperref}


\usepackage[T1]{fontenc}
\usepackage[utf8]{inputenc}
\usepackage{lmodern}
\usepackage{amssymb,tikz-cd}
%\usepackage{natbib}
\usepackage[english]{babel}
\usepackage{nameref}

\usepackage[nameinlink,capitalize]{cleveref}
\usepackage[style=alphabetic,maxnames=99,maxalphanames=5, isbn=false, giveninits=true, doi=false]{biblatex}
\usepackage{lipsum, physics}
\usepackage{ifthen}
\usepackage{microtype}
\usepackage{booktabs}
\usetikzlibrary{calc}
\usepackage{emptypage}
\usepackage{setspace}
\usepackage[margin=0.75cm, font={small,stretch=0.80}]{caption}
\usepackage{subcaption}
\usepackage{url}
\usepackage{bookmark}
\usepackage{graphicx}
\usepackage{dsfont}
\usepackage{enumitem}
\usepackage{mathtools}
\usepackage{csquotes}
\usepackage{silence}
\usepackage{mathrsfs}
\usepackage{bigints}

\WarningFilter{biblatex}{Patching footnotes failed}


\ProcessOptions\relax

\emergencystretch=1em

\hypersetup{
colorlinks=true,
linktoc=all
}

\setcounter{tocdepth}{1}


\hyphenation{archi-medean  Archi-medean Tru-ding-er}

%\captionsetup[table]{position=bottom}   %% or below
\renewcommand{\thefootnote}{\fnsymbol{footnote}}
%\DeclareMathAlphabet{\mathcal}{OMS}{cmsy}{m}{n}
\renewbibmacro{in:}{}

\DeclareFieldFormat[article]{citetitle}{#1}
\DeclareFieldFormat[article]{title}{#1}
\DeclareFieldFormat[inbook]{citetitle}{#1}
\DeclareFieldFormat[inbook]{title}{#1}
\DeclareFieldFormat[incollection]{citetitle}{#1}
\DeclareFieldFormat[incollection]{title}{#1}
\DeclareFieldFormat[inproceedings]{citetitle}{#1}
\DeclareFieldFormat[inproceedings]{title}{#1}
\DeclareFieldFormat[phdthesis]{citetitle}{#1}
\DeclareFieldFormat[phdthesis]{title}{#1}
\DeclareFieldFormat[misc]{citetitle}{#1}
\DeclareFieldFormat[misc]{title}{#1}
\DeclareFieldFormat[book]{citetitle}{#1}
\DeclareFieldFormat[book]{title}{#1} 


%% Define various environments.

\theoremstyle{definition}
\newtheorem{theorem}{Theorem}[section]
\newtheorem{thm}[theorem]{Theorem}
\newtheorem{proposition}[theorem]{Proposition}
\newtheorem{corollary}[theorem]{Corollary}
\newtheorem{lemma}[theorem]{Lemma}
\newtheorem{conjecture}[theorem]{Conjecture}
\newtheorem{question}[theorem]{Question}
\newtheorem{example}[theorem]{Example}
\newtheorem{definition}[theorem]{Definition}
\newtheorem{condition}[theorem]{Condition}

\theoremstyle{remark}
\newtheorem{remark}[theorem]{Remark}
\numberwithin{equation}{section}

%\renewcommand{\thesection}{\thechapter.\arabic{section}}
%\renewcommand{\thetheorem}{\thesection.\arabic{theorem}}
%\renewcommand{\thedefinition}{\thesection.\arabic{definition}}
%\renewcommand{\theremark}{\thesection.\arabic{remark}}


%% Define new operators

\DeclareMathOperator{\rad}{rad}
\DeclareMathOperator{\nd}{nd}
\DeclareMathOperator{\ord}{ord}
\DeclareMathOperator{\Hom}{Hom}
\DeclareMathOperator{\PreSh}{PreSh}
\DeclareMathOperator{\Gr}{Gr}
\DeclareMathOperator{\Homint}{\mathcal{H}\mathrm{om}}
\DeclareMathOperator{\Torint}{\mathcal{T}\mathrm{or}}
\DeclareMathOperator{\Div}{div}
\DeclareMathOperator{\DSP}{DSP}
\DeclareMathOperator{\Diff}{Diff}
\DeclareMathOperator{\MA}{MA}
\DeclareMathOperator{\NA}{NA}
\DeclareMathOperator{\AN}{an}
\DeclareMathOperator{\Rep}{Rep}
\DeclareMathOperator{\Rest}{Res}
\DeclareMathOperator{\DF}{DF}
\DeclareMathOperator{\VCart}{VCart}
\DeclareMathOperator{\PL}{PL}
\DeclareMathOperator{\Bl}{Bl}
\DeclareMathOperator{\Td}{Td}
\DeclareMathOperator{\Fitt}{Fitt}
\DeclareMathOperator{\Ric}{Ric}
\DeclareMathOperator{\coeff}{coeff}
\DeclareMathOperator{\Aut}{Aut}
\DeclareMathOperator{\Capa}{Cap}
\DeclareMathOperator{\loc}{loc}
\DeclareMathOperator{\vol}{vol}
\DeclareMathOperator{\Val}{Val}
\DeclareMathOperator{\ST}{ST}
\DeclareMathOperator{\het}{ht}
\DeclareMathOperator{\Amp}{Amp}
\DeclareMathOperator{\Herm}{Herm}
\DeclareMathOperator{\trop}{trop}
\DeclareMathOperator{\Trop}{Trop}
\DeclareMathOperator{\Cano}{Can}
\DeclareMathOperator{\PS}{PS}
\DeclareMathOperator{\codim}{codim}
\DeclareMathOperator{\Var}{Var}
\DeclareMathOperator{\Psef}{Psef}
\DeclareMathOperator{\Jac}{Jac}
\DeclareMathOperator{\Char}{char}
\DeclareMathOperator{\Red}{red}
\DeclareMathOperator{\Spf}{Spf}
\DeclareMathOperator{\Span}{Span}
\DeclareMathOperator{\Der}{Der}
%\DeclareMathOperator{\Mod}{mod}
\DeclareMathOperator{\Hilb}{Hilb}
\DeclareMathOperator{\triv}{triv}
\DeclareMathOperator{\Frac}{Frac}
\DeclareMathOperator{\diam}{diam}
\DeclareMathOperator{\Spec}{Spec}
\DeclareMathOperator{\Spm}{Spm}
\DeclareMathOperator{\Specrel}{\underline{Sp}}
\DeclareMathOperator{\Sp}{Sp}
\DeclareMathOperator{\reg}{reg}
\DeclareMathOperator{\sing}{sing}
\DeclareMathOperator{\Star}{Star}
\DeclareMathOperator{\relint}{relint}
\DeclareMathOperator{\Cvx}{Cvx}
\DeclareMathOperator{\Int}{Int}
\DeclareMathOperator{\dep}{dep}
\DeclareMathOperator{\pd}{pd}
\DeclareMathOperator{\codep}{codep}
\DeclareMathOperator{\Supp}{Supp}
\DeclareMathOperator{\FS}{FS}
\DeclareMathOperator{\RZ}{RZ}
\DeclareMathOperator{\Ext}{Ext}
\DeclareMathOperator{\Redu}{red}
\DeclareMathOperator{\lct}{lct}
\DeclareMathOperator{\Proj}{Proj}
\DeclareMathOperator{\Sing}{Sing}
\DeclareMathOperator{\Conv}{Conv}
\DeclareMathOperator{\Max}{Max}
\DeclareMathOperator{\Tor}{Tor}
\DeclareMathOperator{\Gal}{Gal}
\DeclareMathOperator{\Frob}{Frob}
\DeclareMathOperator{\coker}{coker}
\DeclareMathOperator{\Sym}{Sym}
\DeclareMathOperator{\CSp}{CSp}
\DeclareMathOperator{\Cov}{Cov}
\DeclareMathOperator{\Img}{Im}


\newcommand{\alg}{\mathrm{alg}}
\newcommand{\Sh}{\mathrm{Sh}}
\newcommand{\fin}{\mathrm{fin}}
\newcommand{\BPF}{\mathrm{BPF}}
\newcommand{\dBPF}{\mathrm{dBPF}}
\newcommand{\divf}{\mathrm{Div}^f}
\newcommand{\nef}{\mathrm{nef}}
\newcommand{\Bir}{\mathrm{Bir}}
\newcommand{\hO}{\hat{\mathcal{O}}}
\newcommand{\bDiv}{\mathrm{Div}^{\mathrm{b}}}
\newcommand{\un}{\mathrm{un}}
\newcommand{\sep}{\mathrm{sep}}
\newcommand{\diag}{\mathrm{diag}}
\newcommand{\Pic}{\mathrm{Pic}}
\newcommand{\GL}{\mathrm{GL}}
\newcommand{\SL}{\mathrm{SL}}
\newcommand{\LS}{\mathrm{LS}}
\newcommand{\GLS}{\mathrm{GLS}}
\newcommand{\GLSi}{\mathrm{GLS}_{\cap}}
\newcommand{\PGLS}{\mathrm{PGLS}}
\newcommand{\Loc}[1][S]{_{\{{#1}\}}}
\newcommand{\cl}{\mathrm{cl}}
\newcommand{\otL}{\hat{\otimes}^{\mathbb{L}}}
\newcommand{\ddpp}{\mathrm{d}'\mathrm{d}''}
\newcommand{\TC}{\mathcal{TC}}
\newcommand{\ddPP}{\mathrm{d}'_{\mathrm{P}}\mathrm{d}''_{\mathrm{P}}}
\newcommand{\PSs}{\mathcal{PS}}
\newcommand{\Gm}{\mathbb{G}_{\mathrm{m}}}
\newcommand{\End}{\mathrm{End}}
\newcommand{\Aff}[1][X]{\mathcal{M}\left(\mathcal{#1}\right)}
\newcommand{\XG}[1][X]{{#1}_{\mathrm{G}}}
\newcommand{\convC}{\xrightarrow{C}}
\newcommand{\Vect}{\mathrm{Vect}}
\newcommand{\abso}[1]{\lvert#1\rvert}
\newcommand{\Mdl}{\mathrm{Model}}
\newcommand{\cn}{\stackrel{\sim}{\longrightarrow}}
\newcommand{\sbc}{\mathbf{s}}
\newcommand{\CH}{\mathrm{CH}}
\newcommand{\GR}{\mathrm{GR}}
\newcommand{\bir}{\mathrm{bir}}
\newcommand{\dc}{\mathrm{d}^{\mathrm{c}}}
\newcommand{\Nef}{\mathrm{Nef}}
\newcommand{\Adj}{\mathrm{Adj}}
\newcommand{\DHm}{\mathrm{DH}}
\newcommand{\An}{\mathrm{an}}
\newcommand{\Rec}{\mathrm{Rec}}
\newcommand{\dP}{\mathrm{d}_{\mathrm{P}}}
\newcommand{\ddp}{\mathrm{d}_{\mathrm{P}}'\mathrm{d}_{\mathrm{P}}''}
\newcommand{\ddc}{\mathrm{dd}^{\mathrm{c}}}
\newcommand{\ddL}{\mathrm{d}'\mathrm{d}''}
\newcommand{\PSH}{\mathrm{PSH}}
\newcommand{\CPSH}{\mathrm{CPSH}}
\newcommand{\PSP}{\mathrm{PSP}}
\newcommand{\WPSH}{\mathrm{WPSH}}
\newcommand{\Ent}{\mathrm{Ent}}
\newcommand{\NS}{\mathrm{NS}}
\newcommand{\QPSH}{\mathrm{QPSH}}
\newcommand{\proet}{\mathrm{pro-ét}}
\newcommand{\XL}{(\mathcal{X},\mathcal{L})}
\newcommand{\ii}{\mathrm{i}}
\newcommand{\Ann}{\mathrm{Ann}}
\newcommand{\ExtFun}{\mathcal{E}\mathrm{xt}}
\newcommand{\Cpt}{\mathrm{Cpt}}
\newcommand{\bp}{\bar{\partial}}
\newcommand{\ddt}{\frac{\mathrm{d}}{\mathrm{d}t}}
\newcommand{\dds}{\frac{\mathrm{d}}{\mathrm{d}s}}
\newcommand{\Ep}{\mathcal{E}^p(X,\theta;[\phi])}
\newcommand{\Ei}{\mathcal{E}^{\infty}(X,\theta;[\phi])}
\newcommand{\infs}{\operatorname*{inf\vphantom{p}}}
\newcommand{\sups}{\operatorname*{sup*}}
\newcommand{\colim}{\operatorname*{colim}}
\newcommand{\ddtz}[1][0]{\left.\ddt\right|_{t={#1}}}
\newcommand{\tube}[1][Y]{]{#1}[}
\newcommand{\ddsz}[1][0]{\left.\ddt\right|_{s={#1}}}
\newcommand{\floor}[1]{\left \lfloor{#1}\right \rfloor }
\newcommand{\dec}[1]{\left \{{#1}\right \} }
\newcommand{\ceil}[1]{\left \lceil{#1}\right \rceil }
\newcommand{\Projrel}{\mathcal{P}\mathrm{roj}}
\newcommand{\Weil}{\mathrm{Weil}}
\newcommand{\Cart}{\mathrm{Cart}}
\newcommand{\bWeil}{\mathrm{b}\mathrm{Weil}}
\newcommand{\bCart}{\mathrm{b}\mathrm{Cart}}
\newcommand{\Cond}{\mathrm{Cond}}
\newcommand{\IC}{\mathrm{IC}}
\newcommand{\IH}{\mathrm{IH}}
\newcommand{\Eq}{\mathrm{Eq}}
\newcommand{\cris}{\mathrm{cris}}
\newcommand{\Zar}{\mathrm{Zar}}
\newcommand{\HvbCat}{\overline{\mathcal{V}\mathrm{ect}}}
\newcommand{\BanModCat}{\mathcal{B}\mathrm{an}\mathcal{M}\mathrm{od}}
\newcommand{\DesCat}{\mathcal{D}\mathrm{es}}
\newcommand{\RingCat}{\mathcal{R}\mathrm{ing}}
\newcommand{\SchCat}{\mathcal{S}\mathrm{ch}}
\newcommand{\AbCat}{\mathcal{A}\mathrm{b}}
\newcommand{\RSCat}{\mathcal{R}\mathrm{S}}
\newcommand{\LRSCat}{\mathcal{L}\mathrm{RS}}
\newcommand{\CLRSCat}{\mathbb{C}\text{-}\LRSCat}
\newcommand{\CRSCat}{\mathbb{C}\text{-}\RSCat}
\newcommand{\CLA}{\mathbb{C}\text{-}\mathcal{L}\mathrm{A}}
\newcommand{\CASCat}{\mathbb{C}\text{-}\mathcal{A}\mathrm{n}}
\newcommand{\LiuCat}{\mathcal{L}\mathrm{iu}}
\newcommand{\BanCat}{\mathcal{B}\mathrm{an}}
\newcommand{\BanAlgCat}{\mathcal{B}\mathrm{an}\mathcal{A}\mathrm{lg}}
\newcommand{\AnaCat}{\mathcal{A}\mathrm{n}}
\newcommand{\LiuAlgCat}{\mathcal{L}\mathrm{iu}\mathcal{A}\mathrm{lg}}
\newcommand{\AlgCat}{\mathcal{A}\mathrm{lg}}
\newcommand{\SetCat}{\mathcal{S}\mathrm{et}}
\newcommand{\ModCat}{\mathcal{M}\mathrm{od}}
\newcommand{\GerCat}{\mathcal{G}\mathrm{er}}
\newcommand{\AnaGerCat}{\mathbb{C}\text{-}\GerCat}
\newcommand{\TopCat}{\mathcal{T}\mathrm{op}}
\newcommand{\CohCat}{\mathcal{C}\mathrm{oh}}
\newcommand{\SolCat}{\mathcal{S}\mathrm{olid}}
\newcommand{\AffCat}{\mathcal{A}\mathrm{ff}}
\newcommand{\AffAlgCat}{\mathcal{A}\mathrm{ff}\mathcal{A}\mathrm{lg}}
\newcommand{\QcohLiuAlgCat}{\mathcal{L}\mathrm{iu}\mathcal{A}\mathrm{lg}^{\mathrm{QCoh}}}
\newcommand{\LiuMorCat}{\mathcal{L}\mathrm{iu}}
\newcommand{\Isom}{\mathcal{I}\mathrm{som}}
\newcommand{\Cris}{\mathcal{C}\mathrm{ris}}
\newcommand{\Pro}{\mathrm{Pro}-}
\newcommand{\Fin}{\mathcal{F}\mathrm{in}}
\newcommand{\norms}[1]{\left\|#1\right\|}
\newcommand{\HPDDiff}{\mathbf{D}\mathrm{iff}}
\newcommand{\Menn}[2]{\begin{bmatrix}#1\\#2\end{bmatrix}}
\newcommand{\Fins}{\widehat{\Vect}^F}
\newcommand\blfootnote[1]{%
  \begingroup
  \renewcommand\thefootnote{}\footnote{#1}%
  \addtocounter{footnote}{-1}%
  \endgroup
}


\makeatletter
\newcommand*{\addFileDependency}[1]{% argument=file name and extension
  \typeout{(#1)}
  \@addtofilelist{#1}
  \IfFileExists{#1}{}{\typeout{No file #1.}}
}
\makeatother



\newcommand*{\myexternaldocument}[2]{%
\externaldocument[#1]{#2}%
\addFileDependency{#2.tex}%
\addFileDependency{#2.aux}%
%\addFileDependency{#2.pdf}%
}


%\iffalse

\myexternaldocument{Introduction-}{Introduction}
\myexternaldocument{Topology-}{Topology-Bornology}
\myexternaldocument{Banach-}{Banach-Rings}
\myexternaldocument{Commutative-}{Commutative-Algebra}



\myexternaldocument{Local-}{Local-Algebras}
\myexternaldocument{Complex-}{Complex-Analytic-Spaces}
\myexternaldocument{ConstructionComplex-}{Constructions-Complex-Spaces}
\myexternaldocument{PropertyComplex-}{Properties-Complex-Spaces}
\myexternaldocument{GPropertyComplex-}{Global-Properties-Complex-Spaces}
\myexternaldocument{Analytic-}{Analytic-Sets}
\myexternaldocument{Morphisms-}{Morphisms-Complex-Spaces}

\myexternaldocument{Affinoid-}{Affinoid-Algebras}
\myexternaldocument{Berkovich-}{Berkovich-Analytic-Spaces}
\myexternaldocument{BerkProperty-}{Properties-Berkovich-Spaces}
%\fi


\bibliography{Ymir}

\endinput
\title{Ymir}
\begin{document}
\maketitle
\tableofcontents

\chapter*{Constructions of complex analytic spaces}\label{chap-constructionComplex}

\section{Introduction}\label{sec-introduction-constructioncomplex}

\section{Analytic spectra}

\begin{proposition}\label{prop-analyticspecrep}
    Let $S$ be a complex analytic space and $\mathcal{A}$ be an $\mathcal{O}_S$-module of finite presentation. Then the presheaf $F_{\mathcal{A}}$ on $\CASCat_{/S}$ defined by
    \[
        F_{\mathcal{A}}(T\xrightarrow{p} S)=\Hom_{\mathcal{O}_T}(p^*\mathcal{A},\mathcal{O}_T)  
    \]
    is representable.
\end{proposition}


\begin{proof}
    By the arguments of \cite[\href{https://stacks.math.columbia.edu/tag/01JJ}{Tag 01JJ}]{stacks-project}, the problem is local in $S$. So we may assume that $\mathcal{A}$ has the following form
    \[
        \mathcal{A}=\mathcal{O}_S[X_1,\ldots,X_n]/\mathcal{I}
    \]   
    for some $n\in \mathbb{N}$ and $\mathcal{I}\subseteq \mathcal{O}_S(S)[X_1,\ldots,X_n]$ an ideal sheaf of finite type. 
    
    \textbf{Step~1}. We first handle the case where $\mathcal{A}=\mathcal{O}_S[X_1,\ldots,X_n]$.

    In this case, we claim that $F_{\mathcal{A}}$ is represented by $S\times \mathbb{C}^n$. In fact, it suffices to observe that 
    \[
        \begin{split}
        F_{\mathcal{A}}(T\xrightarrow{p} S)\cn \Hom_{\mathcal{O}_T}(\mathcal{O}_T[X_1,\ldots,X_n],\mathcal{O}_T)\cn \mathcal{O}_T(T)^n \\
         = \Hom_{\CASCat}(T,\mathbb{C}^n)=\Hom_{\CASCat_{/S}}(T,S\times \mathbb{C}^n).
        \end{split}
    \]
    From this proof, it is easy to see that the universal morphism is 
    \[
        \eta:\mathcal{O}_{S\times \mathbb{C}^n}[X_1,\ldots,X_n]\rightarrow \mathcal{O}_{S\times \mathbb{C}^n}
    \]
    sending $X_i$ to $z_i$, the $i$-th coordinate of $\mathbb{C}^n$.

    \textbf{Step~2}. We handle the general case.
    We have a short exact sequence
    \[
        0\rightarrow \mathcal{I}\rightarrow \mathcal{O}_S[X_1,\ldots,X_n]\rightarrow \mathcal{A}\rightarrow 0.
    \]    
    For any $p:T\rightarrow S$ in $\CASCat$, we have an exact sequence
    \[
        p^*\mathcal{I}\rightarrow \mathcal{O}_T[X_1,\ldots,X_n]\rightarrow p^*\mathcal{A}\rightarrow 0.
    \]
    We then have
    \[
        \begin{aligned}
        F_{\mathcal{A}}(T)\cn & \left\{ h\in \Hom_{\mathcal{O}_T}(\mathcal{O}_T[X_1,\ldots,X_n],\mathcal{O}_T): h|_{p^*\mathcal{I}}=0  \right\}\\
        \cn & \left\{ h\in F_{\mathcal{O}_S[X_1,\ldots,X_n]}(T): h|_{p^*\mathcal{I}}=0  \right\}.
        \end{aligned}
    \]
    Let $\pi:S\times \mathbb{C}^n\rightarrow S$ be the projection. Then $F_{\mathcal{A}}(T)$ is represented by the closed subspace of $S\times \mathbb{C}^n$ defined by the ideal $\eta(\pi^* \mathcal{I})$, which is clearly of finite type.
\end{proof}

\begin{definition}
    Let $S$ be a complex analytic space and $\mathcal{A}$ be an $\mathcal{O}_S$-module of finite presentation. Then the complex analytic space representing the functor $F_{\mathcal{A}}$ in \cref{prop-analyticspecrep} is called the \emph{analytic spectrum} of $\mathcal{A}$. We denote it by $\Spec^{\An}_S \mathcal{A}$. By construction, there is a canonical morphism $\Spec^{\An}_S \mathcal{A}\rightarrow S$.

    By definition, we have a universal morphism $\xi\in F_{\mathcal{A}}(X)=\Hom_{\mathcal{O}_X}(\mathcal{A}_X,\mathcal{O}_X)$ with $X=\Spec^{\An}_S \mathcal{A}$. It defines a morphism of ringed spaces $X\rightarrow (|S|,\mathcal{A})$. The pull-back of an $\mathcal{A}$-module $\mathcal{M}$ is denoted by $\tilde{\mathcal{M}}$. The assignment $\mathcal{M}\mapsto \tilde{\mathcal{M}}$ is functorial in $M$.
\end{definition}
It is easy to see that $\Spec^{\An}_S \mathcal{A}$ is contravaraint in $\mathcal{A}$. 



\begin{proposition}\label{prop-specanbasechange}
    Let $S$ be a complex analytic space and $\mathcal{A}$ be an $\mathcal{O}_S$-module of finite presentation. Consider a morphism $g:S'\rightarrow S$ of complex analytic spaces. Then we have a Cartesian diagram
    \[
        \begin{tikzcd}
            \Spec^{\An}_{S'} g^*\mathcal{A} \arrow[d] \arrow[r] \arrow[rd, "\square", phantom] & \Spec^{\An}_S \mathcal{A} \arrow[d] \\
            S' \arrow[r, "g"]                                                                  & S                                  
        \end{tikzcd}
    \]
\end{proposition}
\begin{proof}
    This is clear at the level of functor of points.
\end{proof}

\begin{corollary}\label{cor-fiberspecan}
    Let $S$ be a complex analytic space and $\mathcal{A}$ be an $\mathcal{O}_S$-module of finite presentation. Take $s\in S$. Then $\Spec^{\An}_{\{s\}} \mathcal{A}_s\cn (\Spec^{\An}_S \mathcal{A})_s$. 
    
    Moreover, the universal morphism $\mathcal{A}_{\Spec^{\An}_{\{s\}} \mathcal{A}_s}\rightarrow \mathcal{O}_{\Spec^{\An}_{\{s\}} \mathcal{A}_s}$ is the reduction of the universal morphism $\mathcal{A}_{\Spec^{\An}_S \mathcal{A}}\rightarrow \mathcal{O}_{\Spec^{\An}_S \mathcal{A}}$ modulo $\mathfrak{m}_s$.
\end{corollary}
\begin{proof}
    This follows from \cref{prop-specanbasechange}.
\end{proof}


\begin{proposition}\label{prop-specanlocalringcompletion}
    Let $S$ be a complex analytic space and $\mathcal{A}$ be an $\mathcal{O}_S$-module of finite presentation. Take $s\in S$.
    Write $X=\Spec^{\An}_S \mathcal{A}$ and $\mathcal{A}_s:=\mathcal{A}\otimes_{\mathcal{O}_S}\mathcal{O}_{S,s}$. Then the map from $X_s$ to 
    \[
        \left\{ \mathfrak{m}\in \Spm_{\mathbb{C}} \mathcal{A}_s : \mathfrak{m}\supseteq\mathfrak{m}_s  \right\}
    \]
    sending $x\in X_s$ to the inverse image of $\mathfrak{m}_x$ with respect to $\mathcal{A}_s\rightarrow \mathcal{O}_{X,x}$ is bijective.

    If $\mathfrak{m}$ corresponds to $x\in X_s$, then the natural homomorphism $\mathcal{A}_s\rightarrow \mathcal{O}_{X,x}$ factorizes through $\mathcal{A}_{s,\mathfrak{m}}\rightarrow  \mathcal{O}_{X,x}$. The completion of the latter
    \[
        \widehat{\mathcal{A}_{s,\mathfrak{m}}}\rightarrow   \widehat{\mathcal{O}_{X,x}}
    \]
    is an isomorphism.
\end{proposition}
\begin{proof}
    By \cref{cor-fiberspecan}, we have natural bijections
    \[
        X_s\cn \Hom_{\{s\}}(\{s\},X_s)\cn \Hom_{\mathbb{C}\text{-}\AlgCat}(\mathcal{A}_s/\mathfrak{m}_s\mathcal{A}_s, \mathbb{C}).
    \]  
    This gives the desired bijection.

    Next we prove the latter part. The problem is local on $S$, we may assume that 
    \[
        \mathcal{A}=\mathcal{O}_S[X_1,\ldots,X_n]/\mathcal{I}
    \]  
    for some $n\in \mathbb{N}$ and some ideal $\mathcal{I}$ of finite type in $\mathcal{O}_S[X_1,\ldots,X_n]$. Recall that the universal morphism
    \[
        \eta:\mathcal{O}_{S\times \mathbb{C}^n}[X_1,\ldots,X_n]\rightarrow \mathcal{O}_{S\times \mathbb{C}^n}
    \]
    sends $X_i$ to $z_i$, the $i$-th coordinate of $\mathbb{C}^n$.

    By construction, we have
    \[
        \mathcal{A}_s\cn \mathcal{O}_{S,s}[X_1,\ldots,X_n]/\mathcal{I}_s  
    \]
    and
    \[
        \mathcal{O}_{X,x}=\mathcal{O}_{S\times \mathbb{C}^n,x}/\mathcal{J}_x,  
    \]
    where
    \[
        \mathcal{J}_x=\eta_x\left( \mathcal{I}_s \mathcal{O}_{S\times \mathbb{C}^n,x}[X_1,\ldots,X_n] \right).   
    \]
    We have a commutative diagram with exact rows
    \[
        \begin{tikzcd}
            0 \arrow[r] & \mathcal{I}_s \arrow[r] \arrow[d] & {\mathcal{O}_{S,s}[X_1,\ldots,X_n]} \arrow[r] \arrow[d] & \mathcal{A}_s \arrow[r] \arrow[d] & 0 \\
            0 \arrow[r] & \mathcal{J}_x \arrow[r]           & {\mathcal{O}_{S\times \mathbb{C}^n,x}} \arrow[r]        & {\mathcal{O}_{X,x}} \arrow[r]     & 0
        \end{tikzcd}.  
    \]
    The middle vertical map is induced by $\eta_x$.
    Let $\mathfrak{p}$ be the inverse image of $\mathfrak{m}_{S\times \mathbb{C}^n,x}$ under the vertical map in the middle. Then $\mathfrak{p}$ is generated by $\mathfrak{m}_s$ and $X_1-x_1,\ldots,X_n-x_n$, where $x_i\in \mathbb{C}$ is the value of $z_i$ at $x$ for $i=1,\ldots,n$.  By localization and completion, we find a commutative diagram with exact rows
    \[
        \begin{tikzcd}
            0 \arrow[r] & \widehat{(\mathcal{I}_s)_{\mathfrak{p}}} \arrow[r] \arrow[d] & {(\mathcal{O}_{S,s}[X_1,\ldots,X_n])^{\hat{}}_{\mathfrak{p}}} \arrow[r] \arrow[d] & \widehat{(\mathcal{A}_s)_{\mathfrak{m}}} \arrow[r] \arrow[d] & 0 \\
            0 \arrow[r] & \widehat{\mathcal{J}_x} \arrow[r]           & \widehat{\mathcal{O}_{S\times \mathbb{C}^n,x}} \arrow[r]        & \widehat{\mathcal{O}_{X,x}} \arrow[r]     & 0
        \end{tikzcd}.  
    \]
    Observe that
    \[
        (\mathcal{O}_{S,s}[X_1,\ldots,X_n])^{\hat{}}_{\mathfrak{p}}  \cong \widehat{\mathcal{O}_{S,s}}[[X_1-x_1,\ldots,X_n-x_n]]
    \]
    and
    \[
        \widehat{\mathcal{O}_{S\times \mathbb{C}^n,x}}\cong \widehat{O_{S,s}}\hat{\otimes}_k  \widehat{\mathcal{O}_{\mathbb{C}^n,(x_1,\ldots,x_n)}}\cong \widehat{\mathcal{O}_{S,s}}[[X_1-x_1,\ldots,X_n-x_n]].
    \]
    It is easy to see that the middle map is an isomorphism. As $\mathcal{J}_x$ is generated by $\mathcal{I}_s$, the first vertical map is also an isomorphism. Our assertion follows. 
\end{proof}


\begin{corollary}\label{cor-finitespectrumidentification}
    Let $S$ be a complex analytic space and $\mathcal{A}$ be a finite $\mathcal{O}_S$-algebra. Write $X=\Spec^{\An}_S\mathcal{A}$.
    Take $s\in S$. Then the fiber $X_s$ is finite and is in bijection with $\Spm_{\mathbb{C}} \mathcal{A}_s$. If $\mathfrak{m}$ corresponds to $x\in X_s$, then we have a natural isomorphism 
    \[
        \mathcal{A}_{s,\mathfrak{m}}\cn \mathcal{O}_{X,x}.  
    \]
\end{corollary}
\begin{proof}
    As $\mathcal{O}_{S,s}\rightarrow \mathcal{A}_s$ is finite, $\mathcal{A}_s$ is semi-local. On the other hand, by \cref{prop-specanlocalringcompletion}, 
    \[
        \mathcal{A}_{s,\mathfrak{m}}\rightarrow   \mathcal{O}_{X,x}
    \]
    is injective and $\mathcal{O}_{X,x}$ is quasi-finite over $\mathcal{O}_{S,s}$. 
    Then $\mathcal{O}_{X,x}$ is finite over $\mathcal{O}_{S,s}$ by \cref{Local-thm-analyticquasifiniteifffinite} in \nameref{Local-chap-local}.
    It follows from Nakayama's lemma that $\mathcal{A}_{s,\mathfrak{m}}\rightarrow   \mathcal{O}_{X,x}$ is also surjective.
\end{proof}

\begin{corollary}\label{cor-suppaffmorphism}
    Let $S$ be a complex analytic space and $\mathcal{A}$ be a finite $\mathcal{O}_S$-algebra. Then the image of $\Spec^{\An}_S\mathcal{A}\rightarrow S$ is $\Supp A$.
\end{corollary}
\begin{proof}
    This follows from \cref{cor-finitespectrumidentification} and the fact that $\Spm_{\mathbb{C}}\mathcal{A}_s=\Spm\mathcal{A}_s$ for all $s\in S$.
\end{proof}


\begin{proposition}\label{prop-moduletildepush}
    Let $S$ be a complex analytic space and $\mathcal{A}$ be a finite $\mathcal{O}_S$-algebra. Write $f:\Spec^{\An}_S \mathcal{A}$ for the structure map.
    Then we have the following assertions:
    \begin{enumerate}
        \item for all $\mathcal{A}$-module $\mathcal{M}$, the natural morphism 
        \[
            \mathcal{M}\rightarrow f_*\tilde{\mathcal{M}}  
        \]
        is an isomorphism, 
    
        In particular, $\mathcal{A}\cn f_*\mathcal{O}_X$.
        \item for all $\mathcal{O}_X$-module $\mathcal{F}$, the canonical morphism
            \[
                \widehat{f_*\mathcal{F}}\rightarrow \mathcal{F}
            \]
            is an isomorphism.
    \end{enumerate}
    In particular, the category of $\mathcal{A}$-modules is equivalent to the cateogory of $\mathcal{O}_X$-modules.
\end{proposition}
\begin{proof}
    By \cref{cor-specanfinitetopfinite}, $f$ is topologically finite. 
    Take $s\in S$. 
    Let $x_1,\ldots,x_n$ be the distinct points of $f^{-1}(s)$ and $\mathfrak{m}_1,\ldots,\mathfrak{m}_n$ denote the maximal ideals of $\mathcal{A}_s$ corresponding to $x_1,\ldots,x_n$.

    (1)    
    By \cref{Topology-cor-pushforwardsheaffinite} in \nameref{Topology-chap-topology} and \cref{cor-finitespectrumidentification},  
    \[
        (f_*\tilde{\mathcal{M}})_s\cong \prod_{i=1}^n \widehat{M}_{x_i}\cong \prod_{i=1}^n \widehat{\mathcal{M}}_s\otimes_{\mathcal{A}_s}\mathcal{O}_{X,x_i}\cong \mathcal{M}_s \otimes_{\mathcal{A}_s}\prod_{i=1}^n \mathcal{A}_{s,\mathfrak{m}_i}\cn \mathcal{M}_s.
    \]

    (2) By \cref{Topology-cor-pushforwardsheaffinite} in \nameref{Topology-chap-topology},
    \[
        f_*\mathcal{F}_s\cong \prod_{i=1}^n \mathcal{F}_{x_i}.  
    \]
    It follows that 
    \[
        \widetilde{f_*\mathcal{M}}_{x_i}\cong f_*\mathcal{F}_s\otimes_{\mathcal{A}_s} \mathcal{O}_{X,x_i}\cong \prod_{j=1}^n \mathcal{F}_{x_j}\otimes_{\mathcal{A}_s} \mathcal{A}_{s,\mathfrak{m}_i}
    \]
    for $i=1,\ldots,n$. But the only non-zero term is when $j=i$, so 
    \[
        \widetilde{f_*\mathcal{M}}_{x_i}\cong \mathcal{F}_{x_i}  
    \]
    for $i=1,\ldots,n$.
\end{proof}
\begin{corollary}\label{cor-specpushcohiscoh}
    Let $S$ be a complex analytic space and $\mathcal{A}$ be a finite $\mathcal{O}_S$-algebra. Write $f:\Spec^{\An}_S \mathcal{A}$ for the structure map. Then for any coherent $\mathcal{O}_X$-module $\mathcal{M}$, $f_*\mathcal{F}$ is coherent.

    Moreover, $f_*$ is exact from $\CohCat(\mathcal{O}_X)$ to
    $\CohCat(\mathcal{O}_Y)$.
\end{corollary}
\begin{proof}
    The exactness of $f_*$ follows from \cref{prop-moduletildepush}.

    We claim that up to shrinking $S$, we may assume that $\mathcal{M}$ has a global presentation. Fix $s\in S$ and let $x_1,\ldots,x_n$ be the distinct points of $f^{-1}(s)$.
    
    For each $j=1,\ldots,n$, we can find an open neighbourhood $U_j$ of $x_j$ in $X$, pairwise disjoint and an exact sequence 
    \[
        \mathcal{O}_{U_j}^{p_j}\rightarrow \mathcal{O}_{U_j}^{q_j}\rightarrow \mathcal{M}|_{U_j}\rightarrow 0
    \]
    for some $p_j,q_j\in \mathbb{Z}_{>0}$.  We may assume that $p_1=\cdots=p_n$ and $q_1=\cdots=q_n$. We denote the common values by $p$ and $q$. Then $U=U_1\cup\cdots\cup U_n$ is a neighbourhood of $f^{-1}(s)$, and we have an exact sequence 
    \[
        \mathcal{O}_{U}^{p}\rightarrow \mathcal{O}_{U}^{q}\rightarrow \mathcal{M}|_{U}\rightarrow 0.
    \]
    By \cref{Topology-lma-opennhfiberclosedmap} in \nameref{Topology-chap-topology},
    we may assume that $U=\pi^{-1}(V)$ for some open neighbourhood $V$ of $s$ in $S$. The induced map $f':U\rightarrow V$ is finite and by \cref{Topology-cor-pushforwardsheaffinite} in \nameref{Topology-chap-topology}.

    Now let us take a presentation
    \[
        \mathcal{O}^{p}\rightarrow \mathcal{O}^{q}\rightarrow \mathcal{M}\rightarrow 0.
    \]
    By \cref{prop-moduletildepush}, we have an exact sequence
    \[
        f_*  \mathcal{O}^{p}\rightarrow f_* \mathcal{O}^{q}\rightarrow f_* \mathcal{M}\rightarrow 0.
    \]
    By \cref{prop-moduletildepush} again, this can be written as
    \[
        \mathcal{A}^p\rightarrow \mathcal{A}^q\rightarrow   f_* \mathcal{M}\rightarrow 0.
    \]
    It follows that $f_*\mathcal{M}$ is coherent.
\end{proof}

\begin{proposition}\label{prop-morphismrelaffspaces}
    Let $S$ be a complex analytic space and $\mathcal{A}$, $\mathcal{B}$ be $\mathcal{O}_S$-algebras of finite presentation. Assume that $\mathcal{A}$ is finite. Then we have a natural bijection
    \[
        \Hom_{\mathcal{O}_S}(\mathcal{B},\mathcal{A})\cn \Hom_{\CASCat_{/S}}(\Spec^{\An}_S \mathcal{A},\Spec^{\An}_S \mathcal{B}).  
    \]
\end{proposition}
\begin{proof}
    Let $f:X:=\Spec^{\An}_S\mathcal{A}\rightarrow S$ be the natural map.
    We construct the bijection as
    \[
        \Hom_{\mathcal{O}_S}(\mathcal{B},\mathcal{A})\cn \Hom_{\mathcal{O}_S}(\mathcal{B},f_*\mathcal{O}_X)\cn \Hom_{\mathcal{O}_X}(\mathcal{B}_X,\mathcal{O}_X)\cn \Hom_{\CASCat_{/S}}(\Spec^{\An}_S \mathcal{A},\Spec^{\An}_S \mathcal{B}).
    \]
    The first map is a bijection by \cref{prop-moduletildepush}
\end{proof}

\begin{definition}
    Let $S$ be a complex analytic space and $\mathcal{E}$ be an $\mathcal{O}_S$-module of finite presentation. We define the \emph{vector bundle} $\mathbf{V}(\mathcal{E})$ generated by $\mathcal{E}$ as
    \[
        \mathbf{V}(\mathcal{E})=\Spec^{\An}_S \Sym \mathcal{E}.
    \]
    We have a natural projection $\mathbf{V}(\mathcal{E})\rightarrow S$.
\end{definition}
We remind the readers that we are following Grothendieck's convention for $\mathbf{V}(\mathcal{E})$, which is different from Fulton's.



\section{Analytic germs}

\begin{definition}\label{def-complexgerm}
    A \emph{pointed complex analytic space} is a pair $(X,x)$ consisting of a complex analytic space $X$ and a point $x\in X$. A morphism between pointed complex analytic spaces $(X,x)$ and $(Y,y)$ is a morphism $f:X\rightarrow Y$ of complex analytic spaces such that $f(x)=y$. The category of pointed complex analytic spaces is denoted by $\CASCat_*$. 
    

    The category of \emph{complex analytic germs} $\AnaGerCat$ is the right category of fractions of $\CASCat$ with respect to the system of morphisms $f:(X,x)\rightarrow (Y,y)$ such that $f:X\rightarrow Y$ is an open immersion. An element in $\AnaGerCat$ is called a \emph{complex analytic germ}. A complex analytic germ represented by $(X,x)$ is denoted by $X_x$.

    Given a complex analytic germ $X_x$, we write $\mathcal{O}_{X,x}$ for the local ring of $X$ at $x$. Clearly, it does not depend on the choice of $(X,x)$. Given any morphism $f:X_x\rightarrow Y_y$ of complex analytic germs, we have an obvious local homomorphism $f^{\#}:\mathcal{O}_{Y,y}\rightarrow \mathcal{O}_{X,x}$.
\end{definition}


\begin{definition}\label{def-closedsubgerm}
    Given a complex analytic germ $X_x$, a \emph{closed subgerm} of $X_x$ is an isomorphism class in $\AnaGerCat_{/X_x}$ of $Y_x$ represented by a closed analytic subspace of $X$ containing $x$ for any representation $(X,x)$ of $X_x$.

    In particular, $X_x$ is a closed subgerm of $X_x$. A closed subgerm $Y_y$ of $X_x$ is \emph{proper} if $Y_y$ is different from $X_x$ as subgerms. 

    Given a closed subgerm $Y_x$ of $X_x$, we have an induced surjective homomorphism $\mathcal{O}_{X,x}\rightarrow \mathcal{O}_{Y,y}$. The kernel is denoted by $I(Y,x)$ or $I_X(Y,x)$.
\end{definition}

\begin{thm}\label{thm-germlocalringequivalence}
    The functor $\AnaGerCat^{\mathrm{op}}\rightarrow \CLA$ defined in \cref{def-complexgerm} is an equivalence.
\end{thm}
\begin{proof}
    \textbf{Step~1}. We show that the functor is faithfully. 
    
    In order words, let $(X,x)$ and $(Y,y)$ be two pointed complex analytic spaces and $f,g:(X,x)\rightarrow (Y,y)$ be two morphsims inducing the same map $\mathcal{O}_{Y,y}\rightarrow  \mathcal{O}_{X,x}$, then $f$ and $g$ coincide on a neighbourhood of $x$ in $X$.

    The question is open on $Y$, so we may reduce to the case where $Y$ is a complex model space.  We then further reduce to the case where $Y$ is a domain in $\mathbb{C}^n$ for some $n\in \mathbb{N}$ and then to $Y=\mathbb{C}^n$.

    By \cref{Complex-thm-hCnidentification} in \nameref{Complex-chap-complex}, $f$ and $g$ can be identified with systems $(f_1,\ldots,f_n)\in \mathcal{O}_X(X)^n$ and $(g_1,\ldots,g_n)\in \mathcal{O}_X(X)^n$. The assumption $f_x^{\#}=g_x^{\#}$ menas $f_{i,x}=g_{i,x}$ for $i=1,\ldots,n$. So $f_i=g_i$ after shrinking $X$. We conclude by  \cref{Complex-thm-hCnidentification} in \nameref{Complex-chap-complex} again.

    \textbf{Step~2}. We show that the functor is fully faithful. 
    
    In other words, let $(X,x)$ and $(Y,y)$ be two pointed complex analytic spaces and $\varphi:\mathcal{O}_{Y,y}\rightarrow \mathcal{O}_{X,x}$ be a morphism in $\CLA$. Then we can find an open neighbourhood $U$ of $x$ in $X$ and a morphism $(U,x)\rightarrow (Y,y)$ inducing $\varphi$.

    The problem is local on $Y$, so we may assuem that $Y$ is a complex model space, say $Y$ is a closed subspace of a domain $V$ in $\mathbb{C}^n$ defined by a coherent ideal $\mathcal{I}$. We write $\psi:\mathcal{O}_{V,y}\rightarrow \mathcal{O}_{X,x}$ the homomorphism induced by $\varphi$, we have a commutative diagram
    \[
        \begin{tikzcd}
            {\mathcal{O}_{V,y}} \arrow[d, two heads] \arrow[r, "\psi"] & {\mathcal{O}_{Y,y}} \\
            {\mathcal{O}_{X,x}} \arrow[ru, "\varphi"]                  &                    
        \end{tikzcd}.  
    \]
    Let $z_1,\ldots,z_n$ be the coordinates on $V$.
    Let $f_{i,x}$ be the image of $z_{i,x}$ under $\psi$ for $i=1,\ldots,n$. Take an open neighbourhood $U$ of $x$ in $X$ so that $f_{i,x}$ lifts to $f_i\in \mathcal{O}_X(U)$ for $i=1,\ldots,n$. By \cref{Complex-thm-hCnidentification} in \nameref{Complex-chap-complex}, $f_1,\ldots,f_n$ then defines a morphism $g:U\rightarrow \mathbb{C}^n$. Clearly $g(x)=y$. But $g_x^{\#}$ and $\psi$ coincide on $z_{i,y}$ so $g_x^{\#}=\psi$ as $\mathcal{O}_{V,y}=\mathbb{C}\{z_{1,y}-a_1,\ldots,z_{n,y}-a_n\}$ with $a_i=\epsilon(z_{i,y})$ for $i=1,\ldots,n$. Therefore, $g_x^{\#}(\mathcal{I}_y)=0$. Up to shrinking $U$, we may guarantee that $g(U)\subseteq V$ and $g^*(\mathcal{I})=0$ on $U$. Namely, $g$ factorizes through $f:U\rightarrow Y$ and $f^*_x=\varphi$.

    \textbf{Step~3}. We show that the functor is essentially surjective.

    In other words, let $A$ be a complex analytic local algebra, then there is a pointed complex analytic space $(X,x)$ with $\mathcal{O}_{X,x}\cong A$ in $\CLA$.

    We may assume that $A=\mathbb{C}\{z_1,\ldots,z_n\}/I$ for some $n\in \mathbb{N}$ and ideal $I$ in $\mathbb{C}\{z_1,\ldots,z_n\}$. Then $I$ is finitely generated as $\mathbb{C}\{z_1,\ldots,z_n\}$ is noetherian. Take finitely many generators $f_1,\ldots,f_m\in I$. We extend $f_1,\ldots,f_m$ to $g_1,\ldots,g_m\in \mathcal{O}_{\mathbb{C}^n}(U)$ for some open neighbourhood $U$ of $0$ in $\mathbb{C}^n$. Then the closed subspace $X$ of $U$ defined by $f_1,\ldots,f_m$ satisfies the required conditions.
\end{proof}


\begin{lemma}\label{lma-spreadfinitelag}
    Let $S$ be a complex analytic space and $s\in S$. For any finite $\mathcal{O}_{S,s}$-algebra $A$, there is an open neighbourhood $U$ of $s$ in $S$ and a finite $\mathcal{O}_U$-algebra such that $\mathcal{A}_s\cong A$.
\end{lemma}
\begin{proof}
    Let $s\in S$, as $\mathcal{A}_s$ is a finite $\mathcal{O}_{S,s}$-algebra, we can find finitely many generators $\sigma_{1,s},\ldots,\sigma_{n,s}$. As $\mathcal{A}_s$ is integral over $\mathcal{O}_{S,s}$, we can find unitary polynomials $F_{i,s}\in \mathcal{O}_{S,s}[X_i]$ such that $F_{i,s}(\sigma_{i,s})=0$ for $i=1,\ldots,n$. Take a sufficient small neighbourhood $U$ of $s$ so that $\sigma_{i,s}$ lifts to $\sigma_i\in \mathcal{O}_S(U)$ and $F_{i,s}$ lifts to a unitary polynomial $F_i\in H^0(U,\mathcal{O}_S[X_i])$ for $i=1,\ldots,n$. Up to shrinking $U$, we may guarantee that $\sigma_1,\ldots,\sigma_n$ generate $\mathcal{A}|_U$ at all points and $F_i(\sigma_i)=0$ for $i=1,\ldots,n$. Then $\mathcal{B}:=\mathcal{O}_U[X_1,\ldots,X_n]/(F_1,\ldots,F_n)$ is coherent and we have a surjective homomorphism $\mathcal{B}\rightarrow \mathcal{A}|_U$ sneding $X_i$ to $\sigma_i$ for $i=1,\ldots,n$. As the kenrel of this homomorphism is of finite ytpe, up to shrinking $U$, we may take finitely many $G_1,\ldots,G_m\in \mathcal{B}(U)$ that generate the kernel. Lift $G_1,\ldots,G_m$ to $H_1,\ldots,H_m\in H^0(U,\mathcal{O}_S[X_1,\ldots,X_m])$, then
    \[
        \mathcal{A}|_U\cong \mathcal{O}_U[X_1,\ldots,X_n]/(F_1,\ldots,F_n,G_1,\ldots,G_m).  
    \]


   This follows from the same arguments of the proof of \cref{thm-germlocalringequivalence} Step~3.
\end{proof}
\begin{corollary}\label{cor-specanfinitetopfinite}
    Let $S$ be a complex analytic space and $\mathcal{A}$ be a finite $\mathcal{O}_S$-algebra, then the map $\Spec^{\An}_S \mathcal{A}\rightarrow S$ is topologically finite.
\end{corollary}
\begin{proof}
    By \cref{cor-finitespectrumidentification}, the fibers of $\Spec^{\An}_S \mathcal{A}\rightarrow S$ is finite. The map $\Spec^{\An}_S \mathcal{A}\rightarrow S$ is separated by construction. It remains to show that the map is closed.

    The problem is local on $S$.
    By the proof of \cref{lma-spreadfinitelag}, we can find a closed immersion over $S$: $\Spec^{\An}_S \mathcal{A}\rightarrow \Spec^{\An}_S\mathcal{B}$, where $\mathcal{B}=\mathcal{O}_S[X_1,\ldots,X_n]/(F_1,\ldots,F_n)$ for some $n\in \mathbb{N}$, where $F_i$ is a unitary polynomial in $\mathcal{O}_S(S)[X_i]$ for $i=1,\ldots,n$. It suffices to show that $\Spec^{\An}_S \mathcal{B}\rightarrow S$ is closed.

    Observe that
    \[
        \Spec^{\An}_S \mathcal{B}\cong \Spec^{\An}_S \prod_{j=1}^n \mathcal{O}_S[X_j]/(F_j)
    \]
    in $\AnaCat_{/S}$ as can be seen from the functor of points. So the problem reduces to showing that 
    \[
        \Spec_S^{\An} \mathcal{O}_S[X]/(F)\rightarrow S  
    \]
    for a unitary polynomial is closed. This is the classical continuity of roots.
\end{proof}


Next we describe the local structure of a complex analytic germ.
\begin{thm}\label{thm-morphismdefinedbygeneratorlocal}
    Let $X_x$ be a complex analytic germ, $n\in \mathbb{Z}_{>0}$ and $f_1,\ldots,f_n\in \mathcal{O}_{X,x}$ be a system of parameters. We have a morphism $X_x\rightarrow \mathbb{C}^n_0$ induced by $f_1,\ldots,f_n$. Then there is an open neighbourhood $U$ of $0$ in $\mathbb{C}^n$ and a finite $\mathcal{O}_U$-algebra $\mathcal{A}$ such that $\mathcal{A}_0\cong \mathcal{O}_{X,x}$. The space $\Spec^{\An}_{U}(\mathcal{A})$ admits a unique point $x'$ over $0$ and $X_x$ is isomorphic to $\Spec^{\An}_{U}(\mathcal{A})_{x'}$ in $\AnaGerCat_{/\mathbb{C}^n_0}$.
\end{thm}
\begin{proof}
    As $f_1,\ldots,f_n$ is a system of parameters, $\mathcal{O}_{X,x}\rightarrow \mathcal{O}_{\mathbb{C}^n,0}$ is finite. By \cref{lma-spreadfinitelag}, we can spread $\mathcal{O}_{X,x}$ to a finite $\mathcal{O}_U$-algebra on an open neighbourhood $U$ of $0$ in $\mathbb{C}^n$. Let $Y=\Spec^{\An}_{U}(\mathcal{A})$. 
    It follows from \cref{cor-finitespectrumidentification} that $Y$ has a unique point $x'$ over $0$. By \cref{thm-germlocalringequivalence}, up to shrinking $U$, we may guarantee that $X_x$ and $Y_{x'}$ are isomorphic over $\mathbb{C}^n_0$.
\end{proof}


\begin{proposition}\label{prop-bijsubgermideal}
    Let $X_x$ be a complex analytic germ.
    The map $Y_x\mapsto I_X(Y,x)$ defines a bijection between the set of closed subgerms of $X_x$ and the set of ideals of $\mathcal{O}_{X,x}$.
\end{proposition}
\begin{proof}
    We construct a reverse map. Given an ideal $I$ of $\mathcal{O}_{X,x}$, as $\mathcal{O}_{X,x}$ is noetherian, $I$ is finitely generated. We can find an open neighbourhood $U$ of $x$ in $X$ and an ideal sheaf of finite type $\mathcal{I}$ of $U$ with $\mathcal{I}_x=I$. Let $Y$ be the closed analytic subspace of $X$ defined by $\mathcal{I}$. We associated $Y_x$ with $I$. 
    
    It is easy to verify that this map is the inverse of the given map.
\end{proof}

\begin{definition}
    Let $X_x$ be a complex analytic germ and $Y_x, Z_x$ be two closed subgerms of $X_x$. We say $Y_x$ is contained in $Z_x$ and write
    $Y_x\subseteq Z_x$ if $I(Y,x)\supseteq I_X(Z,x)$. This defines a partial order on the set of closed subgerms of $X_x$.
\end{definition}

\begin{definition}\label{def-integralgerm}
    A complex analytic germ $X_x$ is \emph{integral} if $\mathcal{O}_{X,x}$ is integral.

    We also say $(X,x)$ is \emph{integral}.
\end{definition}

\begin{thm}[Nullstellensatz]\label{thm-Nullstellensatz}
    Let $X_x$ be an integral complex analytic germ and $Y_y$ be a closed subgerm of $X_x$. Then the following are equivalent:
    \begin{enumerate}
        \item $Y_x$ is a proper closed subgerm of $X_x$;
        \item $|Y|_x$ is a proper closed subgerm of $|X|_x$. 
    \end{enumerate}
\end{thm}
\begin{proof}
    (2) $\implies$ (1): This is obvious.

    (1) $\implies$ (2): Consider a proper closed subgerm $Y_x$ of $X_x$. By \cref{prop-bijsubgermideal}, $I(Y,x)\neq 0$. 
    
    \textbf{Step~1}. We reduce to the case $I(Y,x)=(f)$ for some non-zero element $f\in \mathcal{O}_{X,x}$.

    Take a non-zero element $f\in I(Y,x)$. Let $Y'_x$ be the subgerm of $X_x$ corresponding to the ideal $(f)$ of $\mathcal{O}_{X,x}$. Then $Y_x\subseteq Y'_x$.  It suffices to show that $|Y'|_x\neq |X|_x$. We may replace $Y$ by $Y'$.

    \textbf{Step~2}. We prove that $|Y|_x\neq |X|_x$. 

    Note that $f$ is not a zero-divisor as $\mathcal{O}_{X,x}$ is integral. Write $n=\dim \mathcal{O}_{X,x}$.
    By Krulls Hauptidealsatz, $\dim \mathcal{O}_{X,x}/(f)=n-1$.
    Let $\overline{f_1},\ldots,\overline{f_{n-1}}$ be a system of parameters (\cite[\href{https://stacks.math.columbia.edu/tag/00KU}{Tag 00KU}]{stacks-project}) of $\mathcal{O}_{X,x}/(f)$. Lift them to $f_1,\ldots,f_{n-1}\in \mathcal{O}_{X,x}$. Then $(f_1,\ldots,f_{n-1},f)$ is a system of parameters of $\mathcal{O}_{X,x}$. Let $\varphi:X_x\rightarrow \mathbb{C}^n_0$ and $\psi:Y_x\rightarrow \mathbb{C}^{n-1}_0$ be the morphisms defined by these systems of parameters. We then have a commutative diagram in $\AnaGerCat$:
    \[
        \begin{tikzcd}
            Y_x \arrow[r, hook] \arrow[d, "\psi"] & X_x \arrow[d, "\varphi"] \\
            \mathbb{C}^{n-1}_0 \arrow[r, hook]    & \mathbb{C}^{n}_0        
        \end{tikzcd}
    \]
    It induces a commutative diagram of topological germs:
    \[
        \begin{tikzcd}
            {|Y|_x} \arrow[r, hook] \arrow[d, "|\psi|"] & {|X|_x} \arrow[d, "|\varphi|"] \\
            \mathbb{C}^{n-1}_0 \arrow[r, hook]    & \mathbb{C}^{n}_0        
        \end{tikzcd}
    \]
The morphism of topological germs of $\mathbb{C}^{n-1}_0\rightarrow \mathbb{C}^{n}_0$ is clearly not an isomorphism, so it suffices to show that $|\varphi|:|X|_x\rightarrow \mathbb{C}^{n}_0   $ is surjective, in the sense that if we represent $|\varphi|$ by a morphism $(U,x)\rightarrow (\mathbb{C}^n,0)$ from an open neighbourhood $U$ of $x$ in $X$ to $\mathbb{C}^n$, then its image contains an open neighbourhood of $0$ in $\mathbb{C}^n$.

By \cref{thm-morphismdefinedbygeneratorlocal}, we may assume that $X=\Spec^{\An}_X \mathcal{A}$ for some finite $\mathcal{O}_X$-algebra $\mathcal{A}$ and $X$ has a unique point over $0$. Then by \cref{cor-finitespectrumidentification}, we have $\mathcal{A}_0\cn \mathcal{O}_{X,x}$. By \cref{Local-cor-morphismpowertolocalsurinj} in \nameref{Local-chap-local}, the natural homomorphism
\[
    \varphi^{\#}:\mathcal{O}_{\mathbb{C}^n,0}=\mathbb{C}\{X_1,\ldots,X_n\}\rightarrow \mathcal{A}_0
\]
is injective. 

By \cref{cor-suppaffmorphism}, it remains to show that $\Supp \mathcal{A}$ is a neighbourhood of $s$ in $S$. But the kernel of $\mathcal{O}_S\rightarrow \mathcal{A}$ is $0$ at $s$ hence $0$ in a neighbourhood of $s$ since both $\mathcal{O}_S$ and $\mathcal{A}$ are coherent by \cref{Complex-cor-cohsheaffinitepre} in \nameref{Complex-chap-complex}.
\end{proof}


\begin{corollary}\label{cor-orderrelationsubgermideal}
    Let $X_x$ be a complex analytic germ and $I,J$ be two ideals in $\mathcal{O}_{X,x}$. We let $W(I)$, $W(J)$ denote the topological germs of the closed analytic subgerms of $X_x$ defined by $I$ and $J$ respectively.
    Then the following are equivalent:
    \begin{enumerate}
        \item $W(I)\subseteq W(J)$;
        \item $J\subseteq \sqrt{I}$.
    \end{enumerate}
\end{corollary}
\begin{proof}
    If (2) is true, as $\mathcal{O}_{X,x}$ is noetherian, we can find $n\in \mathbb{Z}_{>0}$ such that $J^n\subseteq I$. Extend $I,J$ to coherent ideals $\mathcal{I},\mathcal{J}$ on $X$ up to shrinking $X$.
    Then $\Supp \mathcal{O}_X/\mathcal{J}\subseteq \Supp \mathcal{O}_X/\mathcal{I}$. Hence, (1) holds.

    Suppose that (1) holds. In order to prove (2), we may assume that $I$ is prime. Then the closed analytic subgerm  $Y_x$ of $X_x$ defined by $I$ is integral. Let $Z_x$ denote the closed analytic subgerm of $X_x$ defined by $J$. The intersection $Y_x\cap Z_x$ of the germs $Y_x$ and $Z_x$ is by definition the closed analytic subgerm of $X_x$ defined by $I+J$. Then
    \[
        |Y_x\cap Z_x|=|Y|_x\cap |Z|_x= W(I).  
    \]
    By \cref{thm-Nullstellensatz}, $Y_x\subseteq Z_x$. Namely, (2) holds.
\end{proof}
\begin{corollary}\label{cor-topgermqunilpotent}
    Let $X_x$ be a complex analytic germ and $Y_x$ be a closed analytic subgerm. Then the following are equivalent:
    \begin{enumerate}
        \item $|X|_x=|Y|_x$;
        \item $I_X(Y,x)$ is nilpotent.
    \end{enumerate}
    In particular, if these conditions hold, $\dim \mathcal{O}_{Y,x}=\dim \mathcal{O}_{X,x}$. 
\end{corollary}
\begin{proof}
    This follows immediately from \cref{cor-orderrelationsubgermideal}.
\end{proof}

\begin{corollary}\label{cor-isolatedpointartin}
    Let $X$ be a complex analytic space and $x\in X$. Then the following are equvalent:
    \begin{enumerate}
        \item $x$ is isolated in $X$;
        \item $\mathcal{O}_{X,x}$ is artinian.
    \end{enumerate}
\end{corollary}
\begin{proof}
    (1) simply means that $X_x=\{x\}_x$. By \cref{cor-topgermqunilpotent}, this holds if and only if $\mathfrak{m}_x$ is nilpotent. As $\mathcal{O}_{X,x}$ is noetherian, the latter is equivalent to that $\mathcal{O}_{X,x}$ is artinian.
\end{proof}

\begin{corollary}\label{cor-analyticsubspacesamesupp}
    Let $X$ be a complex analytic space and $Y$ be a closed analytic subspace defined by a coherent ideal $\mathcal{I}$. Then the following are equivalent:
    \begin{enumerate}
        \item $|X|=|Y|$;
        \item $\mathcal{I}$ is locally nilpotent.
    \end{enumerate} 
\end{corollary}
\begin{proof}
    This follows immediately from \cref{cor-topgermqunilpotent}.
\end{proof}

\begin{corollary}
    Let $X$ be a complex analytic space and $f\in \mathcal{O}_X(X)$. Then the following are equivalent:
    \begin{enumerate}
        \item $f(x)=0$ for all $x\in X$;
        \item $f$ is locally nilpotent.
    \end{enumerate}
\end{corollary}
\begin{proof}
    This follows from \cref{cor-analyticsubspacesamesupp}, where we take $\mathcal{I}$ as the coherent ideal generated by $f$.
\end{proof}

\begin{corollary}[Rückert Nullstellensatz]\label{cor-Rucnull}
    Let $X$ be a complex analytic space and $\mathcal{F}$ be a coherent sheaf of $\mathcal{O}_X$-modules. Let $f\in \mathcal{O}_X(X)$ be a function that vanishes on $\Supp \mathcal{F}$. Then for any $x\in X$, there is an open neighbourhood $U\subseteq X$ of $x$ and $m\in \mathbb{Z}_{>0}$ such that $f^m\mathcal{F}|_U=0$.
\end{corollary}
\begin{proof}
    Let $\mathcal{G}$ be the annihilator sheaf of $\mathcal{F}$: 
    \[
       \mathcal{G}:=\ker \left(\mathcal{O}_X\rightarrow \Homint_{\mathcal{O}_X}(\mathcal{F},\mathcal{F})\right),
    \]
    where the map $\mathcal{O}_X\rightarrow \Homint_{\mathcal{O}_X}(\mathcal{F},\mathcal{F})$ sends a local section $f$ of $\mathcal{O}_X$ to the endohomomorphism of multiplying by $f$ of $\mathcal{F}$.
    Then $\mathcal{G}$ is a coherent sheaf by Oka's coherence theorem \cref{Complex-thm-Okacoh} in \nameref{Complex-chap-complex}. Let $Y$ be the closed analytic subspace defined by $\mathcal{G}$. By our assumption, $f$ is everywhere zero on $Y$, so $f$ is locally nilpotent in $\mathcal{O}_X/\mathcal{G}\subseteq \Homint_{\mathcal{O}_X}(\mathcal{F},\mathcal{F})$. 
\end{proof}

\begin{corollary}
    Let $X$ be a complex analytic space and $\mathcal{I}$ and $\mathcal{J}$ be coherent ideal sheaves on $X$. Then the following are equivalent:
    \begin{enumerate}
        \item $\Supp \mathcal{O}_X/\mathcal{I}\subseteq \Supp \mathcal{O}_X/\mathcal{J}$;
        \item For any $x\in X$, there is an open neighbourhood $U$ of $x$ in $X$ and $n\in \mathbb{Z}_{>0}$ such that
            \[
                J^n|_U\subseteq I|_U.
            \]
    \end{enumerate}
\end{corollary}
\begin{proof}
    This follows immediately from \cref{cor-orderrelationsubgermideal}.
\end{proof}


\section{Analytic subsets}

\begin{definition}
    Let $X$ be a complex analytic space. A subset $A\subseteq X$ is \emph{analytic at $x\in X$} if there is an open neighbourhood $U$ of $x$ in $X$ and finitely many $f_1,\ldots,f_m\in \mathcal{O}_X(U)$ such that 
    \[
        A\cap U=\left\{ x\in U: f_1(x)=\cdots=f_m(x)=0 \right\}.  
    \]
    We will denote the set on the right-hand side as $N_U(f_1,\ldots,f_m)$.
    A subset $A\subseteq X$ is \emph{analytic} in $X$ if it is analytic at all $x\in X$.
\end{definition}
We observe that given $A\subseteq X$, the set of points $x\in X$ such that $A$ is analytic at $x$ is open. Also observe that an analytic set is necessarily closed.
Analytic sets are clearly closed under finite intersection and finite unions.

\begin{example}\label{ex-closedanalyticspaceanalyt}
    Let $X$ be a complex analytic space. 
    The underlying set of a closed analytic subspace of $X$ is an analytic set in $X$.

    In particular, the support of a coherent sheaf of $\mathcal{O}_X$-modules is an analytic set in $X$.
\end{example}

\begin{proposition}\label{prop-analyticsettrans}
    Let $X$ be a complex analytic space and $Y$ be a closed analytic subspace of $X$. Then each analytic set $A$ in $Y$ is also an analytic set in $X$. 

    Conversely, if $A$ is an analytic subset of $X$, then $A\cap Y$ is an analytic set in $Y$.
\end{proposition}
\begin{proof}
    We prove the first part. Let $A$ be an analytic set in $Y$. Then $A$ is closed in $Y$. It follows that $A$ is closed in $X$. Let $a\in A$, we can find an open neighbourhood $V$ of $a$ in $Y$ and finitely many $g_1,\ldots,g_k\in \mathcal{O}_Y(V)$ such that 
    \[
        A\cap V=N_V(g_1,\ldots,g_k).  
    \]
    Up to shrinking $V$, we may find a neighbourhood $U$ of $a$ in $X$ with $V=Y\cap U$ and $f_1,\ldots,f_k\in \mathcal{O}_X(U)$ lifting $g_1,\ldots,g_k$. Then 
    \[
        A\cap U=N_U(f_1,\ldots,f_k)\cap Y.  
    \]
    So by \cref{ex-closedanalyticspaceanalyt}, $A\cap U$ is analytic at $a$ as a subset of $X$.

    The second part is obvious.
\end{proof}

\begin{definition}\label{def-sheafidealanaset}
    Let $X$ be a complex analytic space and $A\subseteq X$ be an analytic set. We define the \emph{sheaf of ideals} $\mathcal{J}_A$ of $A$ as the sheafification of the presheaf of ideals on $X$ defined by 
    \[
        U\mapsto \left\{ f\in \mathcal{O}_X(U): N_U(f)\supseteq M\cap U \right\}
    \]
    for any open subset $U\subseteq X$.
\end{definition}
Observe that $\mathcal{J}_A$ is reduced.

\begin{lemma}\label{lma-germidealanalyticsetcont}
    Let $X$ be a complex analytic space and $A,B\subseteq X$ be analytic sets. Take $x\in X$. Then the following are equivalent:
    \begin{enumerate}
        \item $\mathcal{J}_{A,x}\subseteq \mathcal{J}_{B,x}$;
        \item $A\cap U\supseteq B\cap U$ for some neighbourhood $U$ of $x$ in $X$.
    \end{enumerate}
\end{lemma}
\begin{proof}
    (2) $\implies$ (1): This is trivial.

    (1) $\implies$ (2): Choose a neighbourhood $U$ of $x$ and finitely many $f_1,\ldots,f_k\in \mathcal{O}_X(U)$ such that $A\cap U=N_U(f_1,\ldots,f_k)$. Then $f_{1,x},\ldots,f_{k,x}\in \mathcal{J}_{A,x}\subseteq \mathcal{J}_{B,x}$. Up to shrinking $U$, we may assume that $f_1,\ldots,f_k\in \mathcal{J}_B(U)$. It follows that $A\cap U\supseteq B\cap U$.
\end{proof}


\begin{lemma}\label{lma-analyticsetlocallyidealsheaf}
    Let $X$ be a complex analytic space and $A$ be an analytic set in $X$. Take $a\in A$.
    Let $\mathcal{I}$ be a coherent ideal sheaf on $X$ with $\mathcal{I}_a=\mathcal{J}_{A,a}$. Then there is an open neighbourhood $U$ of $a$ in $X$ such that
    \[
        W(\mathcal{I}|_U)=A\cap U.  
    \]
\end{lemma}

The lemma tells that an analytic set can always be locally wrtten in the form $W(\mathcal{I})$ for some open set $U\subseteq X$ and a coherent ideal $\mathcal{I}$ on $U$.

\begin{proof}
    Choose an open neighbourhood $U$ of $x$ in $X$ and finitely many sections $f_1,\ldots,f_k\in \mathcal{J}_A(U)$ such that 
    \[
        \mathcal{I}|_U=\mathcal{O}_Uf_1+\cdots+\mathcal{O}_U f_k.  
    \]
    After shrinking $U$, we may assume that 
    \[
        A\cap U=N_U(g_1,\ldots,g_l)  
    \]
    for finitely many $g_1,\ldots,g_l\in \mathcal{J}_A(U)$. Then $g_{1,a},\ldots,g_{l,a}\in \mathcal{J}_{A,a}=\mathcal{I}_a$. So up to shrinking $U$, we can find equations for all $j=1,\ldots,l$:
    \[
        g_j=\sum_{i=1}^k a_{ij}f_i  
    \]
    for some $a_{ij}\in \mathcal{O}_X(U)$ with $i=1,\ldots,k$, $j=1,\ldots,l$. This implies that $W(\mathcal{I}|_U)\subseteq A\cap U$. The reverse inclusion is clear.
\end{proof}



\section{Lasker--Noether decomposition}
\begin{definition}
    Let $X$ be a complex analytic space.
    An analytic set $A$ in $X$ is \emph{irreducible} at $a\in A$ if $\mathcal{J}_{A,a}$ is a prime ideal in $\mathcal{O}_{X,a}$.
\end{definition}

\begin{definition}
    Let $X$ be a complex analytic space, $A$ be an analytic set in $X$ and $a\in A$. A \emph{local decomposition} of $A$ at $a$ consists of an open neighbourhood $U$ of $a$ in $X$ and finitely many analytic sets $A_1,\ldots,A_s$ in $U$ such that
    \begin{enumerate}
        \item 
        \[
            A\cap U=A_1\cup\cdots\cup A_s;  
        \]
        \item $A_i$ is irreducible at $a$ for $i=1,\ldots,s$;
        \item for any open neighbourhood $V$ of $a$ in $U$, $A_j\cap V\not\subset A_k\cap V$ for $j,k=1,\ldots,s$, $j\neq k$.
    \end{enumerate}
    We also say $A_1\cup\cdots \cup A_s$ is a \emph{local decomposition} of $A\cap U$.
\end{definition}

\begin{proposition}\label{prop-primedecunique}
    Let $X$ be a complex analytic space, $A$ be an analytic set in $X$ and $a\in A$. 
    Let 
    \[
        \mathcal{J}_{A,a}=\bigcap_{j=1}^s \mathfrak{p}_j
    \]
    be the Lasker--Noether decomposition. Then there is a local decompose of $A$ at $a$:
    \[
        A\cap U=A_1  \cup\cdots\cup A_s
    \] 
    with $\mathcal{J}_{A_j,a}=\mathfrak{p}_j$ for $j=1,\ldots,s$.

    Let $A\cap U'=A_1'\cup\cdots\cup A_r'$ be another local decomposition of $A$ at $a$. Then $r=s$ and we can find an open neighbourhood $W\subseteq U\cap U'$ and a bijection $\sigma:\{1,\ldots,s\}\rightarrow \{1,\ldots,s\}$ such that 
    \[
        A_j'\cap W=A_{\sigma(j)}\cap W  
    \]
    for $j=1,\ldots,s$.
\end{proposition}

\begin{proof}
    We first prove the existence part. Take an open neighbourhood $U$ of $a$ in $X$ and coherent ideal sheaves $\mathcal{I}_1,\ldots,\mathcal{I}_s$ on $U$ such that 
    \[
        \mathcal{I}_{j,a}=\mathfrak{p}_j  
    \]
    for $j=1,\ldots,s$. Define
    \[
        \mathcal{I}=\bigcap_{j=1}^s\mathcal{I}_{j}.
    \]
    Then $\mathcal{I}_a=\mathcal{J}_{A,a}$. By \cref{lma-analyticsetlocallyidealsheaf}, up to shrinking $U$, we may guarantee that 
    \[
        W(\mathcal{I})=A\cap U.  
    \]
    We set $A_j=W(\mathcal{I}_j)$ for $j=1,\ldots,s$. Then $A_j$ is an analytic set in $U$ and 
    \[
        A\cap U=W(\mathcal{I})=\bigcup_{j=1}^s W(\mathcal{I}_j)=A_1\cup\cdots\cup A_s. 
    \]
    Observe that $\mathfrak{p}_j=\mathcal{I}_{j,a}\subseteq \mathcal{J}_{A_j,a}$ for all $j=1,\ldots,s$. We need to prove the reverse inclusion. Assume that this is not true, say it fails for $j=1$.
    Then there is $g_1\in \mathcal{J}_{A_1,a}\setminus \mathfrak{p}_1$. As $\mathfrak{p}_j\not\subset \mathfrak{p}_1$ for $j=2,\ldots,s$, we can find $g_j\in \mathfrak{p}_j\setminus \mathfrak{p}_1$ for $j=2,\ldots,s$. Then
    \[
        g_1\cdots g_s\in \mathcal{J}_{A_1,a}\cap  \cdots \cap \mathcal{J}_{A_s,a}=\mathcal{J}_{A,a}\subseteq \mathfrak{p}_1. 
    \]
    This is a contradiction. So $\mathcal{J}_{A_j,a}=\mathfrak{p}_j$ for $j=1,\ldots,s$. We conclude that $A\cap U=A_1  \cup\cdots\cup A_s$ is a local decomposition by \cref{lma-germidealanalyticsetcont}.

    Next we prove the uniqueness statement. We take $U'$ and $A_1',\ldots,A_r'$ as in the statement of the theorem. Then
    \[
        \mathcal{J}_{A,a}=\mathcal{J}_{A'_1,a}\cap \cdots \cap   \mathcal{J}_{A'_r,a}.
    \]
    By \cref{lma-germidealanalyticsetcont}, we find that this is the Lasker--Noether decomposition of $\mathcal{J}_{A,a}$. The uniqueness follows from the uniqueness of Lasker--Noether decomposition and \cref{lma-germidealanalyticsetcont}.
\end{proof}

\begin{definition}
    Let $X$ be a complex analytic space, $A$ be an analytic set in $X$ and $a\in A$. Let 
    \[
        A\cap U=A_1  \cup\cdots\cup A_s
    \] 
    be a local decomposition of $A$ at $a$. We call $A_{1,a},\ldots,A_{s,a}$ the \emph{prime components} of $A$ at $a$. 
\end{definition}
By \cref{prop-primedecunique}, the prime compoments are uniquely determined by the germ of $X$ at $x$.

\begin{lemma}\label{lma-localprimeavoidance}
    Let $X$ be a complex analytic space, $A$ be an analytic set in $X$ and $a\in A$. Let $A_1,\ldots,A_s$ be the prime components of $A$ at $a$. Then $A_1$ is not contained in $A_{2}\cup\cdots\cup A_s$.
\end{lemma}
\begin{proof}
    If not, we have 
    \[
        \mathcal{J}_{A_1,a}\supseteq  \bigcap_{j=2}^s \mathcal{J}_{A_j,a}.
    \]
    So
    \[
        \mathcal{J}_{A,a}=  \bigcap_{j=2}^s \mathcal{J}_{A_j,a}.
    \]
    This contradicts the uniqueness of the Lasker--Noether decomposition.
\end{proof}

\begin{proposition}
    Let $X$ be a complex analytic space, $A$ be an analytic set in $X$ and $a\in A$. The following are equivalent:
    \begin{enumerate}
        \item $A$ is not irreducible at $a$;
        \item there is an open neighbourhood $U$ of $a$ in $X$ and a decomposition
            \[
                A\cap U=A'\cup A'',    
            \]
            where $A'$ and $A''$ are analytic sets in $U$ such that $A'_a\neq A_a$ and $A''_a\neq A_a$.
    \end{enumerate}
\end{proposition}
\begin{proof}
    (1) $\implies$ (2): Let $A_{1,x},\ldots,A_{s,x}$ be the prime components of $A$ at $a$. Then $s\geq 2$.
    Take an open neighbourhood $U$ of $a$ in $X$ such that $A_{1,x},\ldots,A_{s,x}$ lifts to analytic subsets $A_1,\ldots,A_s$ of $U$. It suffices to let $A'=A_1$ and $A''=A_2\cup\cdots\cup A_s$. By \cref{lma-localprimeavoidance}, $A'$ and $A''$ satisfies the conditions in (2).

    (2) $\implies$ (1): We have $\mathcal{J}_{A,a}\neq \mathcal{J}_{A',a}$ and $\mathcal{J}_{A,a}\neq \mathcal{J}_{A'',a}$. Take $f\in \mathcal{J}_{A',a}\setminus \mathcal{J}_{A,a}$ and $g\in \mathcal{J}_{A'',a}\setminus \mathcal{J}_{A,a}$. Then $fg\in \mathcal{J}_{A',a}\cap \mathcal{J}_{A'',a}=\mathcal{J}_{A,a}$. So $\mathcal{J}_{A,a}$ is not a prime ideal.
\end{proof}


\printbibliography
\end{document}