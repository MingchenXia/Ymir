
\documentclass{amsbook} 
%\usepackage{xr}
\usepackage{xr-hyper}
\usepackage[unicode]{hyperref}


\usepackage[T1]{fontenc}
\usepackage[utf8]{inputenc}
\usepackage{lmodern}
\usepackage{amssymb,tikz-cd}
%\usepackage{natbib}
\usepackage[english]{babel}

\usepackage[nameinlink,capitalize]{cleveref}
\usepackage[style=alphabetic,maxnames=99,maxalphanames=5, isbn=false, giveninits=true, doi=false]{biblatex}
\usepackage{lipsum, physics}
\usepackage{ifthen}
\usepackage{microtype}
\usepackage{booktabs}
\usetikzlibrary{calc}
\usepackage{emptypage}
\usepackage{setspace}
\usepackage[margin=0.75cm, font={small,stretch=0.80}]{caption}
\usepackage{subcaption}
\usepackage{url}
\usepackage{bookmark}
\usepackage{graphicx}
\usepackage{dsfont}
\usepackage{enumitem}
\usepackage{mathtools}
\usepackage{csquotes}
\usepackage{silence}
\usepackage{mathrsfs}
\usepackage{bigints}

\WarningFilter{biblatex}{Patching footnotes failed}


\ProcessOptions\relax

\emergencystretch=1em

\hypersetup{
colorlinks=true,
linktoc=all
}

\setcounter{tocdepth}{1}


\hyphenation{archi-medean  Archi-medean Tru-ding-er}

%\captionsetup[table]{position=bottom}   %% or below
\renewcommand{\thefootnote}{\fnsymbol{footnote}}
%\DeclareMathAlphabet{\mathcal}{OMS}{cmsy}{m}{n}
\renewbibmacro{in:}{}

\DeclareFieldFormat[article]{citetitle}{#1}
\DeclareFieldFormat[article]{title}{#1}
\DeclareFieldFormat[inbook]{citetitle}{#1}
\DeclareFieldFormat[inbook]{title}{#1}
\DeclareFieldFormat[incollection]{citetitle}{#1}
\DeclareFieldFormat[incollection]{title}{#1}
\DeclareFieldFormat[inproceedings]{citetitle}{#1}
\DeclareFieldFormat[inproceedings]{title}{#1}
\DeclareFieldFormat[phdthesis]{citetitle}{#1}
\DeclareFieldFormat[phdthesis]{title}{#1}
\DeclareFieldFormat[misc]{citetitle}{#1}
\DeclareFieldFormat[misc]{title}{#1}
\DeclareFieldFormat[book]{citetitle}{#1}
\DeclareFieldFormat[book]{title}{#1} 


%% Define various environments.

\theoremstyle{definition}
\newtheorem{theorem}{Theorem}[section]
\newtheorem{thm}[theorem]{Theorem}
\newtheorem{proposition}[theorem]{Proposition}
\newtheorem{corollary}[theorem]{Corollary}
\newtheorem{lemma}[theorem]{Lemma}
\newtheorem{conjecture}[theorem]{Conjecture}
\newtheorem{question}[theorem]{Question}
\newtheorem{example}[theorem]{Example}
\newtheorem{definition}[theorem]{Definition}
\newtheorem{condition}[theorem]{Condition}

\theoremstyle{remark}
\newtheorem{remark}[theorem]{Remark}
\numberwithin{equation}{section}

%\renewcommand{\thesection}{\thechapter.\arabic{section}}
%\renewcommand{\thetheorem}{\thesection.\arabic{theorem}}
%\renewcommand{\thedefinition}{\thesection.\arabic{definition}}
%\renewcommand{\theremark}{\thesection.\arabic{remark}}


%% Define new operators

\DeclareMathOperator{\nd}{nd}
\DeclareMathOperator{\ord}{ord}
\DeclareMathOperator{\Hom}{Hom}
\DeclareMathOperator{\PreSh}{PreSh}
\DeclareMathOperator{\Gr}{Gr}
\DeclareMathOperator{\Homint}{\mathcal{H}\mathrm{om}}
\DeclareMathOperator{\Torint}{\mathcal{T}\mathrm{or}}
\DeclareMathOperator{\Div}{div}
\DeclareMathOperator{\DSP}{DSP}
\DeclareMathOperator{\Diff}{Diff}
\DeclareMathOperator{\MA}{MA}
\DeclareMathOperator{\NA}{NA}
\DeclareMathOperator{\AN}{an}
\DeclareMathOperator{\Rep}{Rep}
\DeclareMathOperator{\Rest}{Res}
\DeclareMathOperator{\DF}{DF}
\DeclareMathOperator{\VCart}{VCart}
\DeclareMathOperator{\PL}{PL}
\DeclareMathOperator{\Bl}{Bl}
\DeclareMathOperator{\Td}{Td}
\DeclareMathOperator{\Fitt}{Fitt}
\DeclareMathOperator{\Ric}{Ric}
\DeclareMathOperator{\coeff}{coeff}
\DeclareMathOperator{\Aut}{Aut}
\DeclareMathOperator{\Capa}{Cap}
\DeclareMathOperator{\loc}{loc}
\DeclareMathOperator{\vol}{vol}
\DeclareMathOperator{\Val}{Val}
\DeclareMathOperator{\ST}{ST}
\DeclareMathOperator{\Amp}{Amp}
\DeclareMathOperator{\Herm}{Herm}
\DeclareMathOperator{\trop}{trop}
\DeclareMathOperator{\Trop}{Trop}
\DeclareMathOperator{\Cano}{Can}
\DeclareMathOperator{\PS}{PS}
\DeclareMathOperator{\Var}{Var}
\DeclareMathOperator{\Psef}{Psef}
\DeclareMathOperator{\Jac}{Jac}
\DeclareMathOperator{\Char}{char}
\DeclareMathOperator{\Red}{red}
\DeclareMathOperator{\Spf}{Spf}
\DeclareMathOperator{\Span}{Span}
\DeclareMathOperator{\Der}{Der}
%\DeclareMathOperator{\Mod}{mod}
\DeclareMathOperator{\Hilb}{Hilb}
\DeclareMathOperator{\triv}{triv}
\DeclareMathOperator{\Frac}{Frac}
\DeclareMathOperator{\diam}{diam}
\DeclareMathOperator{\Spec}{Spec}
\DeclareMathOperator{\Spm}{Spm}
\DeclareMathOperator{\Specrel}{\underline{Sp}}
\DeclareMathOperator{\Sp}{Sp}
\DeclareMathOperator{\reg}{reg}
\DeclareMathOperator{\sing}{sing}
\DeclareMathOperator{\Star}{Star}
\DeclareMathOperator{\relint}{relint}
\DeclareMathOperator{\Cvx}{Cvx}
\DeclareMathOperator{\Int}{Int}
\DeclareMathOperator{\Supp}{Supp}
\DeclareMathOperator{\FS}{FS}
\DeclareMathOperator{\RZ}{RZ}
\DeclareMathOperator{\Redu}{red}
\DeclareMathOperator{\lct}{lct}
\DeclareMathOperator{\Proj}{Proj}
\DeclareMathOperator{\Sing}{Sing}
\DeclareMathOperator{\Conv}{Conv}
\DeclareMathOperator{\Max}{Max}
\DeclareMathOperator{\Tor}{Tor}
\DeclareMathOperator{\Gal}{Gal}
\DeclareMathOperator{\Frob}{Frob}
\DeclareMathOperator{\coker}{coker}
\DeclareMathOperator{\Sym}{Sym}
\DeclareMathOperator{\CSp}{CSp}
\DeclareMathOperator{\Img}{Im}


\newcommand{\alg}{\mathrm{alg}}
\newcommand{\Sh}{\mathrm{Sh}}
\newcommand{\fin}{\mathrm{fin}}
\newcommand{\BPF}{\mathrm{BPF}}
\newcommand{\dBPF}{\mathrm{dBPF}}
\newcommand{\divf}{\mathrm{Div}^f}
\newcommand{\nef}{\mathrm{nef}}
\newcommand{\Bir}{\mathrm{Bir}}
\newcommand{\hO}{\hat{\mathcal{O}}}
\newcommand{\bDiv}{\mathrm{Div}^{\mathrm{b}}}
\newcommand{\un}{\mathrm{un}}
\newcommand{\sep}{\mathrm{sep}}
\newcommand{\diag}{\mathrm{diag}}
\newcommand{\Pic}{\mathrm{Pic}}
\newcommand{\GL}{\mathrm{GL}}
\newcommand{\SL}{\mathrm{SL}}
\newcommand{\LS}{\mathrm{LS}}
\newcommand{\GLS}{\mathrm{GLS}}
\newcommand{\GLSi}{\mathrm{GLS}_{\cap}}
\newcommand{\PGLS}{\mathrm{PGLS}}
\newcommand{\Loc}[1][S]{_{\{{#1}\}}}
\newcommand{\cl}{\mathrm{cl}}
\newcommand{\otL}{\hat{\otimes}^{\mathbb{L}}}
\newcommand{\ddpp}{\mathrm{d}'\mathrm{d}''}
\newcommand{\TC}{\mathcal{TC}}
\newcommand{\ddPP}{\mathrm{d}'_{\mathrm{P}}\mathrm{d}''_{\mathrm{P}}}
\newcommand{\PSs}{\mathcal{PS}}
\newcommand{\Gm}{\mathbb{G}_{\mathrm{m}}}
\newcommand{\End}{\mathrm{End}}
\newcommand{\Aff}[1][X]{\mathcal{M}\left(\mathcal{#1}\right)}
\newcommand{\XG}[1][X]{{#1}_{\mathrm{G}}}
\newcommand{\convC}{\xrightarrow{C}}
\newcommand{\Vect}{\mathrm{Vect}}
\newcommand{\abso}[1]{\lvert#1\rvert}
\newcommand{\Mdl}{\mathrm{Model}}
\newcommand{\cn}{\stackrel{\sim}{\longrightarrow}}
\newcommand{\sbc}{\mathbf{s}}
\newcommand{\CH}{\mathrm{CH}}
\newcommand{\GR}{\mathrm{GR}}
\newcommand{\dc}{\mathrm{d}^{\mathrm{c}}}
\newcommand{\Nef}{\mathrm{Nef}}
\newcommand{\Adj}{\mathrm{Adj}}
\newcommand{\DHm}{\mathrm{DH}}
\newcommand{\An}{\mathrm{an}}
\newcommand{\Rec}{\mathrm{Rec}}
\newcommand{\dP}{\mathrm{d}_{\mathrm{P}}}
\newcommand{\ddp}{\mathrm{d}_{\mathrm{P}}'\mathrm{d}_{\mathrm{P}}''}
\newcommand{\ddc}{\mathrm{dd}^{\mathrm{c}}}
\newcommand{\ddL}{\mathrm{d}'\mathrm{d}''}
\newcommand{\PSH}{\mathrm{PSH}}
\newcommand{\CPSH}{\mathrm{CPSH}}
\newcommand{\PSP}{\mathrm{PSP}}
\newcommand{\WPSH}{\mathrm{WPSH}}
\newcommand{\Ent}{\mathrm{Ent}}
\newcommand{\NS}{\mathrm{NS}}
\newcommand{\QPSH}{\mathrm{QPSH}}
\newcommand{\proet}{\mathrm{pro-ét}}
\newcommand{\XL}{(\mathcal{X},\mathcal{L})}
\newcommand{\ii}{\mathrm{i}}
\newcommand{\Cpt}{\mathrm{Cpt}}
\newcommand{\bp}{\bar{\partial}}
\newcommand{\ddt}{\frac{\mathrm{d}}{\mathrm{d}t}}
\newcommand{\dds}{\frac{\mathrm{d}}{\mathrm{d}s}}
\newcommand{\Ep}{\mathcal{E}^p(X,\theta;[\phi])}
\newcommand{\Ei}{\mathcal{E}^{\infty}(X,\theta;[\phi])}
\newcommand{\infs}{\operatorname*{inf\vphantom{p}}}
\newcommand{\sups}{\operatorname*{sup*}}
\newcommand{\colim}{\operatorname*{colim}}
\newcommand{\ddtz}[1][0]{\left.\ddt\right|_{t={#1}}}
\newcommand{\tube}[1][Y]{]{#1}[}
\newcommand{\ddsz}[1][0]{\left.\ddt\right|_{s={#1}}}
\newcommand{\floor}[1]{\left \lfloor{#1}\right \rfloor }
\newcommand{\dec}[1]{\left \{{#1}\right \} }
\newcommand{\ceil}[1]{\left \lceil{#1}\right \rceil }
\newcommand{\Projrel}{\mathcal{P}\mathrm{roj}}
\newcommand{\Weil}{\mathrm{Weil}}
\newcommand{\Cart}{\mathrm{Cart}}
\newcommand{\bWeil}{\mathrm{b}\mathrm{Weil}}
\newcommand{\bCart}{\mathrm{b}\mathrm{Cart}}
\newcommand{\Cond}{\mathrm{Cond}}
\newcommand{\IC}{\mathrm{IC}}
\newcommand{\IH}{\mathrm{IH}}
\newcommand{\cris}{\mathrm{cris}}
\newcommand{\Zar}{\mathrm{Zar}}
\newcommand{\HvbCat}{\overline{\mathcal{V}\mathrm{ect}}}
\newcommand{\BanModCat}{\mathcal{B}\mathrm{an}\mathcal{M}\mathrm{od}}
\newcommand{\DesCat}{\mathcal{D}\mathrm{es}}
\newcommand{\RingCat}{\mathcal{R}\mathrm{ing}}
\newcommand{\SchCat}{\mathcal{S}\mathrm{ch}}
\newcommand{\AbCat}{\mathcal{A}\mathrm{b}}
\newcommand{\RSCat}{\mathcal{R}\mathrm{S}}
\newcommand{\LRSCat}{\mathcal{L}\mathrm{RS}}
\newcommand{\CLRSCat}{\mathbb{C}\text{-}\LRSCat}
\newcommand{\CRSCat}{\mathbb{C}\text{-}\RSCat}
\newcommand{\CLA}{\mathbb{C}\text{-}\mathcal{L}\mathrm{A}}
\newcommand{\CASCat}{\mathbb{C}\text{-}\mathcal{A}\mathrm{n}}
\newcommand{\LiuCat}{\mathcal{L}\mathrm{iu}}
\newcommand{\BanCat}{\mathcal{B}\mathrm{an}}
\newcommand{\BanAlgCat}{\mathcal{B}\mathrm{an}\mathcal{A}\mathrm{lg}}
\newcommand{\AnaCat}{\mathcal{A}\mathrm{n}}
\newcommand{\LiuAlgCat}{\mathcal{L}\mathrm{iu}\mathcal{A}\mathrm{lg}}
\newcommand{\AlgCat}{\mathcal{A}\mathrm{lg}}
\newcommand{\SetCat}{\mathcal{S}\mathrm{et}}
\newcommand{\ModCat}{\mathcal{M}\mathrm{od}}
\newcommand{\TopCat}{\mathcal{T}\mathrm{op}}
\newcommand{\CohCat}{\mathcal{C}\mathrm{oh}}
\newcommand{\SolCat}{\mathcal{S}\mathrm{olid}}
\newcommand{\AffCat}{\mathcal{A}\mathrm{ff}}
\newcommand{\AffAlgCat}{\mathcal{A}\mathrm{ff}\mathcal{A}\mathrm{lg}}
\newcommand{\QcohLiuAlgCat}{\mathcal{L}\mathrm{iu}\mathcal{A}\mathrm{lg}^{\mathrm{QCoh}}}
\newcommand{\LiuMorCat}{\mathcal{L}\mathrm{iu}}
\newcommand{\Isom}{\mathcal{I}\mathrm{som}}
\newcommand{\Cris}{\mathcal{C}\mathrm{ris}}
\newcommand{\Pro}{\mathrm{Pro}-}
\newcommand{\Fin}{\mathcal{F}\mathrm{in}}
\newcommand{\norms}[1]{\left\|#1\right\|}
\newcommand{\HPDDiff}{\mathbf{D}\mathrm{iff}}
\newcommand{\Menn}[2]{\begin{bmatrix}#1\\#2\end{bmatrix}}
\newcommand{\Fins}{\widehat{\Vect}^F}
\newcommand\blfootnote[1]{%
  \begingroup
  \renewcommand\thefootnote{}\footnote{#1}%
  \addtocounter{footnote}{-1}%
  \endgroup
}

\externaldocument[Introduction-]{Introduction}
%One variable complex analysis
%Several variables complex analysis
\externaldocument[Topology-]{Topology-Bornology}
\externaldocument[Banach-]{Banach-Rings}
\externaldocument[Commutative-]{Commutative-Algebra}
\externaldocument[Local-]{Local-Algebras}
\externaldocument[Complex-]{Complex-Analytic-Spaces}
%Properties of space
\externaldocument[Morphisms-]{Morphisms}
%Differential calculus
%GAGA
%Hilbert scheme complex analytic version

%Complex differential geometry

\externaldocument[Affinoid-]{Affinoid-Algebras}
\externaldocument[Berkovich-]{Berkovich-Analytic-Spaces}


\bibliography{Ymir}

\endinput
\title{Ymir}
\begin{document}
\maketitle
\tableofcontents

\chapter*{Global properties of complex analytic spaces}\label{chap-CGlobalProperty}

\section{Introduction}\label{sec-introduction-CGlobalProperty}

\section{Holomorphically convex hulls}

\begin{definition}
    Let $X$ be a complex analytic space and $M$ be a subset of $X$, we define the \emph{holomorphically convex hull} of $M$ in $X$ as
    \[
        \hat{M}^X:=  \left\{ x\in X: |f(x)|\leq \sup_{y\in M}|f(y)|\text{ for all }f\in \mathcal{O}_X(X) \right\}.  
    \]
\end{definition}

\begin{proposition}\label{prop-convexhullprop}
    Let $X$ be a complex analytic space and $M$ be a subset of $X$. Then the following properties hold:
    \begin{enumerate}
        \item $\hat{M}^X$ is closed in $X$;
        \item $M\subseteq \hat{M}^X$ and $\widehat{\hat{M}^X}^X=\hat{M}^X$;
        \item If $M'$ is another subset of $X$ containing $M$, then $\hat{M}^X\subseteq \hat{M'}^X$;
        \item If $f:Y\rightarrow X$ is a morphism of complex analytic spaces, then
             \[ 
                \widehat{f^{-1}(M)}^Y\subseteq f^{-1}(\hat{M}^X);
             \]
        \item If $X'$ is another complex analytic space and $M'$ is a subset of $X'$, then
            \[
                \widehat{M\times M'}^{X\times X'}\subseteq \hat{M}^X\times \hat{M'}^{X'};  
            \]
        \item If $M'$ is another subset of $X$ and $\hat{M}^X=M$, $\hat{M'}^X=M'$, then
            \[
                \widehat{M\cap M'}^X=M\cap M'.    
            \]
    \end{enumerate}
\end{proposition}
\begin{proof}
    (1), (2), (3), (4), (5) are obvious by definition.

    (6) is a consequence of (3).
\end{proof}

\begin{example}\label{ex-tubeholhull}
    Let $Q$ be a compact cube in $\mathbb{C}^n$ for some $n\in \mathbb{N}$, then $\hat{Q}^{\mathbb{C}^n}=Q$.

    In fact, by \cref{prop-convexhullprop}(5), we may assume that $n=1$. Given $p\in \mathbb{C}\setminus Q$, we can take a closed disk $T\subseteq \mathbb{C}$ centered at $a\in \mathbb{C}$ such that $Q\subseteq T$ while $p\not\in T$. Consider $z-a\in \mathcal{O}_{\mathbb{C}}(\mathbb{C})$, then
    \[
        |f(p)|>\sup_{q\in Q}|f(q)|.  
    \]
    So $p\not\in \hat{Q}^{\mathbb{C}}$.
\end{example}

\section{Stones}

\begin{definition}\label{def-stone}
    Let $X$ be a complex analytic space. A \emph{stone} in $X$ is a pair $(P,\pi)$ consisting of 
    \begin{enumerate}
        \item a non-empty compact set $P$ in $X$ and
        \item a morphism $\pi:X\rightarrow \mathbb{C}^n$ for some $n \in \mathbb{N}$
    \end{enumerate}
    such that there is a compact tube $Q$ in $\mathbb{C}^n$ and an open set $W$ in $X$ such that $P=\pi^{-1}(Q)\cap W$.

    We call $P^0:=\pi^{-1}(\Int Q)\cap W$ the \emph{analytic interior} of the stone $(P,\pi)$. It clearly does not depend on the choice of $W$.
\end{definition}
We observe that $\hat{P}^X\cap W=P$. In fact, $P\subseteq \pi^{-1}(Q)$, so
\[
    \hat{P}^X\subseteq \pi^{-1}(\hat{Q}^{\mathbb{C}^n})=\pi^{-1}(Q)=P\cap W=P.  
\]
Here we applied \cref{prop-convexhullprop} and \cref{ex-tubeholhull}.

In general, $P^0\subseteq \Int P$, but they can be different.

\begin{thm}\label{thm-fixedcptsethullloccpt}
    Let $X$ be a Hausdorff complex analytic space and $K\subseteq X$ be a compact subset. Then the following are equivalent:
    \begin{enumerate}
        \item There is an open neighbourhood $W$ of $K$ in $X$ such that $\hat{K}^X\cap W$ is compact;
        \item There is an open relative compact neighbourhood $W$ of $K$ in $X$ such that $\partial W\cap \hat{K}=\emptyset$;
        \item There is a stone $(P,\pi)$ in $X$ with $K\subseteq P^0$.
    \end{enumerate}
\end{thm}
\begin{proof}
    (1) $\implies$ (2): This is trivial, in fact, we may assume that $W$ in (1) is relatively compact in $X$.

    (2) $\implies$ (3): As $\hat{K}^X$ is closed by \cref{prop-convexhullprop}(1) and $\partial W\cap \hat{K}^X=\emptyset$, given $p\in \partial W$, we can find $h\in \mathcal{O}_X(X)$ such that
    \[
        \sup_{x\in K}|h(x)|<1<|h(p)|.
    \]
    We will denote the left-hand side by $|h|_K$.
    Up to raising $h$ to a power, we may assume that 
    \[
        \max\{|\Real h(p)|,|\Imag h(p)| \}>1. 
    \]
    As $\partial W$ is compact, we can find finitely many sections $h_1,\ldots,h_m\in \mathcal{O}_X(X)$ so that
    \[
        \max_{j=1,\ldots,m} \{|\Real h_j|_K,|\Imag h_j|_K\}<1,\quad \max_{j=1,\ldots,m} \{|\Real h_j(p)|,|\Imag h_j(p)|\}>1.  
    \]
    Let 
    \[
        Q:=\left\{ (z_1,\ldots,z_m)\in \mathbb{C}^m: |\Real z_i|\leq 1, |\Imag z_i|\leq 1\text{ for all }i=1,\ldots,m \right\}.  
    \]
    The sections $h_1,\ldots,h_m$ defines a homomorphism $\pi:X\rightarrow \mathbb{C}^m$ by \cref{Complex-thm-hCnidentification} in \nameref{Complex-chap-complex}. Obviously, $P=\pi^{-1}(Q)\cap W$ satisfies our assumptions.

    (3) $\implies$ (1): Let $W$ be the open set as in \cref{def-stone}. As $\hat{P}^X\cap W=P$ and $K\subseteq P$, we have
    \[
        \hat{K}\cap W\subseteq P\cap W=P.  
    \]
    As $P$ is compact, so is $\hat{K}\cap W$.
\end{proof}

\begin{thm}\label{thm-stonelocallyproper}
    Let $X$ be a Hausdorff complex analytic space and $(P,\pi:X\rightarrow \mathbb{C}^n)$ be a stone in $X$. Let $Q$ be the tube in $\mathbb{C}^m$ as in \cref{def-stone}. Then there are open neighbourhoods $U$ and $V$ of $P$ and $Q$ in $X$ and $\mathbb{C}^n$ respectively with $\pi(U)\subseteq V$ and $P=\pi^{-1}(Q)\cap U$ such that $\pi|_U:U\rightarrow V$ is proper.
\end{thm}
%Recall that a map between topological spaces is quasi-proper if the inverse image of each quasi-compact set is quasi-compact.

\begin{proof}
    Let $W\subseteq X$ be the open set as in \cref{def-stone}. We may assume that $W$ is relatively compact. Then $\partial W$ and $\pi(\partial W)$ are also compact. As $\partial W\cap \pi^{-1}(Q)$ is empty, we know that $V:=\mathbb{C}^n\setminus \pi(\partial W)$ is an open neighbourhood of $Q$. The set $U:=W\cap \pi^{-1}(V)=W\setminus \pi^{-1}(\pi(\partial W))$ is open in $X$ and $\pi(U)\subseteq V$. Observe that $\pi|_U:U\rightarrow V$ is proper by \cref{Topology-lma-qcpttoHausdorffproper} in \nameref{Topology-chap-topology}.
    
    Furthermore,
    \[
        \pi^{-1}(Q)\cap U=\pi^{-1}(Q)\cap \left( W\setminus \left(\pi^{-1}(Q)\cap \pi^{-1}\pi(\partial W)\right) \right).  
    \]
    But $\pi^{-1}Q\cap \pi^{-1}\pi(\partial W)$ is empty as $Q\cap \pi(\partial W)$ is. It follows that $\pi^{-1}(Q)\cap U=P$ and hence $U$ is a neighbourhood of $P$.
\end{proof}

\begin{definition}
    Let $X$ be a complex analytic space.
    Let $(P,\pi:X\rightarrow \mathbb{C}^{n})$, $(P',\pi':X\rightarrow \mathbb{C}^{n'})$ be two stones on $X$. We say $(P,\pi)$ \emph{is contained in} $(P',\pi')$ if the following conditions are satisfied:
    \begin{enumerate}
        \item $P$ lies in the analytic interior of $P'$;
        \item $n'\geq n$ and there is $q\in \mathbb{C}^{n'-n}$ such that if $Q\subseteq \mathbb{C}^n$, $\mathbb{Q}'\subseteq \mathbb{C}^{n'}$ be the tubes as in \cref{def-stone}, then
            \[
                Q\times\{q\}\subseteq Q'.
            \]
        \item There is a morphism $\varphi:X\rightarrow \mathbb{C}^{n'-n}$ such that 
            \[
                \pi'=(\pi,\varphi).
            \]
    \end{enumerate}
    We formally write $(P,\pi)\subseteq (P',\pi')$ in this case. Clearly, this defines a partial order on the set of stones on $X$.
\end{definition}

\begin{definition}
    Let $X$ be a complex analytic space. An \emph{exhaustion of $X$ by stones} is a sequence $(P_i,\pi_i)_{i\in \mathbb{Z}_{>0}}$ of stones such that 
    \begin{enumerate}
        \item $(P_i,\pi_i)\subseteq (P_{i+1},\pi_{i+1})$ for all $i\in \mathbb{Z}_{>0}$;
        \item 
            \[
                X=\bigcup_{i=1}^{\infty} P_i^0.
            \]
    \end{enumerate}
    We say $X$ is \emph{weakly holomorphically convex} if it there is an exhaustion of $X$ by stones.
\end{definition}

\begin{thm}\label{thm-weaklyholconvchar}
    Let $X$ be a Hausdorff complex analytic space. Consider the following conditions:
    \begin{enumerate}
        \item $X$ is weakly holomorphically convex;
        \item For any compact subset $K\subseteq X$, there is an open set $W\subseteq X$ such that $\hat{K}^X\cap W$ is compact.
    \end{enumerate}
    Then (1) $\implies$ (2). If $X$ is Lindelöf, then (2) $\implies$ (1).
\end{thm}
\begin{proof}
    (1) $\implies$ (2): It suffices to observe that $K\subseteq P_j^0$ when $j$ is large enough and apply \cref{thm-fixedcptsethullloccpt}.

    Assume that $X$ is Lindelöf.
    (2) $\implies$ (1): 
    Let $(K_i)$ a compact exhaustion of $X$. We construcct the stones $(P_i,\pi_i)_{i\in \mathbb{Z}_{>0}}$ so that 
    \[
        K_i\subseteq P_i^0
    \] 
    for all $i\in \mathbb{Z}_{>0}$
    inductively. Let $P_1$ be an arbitrary stone in $X$ such that $K_1\subseteq P_1^0$. The existence of $P_1$ is guaranteed by \cref{thm-fixedcptsethullloccpt}.

    Assume that we have constructed $(P_{i-1},\pi_{i-1}:X\rightarrow  \mathbb{C}^{n_{i-1}})$ for $i\geq 2$. Let $Q_{i-1}\subseteq \mathbb{C}^{n_{i-1}}$ be the associated tube. By   \cref{thm-fixedcptsethullloccpt} again, take a stone $(P_i,\pi_i^*:X\rightarrow \mathbb{C}^n)$ with $K_i\cup P_{i-1}\subseteq P_i^0$. Let $Q_i^*\subseteq \mathbb{C}^n$ be the associated tube. Let $W$ be an open subset of $X$ with 
    \[
        P_i=\pi_i^{*,-1}(Q_i^*)\cap W.  
    \]
    Choose a tube $Q_i'\subseteq \mathbb{C}^{n_{i-1}}$ with $Q_{i-1}\subseteq \Int Q_i'$ so that 
    \[
        \pi_{i-1}(P_i)\subseteq \Int Q_i'.
    \]
    Let $\pi_i:=(\pi_{i-1},\pi_i^*):X\rightarrow \mathbb{C}^{n_{i-1}+n}$ and $Q_i:=Q_i'\times Q_i^*$. Then $(P_i,\pi_i)$ is a stone and $(P_{i-1},\pi_{i-1})\subseteq (P_i,\pi_i)$.
\end{proof}


\section{Holomorphical separabble spaces}
\begin{definition}
    Let $X$ be a complex analytic space. We say $X$ is \emph{holomorphically separable} if for any $x,y\in X$ with $x\neq y$, there is $f\in \mathcal{O}_X(X)$ with $f(x)\neq f(y)$.
\end{definition}
Here we regard $f$ as a continuous function $X\rightarrow \mathbb{C}$. In particular, a holomorphically separable space is Hausdorff.


\iffalse
\begin{thm}[Grauert]
    Let $X$ be a holomorphically spreadable complex analytic space. Then there is $n\in \mathbb{N}$ a quasi-finite morphism $f:X\rightarrow \mathbb{C}^n$.
\end{thm}
\begin{proof}
    
\end{proof}
\fi


\begin{definition}
    Let $X$ be a complex analytic space. We say $X$ is \emph{holomorphically convex} if $|X|$ is Hausdorff and for any compact set $K\subseteq X$, $\hat{K}^X$.

    We say $X$ is \emph{weakly holomorphically convex} if for any quasi-compact set $K\subseteq X$, the connected components of $\hat{K}^X$ are all quasi-compact.
\end{definition}

\begin{proposition}
    Let $X$ be a holomorphically convex complex analytic space. Then $X^{\Red}$ is holomorphically convex.
\end{proposition}
\begin{proof}
    This follows immediately from the definition.
\end{proof}

\begin{proposition}\label{prop-holoconvchar}
    Let $X$ be a Hausdorff complex analytic space. Consider the following conditions:
    \begin{enumerate}
        \item $X$ is holomorphically convex;
        \item For any sequence $x_i\in X$ ($i\in \mathbb{Z}_{>0}$) without accumulation points, there is $f\in \mathcal{O}_X(X)$ such that $|f(x_i)|$ is unbounded.
    \end{enumerate}
    Then (2) $\implies$ (1) if $X$ is Lindelöf.
\end{proposition}
\begin{proof}
    (2) $\implies$ (1): For a Lindelöf Hausdorff space, sequential compactness implies compactness. 

    %Recall that a complex analytic space is always first countable. For a first countable Hausdorff space, compactness implies sequential compactness.
\end{proof}

\begin{corollary}
    Let $n\in \mathbb{N}$ and $\Omega$ be a domain in $\mathbb{C}^n$. Assume that for each $p\in \partial \Omega$, there is a holomorphic function $f$ on an open neighbourhood $U$ of $\bar{\Omega}$ such that $f(p)=0$ and $f$ is non-zero on $\Omega$. Then $\Omega$ is holomorphically convex.
\end{corollary}
\begin{proof}
    Let $x_i\in \Omega$ ($i\in \mathbb{Z}_{>0}$) be a sequence without accumulation points in $\Omega$. We need to construct $f\in \mathcal{O}_{\Omega}(\Omega)$ such that $(|f(x_i)|)_{i\in \mathbb{Z}_{>0}}$ is unbounded. This is clear if $x_i$ itself is unbounded. Assume that $x_i$ is bounded. Then up to passing to a subsequence, we may assume that $x_i\to p\in \partial \Omega$ as $i\to\infty$. The inverse of the function $f$ in our assumption of the corollary works.
\end{proof}




\section{Stein sets}


\begin{definition}
    Let $X$ be a complex analytic space and $P$ be a closed subset of $X$. We say $P$ is a \emph{Stein set} in $X$ if for any coherent $\mathcal{O}_U$-module $\mathcal{F}$ for some open neighbourhood $U$ of $P$ in $X$, we have
    \[
        H^i(P,\mathcal{F})=0\quad \text{for all }i\in \mathbb{Z}_{>0}.  
    \]

    A \emph{coherent $\mathcal{O}_P$-module} is a coherent $\mathcal{O}_U$-module for some open neighbourhood $U$ of $P$ in $X$. Two coherent $\mathcal{O}_P$-modules are isomorphic if there is a small enough open neighbourhood $V$ of $P$ in $X$ such that they are isomorphic when restricted to $V$. In particular, $\mathcal{O}_P$ denotes the coherent $\mathcal{O}_P$-module defined by $\mathcal{O}_X$ on $X$.
    
    The germ-wise notions obviously make sense for coherent $\mathcal{O}_P$-modules.
\end{definition}

The given condition is usually known as \emph{Cartan's Theorem~B}. It implies \emph{Cartan's Theorem~A}:
\begin{thm}[Cartan's Theorem~A]\label{thm-CartanA}
    Let $X$ be a complex analytic space and $P$ be a Stein set in $X$. Let $\mathcal{F}$ be a coherent $\mathcal{O}_U$-module for some open neighbourhood $U$ of $P$ in $X$. Then $H^0(P,\mathcal{F})$ generates $\mathcal{F}_x$ for each $x\in P$.
\end{thm}
\begin{proof}
    Fix $x\in P$. Let $\mathcal{M}$ be the coherent ideal sheaf on $U$ consisting of holomorphic functions vanishing at $x$. Then $\mathcal{F}\mathcal{M}$ is a coherent $\mathcal{O}_U$-module. It follows from Theorem~B that
    \[
        H^0(P,\mathcal{F})\rightarrow H^0(P,\mathcal{F}/\mathcal{F}\mathcal{M})
    \]   
    is surjective. Note that we can identify this map with the natural map
    \[
        H^0(P,\mathcal{F})\rightarrow   \mathcal{F}_x/\mathfrak{m}_x \mathcal{F}_x.
    \]

    Let $e_1,\ldots,e_m$ be a basis of $\mathcal{F}_x/\mathfrak{m}_x \mathcal{F}_x$. Lift them to $s_1,\ldots,s_m\in H^0(P,\mathcal{F})$. By Nakayama's lemma, $s_{1x},\ldots,s_{mx}$ generate the $\mathcal{O}_{X,x}$-module $\mathcal{F}_x$.
\end{proof}


\begin{corollary}\label{cor-thmacohsheafquotient}
    Let $X$ be a complex analytic space and $P$ be a quasi-compact Stein set in $X$.  Let $\mathcal{F}$ be a coherent $\mathcal{O}_P$-module.
    Then there is $n\in \mathbb{Z}_{>0}$ and an epimorphism
    \[
        \mathcal{O}_P^n\rightarrow \mathcal{F}.  
    \]
\end{corollary}
\begin{proof}
    By \cref{thm-CartanA}, we can find an open covering $\{U_i\}_{i\in I}$ of $P$ such that there are homomorphisms
    \[
        h_i:\mathcal{O}_P^{n_i}\rightarrow \mathcal{F}  
    \]
    for some $n_i\in \mathbb{Z}_{>0}$, which is surjective on $U_i$ for each $i\in I$. By the quasi-compactness of $P$, we may assume that $I$ is a finite set. Then it suffices to set $n=\sum_{i\in I}n_i$ and consider the epimorphism $\mathcal{O}_P^n\rightarrow \mathcal{F}$ induced by the $h_i$'s.
\end{proof}

\begin{thm}\label{thm-blocktostein}
    Let $X$ be a complex analytic space and $P\subseteq X$ be a set with the following properties:
    \begin{enumerate}
        \item there is an open neighbourhood $U$ of $P$ in $X$, a domain $V$ in $\mathbb{C}^m$ for some $m\in \mathbb{N}$ and a finite holomorphic morphism $\tau:U\rightarrow V$;
        \item There exists a compact tube in $\mathbb{C}^m$ contained in $V$ such that $P=\tau^{-1}(Q)$.
    \end{enumerate}
    Then $P$ is a compact Stein set in $X$.
\end{thm}
\begin{proof}
    As $P=\tau^{-1}(Q)$ and $\tau$ is proper, we see that $P$ is compact.
    
    It remains to show that $P$ is a Stein set in $X$. Let $\mathcal{F}$ be a coherent $\mathcal{O}_P$-module.

    \textbf{Step~1}. We first reduce to the case where $\mathcal{F}$ is defined by a coherent $\mathcal{O}_U$-module.

    Take an open neighbourhood $U'$ of $P$ in $X$ contained in $U$ such that $\mathcal{F}$ is defined by a coherent $\mathcal{O}_{U'}$-module. By \cref{Topology-lma-opennhfiberclosedmap} in \nameref{Topology-chap-topology}, we can take an open neighbourhood $V'$ of $Q$ in $V$ such that $\tau^{-1}(V')\subseteq U'$. The restriction of $\tau$ to $\tau^{-1}(V')\rightarrow V'$ is again finite.

    \textbf{Step~2}. By Leray spectral sequence,
    \[
        H^i(P,\mathcal{F})\cong H^i(Q,(\tau|_P)_*\mathcal{F})
    \]
    for all $i\geq 0$. By \cref{Morphisms-cor-finitepushcoh}  in \nameref{Morphisms-chap-morphismscomplex}, $(\tau|_P)_*\mathcal{F}$ is a coherent $\mathcal{O}_Q$-module, so we are reduced to show that $Q$ is a Stein set in $\mathbb{C}^m$, which is well-known.
\end{proof}




\begin{definition}\label{def-Steinexhaus}
    Let $X$ be a Hausdorff complex analytic space and $\mathcal{F}$ be a coherent $\mathcal{O}_X$-module. 
    A \emph{Stein exhaustion of $X$ relative to $\mathcal{F}$} is a compact exhaustion $(P_i)_{i\in \mathbb{Z}_{>0}}$ such that the following conditions are satisfied:
    \begin{enumerate}
        \item  $P_i$ is a Stein set in $X$ for each $i\in \mathbb{Z}_{>0}$;
        \item the $\mathbb{C}$-vector space $H^0(P_i,\mathcal{F})$ admits a semi-norm $|\bullet|_i$ such that the restriction map 
        \[
            H^0(X,\mathcal{F})\rightarrow H^0(P_i,\mathcal{F})
        \] 
        has dense image with respect to the topological defined by $|\bullet|_i$ for each $i\in \mathbb{Z}_{>0}$;
        \item The restriction map
            \[
                H^0(P_{i+1},\mathcal{F})\rightarrow H^0(P_i,\mathcal{F})
            \]
            is bounded for each $i\in \mathbb{Z}_{>0}$;
        \item Let $i\in \mathbb{Z}_{\geq 2}$. Suppose that $(s_j)_{j\in \mathbb{Z}_{>0}}$ is a Cauchy sequence in $H^0(P_i,\mathcal{F})$, then the restricted sequence $s_j|_{P_{i-1}}$ has a limit in $H^0(P_{i-1},\mathcal{F})$;
        \item Let $i\in \mathbb{Z}_{\geq 2}$. If $s\in H^0(P_i,\mathcal{F})$ and $|s|_i=0$, then $s|_{P_{i-1}}=0$.
    \end{enumerate}

    A \emph{Stein exhaustion of $X$} is a compact exhaustion of $X$ that is a Stein exhaustion of $X$ relative to any coherent $\mathcal{O}_X$-module.
\end{definition}

\begin{thm}\label{thm-Steinexhcohvan}
    Let $X$ be a Hausdorff complex analytic space and $\mathcal{F}$ be a coherent $\mathcal{O}_X$-module. Assume that $(P_i)_{i\in \mathbb{Z}_{>0}}$ is a Stein exhaustion of $X$ relative to $\mathcal{F}$. Then
    \[
        H^q(X,\mathcal{F})=0\quad \text{for any }q\in \mathbb{Z}_{>0}.  
    \]
\end{thm}
\begin{proof}
    When $q\geq 2$, this follows from the general facts proved in \cref{Topology-lma-exhaustioncoh} in \nameref{Topology-chap-topology}. We will assume that $q=1$.

    We may assume that $X$ is connected.
    First observe that $X$ is necessarily paracompact. This follows from \cref{Topology-prop-paracptrefinement} in \nameref{Topology-chap-topology}. In particular, we can take a flabby resolution
    \[
        0\rightarrow \mathcal{F}\rightarrow \mathcal{G}^0\rightarrow \mathcal{G}^1\rightarrow \cdots.  
    \]
    Taking global sections, we get a complex
    \[
        0\rightarrow H^0(X,\mathcal{F})\xrightarrow{i} H^0(X,\mathcal{G}^0)\xrightarrow{d_0} H^0(X,\mathcal{G}^1)\xrightarrow{d_1}H^0(X,\mathcal{G}^2)\xrightarrow{d_2} \cdots.  
    \]
    We need to show that $\ker d_1=\Img d_0$. Let $\alpha\in \ker d_1$. We need to construct $\beta\in H^0(X,\mathcal{G}^0)$ with $d_0 \beta=\alpha$.

    We take semi-norms $|\bullet|_i$ on $H^0(P_i,\mathcal{F})$ for each $i\in \mathbb{Z}_{>0}$ satisfying the conditions in \cref{def-Steinexhaus}. We may furthermore assume that the restriction $H^0(P_{i+1},\mathcal{F})\rightarrow H^0(P_i,\mathcal{F})$ is a contraction for each $i\in \mathbb{Z}_{>0}$.

    For each $j\in \mathbb{Z}_{\geq 2}$, we will construct $\beta_j\in H^0(P_j,\mathcal{G}^0)$ and $\delta_j\in H^0(P_{j-1},\mathcal{F})$ such that
    \begin{enumerate}
        \item $(d_0|_{P_j})\beta_j=\alpha|_{P_j}$;
        \item $(\beta_{j+1}+\delta_{j+1})|_{P_{j-1}}=(\beta_{j}+\delta_{j})|_{P_{j-1}}$.
    \end{enumerate}
    It suffices to take $\beta\in H^0(X,\mathcal{G}^0)$ as the section defined by the $\beta_j+\delta_j$'s.

    We first construct $\beta_j$. Choose a sequence $\beta_j'\in H^0(P_j,\mathcal{G}^0)$ with 
    \[
        (d_0|_{P_j})\beta_j'=\alpha|_{P_j}  
    \]
    for each $j\in \mathbb{Z}_{>0}$. This is possible because $P_j$ is Stein. We define $\beta_j$ satisfying Condition~(1) for $j\in \mathbb{Z}_{>0}$ inductively. We begin with $\beta_1=\beta_1'$. Assume that $\beta_1,\ldots,\beta_j$ have been constructed. Let
    \[
        \gamma_j':=\beta'_{j+1}|_{P_j}-\beta_j. 
    \]
    Then
    \[
        (d_0|_{P_j})\gamma_j'=0.        
    \]
     It follows that $\gamma_j'\in H^0(P_j,\mathcal{F})$. Take $\gamma_j\in H^0(X,\mathcal{F})$ with 
    \[
        |\gamma_j'-\gamma_j|_{P_j}|_j\leq 2^{-j}.  
    \]
    Define
    \[
        \beta_{j+1}=\beta'_{j+1}-\gamma_i|_{P_{j+1}}. 
    \]
    Then clearly $\beta_{j+1}$ satisfies (1).

    Next we construct the sequence $\delta_j$.

    We observe that for each $j\in \mathbb{Z}_{>0}$,
    \[
        \left|\beta_{j+1}|_{P_j}-\beta_j  \right|_j\leq 2^{-j}.
    \]
    Let
    \[
        s^j_{k}:=\beta_{j+k}|_{P_j}-\beta_j  \in H^0(P_j,\mathcal{F})
    \]
    for all $j\in \mathbb{Z}_{>0}$ and $k\in \mathbb{N}$.
    By definition,
    \[
        s^j_k-s^{j+1}_{k-1}|_{P_j}  =\beta_{j+1}|_{P_j}-\beta_j
    \]
    for all $j\in \mathbb{Z}_{>0}$ and $k\in \mathbb{Z}_{>0}$.

    We claim that $(s^j_k|_{P_{j-1}})_k$ converges in $H^0(P_{j-1},\mathcal{F})$ as $k\to\infty$. 
    By our assumption, it suffices to show that $(s^j_k)_k$ is a Cauchy sequence in $H^0(P_j,\mathcal{F})$ for each $j\in \mathbb{Z}_{>1}$.
    We first compute
    \[
        \left|\beta_{j+l}|_{P_j}-\beta_{j+l-1}|_{P_j}\right|_j\leq   \left|\beta_{j+l}|_{P_{j+l-1}}-\beta_{j+l-1}\right|_{j+l-1}\leq 2^{1-j-l}
    \]
    for all $l\in \mathbb{Z}_{>0}$ and $j\in \mathbb{Z}_{>0}$. As a consequence for $k'>k\geq 1$, we have
    \[
        |s^j_k-s^j_{k'}|_j\leq \sum_{l=k+1}^k 2^{1-j-l}\leq 2^{1-j+k}.  
    \]
    So we conclude our claim.

    Let $\delta_j$ be the limit of $s^j_k|_{P_{j-1}}$ as $k\to\infty$ for each $j\in \mathbb{Z}_{\geq 2}$. 
    Then
    \[
        \lim_{k\to\infty}\left(s^j_k-s^{j+1}_{k-1}\right)|_{P_{j-1}}=\left(\delta_j-\delta_{j+1}\right)|_{P_{j-1}}  
    \]
    for each $j\in \mathbb{Z}_{\geq 2}$.
    The desired identity is clear.
\end{proof}


\section{Analytic blocks}

\begin{definition}\label{def-anablock}
    Let $X$ be a Hausdorff complex analytic space. A stone $(P,\pi:X\rightarrow \mathbb{C}^n)$ on $X$ is an \emph{analytic block} in $X$ if there are open neighbourhoods $U$ and $V$ of $P$ and $Q$ in $X$ and $Y$ respectively, where $Q\subseteq \mathbb{C}^n$ denotes the tube associated with the stone, such that
    \begin{enumerate}
        \item $\pi(U)\subseteq V$;
        \item $P=\pi^{-1}(Q)\cap U$;
        \item $U\rightarrow V$ induced by $\pi$ is a finite morphism.
    \end{enumerate}
\end{definition}
Recall that by \cref{thm-stonelocallyproper}, we can always assume that $U\rightarrow V$ is proper.

\begin{proposition}
    Let $X$ be a Hausdorff complex analytic space and $(P,\pi)$ be an analytic block in $X$. Then $P$ is a compact Stein set in $X$.
\end{proposition}
\begin{proof}
    This follows from \cref{thm-blocktostein} applied to $U\rightarrow V$ in \cref{def-anablock}.
\end{proof}

\begin{proposition}\label{prop-stoneisblockunderass}
    Let $X$ be a complex analytic space such that each compact analytic set in $X$ is finite, then every stone in $X$ is an analytic block in $X$.
\end{proposition}
\begin{proof}
    Let $(P,\pi:X\rightarrow \mathbb{C}^n)$ be a stone in $X$.
    We consider the proper morphism $\tau:U\rightarrow V$ as in \cref{thm-stonelocallyproper}. Each fiber of $\tau$ is a compact subset of $U$ and hence a compact subset of $X$. By our assumption, it is finite. It suffices to apply \cref{Topology-prop-topfinitechar} in \nameref{Topology-chap-topology} to conclude that $\tau$ is finite.
\end{proof}

\section{Holomorphically spreadable spaces}
\begin{definition}
    Let $X$ be a complex analytic space. We say $X$ is \emph{holomorphically spreadable} if $|X|$ is Hausdorff and for any $x\in X$, we can find an open neighbourhood $U$ of $x$ in $X$ such that
    \[
        \left\{y\in U:f(x)=f(y)\text{ for all }f\in \mathcal{O}_X(X) \right\} =\{x\}.
    \]
\end{definition}
A holomorphically separable space is clearly holomorphically spreadable.



\begin{proposition}\label{prop-holspreadablelocallyisopt}
    Let $X$ be a holomorphically spreadable complex analytic space and $x\in X$. Then there exist finitely many $f_1,\ldots,f_n\in \mathcal{O}_X(X)$ such that $x$ is an isolated point of $W(f_1,\ldots,f_n)$.
\end{proposition}
\begin{proof}
    By induction on $\dim_x X$, it suffices to prove the following claim: if $A$ is an analytic set in $X$ and $a\in A$ such that $\dim_a A\geq 1$. Then there is $f\in \mathcal{O}_X(X)$ such that $\dim_a (A\cap W(f))=\dim_a A-1$.

    To prove the claim, let $A_1,\ldots,A_k$ be the irreducible components of $A$. We may assume that all of them contain $a$. Choose $a_j\in A_j$ for each $j=1,\ldots,k$ so that $a,a_1,\ldots,a_k$ are pairwise different. Then there is a function $f\in \mathcal{O}_X(X)$ with $f(a)=0$ while $f(a_j)\neq 0$ for $j=1,\ldots,k$. Then $a\in W(f)$ while $f|_{A_j}$ is not identically $0$. By Krulls Hauptidealsatz, $\dim_a (A_j\cap W(f))=\dim_a A_j-1$ for all $j=1,\ldots,k$. Observe that $A\cap W(f)$ and $\bigcup_{j=1}^k (A_j\cap W(f))$ coincide near $a$, so
    \[
        \dim_a (A\cap W(f))=\max_{j=1,\ldots,k} \dim_a (A_j\cap W(f))=\max_{j=1,\ldots,k}(\dim_a A_j-1)=\dim_a A-1.
    \] 
\end{proof}

\begin{proposition}
    Let $X$ be an irreducible holomorphically spreadable complex analytic space. Then $X$ has countable basis.
\end{proposition}
The statement of this proposition in \cite[Proposition~0.37]{Fis76} is clearly wrong. I do not understand the argument of either \cite{Jur59} or \cite{Gra55}, where they claim that this result holds for connected holomorphically spreadable complex analytic spaces.
\begin{proof}
    We may assume that $X$ is connected. Recall that by \cref{PropertyComplex-cor-caslocallyconnected} in \nameref{PropertyComplex-chap-propcomplex}, $X$ is locally connected.
    Let $F:X\rightarrow \mathbb{C}^{\mathcal{O}_X(X)}$ be the map sending $x\in X$ to $(f(x))_{f\in \mathcal{O}_X(X)}$. By our assumption, $F$ is continuous and has discrete fibers.     
    In particular, for each $x\in X$, we may assume take finitely many $f_1,\ldots,f_n\in \mathcal{O}_X(X)$ so that the induced morphism $F':X\rightarrow \mathbb{C}^n$ is quasi-finite at $x$. By \cref{Analytic-cor-quasifinitelocuscoana} in \nameref{Analytic-chap-AS}, we can find a nowhere dense analytic set $A$ in $X$ such that the map $X\setminus A\rightarrow \mathbb{C}^n$ induced by $F'$ is quasi-finite. Now we endow $\mathcal{O}_X(X)$ with the compact-open topology. It is a metric space. By \cref{Topology-prop-fiberdiscountablebasisinh} in \nameref{Topology-chap-topology}, $X\setminus A$ has countable basis. It follows that $\mathcal{O}_X(X\setminus A)$ is a separable metric space. Hence, so it $\mathcal{O}_X(X)$. In particular, there is a continous map with discrete fibers
    \[
        X\rightarrow  \mathbb{C}^{\omega}.  
    \]
    It follows again from \cref{Topology-prop-fiberdiscountablebasisinh} in \nameref{Topology-chap-topology} that $X$ has countable basis.
\end{proof}

\begin{proposition}\label{prop-cptanainspreadablefinite}
    Let $X$ be a holomorphically spreadable complex analytic space. Then any compact analytic set $A$ in $X$ is finite.
\end{proposition}
\begin{proof}
    Let $B$ be a connected component of $A$ and $p\in B$. We need to show that $B=\{p\}$. Take finitely many $f_1,\ldots,f_n\in \mathcal{O}_X(X)$ so that $p$ is an isolated point of $W(f_1,\ldots,f_n)$. This is possible by \cref{prop-holspreadablelocallyisopt}.
    As $f_i$ vanishes on $B$ for each $i=1,\ldots,n$, we have $B=\{p\}$.
\end{proof}
\begin{corollary}\label{cor-anasetinblockcptimplfinite}
    Let $X$ be a complex analytic space and $A$ be a compact analytic subset of $X$. Suppose that there exists an analytic block $(P,\pi:X\rightarrow \mathbb{C}^n)$ in $X$ with $A\subseteq P$, then $A$ is finite.
\end{corollary}
\begin{proof}
    Take $U\subseteq X,V\subseteq \mathbb{C}^n$ as in \cref{def-anablock} so that $U\rightarrow V$ is finite. Then $U$ is clearly holomorphically spreadable. By \cref{prop-cptanainspreadablefinite}, $A$ is finite.
\end{proof}

\section{Holomorphically complete spacs}

\begin{definition}
    Let $X$ be a complex analytic space. An \emph{exhaustion of $X$ by analytic blocks} is an exhaustion of $X$ by stones $(P_i,\pi_i)_{i\in \mathbb{Z}_{>0}}$ such that $(P_i,\pi_i)$ is an analytic block for each $i\in \mathbb{Z}_{>0}$.

    We say $X$ is \emph{holomorphically complete} if $X$ is Hausdorff and there is an exhaustion of $X$ by analytic stones.
\end{definition}
\begin{thm}\label{thm-holcomchar}
    Let $X$ be a Hausdorff complex analytic space. Then the following are equivalent:
    \begin{enumerate}
        \item $X$ is holomorphically complete;
        \item $X$ is weakly holomorphically convex and every compact analytic subset of $X$ is finite.
    \end{enumerate}
\end{thm}
\begin{proof}
    (1) $\implies$ (2): $X$ is weakly holomorphically convex by definition. Each compact analytic subset $A$ of $X$ is contained in some analytic block, hence finite by \cref{cor-anasetinblockcptimplfinite}.

    (2) $\implies$ (1): This follows from \cref{prop-stoneisblockunderass}.
\end{proof}

\begin{lemma}\label{lma-coherentsheafclosed}
    Let $X$ be a complex manifold and $\mathcal{I}$ be a coherent subsheaf of $\mathcal{O}_X^l$ for some $l\in \mathbb{Z}_{>0}$. Then $\mathcal{I}(X)$ is a closed subspace of $\mathcal{O}_X(X)^l$ endowed with the compact-open topology.
\end{lemma}
\begin{proof}
   Let $(f_j\in \mathcal{I}(X))_{j\in \mathbb{Z}_{>0}}$ be a sequence with a limit $f\in \mathcal{O}^l_X(X)$. Let $x\in X$. It suffices to show that $f_x\in \mathcal{I}_x$. Observe that $f_x$ is the limit of $f_{jx}$ as $j\to\infty$. As $\mathcal{O}_{X,x}$ is noetherian, the submodule $\mathcal{I}_x$ of $\mathcal{O}_x^l$ is closed by \cref{Banach-cor-idealclosed} in \nameref{Banach-chap-banach}. We conclude.
\end{proof}


\begin{definition}\label{def-goodseminorm}
    Let $X$ be a complex analytic space and $\mathcal{F}$ be a coherent $\mathcal{O}_X$-module.
Let $(P,\pi:X\rightarrow \mathbb{C}^n)$ be an analytic block on $X$ with a non-zero associated tube $Q\subseteq \mathbb{C}^n$. 

Choose $U\subseteq X,V\subseteq \mathbb{C}^n$ as in \cref{def-anablock} so that $\tau:U\rightarrow V$ induced by $\pi$ is finite. Then $\mathcal{G}:=\tau_*(\mathcal{F}|_U)$  is a coherent $\mathcal{O}_V$-module. By \cref{cor-thmacohsheafquotient}, we can find $l\in \mathbb{Z}_{>0}$ and an epimorphism $\mathcal{O}_Q^l\rightarrow \mathcal{G}|_Q$. It induces an epimorphism $\epsilon:H^0(Q,\mathcal{O}_{\mathbb{C}^n})^l\rightarrow H^0(Q,\mathcal{G})\cn H^0(P,\mathcal{F})$. We define a semi-norm $|\bullet|$ on $H^0(P,\mathcal{F})$ as the quotient semi-norm induced by the sup seminorm on $H^0(Q,\mathcal{O}_{\mathbb{C}^n})^l$. 

A seminorm on  $H^0(P,\mathcal{F})$ defined in this way is called a \emph{good semi-norm} on  $H^0(P,\mathcal{F})$ with respect to $(P,\pi)$.
\end{definition}
\begin{lemma}\label{lma-goodseminormmetric}
    Let $X$ be a complex analytic space and $\mathcal{F}$ be a coherent $\mathcal{O}_X$-module.
Let $(P,\pi)$ be an analytic block on $X$. A good semi-norm on  $H^0(P,\mathcal{F})$ induces a metric on $H^0(P^0,\mathcal{F})$.
\end{lemma}
\begin{proof}
We need to show that if $|s|=0$ for some $s\in H^0(P,\mathcal{F})$, then $s|_{P^0}=0$, where $P^0$ is the analytic interior of $P$. 

We use the same notations as in \cref{def-goodseminorm}.
We can take $h\in H^0(Q,\mathcal{O}_{\mathbb{C}^n})^l$ and $h_j\in \ker \epsilon$ for each $j\in \mathbb{Z}_{>0}$ so that $\epsilon(h)=s$ and $\|h_j-h\|_{L^{\infty}}\to 0$. So $h_j|_Q\to h|_Q$ with respect to the compact-open topology. From  \cref{lma-coherentsheafclosed}, we conclude that the image of $h|_{\Int Q}$ is $0$. Namely, $s$ vanishes on $P^0=\tau^{-1}(\Int Q)$.
\end{proof}


\begin{lemma}\label{lma-comptopologies1}
    Let $X$ be a complex analytic space and $\mathcal{F}$ be a coherent $\mathcal{O}_X$-module.
Let $(P,\pi:X\rightarrow \mathbb{C}^n)$ be an analytic block on $X$ with a non-zero associated tube $Q\subseteq \mathbb{C}^n$. Consider the epimorphism of sheaves 
\[
    \mathcal{O}_Q^l\rightarrow \pi_*(\mathcal{F}|_P)
\]
as in \cref{def-goodseminorm} and endow $H^0(P^0,\mathcal{F})$ with the metric induced by the corresponding good semi-norm. Let 
\[
    Q_1\subseteq Q_2\subseteq \cdots
\]
be a compact exhaustion of $\Int Q$ by tubes with the same centers in $\mathbb{C}^n$. We get an induced map
\[
    \epsilon_j:H^0(Q_j,\mathcal{O}_{\mathbb{C}^n}^l) \rightarrow \pi_*(\mathcal{F}|_P)(Q_j) 
\]
for each $j\in \mathbb{Q}_{>0}$. We therefore get good semi-norms $|\bullet|_j$ on $H^0(P^0, \mathcal{F})$ for each $j\in \mathbb{Z}_{>0}$. Let
\[
    d(s_1,s_2):=\sum_{j=1}^{\infty}  2^{-j}\frac{|s_1-s_2|_j}{1+|s_1-s_2|_j}
\]
for each $s_1,s_2\in H^0(P^0, \mathcal{F})$. Then $d$ is a metric on  $H^0(P^0, \mathcal{F})$ and $H^0(P^0, \mathcal{F})$ is a Fréchet space with respect to this topology.

Moreover, the topology does not depend on the choice of $\pi$, $\epsilon$ and the exhaustion.
\end{lemma}
\begin{proof}
By \cref{lma-goodseminormmetric}, each $|\bullet|_{\nu}$ is a norm on $H^0(P^0, \mathcal{F})$. It follows that $d$ is a metric. Next we show that $H^0(P^0, \mathcal{F})$ is Fréchet. 
Let $(s_j)_{j\in \mathbb{Z}_{>0}}$ be a Cauchy sequence in $H^0(P^0, \mathcal{F})$. We can find bounded sequences $(f_{jk}\in H^0(Q_k,\mathcal{O}^l_{\mathbb{C}^n}))_{k\in \mathbb{Z}_{>0}}$ so that $\epsilon_k(f_{jk})=s_j|_{\pi^{-1}(Q_k)\cap P}$ ($k\in \mathbb{Z}_{>0}$) for each $j\mathbb{Z}_{>0}$. By Montel's theorem, there is a subsequence of $(f_{jk})_j$ which converges uniformly on $Q_{k-1}$ to $f_k\in H^0(Q_{k-1},\mathcal{O}_{\mathbb{C}^n}^l)$. Then $\epsilon_{k-1}(f_{k+1})|_{\Int Q_{k-1}}=\epsilon_{k-1}(f_k)|_{\Int Q_{k-1}}$ for each $k\in \mathbb{Z}_{\geq 2}$. So we can glue the $f_k$'s to $s\in H^0(P^0,\mathcal{F})$. Clearly, $s_k\to s$ as $k\to\infty$.

Next we show that the topology is independent of the choice of $\pi$, $\epsilon$ and the exhaustion. 
The independence of the exhaustion is obvious. We prove the other two independence.
Let $(P,\pi':X\rightarrow \mathbb{C}^{n'})$ be another analytic block with $\pi'=(\pi,\varphi):X\rightarrow \mathbb{C}^n\times \mathbb{C}^m$, $n'=n+m$. Let $Q^*\subseteq \mathbb{C}^m$ be a tube such that $\varphi(P)\subseteq Q^*$. Then $P=\pi'^{-1}(Q\times Q^*)\cap U$. We can find an open neighbourhood $U'$ of $P$ in $X$ and $V'$ of $Q\times Q^*$ in $\mathbb{C}^{n'}$ for which the induced map $\tau':U'\rightarrow V'$ is finite by \cref{def-anablock}. Fix an epimorphism $\mathcal{O}_{\mathbb{C}^{n'}}^{l'}|_{Q\times Q^*}\rightarrow \pi'_*(\mathcal{F}|_P)$ for some $l'\in \mathbb{Z}_{>0}$. Construct an exhanstion of $\Int Q\times \Int Q^*$ of the product type: $(Q_j\times Q_j^*)_{j\in \mathbb{Z}_{>0}}$ as in the lemma. Let $d'$ denote the induced metric on $H^0(\Int P,\mathcal{F})$.


We will show that $d'$ and $d$ induce the same topology. Let $e_1,\ldots,e_l\in H^0(Q,\mathcal{O}_{\mathbb{C}^n}^l)$ be the standard basis.  Let $e_1',\ldots,e_l'$ be the preimages of $\epsilon(e_1),\ldots,\epsilon(e_l)\in \pi_* (\mathcal{F}|P)(Q)=\pi'_*(\mathcal{F}|_P)(Q\times Q^*)$ in $\mathcal{O}_{\mathbb{C}^{n'}}(Q\times Q^*)^{l'}$ under $\epsilon'$. Further, for $f\in \mathcal{O}_{\mathbb{C}^n}(Q_j)$, we denote by $f'\in\mathcal{O}_{\mathbb{C}^{n'}}(Q_j\times Q_j^*)$ the holomorphic extension of $f$ to $Q_j\times Q_j^*$ constant along $\{q\}\times Q_j^*$ for each $q\in Q_j$  for each $j\in \mathbb{Z}_{>0}$. The norms of 
\[
\mathcal{O}_{\mathbb{C}^n}(Q_j)^l\rightarrow \mathcal{O}_{\mathbb{C}^{n'}}(Q_j\times Q_j^*)^l,\quad \sum_{i=1}^l f_i e_i\mapsto \sum_{i=1}^l f_i' e_i'    
\]
for $j\in \mathbb{Z}_{>0}$ are bounded by a constant independent of $j$. Therefore, the identity map 
\[
  (H^0(P^0,\mathcal{F}),d)  \rightarrow (H^0(P^0,\mathcal{F}),d')
\] 
is continuous. By open mapping theorem, the map is a homeomorphism.
\end{proof}

\begin{thm}\label{thm-compseminor2}
    Let $X$ be a complex analytic space and $(P,\pi)\subseteq (P',\pi')$ be two analytic blocks on $X$ and $\mathcal{F}$ be a coherent $\mathcal{O}_X$-module, then the restriction map
    \[
        H^0(P',\mathcal{F}) \rightarrow H^0(P,\mathcal{F}) 
    \]
    with respect to any good semi-norms.
\end{thm}
\begin{proof}
    We claim that there exists an analytic block $(P_1,\pi)$ such that
    \[
        (P,\pi)\subseteq (P_1,\pi)\subseteq (P',\pi').  
    \]
    Assume this claim, then we have a decomposition of the restriction map
    \[
        H^0(P',\mathcal{F}) \rightarrow H^0(P_1^0,\mathcal{F})\rightarrow H^0(P,\mathcal{F}).
    \]
    The first map is continuous if we endow $H^0(P_1^0,\mathcal{F})$ with the topology induced by $\pi'$, the second is continuous if we endow $H^0(P_1^0,\mathcal{F})$ with the topology induced by $\pi$. These topologies are identical by \cref{lma-comptopologies1}.
    Our assertion follows.

    To argue the claim, let us write $\pi:X\rightarrow \mathbb{C}^n$ and $\pi'=(\pi,\varphi):X\rightarrow \mathbb{C}^n\times \mathbb{C}^m$. Take $q\in \mathbb{C}^m$ with $Q\times\{q\}\subseteq \Int Q'$. Let $Q'':=Q'\cap (\mathbb{C}^n\times \{q\})$ and identify it with a subset of $\mathbb{C}^n$. Let $Q^*$ be the image of $Q'$ under the projection $\mathbb{C}^{n+m}\rightarrow \mathbb{C}^m$. 

    Choose open neighbourhoods $U\subseteq P'^0$, $V\subseteq Q'$ of $P$ and $Q$ respectively such that $\tau:U\rightarrow V$ is finite and $U\cap \pi^{-1}(Q)=P$. Take a tube $Q_1\subseteq \mathbb{C}^n$ such that
    \[
        Q\subseteq \Int Q_1\subseteq Q_1\subseteq \Int Q''.  
    \]
    Now it suffices to set $P_1:=\pi^{-1}(Q_1)\cap U$.
\end{proof}

\begin{corollary}\label{cor-convthm}
    Let $X$ be a complex analytic space and $\mathcal{F}$ be a coherent $\mathcal{O}_X$-module. Let
    $(P,\pi)\subseteq (P',\pi')$ be analytic blocks in $X$. Then for any Cauchy sequence $(s_j)_{j\in \mathbb{Z}_{>0}}$ in $H^0(P',\mathcal{F})$, the restriction sequence $(s_j|_P)_{j\in \mathbb{Z}_{>0}}$ has a limit in $H^0(P,\mathcal{F})$.
\end{corollary}
\begin{proof}
    Choose an analytic block $(P_1,\pi)$ such that
    \[
        (P,\pi)\subseteq (P_1,\pi)\subseteq (P',\pi').  
    \]
    The existence of the block $(P_1,\pi)$ is argued in the proof of \cref{thm-compseminor2}. We have a decomposition of the restriction map
    \[
        H^0(P',\mathcal{F}) \rightarrow H^0(P_1^0,\mathcal{F})\rightarrow H^0(P,\mathcal{F}).
    \]
    The first map is bounded, so the images of $(s_j)_{j\in \mathbb{Z}_{>0}}$ in $H^0(P_1^0,\mathcal{F})$ is a Cauchy sequence. As we have shown that $H^0(P_1^0,\mathcal{F})$ is a Fréchet space in \cref{lma-comptopologies1}, the sequence converges. As the second map is also continuous, it follows that $(s_j|_P)_{j\in \mathbb{Z}_{>0}}$ has a limit in $H^0(P,\mathcal{F})$.
\end{proof}

\begin{lemma}\label{lma-denseimage}
    Let $X$ be a complex analytic space and $\mathcal{F}$ be a coherent $\mathcal{O}_X$-module. Let
    $(P,\pi:X\rightarrow  \mathbb{C}^n)\subseteq (P',\pi':X\rightarrow \mathbb{C}^n\times \mathbb{C}^m)$ be analytic blocks in $X$ with tubes $Q$ and $Q'$. Choose $U'\subseteq X$ and $V'\subseteq \mathbb{C}^{n+m}$ of $P'$ and $Q'$ respectively as in \cref{def-anablock} such that $U'\rightarrow V'$ is finite. Set
    \[
        Q_1:=(Q\times \mathbb{C}^m)\cap Q',\quad P_1=\pi'^{-1}(Q_1)\cap U'.  
    \]
    Then $(P_1.\pi')$ is an analytic block in $X$ with block $Q_1$ and 
    $H^0(P',\mathcal{F})\rightarrow H^0(P_1,\mathcal{F})$ has dense image. Here we take an epsimorphism 
    \[
        \mathcal{O}_{\mathbb{C}^{n+m}}^{l'}|_{Q'}\rightarrow \left(\tau' (\mathcal{F}|_{U'}) \right)_{Q'} 
    \]
    and it induces 
    \[
        \mathcal{O}_{\mathbb{C}^{n+m}}^{l'}|_{Q_1}\rightarrow \left(\tau' (\mathcal{F}|_{U'}) \right)_{Q_1},
    \]
    which in turn induces a good semi-norm on $H^0(P_1,\mathcal{F})$. This is the semi-norm we are using.

    Moreover, there is a compact set $\tilde{P}\subseteq X$ disjoint from $P$ such that
    \[
        P_1=P\cup \tilde{P}.  
    \]
\end{lemma}
\begin{proof}
    We have a commutative diagram in the category of topological linear spaces:
    \[
        \begin{tikzcd}
            {H^0(Q',\mathcal{O}_{\mathbb{C}^{m+n}}^l)} \arrow[r] \arrow[d] & {H^0(P',\mathcal{F})} \arrow[d] \\
            {H^0(Q_1,\mathcal{O}_{\mathbb{C}^{m+n}}^l)} \arrow[r]          & {H^0(P_1,\mathcal{F})}         
        \end{tikzcd}.
    \]
    In order to show that the right vertical map has dense image, it is enough to show that the map on the left-hand side has dense images, which is the Runge approximation.

    For the last assertion, as $Q_1=(Q\times \mathbb{C}^m)\cap Q'$, we have
    \[
        P_1=\pi^{-1}(Q)\cap P'.  
    \]
    As $P\subseteq P'$ and $P\subseteq \pi^{-1}(Q)$, it follows that $P\subseteq P_1$. But there is an open neighbourhood $U$ of $P$ in $X$ so that $P=\pi^{-1}(Q)\cap U$. Hence, $\tilde{P}=P_1\setminus P$ is compact.
\end{proof}

\begin{thm}[Runge approximation]\label{thm-Rungapp}
    Let $X$ be a complex analytic space and $\mathcal{F}$ be a coherent $\mathcal{O}_X$-module. Let
    $(P,\pi:X\rightarrow  \mathbb{C}^n)\subseteq (P',\pi':X\rightarrow \mathbb{C}^n\times \mathbb{C}^m)$ be analytic blocks in $X$ with tubes $Q$ and $Q'$. Then the map 
    \[
        H^0(P',\mathcal{F})\rightarrow H^0(P,\mathcal{F})  
    \]
    has dense image with respect to a good semi-norm.
\end{thm}
\begin{proof}
    We use the notations of \cref{lma-denseimage}. We extend $Q,Q_1,Q'$ to tubes $\hat{Q},\hat{Q}_1,\hat{Q}'$ and get $\hat{P},\hat{P_1},\hat{P'}$ corresponding to the original $P,P_1,P'$. The restriction map
    \[
        H^0(\hat{P_1}^0,\mathcal{F})\rightarrow H^0(\hat{P}^0,\mathcal{F})
    \]
    is a continuous morphism of Fréchet spaces.

    Let $s\in H^0(P,\mathcal{F})$ be a section. Lift $s$ to $s_1\in H^0(P_1,\mathcal{F})$. Up to a suitable modification of the tubes, we can extend $s_1$ to $\hat{s_1}\in H^0(\hat{P_1},\mathcal{F})$. Then there is a sequence $(s^j\in H^0(\hat{P'},\mathcal{F}))_{j\in \mathbb{Z}_{>0}}$ such that $s^j|_{\hat{P_1}}\rightarrow \hat{s_1}$ as $j\to\infty$ in $H^0(\hat{P_1},\mathcal{F})$. It follows that $s^j|_{\hat{P}^0}\to \hat{s}_1|_{\hat{P}^0}$ in $H^0(\hat{P}^0,\mathcal{F})$. It follows that $s^j|_P\to s_1|_P=s$ sa $j\to\infty$.
\end{proof}

\begin{thm}\label{thm-AllanalybloexhisStein}
    Let $X$ be a complex analytic space.
    Each exhaustion of $X$ by analytic blocks is a Stein exhaustion.
\end{thm}
\begin{proof}
    Let $(P_i,\pi_i)_{i\in \mathbb{Z}_{>0}}$ be an exhaustion of $X$ by analytic blocks. Take a coherent $\mathcal{O}_X$-module $\mathcal{F}$.

    We verify the conditions in \cref{def-Steinexhaus}.
    By \cref{thm-blocktostein}, $P_i$ is a compact Stein set for each $i\in \mathbb{Z}_{>0}$. So (1) is satisfied.
    
    On $H^0(P_i,\mathcal{F})$, we fix a good semi-norm $|\bullet|_i$ for each $i\in \mathbb{Z}_{>0}$. We may assume that $H^0(P_{i+1},\mathcal{F})\rightarrow H^0(P_i,\mathcal{F})$ is contractive for $i\in \mathbb{Z}_{>0}$.

    We have already verified (3), (4) and (5).

    We verify (2). It suffices to show that
    \[
        H^0(X,\mathcal{F})\rightarrow H^0(P_1,\mathcal{F})  
    \]
    has dense image. Let $s\in H^0(P_1,\mathcal{F})$ and $\delta>0$. By \cref{thm-Rungapp}, we can find $s_i\in H^0(P_i,\mathcal{F})$ for $i\in \mathbb{Z}_{>0}$ such that $s_1=s$, 
    \[
        \left| s_{i+1}|_{P_i}-s_i \right|_i<2^{-i}\delta
    \]
    for $i\in \mathbb{Z}_{>0}$. By \cref{cor-convthm}, $(s_j|_{P_i})_{j\in \mathbb{Z}_{>0}}$ has a limit $t_i\in H^0(P_i,\mathcal{F})$ for each $i\in \mathbb{Z}_{>0}$. As $H^0(P_{i+1},\mathcal{F})\rightarrow H^0(P_{i},\mathcal{F})$ is continuous for $i\in \mathbb{Z}_{>0}$, the $t_{i+1}|_{P_i}$'s are compatible and defines $t\in H^0(X,\mathcal{F})$. It is easy to see that $|t|_{P_1}-s|_1<\delta$. Thus condition (2) is satisfied.
\end{proof}


\section{Stein spaces}

\begin{definition}
    Let $X$ be a complex analytic space. We say that $X$ is a Stein space if $X$ is a Stein set in $X$ and $|X|$ is paracompact and Hausdorff.
\end{definition}

\begin{definition}
    Let $X$ be a complex analytic space. An \emph{effective formal $0$-cycle} on $X$ consists of 
    \begin{enumerate}
        \item A disrete set $D\subseteq X$;
        \item An integer $n_x$ for each $x\in D$.
    \end{enumerate}
    We write the effective formal $0$-cycle as $\sum_{x\in D}n_x x$. We define the \emph{ideal sheaf} $\mathcal{O}_X(-\sum_{x\in D}n_x x)$ of an effective formal $0$-cycle as $\sum_{x\in D}n_x x$ as 
    \[
        \mathcal{O}_X(-\sum_{x\in D}n_x x)(U)=\left\{f\in H^0(U,\mathcal{O}_X): f_x\in \mathfrak{m}_x^{n_x}\text{ for each }x\in D\cap U \right\}
    \]
    for each open subset $U\subseteq X$.
\end{definition}
Observe that $\mathcal{O}_X(-\sum_{x\in D}n_x x)$ is a coherent $\mathcal{O}_X$-module. In fact, the problem is local, so we may assume that $D$ is finite. In this case, $D$ is an effective $0$-cycle and the result is clear.

\begin{lemma}\label{lma-localassigngerm}
    Let $X$ be a complex analytic space and $\sum_{x\in D}n_x x$ be an effective formal $0$-cycle on $X$. Assume that 
    \[
        H^0(X,\mathcal{O}_X)\rightarrow   H^0(X,\mathcal{O}_X/\mathcal{O}_X(-\sum_{x\in D}n_x x))
    \]
    is surjective. Suppose that for each $x\in D$, we assign $g_x\in \mathcal{O}_{X,x}$. Then there is $f\in H^0(X,\mathcal{O}_X)$ such that
    \[
        f_x-g_x\in \mathfrak{m}_x^{n_x}  
    \]
    for all $x\in D$.
\end{lemma}
\begin{proof}
    We define $s\in H^0(X,\mathcal{O}_X/\mathcal{O}_X(-\sum_{x\in D}n_x x))$ by $s_x=g_x$ for each $x\in D$. Lift $s$ to $f\in H^0(X,\mathcal{O}_X)$. Then $f$ clearly satisfies the required properties.
\end{proof}

\begin{proposition}\label{prop-assignvalue}
    Let $X$ be a complex analytic space. Assume that $H^1(X,\mathcal{I})=0$ for each coherent ideal sheaf $\mathcal{I}$ on $X$. Let $(x_i\in X)_{i\in \mathbb{Z}_{>0}}$ be a sequence without accumulation points and $(c_i)_{i\in \mathbb{Z}_{>0}}$ be a sequence in $\mathbb{C}$. Then there is $f\in \mathcal{O}_X(X)$ with $f(x_i)=c_i$ for each $i\in \mathbb{Z}_{>0}$.
\end{proposition}
\begin{proof}
    Consider the formal cycle $\sum_{i=1}^{\infty} x_i$. Apply \cref{lma-localassigngerm} with $g_{x_i}=c_i$.
\end{proof}


\begin{thm}\label{thm-Steinchar}
    Let $X$ be a complex analytic space. Then the following are equivalent:
    \begin{enumerate}
        \item $X$ is a Stein space;
        \item For any coherent ideal sheaf $\mathcal{I}$ on $X$, we have $H^1(X,\mathcal{I})=0$;
        \item $X$ is holomorphically separable and holomorphically convex;
        \item $X$ is holomorphically spreadable and weakly holomorphically convex; %and for each $x\in X$, there are $e=\dim_{\mathbb{C}}\mathfrak{m}_X/\mathfrak{m}_X^2$ functions $f_1,\ldots,f_e\in H^0(X,\mathcal{O}_X)$ such that $f_{1x},\ldots,f_{ex}$ generate $\mathfrak{m}_x$ as an $\mathcal{O}_{X,x}$-module;
        \item $X$ is holomorphically complete.
    \end{enumerate}
\end{thm}
\begin{proof}
    (1) $\implies$ (2): This is trivial.

    (2) $\implies$ (3): $X$ is holomorphicaly convex by \cref{prop-assignvalue} and \cref{prop-holoconvchar}.
    $X$ is holomorphically separable by \cref{prop-assignvalue}.

    (3) $\implies$ (4): $X$ is holomorphically spreadable and weakly holomorphically convex by definition. %Let $h_1,\ldots,h_e\in \mathfrak{m}_x$ be some generators of $\mathfrak{m}_x$, whose existence is guaranteed by Nakayama's lemma. Applying  \cref{lma-localassigngerm} to the cycle $2x$, we get $f_1,\ldots,f_e\in H^0(X,\mathcal{O}_X)$ with $f_{ix}-h_i\in\mathfrak{m}_x^2$ for each $i=1,\ldots,e$. Then $f_1,\ldots,f_e$ satisfy the required properties.

    (4) $\implies$ (5): This follows from \cref{thm-holcomchar} and \cref{prop-cptanainspreadablefinite}.

    (5) $\implies$ (1):This follows from \cref{thm-AllanalybloexhisStein} and \cref{thm-Steinexhcohvan}.
\end{proof}

\begin{corollary}
    Let $X$ be a complex analytic space. Then the following are equivalent:
    \begin{enumerate}
        \item $X$ is Stein;
        \item $X^{\Red}$ is Stein;
        \item The normalization $\overline{X^{\Red}}$ is Stein.
    \end{enumerate}
\end{corollary}
The equivalence of (1) and  (2) is due to Grauert PIHES. The difficult direction (3) implies  (2) is claimed in Grauert--Remmert, where the proof is nonsense. We follow the argument of Narasimhan here.

\begin{proof}
    (1) $\implies$ (2): This follows from Leray's spectral sequence.

    (2) $\implies$ (1): By \cref{thm-Steinchar}(3), it suffices to show that the restriction map $H^0(X,\mathcal{O}_X)\rightarrow H^0(X^{\Red},\mathcal{O}_{X^{\Red}})$ is surjective.

    Let $\mathcal{I}$ be the nilradical of $\mathcal{O}_X$. It is coherent by Cartan--Oka theorem.  For each $i\in \mathbb{Z}_{>0}$, we have a short exact sequence
    \[
        0\rightarrow \mathcal{I}^i/\mathcal{I}^{i+1}\rightarrow \mathcal{O}_X/\mathcal{I}^{i+1}\rightarrow \mathcal{O}_X/\mathcal{I}^{i}\rightarrow 0.
    \]
    As $\mathcal{I}^i/\mathcal{I}^{i+1}$ is a coherent $\mathcal{O}_{X^{\Red}}$-module, we conclude that 
    \[
        \varphi_i: H^0(X,\mathcal{O}_X/\mathcal{I}^{i+1})\rightarrow H^0(X,\mathcal{O}_X/\mathcal{I}^{i})  
    \]
    is surjective for each  $i\in \mathbb{Z}_{>0}$. Let $h_1\in H^0(X,\mathcal{O}_X/\mathcal{I})=H^0(X^{\Red},\mathcal{O}_{X^{\Red}})$. We want to lift it to $h\in H^0(X,\mathcal{O}_X)$.

    We successively lift $h_1$ to $h_i\in H^0(X,\mathcal{O}_X/\mathcal{I}^i)$ for each $i\in \mathbb{Z}_{>0}$. Let $X_i=X\setminus \Supp \mathcal{I}^i$ of each $i\in \mathbb{Z}_{>0}$. Then clearly
    \[
        X=\bigcup_{i=1}^{\infty}X_i.  
    \] 
    It is easy to see that 
    \[
        h_{i+1}|_{X_i}=h_i|_{X_i}  
    \]
    for each $i\in \mathbb{Z}_{>0}$. It follows that we can glue the $h_i|_{X_i}$'s
    to $h\in H^0(X,\mathcal{O}_X)$ which restricts to $h_1$.

    (2) $\implies$ (3): This follows from Leray's spectral sequence as $\overline{X^{\Red}}\rightarrow X^{\Red}$ is finite by \cref{PropertyComplex-prop-reducedspacenormalizationchar} in \nameref{PropertyComplex-chap-propcomplex}.

    (3) $\implies$ (2): \textcolor{red}{Reproduce Narasimhan's work}
\end{proof}

\printbibliography
\end{document}