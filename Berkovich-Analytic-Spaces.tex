
\documentclass{amsbook} 



%\usepackage{xr}
\usepackage{xr-hyper}
\usepackage[unicode]{hyperref}


\usepackage[T1]{fontenc}
\usepackage[utf8]{inputenc}
\usepackage{lmodern}
\usepackage{amssymb,tikz-cd}
%\usepackage{natbib}
\usepackage[english]{babel}
\usepackage{nameref}

\usepackage[nameinlink,capitalize]{cleveref}
\usepackage[style=alphabetic,maxnames=99,maxalphanames=5, isbn=false, giveninits=true, doi=false]{biblatex}
\usepackage{lipsum, physics}
\usepackage{ifthen}
\usepackage{microtype}
\usepackage{booktabs}
\usetikzlibrary{calc}
\usepackage{emptypage}
\usepackage{setspace}
\usepackage[margin=0.75cm, font={small,stretch=0.80}]{caption}
\usepackage{subcaption}
\usepackage{url}
\usepackage{bookmark}
\usepackage{graphicx}
\usepackage{dsfont}
\usepackage{enumitem}
\usepackage{mathtools}
\usepackage{csquotes}
\usepackage{silence}
\usepackage{mathrsfs}
\usepackage{bigints}

\WarningFilter{biblatex}{Patching footnotes failed}


\ProcessOptions\relax

\emergencystretch=1em

\hypersetup{
colorlinks=true,
linktoc=all
}

\setcounter{tocdepth}{1}


\hyphenation{archi-medean  Archi-medean Tru-ding-er}

%\captionsetup[table]{position=bottom}   %% or below
\renewcommand{\thefootnote}{\fnsymbol{footnote}}
%\DeclareMathAlphabet{\mathcal}{OMS}{cmsy}{m}{n}
\renewbibmacro{in:}{}

\DeclareFieldFormat[article]{citetitle}{#1}
\DeclareFieldFormat[article]{title}{#1}
\DeclareFieldFormat[inbook]{citetitle}{#1}
\DeclareFieldFormat[inbook]{title}{#1}
\DeclareFieldFormat[incollection]{citetitle}{#1}
\DeclareFieldFormat[incollection]{title}{#1}
\DeclareFieldFormat[inproceedings]{citetitle}{#1}
\DeclareFieldFormat[inproceedings]{title}{#1}
\DeclareFieldFormat[phdthesis]{citetitle}{#1}
\DeclareFieldFormat[phdthesis]{title}{#1}
\DeclareFieldFormat[misc]{citetitle}{#1}
\DeclareFieldFormat[misc]{title}{#1}
\DeclareFieldFormat[book]{citetitle}{#1}
\DeclareFieldFormat[book]{title}{#1} 


%% Define various environments.

\theoremstyle{definition}
\newtheorem{theorem}{Theorem}[section]
\newtheorem{thm}[theorem]{Theorem}
\newtheorem{proposition}[theorem]{Proposition}
\newtheorem{corollary}[theorem]{Corollary}
\newtheorem{lemma}[theorem]{Lemma}
\newtheorem{conjecture}[theorem]{Conjecture}
\newtheorem{question}[theorem]{Question}
\newtheorem{example}[theorem]{Example}
\newtheorem{definition}[theorem]{Definition}
\newtheorem{condition}[theorem]{Condition}

\theoremstyle{remark}
\newtheorem{remark}[theorem]{Remark}
\numberwithin{equation}{section}

%\renewcommand{\thesection}{\thechapter.\arabic{section}}
%\renewcommand{\thetheorem}{\thesection.\arabic{theorem}}
%\renewcommand{\thedefinition}{\thesection.\arabic{definition}}
%\renewcommand{\theremark}{\thesection.\arabic{remark}}


%% Define new operators

\DeclareMathOperator{\rad}{rad}
\DeclareMathOperator{\nd}{nd}
\DeclareMathOperator{\ord}{ord}
\DeclareMathOperator{\Hom}{Hom}
\DeclareMathOperator{\PreSh}{PreSh}
\DeclareMathOperator{\Gr}{Gr}
\DeclareMathOperator{\Homint}{\mathcal{H}\mathrm{om}}
\DeclareMathOperator{\Torint}{\mathcal{T}\mathrm{or}}
\DeclareMathOperator{\Div}{div}
\DeclareMathOperator{\DSP}{DSP}
\DeclareMathOperator{\Diff}{Diff}
\DeclareMathOperator{\MA}{MA}
\DeclareMathOperator{\NA}{NA}
\DeclareMathOperator{\AN}{an}
\DeclareMathOperator{\Rep}{Rep}
\DeclareMathOperator{\Rest}{Res}
\DeclareMathOperator{\DF}{DF}
\DeclareMathOperator{\VCart}{VCart}
\DeclareMathOperator{\PL}{PL}
\DeclareMathOperator{\Bl}{Bl}
\DeclareMathOperator{\Td}{Td}
\DeclareMathOperator{\Fitt}{Fitt}
\DeclareMathOperator{\Ric}{Ric}
\DeclareMathOperator{\coeff}{coeff}
\DeclareMathOperator{\Aut}{Aut}
\DeclareMathOperator{\Capa}{Cap}
\DeclareMathOperator{\loc}{loc}
\DeclareMathOperator{\vol}{vol}
\DeclareMathOperator{\Val}{Val}
\DeclareMathOperator{\ST}{ST}
\DeclareMathOperator{\het}{ht}
\DeclareMathOperator{\Amp}{Amp}
\DeclareMathOperator{\Herm}{Herm}
\DeclareMathOperator{\trop}{trop}
\DeclareMathOperator{\Trop}{Trop}
\DeclareMathOperator{\Cano}{Can}
\DeclareMathOperator{\PS}{PS}
\DeclareMathOperator{\codim}{codim}
\DeclareMathOperator{\Var}{Var}
\DeclareMathOperator{\Psef}{Psef}
\DeclareMathOperator{\Jac}{Jac}
\DeclareMathOperator{\Char}{char}
\DeclareMathOperator{\Red}{red}
\DeclareMathOperator{\Spf}{Spf}
\DeclareMathOperator{\Span}{Span}
\DeclareMathOperator{\Der}{Der}
%\DeclareMathOperator{\Mod}{mod}
\DeclareMathOperator{\Hilb}{Hilb}
\DeclareMathOperator{\triv}{triv}
\DeclareMathOperator{\Frac}{Frac}
\DeclareMathOperator{\diam}{diam}
\DeclareMathOperator{\Spec}{Spec}
\DeclareMathOperator{\Spm}{Spm}
\DeclareMathOperator{\Specrel}{\underline{Sp}}
\DeclareMathOperator{\Sp}{Sp}
\DeclareMathOperator{\reg}{reg}
\DeclareMathOperator{\sing}{sing}
\DeclareMathOperator{\Star}{Star}
\DeclareMathOperator{\relint}{relint}
\DeclareMathOperator{\Cvx}{Cvx}
\DeclareMathOperator{\Int}{Int}
\DeclareMathOperator{\dep}{dep}
\DeclareMathOperator{\pd}{pd}
\DeclareMathOperator{\codep}{codep}
\DeclareMathOperator{\Supp}{Supp}
\DeclareMathOperator{\FS}{FS}
\DeclareMathOperator{\RZ}{RZ}
\DeclareMathOperator{\Ext}{Ext}
\DeclareMathOperator{\Redu}{red}
\DeclareMathOperator{\lct}{lct}
\DeclareMathOperator{\Proj}{Proj}
\DeclareMathOperator{\Sing}{Sing}
\DeclareMathOperator{\Conv}{Conv}
\DeclareMathOperator{\Max}{Max}
\DeclareMathOperator{\Tor}{Tor}
\DeclareMathOperator{\Gal}{Gal}
\DeclareMathOperator{\Frob}{Frob}
\DeclareMathOperator{\coker}{coker}
\DeclareMathOperator{\Sym}{Sym}
\DeclareMathOperator{\CSp}{CSp}
\DeclareMathOperator{\Cov}{Cov}
\DeclareMathOperator{\Img}{Im}


\newcommand{\alg}{\mathrm{alg}}
\newcommand{\Sh}{\mathrm{Sh}}
\newcommand{\fin}{\mathrm{fin}}
\newcommand{\BPF}{\mathrm{BPF}}
\newcommand{\dBPF}{\mathrm{dBPF}}
\newcommand{\divf}{\mathrm{Div}^f}
\newcommand{\nef}{\mathrm{nef}}
\newcommand{\Bir}{\mathrm{Bir}}
\newcommand{\hO}{\hat{\mathcal{O}}}
\newcommand{\bDiv}{\mathrm{Div}^{\mathrm{b}}}
\newcommand{\un}{\mathrm{un}}
\newcommand{\sep}{\mathrm{sep}}
\newcommand{\diag}{\mathrm{diag}}
\newcommand{\Pic}{\mathrm{Pic}}
\newcommand{\GL}{\mathrm{GL}}
\newcommand{\SL}{\mathrm{SL}}
\newcommand{\LS}{\mathrm{LS}}
\newcommand{\GLS}{\mathrm{GLS}}
\newcommand{\GLSi}{\mathrm{GLS}_{\cap}}
\newcommand{\PGLS}{\mathrm{PGLS}}
\newcommand{\Loc}[1][S]{_{\{{#1}\}}}
\newcommand{\cl}{\mathrm{cl}}
\newcommand{\otL}{\hat{\otimes}^{\mathbb{L}}}
\newcommand{\ddpp}{\mathrm{d}'\mathrm{d}''}
\newcommand{\TC}{\mathcal{TC}}
\newcommand{\ddPP}{\mathrm{d}'_{\mathrm{P}}\mathrm{d}''_{\mathrm{P}}}
\newcommand{\PSs}{\mathcal{PS}}
\newcommand{\Gm}{\mathbb{G}_{\mathrm{m}}}
\newcommand{\End}{\mathrm{End}}
\newcommand{\Aff}[1][X]{\mathcal{M}\left(\mathcal{#1}\right)}
\newcommand{\XG}[1][X]{{#1}_{\mathrm{G}}}
\newcommand{\convC}{\xrightarrow{C}}
\newcommand{\Vect}{\mathrm{Vect}}
\newcommand{\abso}[1]{\lvert#1\rvert}
\newcommand{\Mdl}{\mathrm{Model}}
\newcommand{\cn}{\stackrel{\sim}{\longrightarrow}}
\newcommand{\sbc}{\mathbf{s}}
\newcommand{\CH}{\mathrm{CH}}
\newcommand{\GR}{\mathrm{GR}}
\newcommand{\bir}{\mathrm{bir}}
\newcommand{\dc}{\mathrm{d}^{\mathrm{c}}}
\newcommand{\Nef}{\mathrm{Nef}}
\newcommand{\Adj}{\mathrm{Adj}}
\newcommand{\DHm}{\mathrm{DH}}
\newcommand{\An}{\mathrm{an}}
\newcommand{\Rec}{\mathrm{Rec}}
\newcommand{\dP}{\mathrm{d}_{\mathrm{P}}}
\newcommand{\ddp}{\mathrm{d}_{\mathrm{P}}'\mathrm{d}_{\mathrm{P}}''}
\newcommand{\ddc}{\mathrm{dd}^{\mathrm{c}}}
\newcommand{\ddL}{\mathrm{d}'\mathrm{d}''}
\newcommand{\PSH}{\mathrm{PSH}}
\newcommand{\CPSH}{\mathrm{CPSH}}
\newcommand{\PSP}{\mathrm{PSP}}
\newcommand{\WPSH}{\mathrm{WPSH}}
\newcommand{\Ent}{\mathrm{Ent}}
\newcommand{\NS}{\mathrm{NS}}
\newcommand{\QPSH}{\mathrm{QPSH}}
\newcommand{\proet}{\mathrm{pro-ét}}
\newcommand{\XL}{(\mathcal{X},\mathcal{L})}
\newcommand{\ii}{\mathrm{i}}
\newcommand{\Ann}{\mathrm{Ann}}
\newcommand{\ExtFun}{\mathcal{E}\mathrm{xt}}
\newcommand{\Cpt}{\mathrm{Cpt}}
\newcommand{\bp}{\bar{\partial}}
\newcommand{\ddt}{\frac{\mathrm{d}}{\mathrm{d}t}}
\newcommand{\dds}{\frac{\mathrm{d}}{\mathrm{d}s}}
\newcommand{\Ep}{\mathcal{E}^p(X,\theta;[\phi])}
\newcommand{\Ei}{\mathcal{E}^{\infty}(X,\theta;[\phi])}
\newcommand{\infs}{\operatorname*{inf\vphantom{p}}}
\newcommand{\sups}{\operatorname*{sup*}}
\newcommand{\colim}{\operatorname*{colim}}
\newcommand{\ddtz}[1][0]{\left.\ddt\right|_{t={#1}}}
\newcommand{\tube}[1][Y]{]{#1}[}
\newcommand{\ddsz}[1][0]{\left.\ddt\right|_{s={#1}}}
\newcommand{\floor}[1]{\left \lfloor{#1}\right \rfloor }
\newcommand{\dec}[1]{\left \{{#1}\right \} }
\newcommand{\ceil}[1]{\left \lceil{#1}\right \rceil }
\newcommand{\Projrel}{\mathcal{P}\mathrm{roj}}
\newcommand{\Weil}{\mathrm{Weil}}
\newcommand{\Cart}{\mathrm{Cart}}
\newcommand{\bWeil}{\mathrm{b}\mathrm{Weil}}
\newcommand{\bCart}{\mathrm{b}\mathrm{Cart}}
\newcommand{\Cond}{\mathrm{Cond}}
\newcommand{\IC}{\mathrm{IC}}
\newcommand{\IH}{\mathrm{IH}}
\newcommand{\Eq}{\mathrm{Eq}}
\newcommand{\cris}{\mathrm{cris}}
\newcommand{\Zar}{\mathrm{Zar}}
\newcommand{\HvbCat}{\overline{\mathcal{V}\mathrm{ect}}}
\newcommand{\BanModCat}{\mathcal{B}\mathrm{an}\mathcal{M}\mathrm{od}}
\newcommand{\DesCat}{\mathcal{D}\mathrm{es}}
\newcommand{\RingCat}{\mathcal{R}\mathrm{ing}}
\newcommand{\SchCat}{\mathcal{S}\mathrm{ch}}
\newcommand{\AbCat}{\mathcal{A}\mathrm{b}}
\newcommand{\RSCat}{\mathcal{R}\mathrm{S}}
\newcommand{\LRSCat}{\mathcal{L}\mathrm{RS}}
\newcommand{\CLRSCat}{\mathbb{C}\text{-}\LRSCat}
\newcommand{\CRSCat}{\mathbb{C}\text{-}\RSCat}
\newcommand{\CLA}{\mathbb{C}\text{-}\mathcal{L}\mathrm{A}}
\newcommand{\CASCat}{\mathbb{C}\text{-}\mathcal{A}\mathrm{n}}
\newcommand{\LiuCat}{\mathcal{L}\mathrm{iu}}
\newcommand{\BanCat}{\mathcal{B}\mathrm{an}}
\newcommand{\BanAlgCat}{\mathcal{B}\mathrm{an}\mathcal{A}\mathrm{lg}}
\newcommand{\AnaCat}{\mathcal{A}\mathrm{n}}
\newcommand{\LiuAlgCat}{\mathcal{L}\mathrm{iu}\mathcal{A}\mathrm{lg}}
\newcommand{\AlgCat}{\mathcal{A}\mathrm{lg}}
\newcommand{\SetCat}{\mathcal{S}\mathrm{et}}
\newcommand{\ModCat}{\mathcal{M}\mathrm{od}}
\newcommand{\GerCat}{\mathcal{G}\mathrm{er}}
\newcommand{\AnaGerCat}{\mathbb{C}\text{-}\GerCat}
\newcommand{\TopCat}{\mathcal{T}\mathrm{op}}
\newcommand{\CohCat}{\mathcal{C}\mathrm{oh}}
\newcommand{\SolCat}{\mathcal{S}\mathrm{olid}}
\newcommand{\AffCat}{\mathcal{A}\mathrm{ff}}
\newcommand{\AffAlgCat}{\mathcal{A}\mathrm{ff}\mathcal{A}\mathrm{lg}}
\newcommand{\QcohLiuAlgCat}{\mathcal{L}\mathrm{iu}\mathcal{A}\mathrm{lg}^{\mathrm{QCoh}}}
\newcommand{\LiuMorCat}{\mathcal{L}\mathrm{iu}}
\newcommand{\Isom}{\mathcal{I}\mathrm{som}}
\newcommand{\Cris}{\mathcal{C}\mathrm{ris}}
\newcommand{\Pro}{\mathrm{Pro}-}
\newcommand{\Fin}{\mathcal{F}\mathrm{in}}
\newcommand{\norms}[1]{\left\|#1\right\|}
\newcommand{\HPDDiff}{\mathbf{D}\mathrm{iff}}
\newcommand{\Menn}[2]{\begin{bmatrix}#1\\#2\end{bmatrix}}
\newcommand{\Fins}{\widehat{\Vect}^F}
\newcommand\blfootnote[1]{%
  \begingroup
  \renewcommand\thefootnote{}\footnote{#1}%
  \addtocounter{footnote}{-1}%
  \endgroup
}


\makeatletter
\newcommand*{\addFileDependency}[1]{% argument=file name and extension
  \typeout{(#1)}
  \@addtofilelist{#1}
  \IfFileExists{#1}{}{\typeout{No file #1.}}
}
\makeatother



\newcommand*{\myexternaldocument}[2]{%
\externaldocument[#1]{#2}%
\addFileDependency{#2.tex}%
\addFileDependency{#2.aux}%
%\addFileDependency{#2.pdf}%
}


%\iffalse

\myexternaldocument{Introduction-}{Introduction}
\myexternaldocument{Topology-}{Topology-Bornology}
\myexternaldocument{Banach-}{Banach-Rings}
\myexternaldocument{Commutative-}{Commutative-Algebra}



\myexternaldocument{Local-}{Local-Algebras}
\myexternaldocument{Complex-}{Complex-Analytic-Spaces}
\myexternaldocument{ConstructionComplex-}{Constructions-Complex-Spaces}
\myexternaldocument{PropertyComplex-}{Properties-Complex-Spaces}
\myexternaldocument{GPropertyComplex-}{Global-Properties-Complex-Spaces}
\myexternaldocument{Analytic-}{Analytic-Sets}
\myexternaldocument{Morphisms-}{Morphisms-Complex-Spaces}

\myexternaldocument{Affinoid-}{Affinoid-Algebras}
\myexternaldocument{Berkovich-}{Berkovich-Analytic-Spaces}
\myexternaldocument{BerkProperty-}{Properties-Berkovich-Spaces}
%\fi


\bibliography{Ymir}

\endinput
\title{Ymir}
\begin{document}
\maketitle
\tableofcontents

\chapter*{Berkovich analytic spaces}\label{chap-Berkovich}

\section{Introduction}\label{sec-introduction}


The main references of this chapter: \cite{Berk93}, \cite{Berk12}, \cite{Tem04}, \cite{Tem00}, \cite{Duc18}.


\section{The category of Berkovich analytic spaces}
Let $(k,|\bullet|)$ be a complete non-Archimedean valued field and $H$ be a subgroup of $\mathbb{R}_{>0}$ such that $|k^{\times}|\cdot H\neq \{1\}$.
 
\begin{definition}
    Let $X$ be a locally Hausdorff space and $\tau$ be a net of compact subsets. A \emph{$k_H$-affinoid atlas} $\mathcal{A}$ on $X$ with the net $\tau$ is a map which assigns 
    \begin{enumerate}
        \item to each $V\in \tau$, a $k_H$-affinoid algebra $A_V$ and a homeomorphism $\varphi_V:\Sp A_V\rightarrow V$;
        \item to each $U,V\in \tau$, $U\subseteq V$, a morphism of $k_H$-affinoid algebras $\alpha_{V/U}:A_V\rightarrow A_U$ representing a $k_H$-affinoid domain $\Sp A_U$ in $\Sp A_V$ such that the following diagram commutes
        \[
            \begin{tikzcd}
                \Sp A_U \arrow[d, "\varphi_U"] \arrow[r, "\Sp \alpha_{V/U}"] & \Sp A_V \arrow[d, "\varphi_V"] \\
                U \arrow[r]                                                  & V                             
            \end{tikzcd}.  
        \]
    \end{enumerate}
    The triple $(X,\mathcal{A},\tau)$ as above is called a \emph{$k_H$-analytic space}.

    A \emph{morphism} between atlases $\mathcal{A}$ and $\mathcal{A}'$ on $X$ with the net $\tau$ is an assignment that with each $V\in \tau$, one associates a morphism of $k_H$-affinoid algebras $\beta_V:A_V\rightarrow A_V'$ such that
    \begin{enumerate}
        \item for each $V\in \tau$, the following diagram is commutative:
        \[
            \begin{tikzcd}
                \Sp A_V' \arrow[d, "\varphi_V'"] \arrow[r, "\Sp \beta_V"] & \Sp A_V \arrow[ld, "\varphi_V"] \\
                V                                                         &                                
                \end{tikzcd};
        \]
        \item for each $U,V\in \tau$, $U\subseteq V$, the following diagram is commutative:
        \[
            \begin{tikzcd}
                A_V \arrow[r, "\alpha_{V/U}"] \arrow[d, "\beta_V"] & A_U \arrow[d, "\beta_U"] \\
                A_V' \arrow[r, "\alpha'_{V/U}"]                    & A_U'                    
            \end{tikzcd}
        \]
    \end{enumerate}
    Here we have denoted the data associated with $\mathcal{A}'$ with a prime. In this way, the atlases on $X$ with the net $\tau$ form a category.
\end{definition}
We remind the readers that by our convention a compact space is Hausdorff. 

By Condition~(2), it $W\subseteq U\subseteq V$ are three sets in $\tau$, then $\alpha_{V/U}\circ\alpha_{U/W}=\alpha_{V/W}$.

\begin{remark}\label{rmk-affdomainident}
    As a convention, we will denote the atlas by capital  letters in caligraphic font and the affinoid algebras by the same letter in roman font. We will usually omit the maps $\varphi_U$'s by identifying $\Sp A_U$ with $U$. We will say $U$ is a $k_H$-affinoid domain in $V$.    
\end{remark}


\begin{remark}
    Our definition is a special case of the original definitions in \cite{Berk93}. This seems to be the most important case though.
\end{remark}

\begin{lemma}\label{lma-innetaffinoneaffinall}
    Let $(X,\mathcal{A},\tau)$ be a $k_H$-analytic space, $U\in \tau$ and $W$ is a $k_H$-affinoid domain in $U$. Then for any $V\in \tau$ containing $W$, $W$ is a $k_H$-affinoid domain in $V$.
\end{lemma}
\begin{proof}
    As $\tau|_{U\cap V}$ is a net and $W$ is compact, we can find $U_1,\ldots,U_n\in \tau_{U\cap V}$ with $W\subseteq U_1\cup \cdots\cup  U_n$. As $W$, $U_i$ are $k_H$-affinoid domains in $U$, $W_i=W\cap U_i$ is a $k_H$-affinoid domain in $U_i$ for all $i=1,\ldots,n$ by \cref{Affinoid-cor-fiberproductaffdomain} in \nameref{Affinoid-chap-affinoid}. It follows from \cref{Affinoid-cor-affinoiddomaintransi} and \cref{Affinoid-cor-fiberproductaffdomain} in \nameref{Affinoid-chap-affinoid} that $W_i$ and $W_i\cap W_j$ are both $k_H$-affinoid domains in $V$ for $i,j=1,\ldots,n$. So $W$ is a compact $k_H$-analytic domain in $V$.

    By \cref{Affinoid-prop-compactanalydomainaffinoid} in \nameref{Affinoid-chap-affinoid},
    \[
        A_W:=\ker  \left( \prod_{i=1}^n A_{W_i} \rightarrow \prod_{i,j=1}^n A_{W_i\cap W_j}\right)
    \]
    is $k_H$-affinoid and $\Sp A_W\rightarrow \Sp A$ induces a hoemomorphism $\Sp A_W\rightarrow W$ by \cref{Affinoid-prop-affdomainhoemo} in \nameref{Affinoid-chap-affinoid}. By \cref{Affinoid-prop-compactanalydomainaffinoid} in \nameref{Affinoid-chap-affinoid} again, $W$ is affinoid in $V$. 
\end{proof}

\begin{definition}
    Let $(X,\mathcal{A},\tau)$ be a $k_H$-analytic space. We define $\bar{\tau}$ as the set of all $W\subseteq X$ such that there is $U\in \tau$ containing $W$ and $W$ is $k_H$-affinoid in $U$.
\end{definition}

\begin{lemma}
    Let $(X,\mathcal{A},\tau)$ be a $k_H$-analytic space. Then $\bar{\tau}$ is a net on $X$ and there is a $k_H$-affinoid atlas $\overline{\mathcal{A}}$ on $X$ with the net $\bar{\tau}$  extending $\mathcal{A}$. Moreover, the $k_H$-affinoid atlas $\overline{\mathcal{A}}$ on $X$ with the net $\bar{\tau}$  extending $\mathcal{A}$ is unique up to a canonical isomorphism.
\end{lemma}
\begin{proof}
    \textbf{Step~1}.
    We first show that $\bar{\tau}$ is a net. Let $U,V\in \bar{\tau}$ and $x\in U\cap V$. Take $U',V'\in \tau$ containing $U$ and $V$ respectively. Take $n\in \mathbb{Z}_{>0}$ and $W_1,\ldots,W_n\in \tau$ such that 
    \begin{enumerate}
        \item $x\in W_1\cap \cdots \cap W_n$;
        \item $W_1\cup\cdots\cup W_n$ is a neighbourhood of $x$ in $U'\cap V'$.
    \end{enumerate}
    This is possible because $\tau|_{U'\cap V'}$ is a quasi-net by assumption.

    By \cref{lma-innetaffinoneaffinall}, $U$ (resp. $V$) and $W_1,\ldots,W_n$ are $k_H$-affinoid domains in $U'$ (resp. $V'$).

    By \cref{Affinoid-cor-fiberproductaffdomain} in \nameref{Affinoid-chap-affinoid}, $U_i:=U\cap W_i$ (resp. $V_i:=V\cap W_i$) is a $k_H$-affinoid domain in $W_i$  for $i=1,\ldots,n$. By \cref{Affinoid-cor-fiberproductaffdomain} in \nameref{Affinoid-chap-affinoid} again, $U_i\cap V_i$ is a $k_H$-affinoid domain in $W_i$ for $i=1,\ldots,n$. So $U_i\cap V_i\in \bar{\tau}|_{U\cap V}$ for $i=1,\ldots,n$. But
    \[
        \bigcup_{i=1}^n U_i\cap V_i=(U\cap V)\cap \bigcup_{i=1}^n W_i,  
    \]
    so $\bigcup_{i=1}^n U_i\cap V_i$ is a neighbourhood of $x$ in $U\cap V$ and $x\in \bigcap_{i=1}^n U_i\cap V_i$. It follows that $\bar{\tau}$ is a net.

    \textbf{Step~2}. We extend the $k_H$-affinoid atlas $\mathcal{A}$.

    For each $V\in \bar{\tau}$, we fix a $V'\in \tau$ containing $V$. 
    
    By \cref{lma-innetaffinoneaffinall}, $V$ is a $k_H$-affinoid domain in $V'$. Let $A_{V'}\rightarrow A_V$ be the morphism of $k_H$-affinoid algebras representing the $k_H$-affinoid domain $V$ in $\Sp A_{V'}$. We define the homeomorphism $\varphi_V:\Sp A_V\rightarrow V$ as the morphism induced by $\Sp A_V\rightarrow \Sp A$.


    For $U,V\in \bar{\tau}$ with $U\subseteq V$, we want to define $\alpha_{V/U}:A_V\rightarrow A_U$. We handle two cases. When $V\in \tau$, as $\tau|_{U'\cap V}$ is a quasi-net, we can find $n\in \mathbb{Z}_{>0}$ and $U_1,\ldots,U_n\in \tau|_{U'\cap V}$ such that
    \[
        U=\bigcup_{i=1}^n U_i.  
    \]
    By \cref{lma-innetaffinoneaffinall}, $U_1,\ldots,U_n$ are $k_H$-affinoid domains in $U'$ and in $V$. By \cref{Affinoid-thm-Tateacyc} in \nameref{Affinoid-chap-affinoid},
    \[
        A_U\cn \ker \left( \prod_{i=1}^n A_{U_i}\rightarrow \prod_{i,j=1}^n A_{U_i\cap U_j} \right).  
    \]
    So the morphism $\alpha_{V/U_i}:A_V\rightarrow A_{U_i}$ and $A_{V/U_{i}\cap U_j}:\alpha_{V/U_i}:A_V\rightarrow A_{U_i\cap U_j}$ for $i=1,\ldots,n$ and $j=1,\ldots,n$ induces a morphism $\alpha_{V/U}:A_V\rightarrow A_U$. Observe that $\alpha_{V/U}$ represents the $k_H$-affinoid domain $U$ in $V$, so it is independent of the choice of $U_1,\ldots,U_n$.

    More generally, when $V\in \bar{\tau}$, we have constructed a morphism $\alpha_{V'/U}:A_{V'}\rightarrow A_U$ representing the $k_H$-affinoid domain $U$ in $V'$, it follows that $U$ is a $k_H$-affinoid domain in $V$, and we therefore get the desired morphism $\alpha_{V/U}:A_V\rightarrow A_U$.

    It is easy to verify that the constructions gives a $k_H$-affinoid atlas with the net $\bar{\tau}$ extending $\mathcal{A}$. The uniqueness of the extension is immediate.
\end{proof}

\begin{definition}
    Let $(X,\mathcal{A},\tau)$ and $(X',\mathcal{A}',\tau')$ be $k_H$-analytic spaces. A \emph{strong morphism} $\varphi:(X,\mathcal{A},\tau)\rightarrow (X',\mathcal{A}',\tau')$ is a pair consisting of 
    \begin{enumerate}
        \item a continuous map $\varphi:X\rightarrow X'$ such that for each $V\in \tau$, there is $V'\in \tau'$ with $\varphi(V)\subseteq V'$;
        \item for each $V\in \tau$, $V'\in \tau'$ with $\varphi(V)\subseteq V'$, a morphism of $k_H$-affinoid spectra $\varphi_{V/V'}:V\rightarrow V'$ 
    \end{enumerate}
    such that for each $V,W\in \tau$, $V',W'\in \tau'$ satisfying $V\subseteq W$, $W'\subseteq W'$, $\varphi(V)\subseteq V'$ and $\varphi(W)\subseteq W'$, the following diagram commutes:
    \[
        \begin{tikzcd}
            V \arrow[r, "\varphi_{V/V'}"] \arrow[d] & V' \arrow[d] \\
            W \arrow[r, "\varphi_{W/W'}"]           & W'          
        \end{tikzcd}.    
    \]
\end{definition}
Recall our convention \cref{rmk-affdomainident}, the morphism $\varphi_{V/V'}$ means a morphism $A'_{V'}\rightarrow A_V$ of $k_H$-affinoid algebras making the following diagram commutative
\[
    \begin{tikzcd}
        \Sp A_V \arrow[d, "\varphi_V"] \arrow[r] & \Sp A'_{V'} \arrow[d, "\varphi'_{V'}"] \\
        V \arrow[r, "\varphi"]                   & V'                                    
    \end{tikzcd}.  
\]
We will continue our identifications as in \cref{rmk-affdomainident} to simplify our notations.

\begin{proposition}\label{prop-strongmorphismext}
    Let $(X,\mathcal{A},\tau)$ and $(X',\mathcal{A}',\tau')$ be $k_H$-analytic spaces. Let $\varphi:(X,\mathcal{A},\tau)\rightarrow (X',\mathcal{A}',\tau')$ be a strong morphism. Then $\varphi$ extends uniquely to a strong morphism $\varphi:(X,\bar{\mathcal{A}},\bar{\tau})\rightarrow (X',\overline{\mathcal{A}'},\overline{\tau'})$.
\end{proposition}
\begin{proof}
    Let $U\in \bar{\tau}$, $U'\in \overline{\tau'}$ with $\varphi(U)\subseteq U'$. Take $V\in \tau$ and $V'\in \tau'$ containing $U$ and $U'$ respectively. By \cref{lma-innetaffinoneaffinall}, $U$ (resp. $V$) is a $k_H$-affinoid domain in $V$ (resp. $V'$). Take $W\in \tau'$ with $\varphi(V)\subseteq W'$. Then in particular, $\varphi(U)\subseteq W'$. As $\tau'|_{V'\cap W'}$ is a quasi-net and $\varphi(U)$ is compact, we can find $n\in \mathbb{Z}_{>0}$ and $W_1,\ldots,W_n\in \tau'|_{V'\cap W}$ such that 
    \[
        \varphi(U)\subseteq W_1\cup\cdots\cup W_n.
    \]
    Now $W_i$ is a $k_H$-affinoid domain in $W'$ by \cref{lma-innetaffinoneaffinall}, so $V_i:=\varphi^{-1}_{V/W'}(W_i)$ is an affinoid domain in $V$ by \cref{Affinoid-cor-fiberproductaffdomain} in \nameref{Affinoid-chap-affinoid}, and we have an induced morphism $V_i\rightarrow W_i$ for $i=1,\ldots,n$. This morphism in turn induces a morphism of $k_H$-affinoid spectra
    \[
        U_i:=U\cap V_i\rightarrow U_i':=U'\cap W_i  \rightarrow U'
    \]
    for $i=1,\ldots,n$. These morphisms are compatible on their intersections by construction. So by \cref{Affinoid-thm-Tateacyc} in \nameref{Affinoid-chap-affinoid}, they glue together to a morphism of $k_H$-affinoid spectra $\bar{\varphi}_{U/U'}:U\rightarrow U'$. It is easy to see that this construction defines a strong morphism.

    As for the uniqueness, it suffices to show that the morphism $U_i\rightarrow U'_i$ is uniquely determined for $i=1,\ldots,n$. In other words, we need to show that the dotted arrow that makes the following diagram commutes is unique:
    \[
        \begin{tikzcd}
            U_i \arrow[d] \arrow[r, dotted] & U'_i \arrow[d] \\
            V \arrow[r, "\varphi_{V/W'}"]   & W'            
        \end{tikzcd}  
    \]
    for $i=1,\ldots,n$.
    It suffices to apply the universal property of the $k_H$-affinoid domain $U_i'\rightarrow W'$.
\end{proof}

\begin{definition}\label{def-strongmorphismcompo}
    Let $(X,\mathcal{A},\tau)$, $(X',\mathcal{A}',\tau')$, $(X'',\mathcal{A}'',\tau'')$ be $k_H$-analytic spaces. Let 
    \[
        \varphi:  (X,\mathcal{A},\tau)\rightarrow (X',\mathcal{A}',\tau'),\quad \psi: (X',\mathcal{A}',\tau')\rightarrow (X'',\mathcal{A}'',\tau'')
    \]
    be strong morphisms. We will define their \emph{composition} $\chi=\psi\circ \varphi$ as follows. The underlying map of topological spaces is just the composition of the unlerlying maps of topological spaces corresponding to $\psi$ and $\varphi$.
    
    Let $\bar{\varphi}$ and $\bar{\psi}$ be the extensions of  $\varphi$ and $\psi$ to $\bar{\tau}$ and $\overline{\tau'}$ as in \cref{prop-strongmorphismext}. 
    
    Given $V\in \tau$ and $V''\in \tau''$ with $\chi(V)\subseteq V''$, we need to define a morphism of $k_H$-affinoid spectra $\chi_{V/V''}:V\rightarrow V''$. Take $V'\in \tau'$ and $U''\in \tau''$ such that $\varphi(V)\subseteq V'$ and $\psi(V')\subseteq U''$. Since $\chi(V)\subseteq U''\cap V''$ and $V$ is compact, we can take $n\in \mathbb{Z}_{>0}$ and $V_1'',\ldots,V_n''\in \tau''|_{U''\cap V''}$ with $\chi(V)\subseteq V_1''\cup\cdots\cup V_n''$. Then $V_i':=\psi^{-1}_{V'/U''}(V_i'')$ and $V_i:=\varphi_{V/V'}^{-1}(V_i')$ are $k_H$-affinoid domains in $V'$ and $V$ respectively for $i=1,\ldots,n$ and $V=V_1\cup\cdots\cup V_n$. The morphisms $\bar{\varphi}$ and $\bar{\psi}$ then induce a morphism $V_i\rightarrow V_i''\rightarrow V$ of $k_H$-affinoid spectra. These morphisms are clearly compatible on the intersections and hence induce a morphism $V\rightarrow V''$ of $k_H$-affinoid spectra by \cref{Affinoid-thm-Tateacyc} in \nameref{Affinoid-chap-affinoid}.

    It is easy to verify that $\psi\circ \varphi$ is a strong morphism.

    In this way, we get a category $k_H\text{-}\widetilde{\AnaCat}$ of $k_H$-analytic spaces. 
\end{definition}

\begin{definition}
    Let $(X,\mathcal{A},\tau)$ and $(X',\mathcal{A}',\tau')$ be $k_H$-analytic spaces. A strong morphism $\varphi:(X,\mathcal{A},\tau)\rightarrow (X',\mathcal{A}',\tau')$ is said to be a \emph{quasi-isomorphism} if 
    \begin{enumerate}
        \item $\varphi$ is a homeomorphism between $X$ and $X'$;
        \item for any pair $V\in \tau$ and $V'\in \tau'$ with $\varphi(V)\subseteq V'$, $\Sp \varphi_{V/V'}$ identifies $V$ with an affinoid domain in $V'$.
    \end{enumerate}
\end{definition}

\begin{lemma}\label{lma-strongmorphisminversecptanalyticdomain}
    Let $(X,\mathcal{A},\tau)$ and $(X',\mathcal{A}',\tau')$ be $k_H$-analytic spaces and $\varphi:(X,\mathcal{A},\tau)\rightarrow (X',\mathcal{A}',\tau')$ be a strong morphism. Then for any $V\in \bar{\tau}$ and $V'\in \overline{\tau'}$, the intersection $V\cap \varphi^{-1}(V')$ is a compact $k_H$-analytic domain in $V$.
\end{lemma}
\begin{proof}
    Take $U'\in \overline{\tau'}$ with $\varphi(V)\subseteq U'$. As $\tau|_{U'\cap V'}$ is a quasi-net, we can find $n\in \mathbb{Z}_{>0}$ and $U_1',\ldots,U_n'\in \tau|_{U'\cap V'}$ with $\varphi(V)\subseteq U_1'\cup \cdots \cup U_n'$ and 
    \[
        V\cap \varphi^{-1}(V')=\bigcup_{i=1}^n \varphi^{-1}_{V/U}(U_i').
    \]
\end{proof}

\begin{lemma}\label{lma-quasiisomrightmultiplicative}
    The system of quasi-isomorphisms in $k_H\text{-}\widetilde{\AnaCat}$ is a right multiplicative system.
\end{lemma}
For the notion of right multiplicative system, we refer to \cite[\href{https://stacks.math.columbia.edu/tag/04VC}{Tag 04VC}]{stacks-project}.

\begin{proof}
    We verify the three axioms as in \cite[\href{https://stacks.math.columbia.edu/tag/04VC}{Tag 04VC}]{stacks-project}.

    \textbf{RMS1}. The identity is clear a quasi-isomorphism. It remains to verify that the composition of quasi-isomorphisms is still a quasi-isomorphism.

    We take $\varphi,\psi$ as in \cref{def-strongmorphismcompo}. We will use the same notations as in \cref{def-strongmorphismcompo}. We need to show that $V\rightarrow V''$ identifies $V$ with a $k_H$-affinoid domain in $V''$. From the construction, we know that $\varphi$ identifies $V_i$ with a $k_H$-affinoid domain in $V_i'$ and $\psi$ identifies $V_i'$ with a $k_H$-affinoid domain in $V_i''$ for $i=1,\ldots,n$. In particular, $\chi(V)$ is a compact $k_H$-analytic domain in $V''$. It follows from \cref{Affinoid-prop-compactanalydomainaffinoid} in \nameref{Affinoid-chap-affinoid} that $\chi(V)$ is a $k_H$-affinoid domain in $V''$.

    \textbf{RMS2}.
    If $\varphi:(X,\mathcal{A},\tau)\rightarrow (X',\mathcal{A}',\tau')$ and $f:(\widetilde{X'},\widetilde{\mathcal{A}'},\widetilde{\tau'})\rightarrow (X',\mathcal{A}',\tau')$ are given strong morphisms of $k_H$-analytic spaces and $g$ is a quasi-isomorphism, then there are $k_H$-analytic space $(\widetilde{X},\widetilde{\mathcal{A}},\widetilde{\tau})$ and strong morphisms $\tilde{\varphi}:(\widetilde{X},\widetilde{\mathcal{A}},\widetilde{\tau})\rightarrow (\widetilde{X'},\widetilde{\mathcal{A}'},\widetilde{\tau'})$ and $f:(\widetilde{X},\widetilde{\mathcal{A}},\widetilde{\tau})\rightarrow (X,\mathcal{A},\tau)$ such that $f$ is a quasi-isomorphism and the following diagram commutes:  
    \[
        \begin{tikzcd}
            {(\widetilde{X},\widetilde{\mathcal{A}},\widetilde{\tau})} \arrow[r, "\tilde{\varphi}", dotted] \arrow[d, "f", dotted] & {(\widetilde{X'},\widetilde{\mathcal{A}'},\widetilde{\tau'})} \arrow[d, "g"] \\
            {(X,\mathcal{A},\tau)} \arrow[r, "\varphi"]                                                                            & {(X',\mathcal{A}',\tau')}                                                   
        \end{tikzcd}.
    \]  
    We may assume that $\widetilde{X'}=X'$. Then $\widetilde{\tau'}\subseteq \overline{\tau'}$. We let $\tilde{X}=X$. Let $\tilde{\tau}$ be the family of all $V\in \bar{\tau}$ for which there is $\widetilde{V'}\in\widetilde{\tau'}$ with $\varphi(V)\subseteq \widetilde{V'}$. By \cref{lma-strongmorphisminversecptanalyticdomain}, $\tilde{\tau}$ is a net on $\tilde{X}$. The $k_H$-atlas $\bar{\mathcal{A}}$ defines a $k_H$-affinoid atlas $\tilde{\mathcal{A}}$ with the net $\tilde{\tau}$. The strong morphism $\bar{\varphi}$ induces $\tilde{\varphi}$. The morphism $f$ is the canonical quasi-isomorphism. It is immediate that these constructions satisfy the desired conditions.

    \textbf{RMS3}. If $\varphi,\psi:(X,\mathcal{A},\tau)\rightarrow (X',\mathcal{A}',\tau')$ are strong morphisms of $k_H$-analytic spaces and there is a quasi-isomorphism $g:(X',\mathcal{A}',\tau')\rightarrow (\widetilde{X'},\widetilde{\mathcal{A}'},\widetilde{\tau'})$ of $k_H$-analytic spaces such that $g\circ \varphi=g\circ \psi$, then there is a quasi-isomorphism $f:(\tilde{X},\tilde{\mathcal{A}},\tilde{\tau})\rightarrow (X,\mathcal{A},\tau)$ with $\varphi\circ f=\psi\circ f$.

    We will in fact show that $\varphi=\psi$. It is clear that they coincide as maps of topological spaces. Let $V\in \tau$, $V'\in \tau'$ such that $\varphi(V)\subseteq V'$. Take $\widetilde{V'}\in \widetilde{\tau'}$ with $g(V')\subseteq \widetilde{V'}$. Then we have two morphisms of $k$-affinoid spectra $\varphi_{V/V'},\psi_{V/V'}:V\rightarrow V'$ such that their compositions with $g_{V'/\widetilde{V'}}$ coincide.  As $V'$ is an affinoid domain in $\widetilde{V'}$, it follows that $\varphi_{V/V'}=\psi_{V/V'}$ by the universal property.
\end{proof}


\begin{definition}
    The category $k_H\text{-}\AnaCat$ is the right category of fractions of $k_H\text{-}\widetilde{\AnaCat}$ with respect to the system of quasi-isomorphisms.  A morphism in $k_H\text{-}\AnaCat$ is called a \emph{morphism} between $k_H$-analytic spaces.
\end{definition}
We refer to \cite[\href{https://stacks.math.columbia.edu/tag/04VB}{Tag 04VB}]{stacks-project} for the definition of right category of fractions. 

For later references, we explicitly write down the morphisms in $k_H\text{-}\AnaCat$. 
\begin{lemma}
    Let $\varphi:(X,\mathcal{A},\tau)\rightarrow (X',\mathcal{A}',\tau')$ be a morphism of $k_H$-analytic spaces. We define a partial order on the set of nets on $X$: $\tau_1\preceq \tau_0$ if $\tau_1\subseteq \overline{\tau_0}$. Then the set of nets is a directed set and
    \[
        \Hom_{k_H\text{-}\AnaCat}\left((X,\mathcal{A},\tau),(X',\mathcal{A}',\tau') \right)=\varinjlim_{\sigma\preceq \tau}\Hom_{k_H\text{-}\widetilde{\AnaCat}}\left((X,\mathcal{A}_{\sigma},\sigma),(X',\mathcal{A}',\tau') \right)
    \]
    in the category of sets, where $\mathcal{A}_{\sigma}$ is induced by $\overline{\mathcal{A}}$. The transition maps are all injective.
\end{lemma}
\begin{proof}
    This follows immediately from the definition.
\end{proof}

\begin{definition}
    Let  $(X,\mathcal{A},\tau)$ be a $k_H$-analytic space. We say a subset $W\subseteq X$ is \emph{$\tau$-special} if it is compact and there exist $n\in \mathbb{Z}_{>0}$ and a covering $W=W_1\cup \cdots \cup W_n$  with $W_i\in \tau$, $W_i\cap W_j\in \tau$ for all $i,j=1,\ldots,n$ and the natural map
    \[
        A_{W_i}\hat{\otimes}_k A_{W_j}\rightarrow A_{W_i\cap W_j}  
    \]
    is an admissible epimorphism.

    The covering $W_1,\ldots,W_n$ is called a \emph{$\tau$-special covering} of $W$.
\end{definition}
Under our convention, the assumption means that $W_i\cap W_j\rightarrow W_i\times W_j$ is a closed immersion of $k_H$-affinoid spectra.


\begin{example}
    Let  $(X,\mathcal{A},\tau)$ be a $k_H$-analytic space.
    Suppose that $V\in \tau$ and $W$ is a compact $k_H$-analytic domain in $V$. Let $n\in \mathbb{Z}_{>0}$ and $W=W_1\cup \cdots \cup W_n$  with $W_i\in \tau$, $W_i\cap W_j\in \tau$ for all $i,j=1,\ldots,n$. Then $\{W_i\}_i$ is a $\tau$-special covering of $W$. This follows from \cref{Affinoid-cor-spcatclosedimm} in \nameref{Affinoid-chap-affinoid}.
\end{example}


\begin{lemma}\label{lma-tauspecialanycov}
    Let  $(X,\mathcal{A},\tau)$ be a $k_H$-analytic space and $W$ be a $\tau$-special subset of $X$. If $U,V\in \tau|_W$, then $U\cap V\in \bar{\tau}$ and the natural map
    \[
        A_U\hat{\otimes}_k A_V\rightarrow A_{U\cap V}
    \]
    is an admissible epimorphism.
\end{lemma}

\begin{proof}
    Let $n\in \mathbb{Z}_{>0}$ and $W_1,\ldots,W_n$ be a $\tau$-special covering of $W$.
    As $U\cap W_i$ and $V\cap W_i$ are compact for $i=1,\ldots,n$, we can find $m_i\in \mathbb{Z}_{>0}$ (resp. $k_i\in \mathbb{Z}_{>0}$) and finite coverings $U_{i1},\ldots,U_{im_i}\in \tau$ of $U\cap W_i$ (resp. $V_{i1},\ldots,V_{ik_i}\in \tau$ of $V\cap W_i$). 

    Observe that $U_{ik}\cap V_{jl}$ is a $k_H$-affinoid domain in $U\cap V$, hence $U_{ik}\cap V_{jl}\in \bar{\tau}$ for any $i,j=1,\ldots,n$, $k=1,\ldots,m_i$ and $l=1,\ldots,k_l$. By \cref{Affinoid-prop-closedimmbasechange} in \nameref{Affinoid-chap-affinoid}, $U_{ik}\cap V_{jl}\rightarrow U_{ik}\times V_{jl}$ is a closed immersion since $W_i\cap W_j\rightarrow W_i\times W_j$ is by our assumption. 
    
    Consider the finite convering 
    \[
        \mathcal{U}:=\left\{U_{ik}\times V_{jl}:i,j=1,\ldots,n; k=1,\ldots,m_i; l=1,\ldots,k_l \right\}
    \]
    of $U\times V$. For each tuple $(i,j,k,l)$, $A_{U_{ik}\cap V_{jl}}$ is a finite $A_{U_{ik}\times V_{jl}}$-algebra. By \cref{Affinoid-thm-Kiehl} in \nameref{Affinoid-chap-affinoid}, we can construct a finite $A_{U\times V}$-algebra $A_{U\cap V}$ inducing all of these $A_{U_{ik}\cap V_{jl}}$'s. By \cref{Affinoid-prop-strictlykafffinite2} in \nameref{Affinoid-chap-affinoid}, $A_{U\cap V}$ is $k_H$-affinoid.

    As $\mathcal{U}$ is a finite $k_H$-affinoid covering of $U\times V$, $\{A_{U_{ik}\cap V_{jl}}\}_{i,k,j,l}$ is a finite $k_H$-affinoid covering of $U\cap V$ by \cref{Affinoid-cor-fiberproductaffdomain} in \nameref{Affinoid-chap-affinoid}.
    In particular, we have a natural homeomorphism
    \[
        \Sp A_{U\cap V}\cn U\cap V.  
    \]
    Observe that $A_U\hat{\otimes}_k A_V \rightarrow A_{U\cap V}$ is surjective. We endow $A_{U\cap V}$ with the structure of finite $A_U\hat{\otimes}_k A_V$-Banach algebras by \cref{Affinoid-prop-Banachalgebrafiniteforget} in \nameref{Affinoid-chap-affinoid}. Then $A_U\hat{\otimes}_k A_V \rightarrow A_{U\cap V}$ is an admissible epimorphism by \cref{Affinoid-prop-finitemodulemapadmi} in \nameref{Affinoid-chap-affinoid}.

    On the other hand $U\cap V$ is a compact $k_H$-analytic domain in $U$, so by \cref{Affinoid-prop-compactanalydomainaffinoid} in \nameref{Affinoid-chap-affinoid}, $U\cap V$ is a $k_H$-affinoid in $U$. In particular, $U\cap V\in \bar{\tau}$.
\end{proof}

\begin{lemma}\label{lma-AWspecial}
    Let  $(X,\mathcal{A},\tau)$ be a $k_H$-analytic space and $W\subseteq X$ be a $\tau$-special set. Then for any finite covering $\{W_i\}_{i\in I}$ of $W$ with $W_i\in \tau$ for $i\in I$, the Banach $k$-algebra
    \[
        A_W:=\ker\left(\prod_{i\in I}A_{W_i}\rightarrow A_{W_i\cap W_j}  \right)
    \]
    does not depend on the choice of $\{W_i\}_{i\in I}$ up to canonical isomorphisms.

    Moreover, we have a canonical map $W\rightarrow \Sp A_W$, which does not depend on the choice of the covering modulo the canonical isomorphism between $A_W$.
\end{lemma}
\begin{proof}
    It follows from \cref{lma-tauspecialanycov} that the covering $\{W_i\}_{i\in I}$ is $\tau$-special. It suffices to apply the same argument of \cref{Affinoid-lma-compactanalyticdomainring} in \nameref{Affinoid-chap-affinoid}.
\end{proof}

\begin{definition}\label{def-tauhat}
    Let  $(X,\mathcal{A},\tau)$ be a $k_H$-analytic space. Let $\hat{\tau}$ denote the collection of $\bar{\tau}$-special subsets $W\subseteq X$ such that 
    \begin{enumerate}
        \item $A_W$ is $k$-affinoid;
        \item the natural map $W\rightarrow \Sp A_W$ is bijective;
        \item there is a $\bar{\tau}$-special covering $\{W_i\}_{i\in I}$ of $W$ such that $W_i$ is a $k$-affinoid domain in $W$ for $i\in I$.
    \end{enumerate}

    The sets from $\hat{\tau}$ are called \emph{$k_H$-affinoid domains in $(X,\mathcal{A},\tau)$}.
\end{definition}
Observe that $W$ is $k_H$-affinoid and $W_i$ is a $k_H$-affinoid domain in $W$ by \cref{Affinoid-cor-coveringkhimplykh} in \nameref{Affinoid-chap-affinoid}. Condition~(3) holds for any $\bar{\tau}$-special covering.




\begin{proposition}\label{prop-tauhatbasic}
    Let  $(X,\mathcal{A},\tau)$ be a $k_H$-analytic space. Then $\hat{\tau}$ is a net. For any net $\sigma$ on $X$ contained in $\bar{\tau}$, we have $\hat{\sigma}=\hat{\tau}$.

    Moreover, $\hat{\hat{\tau}}=\hat{\tau}$.
\end{proposition}
\begin{proof}
    Let $U,V\in \hat{\tau}$. Take $\bar{\tau}$-special coverings $\{U_i\}_{i\in I}$, $\{V_j\}_{j\in J}$ of $U$ and $V$ respectively. In order to show that $\hat{\tau}|_{U\cap V}$ is a quasi-net, it suffices to show that $\hat{\tau}|_{U_i\cap V_j}$ is for any $i\in I$ and $j\in J$. This follows simply from the fact that $\bar{\tau}|_{U_i\cap V_j}$ is a quasi-net. Similarly, as $\hat{\tau}$ is a quasi-net as $\bar{\tau}$ is. So $\hat{\tau}$ is a net.

    Let $\sigma$ be a net on $X$ contained in $\bar{\tau}$. By \cref{lma-tauspecialanycov}, it suffices to verify that for any $V\in \bar{\tau}$, there are $n\in \mathbb{Z}_{>0}$ and $U_1,\ldots,U_n\in \bar{\sigma}$ with $V=U_1\cup \cdots \cup U_n$. As $\sigma$ is a net on $X$, we can find $m\in \mathbb{Z}_{>0}$, $W_1,\ldots,W_m\in \sigma$ such that
    \[
        V\subseteq W_1\cup \cdots\cup W_m.  
    \]
    As $V,W_j\in \bar{\tau}$ for $j=1,\ldots,m$, by \cref{Topology-lma-netproperty1} in \nameref{Topology-chap-topology}, we can find $U_1,\ldots,U_n\in \bar{\tau}$ such that $V=U_1\cup\cdots\cup U_n$ and each $U_i$ is contained in some $W_j$. As $W_j\in \sigma$ for $j=1,\ldots,m$, it follows that $U_i\in \bar{\sigma}$ for $i=1,\ldots,n$.

    By \cref{lma-tauspecialanycov}, 
    \[
      \overline{\hat{\tau}} =\hat{\tau}. 
    \]
    Let $V\in \hat{\hat{V}}$.
    Let $\{V_i\}_{i\in I}$ be a $\hat{\tau}$-special covering of $V$. For each $i\in I$, take a $\bar{\tau}$-special covering $\{V_{ij}\}_{j\in J_i}$ of $V_i$. Then $\{V_{ij}\}_{i,j}$ is a $\bar{\tau}$-special covering of $V$. It follows that $V\in \hat{\tau}$.
\end{proof}

\begin{proposition}
    Let  $(X,\mathcal{A},\tau)$ be a $k_H$-analytic space. There is a $k_H$-analytic atlas $\hat{\mathcal{A}}$ on $X$ with the net $\hat{\tau}$ extending $\mathcal{A}$. Moreover, $\hat{\mathcal{A}}$ is unique up to a canonical isomorphism.
\end{proposition}
\begin{proof}
    For each $V\in \hat{\tau}$, Fix a $\bar{\tau}$-special covering $\{V_i\}_{i\in I_V}$.
    
    We define $A_V$ using this covering as in \cref{lma-AWspecial}. By definition, the canonical map $V\rightarrow \Sp A_V$ is a homeomorphism.

    Next take $U,V\in \hat{\tau}$ with $U\subseteq V$. We want to identify $U$ with a $k_H$-affinoid domain in $V$. First assume that $U\in \tau$, then $U\cap V_i$ is a $k_H$-affinoid domain in $V_i$ for $i\in I_V$ by \cref{lma-tauspecialanycov}. Hence, $U$ is a $k_H$-affinoid domain in $V$. If we only know $U\in \hat{\tau}$,  we know that $U_i$ is a $k_H$-affinoid domain in $V$ for any $i\in I_U$. It follows that $U$ is a $k_H$-affinoid domain in $V$ by \cref{Affinoid-prop-compactanalydomainaffinoid} in \nameref{Affinoid-chap-affinoid}.
    
    The uniqueness is immediate.
\end{proof}

\begin{definition}
    Let  $(X,\mathcal{A},\tau)$ be a $k_H$-analytic space. A $\hat{\tau}$-special set is called a \emph{$k_H$-special domain} in $X$.
\end{definition}
Observe that a $k_H$-special domain inherits a structure of $k_H$-analytic space from $(X,\mathcal{A},\tau)$.

\begin{proposition}
    Let $\varphi:(X,\mathcal{A},\tau)\rightarrow (X',\mathcal{A}',\tau')$ be a morphism of $k_H$-analytic spaces. Then for any $k_H$-affinoid domains $V\subseteq X$ and $V'\subseteq X'$, the intersection $V\cap \varphi^{-1}(V')$ is a $k_H$-special domain in $X$.
\end{proposition}
\begin{proof}
    By \cref{prop-tauhatbasic}, we may assume that $\varphi$ is a strong morphism. In this case, it suffices to apply \cref{lma-strongmorphisminversecptanalyticdomain}.
\end{proof}

\begin{lemma}\label{lma-extstrongmorphismhat}
    Let $(X,\mathcal{A},\tau)$ and $(X',\mathcal{A}',\tau')$ be $k_H$-analytic spaces. Let $\varphi:(X,\mathcal{A},\tau)\rightarrow (X',\mathcal{A}',\tau')$ be a strong morphism. Then $\varphi$ extends uniquely to a strong morphism $\varphi:(X,\hat{\mathcal{A}},\hat{\tau})\rightarrow (X',\widehat{\mathcal{A}'},\widehat{\tau'})$.
\end{lemma}
\begin{proof}
    Let $V\in \hat{\tau}$ and $V'\in \widehat{\tau'}$ with $\varphi(V)\subseteq V'$. We want to define $\varphi_{V/V'}:V\rightarrow V'$ of $k_H$-affinoid spectra. By \cref{prop-strongmorphismext}, we may extend $\varphi$ uniquely to $\bar{\tau}$. Take a $\bar{\tau}$-special covering of $V$, we may reduce to the case where $V\in\bar{\tau}$. Take $W'\in \tau'$ such that $\varphi(V)\subseteq W'$. As $\tau|_{W'\cap V'}$ is a quasi-net, we can find $n\in \mathbb{Z}_{>0}$ and $W_1,\ldots,W_n\in \tau'|_{V'\cap W}$ such that $\varphi(V)\subseteq W_1\cup \cdots\cup W_n$. Considering the inverse images of $W_i$'s and $W_i\cap W_j$'s using \cref{lma-tauspecialanycov}, we are reduced to the case where $V'\in \overline{\tau'}$. This is already handled in \cref{prop-strongmorphismext}. The uniqueness of the extension is clear.
\end{proof}

\begin{proposition}\label{prop-morphismkhanalyt}
    Let $(X,\mathcal{A},\tau)$, $(X',\mathcal{A}',\tau')$ be $k_H$-analytic spaces. 
    \begin{enumerate}
        \item There is a canonical bijection between 
        \[
            \Hom_{k_H\text{-}\AnaCat}((X,\mathcal{A},\tau),(X',\mathcal{A}',\tau'))
        \]
        and the set of pairs consisting of 
        \begin{enumerate}
            \item a continuous map $\varphi:X\rightarrow X'$ such that for all $x\in X$, there exist $n\in \mathbb{Z}_{>0}$, neighbourhoods $V_1\cup \cdots\cup V_n$ of $x$ and $V_1'\cup\cdots\cup V_n'$ of $\varphi(x)$ with $x\in V_1\cap \cdots\cap V_n$ and $\varphi(V_i)\subseteq V_i'$ for $i=1,\ldots,n$, where $V_i\subseteq X$ and $V_i'\subseteq X'$ are $k_H$-affinoid domains;
            \item for each pair of $k_H$-affinoid domains $V\subseteq X$, $V'\subseteq X'$ with $\varphi(V)\subseteq V'$, a morphism of $k_H$-affinoid spectra $\varphi_{V/V'}:V\rightarrow V'$
        \end{enumerate}
        such that if $V,W\subseteq X$ and $V',W'\subseteq X'$ are $k_H$-affinoid domains with $\varphi(V)\subseteq V'$, $\varphi(W)\subseteq W'$, the diagram below commutes
        \[
            \begin{tikzcd}
                V \arrow[r, "\varphi_{V/V'}"] \arrow[d] & V' \arrow[d] \\
                W \arrow[r, "\varphi_{W/W'}"]           & W'          
            \end{tikzcd}.  
        \]
        \item Under the bijection in (1), an isomorphism corresponds to the pair where $\varphi$ is a hoemomorphism such that $\varphi(\hat{\tau})=\widetilde{\tau'}$ and for any $V\in \hat{\tau}$, $\varphi_{V/\varphi(V)}$ is an isomorphism of $k_H$-affinoid spectra.
    \end{enumerate}
\end{proposition}
\begin{proof}
    (2) follows immediately from (1). So it suffices to prove (1).

    We construct the forward map. Let $\varphi:(X,\mathcal{A},\tau)\rightarrow (X',\mathcal{A}',\tau')$ be a morphism. Take a subnet $\sigma$ of $\bar{\tau}$ such that $\varphi$ is represented by a strong morphism
    \[
        \varphi:(X,\mathcal{A}_{\sigma},\sigma)\rightarrow (X',\mathcal{A}',\tau').
    \]  
    By \cref{lma-extstrongmorphismhat}, this extends to a strong morphism
    \[
        \varphi:(X,\widehat{\mathcal{A}_{\sigma}},\hat{\sigma})\rightarrow (X',\widehat{\mathcal{A}'},\widehat{\tau'}).
    \] 
    We get an injective map from the first set into the second set.

    Conversely, we need to show that any given map from the second map comes from the first set. It suffices to show that
    \[
        \sigma:=\left\{ V\in \hat{\tau} : \varphi(V)\subseteq V'\text{ for some } V'\in \widehat{\tau'} \right\}
    \]
    is a net. Take $x\in X$ and neighbourhoods $V_1\cup\cdots\cup V_n$ of $x$ and $V_1'\cup\cdots\cup V_n'$ of $\varphi(x)$ as in the statement of (1). Then $V_i\in \sigma$, so we conclude. 
\end{proof}

In practice, we do not distinguish a $k_H$-analytic space from the isomorphic $k_H$-analytic spaces. In particular, we will write $(X,\mathcal{A},\tau)$ as $X$ and always endow it with the strucutre $(X,\hat{\mathcal{A}},\hat{\tau})$ of $k_H$-analytic space. If necessarily, we will write $|X|$ for the underlying topological space.

\begin{corollary}
    The natural functor $k_H\text{-}\AffCat\rightarrow k_H\text{-}\AnaCat$ is fully faithful.
\end{corollary}
\begin{proof}
    Let $X=\Sp A$ be a $k_H$-affinoid spectrum. We endow it with the net $\tau= \{X\}$. The $k_H$-atlas with the net $\tau$ assigns $X\in \tau$ with $A$. It is easily verified that this is a functor. By \cref{prop-morphismkhanalyt}, the functor is fully faithful.
\end{proof}

\begin{definition}
    A \emph{$k_H$-affinoid space} is an object of $k_H\text{-}\AnaCat$ lying in the essential image of the functor $k_H\text{-}\AffCat\rightarrow k_H\text{-}\AnaCat$.

    The category of $k_H$-affinoid spaces is denoted by $k_H\text{-}\AffCat$.
\end{definition}
The notation for the category of $k_H$-affinoid spaces is the same as the notation for the category of $k_H$-affinoid spectra, as the two categories are canonically equivalent.

\begin{definition}
    A $k_H$-analytic space $X$ is \emph{good} if any point $x\in X$ admits a $k_H$-affinoid neighbourhood.
\end{definition}

\begin{example}\label{ex-An}
    Fix $n\in \mathbb{N}$.
    Let $\mathbb{A}^n_k$ denote the set of all semi-valuations on $k[T_1,\ldots,T_n]$ whose restriction to $k$ coincides with the given valuation on $k$. We provide $\mathbb{A}^n_k$ with the weakest topology such that for any $f\in k[T_1,\ldots,T_n]$, the map $|\bullet|\mapsto |f|$ is continuous.

    Observe that as a topological space, 
    \begin{equation}\label{eq-Ancolimit}
        \mathbb{A}^n_k\cn \varinjlim{r\in \mathbb{R}^n_{>0}}\Sp k\{r^{-1}T\}.  
    \end{equation} 
    As a set, this is clear: if $|\bullet|\in \mathbb{A}^n_k$, we take $r=(|T_1|,\ldots,|T_n|)$, then $|\bullet|\leq \|\bullet\|_r$, so $|\bullet|\in \Sp k\{r^{-1}T\}$. 
    As
    \[
        \bigcap_{r\in \mathbb{R}^n_{>0}} k\{r^{-1}T\}=k[T_1,\ldots,T_n],
    \]
    so the topology on the right-hand side of \eqref{eq-Ancolimit} is the weakest topology making $|\bullet|\mapsto |f|$ continuous for any $f\in k[T_1,\ldots,T_n]$.
    It follows immediately that \eqref{eq-Ancolimit} is an identification of topological spaces. 

    It is clear that $\mathbb{A}^n_k$ has a structure of good $k_H$-analytic space.
\end{example}

\iffalse
\begin{lemma}
    Let $A$ be a $k$-affinoid algebra. Assume that the 
\end{lemma}

\begin{thm}
    Let $A$ be a $k$-affinoid algebra. Assume that $\Sp A$ is connected, then $\Sp A$ is path-connected.
\end{thm}
\begin{proof}
    
\end{proof}
\fi

\begin{proposition}\label{prop-kanalyticspacelocal}
    Let $X$ be a $k_H$-analytic space, $x\in X$ and $U$ be a neighbourhood of $x$ in $X$. Then there is a neighbourhood $V$ of $x$ in $X$ contained in $U$ such that $V$ is open connected locally compact paracompact and Hausdorff. Moreover, we can guarantee that $\bar{V}\subseteq U$ and $V$ is a countable union of $k_H$-affinoid domains.
\end{proposition}
\begin{proof}
    Take $n\in \mathbb{Z}_{>0}$ and $k_H$-affinoid spaces $V_1,\ldots,V_n$ containing $x$ and $V_1\cup\cdots\cup V_n$ is a neighbourhood of $x$ in $X$. If we have proved the proposition for $V_i$ in place of $X$ and $U\cap V_i$ in place of $U$ for $i=1,\ldots,n$, namely, if we have found open connected locally compact paracompact and Hausdorff sets $W_i$ containing $x$ and contained in $U\cap V_i$ whose closure in $V_i$ is contained in $U\cap V_i$,
    then we can take $V=W_1\cup\cdots\cup W_n$.

    So we may assume that $X$ is a $k_H$-affinoid space, say $X=\Sp A$. Choose a $k_H$-rational neighbourhood 
    \[
        W=\Sp A\{r^{-1}\frac{f}{g}\}  
    \]
    of $x$ in $U$, where $n\in \mathbb{N}$,  $f=(f_1,\ldots,f_n)\in A^n$, $r\in \sqrt{|k^{\times}|\cdot H}^n$, $g\in A$ and $f_1,\ldots,f_n,g$ generate the unit ideal in $A$. This is possible by \cref{Affinoid-cor-Laurentdomainrational} and \cref{Affinoid-prop-laurentdomainfundamental} in \nameref{Affinoid-chap-affinoid}. 
    Take $\delta>0$ so that $x\in \Sp A\{((1-\delta)r)^{-1}\frac{f}{g}\}$.
    Choose a strictly increasing sequence $\epsilon_i\in (0,1)\cap \sqrt{|k^{\times}|\cdot H}$ converging to $1-\delta/2$ for $i\in \mathbb{Z}_{>0}$. Let 
    \[
        W_i=\Sp A\left\{ (\epsilon_i r)^{-1}\frac{f}{g} \right\}  
    \]
    for $i\in \mathbb{Z}_{>0}$.
    Then $W_i$ lies in the interior of $W_{i+1}$ for $i\in \mathbb{Z}_{>0}$. Choose a connected component $V_i$ of $W_i$ so that $V_1\subseteq V_2\subseteq \cdots$ and $x\in V:=\bigcup_{i=1}^{\infty} V_i$. If $x\in V_i$ for some $i\in \mathbb{Z}_{>0}$, then $x$ lies in the topological interior of $V_{i+1}$. Hence, $x$ lies in the interior of $V$. By construction, $V$ is open connected paracompact locally compact and Hausdorff. Moreover, $\bar{V}\subseteq U$ by our construction.
\end{proof}


\begin{proposition}\label{prop-gluing}
    Let $\{X_i\}_{i\in I}$ be a family of $k_H$-analytic spaces. Suppose that for $i,j\in I$, we are given a $k_H$-analytic domain $X_{ij}\subseteq X_i$ and an  isomorphism $\nu_{ij}:X_{ij}\rightarrow X_{ji}$ satisfying the \emph{cocycle condition}: $X_{ii}=X_i$, $\nu_{ij}(X_{ij}\cap X_{il})=X_{ji}\cap X_{jl}$ and $\nu_{il}=\nu_{jl}\circ\nu_{ij}$ on $X_{ij}\cap X_{il}$ for $i,j,l\in I$. 

    Assume that either of the followinig conditions holds:
    \begin{enumerate}
        \item $X_{ij}$ is open in $X_i$ for all $i,j\in I$;
        \item for any $i\in I$, all $X_{ij}$'s are closed in $X_i$ and the number of $j\in I$ with $X_{ij}\neq\emptyset$ is finite.  
    \end{enumerate}
    Then there is a $k_H$-analytic space $X$ and morphisms $\mu_i:X_i\rightarrow X$ for $i\in I$ such that 
    \begin{enumerate}
        \item $\mu_i$ is an isomorphism of $X_i$ with a $k_H$-analytic domain in $X$;
        \item $X=\bigcup_{i\in I}\mu_i(X_i)$;
        \item $\mu_i(X_{ij})=\mu_i(X_i)\cap \mu_j(X_j)$ for $i,j\in I$;
        \item $\mu_i=\mu_j\circ \nu_{ij}$ on $X_{ij}$ for $i,j\in I$.
    \end{enumerate}
    The space $X$ is unique up to a canonical isomorphism. Moreover, under Condition~(1), $\mu_i(X_{i})$ is open in $X$ for $i\in I$; under Condition~(2), $\mu_i(X_i)$ is closed in $X$ for $i\in I$.
    
    Under both conditions, if all $X_i$'s are Hausdorff (resp. paracompact), then so is $X$.
\end{proposition}
We will call $X$ the \emph{gluing} of the $X_i$'s along the $X_{ij}$'s.
\begin{proof}
    By \cref{prop-gtopsubcanonical}, the uniquenss of $X$ is clear.

    Let 
    \[
        \tilde{X}=\coprod_{i\in I}X_i  
    \]
    in $k_H\text{-}\AnaCat$. Observe that
    \[
        |\tilde{X}|=\coprod_{i\in I}|X_i|  
    \]
    in the category $\TopCat$. The system $\nu_{ij}$'s defines an equivalence relation $R$ on $|\tilde{X}|$. Let $|X|=|\tilde{X}|/R$ and $\mu_i:|X_i|\rightarrow |X|$ be the induced map for $i\in I$.

    Under Condition~(1), $\mu_i(|X_i|)$ is open in $|X|$ for $i\in I$. Under Condition~(2), $\mu_{i}(|X_i|)$ is closed in $X$ for $i\in I$. 

    Under both conditions, the map $\mu_i$ induces a homeomorphism $|X_i|\rightarrow \mu_i(|X_i|)$ for $i\in I$. If all $|X_i|$'s are Hausdorff (resp. paracompact), so is $|X|$.

    All these claims follow from well-known results in general topology. 

    We will endow $|X|$ with a structure of $k_H$-analytic space. Let $\tau$ be the set of $V\subseteq |X|$ for which there is $i\in I$ such that $V\subseteq \mu_i(X_i)$ and $\mu_i^{-1}(V)$ is a $k_H$-affinoid domain in $X_i$. Then $\tau$ is a net on $X$. There is an obvious $k$-affinoid atlas on $X$ with the net $\tau$. All properties in the proposition are satisfied by $X=(|X|,\mathcal{A},\tau)$.
\end{proof}

\begin{definition}
    Let $X$ be a $k_H$-analytic space and $x\in X$, take a $k_H$-affinoid domain $\Sp A$ in $X$ containing $x$, we define the \emph{completed residue field} $\mathscr{H}(x)$ of $x$ in $X$ as the completed residue field of $x$ in $\Sp A$.
\end{definition}
By \cref{Affinoid-cor-affdomainresidueequal} in \nameref{Affinoid-chap-affinoid}, $\mathscr{H}(x)$ does not depend on the choice of $\Sp A$ up to an isomorphism of complete valuation fields over $k$.


\section{Analytic domains}
Let $(k,|\bullet|)$ be a complete non-Archimedean valued field and $H$ be a subgroup of $\mathbb{R}_{>0}$ such that $|k^{\times}|\cdot H\neq \{1\}$.

\begin{definition}\label{def-khanalyticdomain}
    Let $X$ be a $k_H$-analytic space. A subset $Y\subseteq X$ is called a \emph{$k_H$-analytic domain} if for any $y\in Y$, there exist $n\in \mathbb{Z}_{>0}$, $k_H$-affinoid domains $V_1,\ldots,V_n$ contained in $Y$ such that
    \begin{enumerate}
        \item $y\in V_1\cap \cdots\cap V_n$;
        \item $V_1\cup\cdots\cup V_n$ is a neighbourhood of $y$ in $Y$.
    \end{enumerate}
\end{definition}
Observe that the net of $k_H$-affinoid domains in $X$ that are contained in $Y$ form a net on $Y$. In particular, $Y$ inherits a $k_H$-analytic space structure from $X$, and we have a canonical morphism $Y\rightarrow X$ in $k_H\text{-}\AnaCat$.

\begin{lemma}
    Let $X$ be a $k_H$-analytic space, $Y$ be a $k_H$-analytic domain in $X$ and $x\in Y$. Then the completed residue field of $x$ in $X$ is the same as the completed residue field of $x$ in $Y$ modulo isomorphisms of completed valuation fields over $k$.
\end{lemma}
\begin{proof}
    This follows immediately from \cref{Affinoid-cor-affdomainresidueequal} in \nameref{Affinoid-chap-affinoid}.
\end{proof}

\begin{proposition}\label{prop-liftelementsinHxred}
    Let $X$ be a $k$-analytic space and $x\in X$. Let $\lambda\in \widetilde{\mathscr{H}(x)}$ be a non-zero homogeneous element. Then we can find a $k$-affinoid domain $\Sp B$ of $x$ in $X$ and an inveritble function $f\in B$ such that 
    \[
        \lambda=\widetilde{f(x)}.  
    \]
    If $X$ is good, we may assume that $\Sp B$ is a $k$-affinoid neighbourhood of $x$ in $X$.
\end{proposition}
\begin{proof}
    We may assume that $X$ is $k$-affinoid, say $X=\Sp A$. Let $\chi_x:A\rightarrow \mathscr{H}(x)$ be the character corresponding to $x$. Let $|\bullet|_x$ be the bounded semi-valuation on $A$ corresponding to $x$.
    As $\Frac A/\ker|\bullet|_x$ is dense in $\mathscr{H}(x)$ by definition, we can find $g,h\in A$ such that $g(x)\neq 0$, $h(x)\neq 0$ and
    \[
        \lambda=\widetilde{g(x)}/\widetilde{h(x)}.  
    \]
    Let $Y$ be the open $k$-analytic domain in $X$ defined by $g(x)\neq 0$ and $h(x)\neq 0$. We take a $k$-affinoid domain $\Sp B$ of $X$ containing $x$ such that $\Sp B\subseteq Y$. If $X$ is good, we may assume that $\Sp B$ is a neighbourhood of $x$ in $X$. Then the images of $g$ and $h$ in $B$ are invertible by \cref{Banach-cor-invertibleintermsofsp} in \nameref{Banach-chap-Banach}. Now $f=g/h\in B$ satisfies our assumptions. 
\end{proof}


\begin{example}\label{ex-opensubspace}
    Let $X$ be a $k_H$-analytic space. Then any open subset $U$ of $X$ is a $k_H$-analytic domain. 

    In fact, for $x\in U$, take $V_1,\ldots,V_n$ as in \cref{def-khanalyticdomain}. By \cref{Affinoid-prop-laurentdomainfundamental} in \nameref{Affinoid-chap-affinoid}, up to replacing $V_i$'s by $k_H$-Laurent domains in them, we may guarantee that $V_i\subseteq U$ for all $i=1,\ldots,n$.
\end{example}


\begin{proposition}\label{prop-analyticdomainprop}
    Let $X$, $X'$ be $k_H$-analytic spaces and $\varphi:X'\rightarrow X$ a morphism of $k_H$-analytic spaces. 
    \begin{enumerate}
        \item Let $Y,Z$ be $k_H$-analytic domains in $X$, then so is $Y\cap Z$.
        \item Let $Y$ be a $k_H$-analytic domain in $X$, then $\varphi^{-1}(Y)$ is a $k_H$-analytic domain in $X'$. 
    \end{enumerate}
\end{proposition}
\begin{proof}
    (1) Let $x\in Y\cap Z$. Take $k_H$-affinoid domains $V_1,\ldots,V_n$ contained in $Y$ and $k_H$-affinoid domains $W_1,\ldots,W_m$ contained in $Z$ such that
    \[
        x\in V_1\cap \cdots\cap V_n,\quad x\in W_1\cap \cdots \cap W_m
    \]
    and $V_1\cup\cdots\cup V_n$ is a neighbourhood of $x$ in $Y$, $W_1\cup\cdots\cup W_m$ is a neighbourhood of $x$ in $Z$. For each $i=1,\ldots,n$ and $j=1,\ldots,m$, $\hat{\tau}|_{V_i\cap W_j}$ is a quasi-net, so we can find a neighbourhood of $x$ in $V_{i}\cap W_j$ of the form $U^{ij}_1\cup \cdots\cup U^{ij}_{m_{ij}}$ with $U^{ij}_1,\ldots,U^{ij}_{m_{ij}}$ being $k_H$-affinoid domains in $X$ containing $x$. Then each element in the collection $\{U^{ij}_k\}$ contains $x$ and the union is a neighbourhood of $x$ in $Y\cap Z$.

    (2) Let $x'\in \varphi^{-1}(Y)$ and $x=\varphi(x')$.
    By \cref{prop-morphismkhanalyt}, we can find $n\in \mathbb{Z}_{>0}$, $k_H$-affinoid domains $V_1',\ldots,V_n'$ on $X'$ and $k_H$-affinoid domains $V_1,\ldots,V_n$ on $X$ such that 
    \[
        \begin{aligned}
        x'\in V_1'\cap \cdots\cap V_n',\quad x\in V_1\cap \cdots\cap V_m,  \\
        \varphi(V_i')\subseteq V_i\text{ for }i=1,\ldots,n,
        \end{aligned}
    \]
    and $V'_1\cup\cdots\cup V'_n$ (resp. $V_1\cup\cdots\cup V_n$) is a neighbourhood of $x'$ (resp. $x$) in $X'$ (resp. $X$). 
    Take $k_H$-affinoid domains $W_1,\ldots,W_m$ in $X$ contained in $Y$, each containing $x$ such that $W_1\cup\cdots\cup W_m$ is a neighbourhood of $x$ in $Y$. 
    
    Then for each $i=1,\ldots,n$, $j=1,\ldots,m$, we can find $k_H$-affinoid domains $W_{ij}^k$ for $k=1,\ldots,r_{ij}$ contained in $W_j\cap V_i$ and  containing $x$ such that $\cup_k W_{ij}^k$ is a neighbourhood of $x$ in $W_j\cap V_i$. Thus, $\cup_{j,k} W_{ij}^k$ is a neighbourhood of $x$ in $V_i\cap Y$.
    Then $U_{ij}^k:=\varphi^{-1}(V_{ij}^k)\cap V_{i}'$ is a $k_H$-affinoid domain in $V_i'$ by \cref{Affinoid-cor-fiberproductaffdomain} in \nameref{Affinoid-chap-affinoid}. Moreover, $\cup_{j,k} U_{ij}^k$ is a neighbourhood of $x'$ in $V_i'\cap Y'$. So $\cup_{i,j,k} U_{ij}^k$ is a neighbourhood of $x'$ in $Y'$.
\end{proof}

\begin{proposition}\label{prop-univpropanalyticdomain}
    Let $X$ be a $k_H$-analytic space and $Y$ be a $k_H$-analytic domain in $X$. Then for any $k_H$-analytic space $Z$ and any morphism $\varphi:Z\rightarrow X$ whose image is contained in $Y$, there is a unique morphism $\psi:Z\rightarrow Y$ such that the following diagram commutes:
    \[
        \begin{tikzcd}
            Z \arrow[d, "\psi", dotted] \arrow[rd, "\varphi"] &   \\
            Y \arrow[r]                                       & X
        \end{tikzcd}.  
    \]
\end{proposition}
\begin{proof}
    The uniqueness of $\psi$ is obvious. We only need to prove the existence. 
    This is an immediate consequence of \cref{prop-morphismkhanalyt} and \cref{prop-analyticdomainprop}.

    To be more precise, assume that $\varphi$ is given by a data as in \cref{prop-morphismkhanalyt}, we only have to show that each $k_H$-affinoid domain $V$ in $X$, $V\cap Y$ is a $k_H$-affinoid domain in $Y$. This follows from \cref{prop-analyticdomainprop}.
\end{proof}

\begin{corollary}\label{cor-fiberprodanalyticdomain}
    Let $\varphi:X'\rightarrow X$ be a morphism of $k_H$-analytic spaces and $Y$ be a $k_H$-analytic domain in $X$. Then $X'\times_Y X$ in the category $k_H\text{-}\AnaCat$ exists and
    $\varphi^{-1}(Y)$ represents $X'\times_Y X$.
\end{corollary}
\begin{proof}
    This follows from \cref{prop-univpropanalyticdomain} and \cref{prop-analyticdomainprop}.    
\end{proof}

\begin{corollary}\label{cor-A1ringed}
    Let $\Sp B$ be a  $k_H$-affinoid space, then we have a functorial isomorphism
    \[
        \Hom_{k_H\text{-}\AnaCat}(\Sp B,\mathbb{A}^1_k)\cn B. 
    \]
\end{corollary}
\begin{proof}
    As $\Sp B$ is compact as a topological space, its image in $\mathbb{A}^1_k$ is contained in $\Sp k\{r^{-1}T\}$ for some $r>0$.  By \cref{prop-univpropanalyticdomain}, we have natural bijections
    \[
        \Hom_{k_H\text{-}\AnaCat}(\Sp B,\mathbb{A}^1_k)\cn \varinjlim_{r>0}\Hom_{k_H\text{-}\AnaCat}(\Sp B,\Sp k\{r^{-1}T\})\cn  \varinjlim_{r>0}\Hom_{k\text{-}\AffAlgCat}(k\{r^{-1}T\}, B).  
    \]
    By \cref{Affinoid-cor-univpropTatenonstrict} in \nameref{Affinoid-chap-affinoid}, the right-hand side is identified with $B$.
\end{proof}

\begin{proposition}\label{prop-analyticdomaininanalyticdomain}
    Let $X$ be a $k_H$-analytic space, $Y$ be a $k_H$-analytic domain in $X$. For a subset $Z\subseteq Y$, the following are equivalent:
    \begin{enumerate}
        \item $Z$ be a $k_H$-analytic domain in $X$;
        \item $Z$ is a $k_H$-analytic domain in $Y$.
    \end{enumerate}
\end{proposition}
\begin{proof}
    (1) $\implies$ (2):
    Let $z\in Z$, we can find $n\in \mathbb{Z}_{>0}$ and $k_H$-affinoid domains $V_1,\ldots,V_n$ in $X$ containing $x$ and contained in $Z$ such that $V_1\cup\cdots\cup V_n$ is a neighbourhood of $z$ in $Z$. But observe that $V_1,\ldots,V_n$ are $k_H$-affinoid domains in $Y$ as well, so we conclude.

    (2) $\implies$ (1): This follows from the same argument. It suffices to observe that a $k_H$-affinoid domain in $Y$ is necessarily $k_H$-affinoid in $X$, as can be seen from \cref{def-tauhat}.
\end{proof}


\begin{definition}
    Let $X,Y$ be $k_H$-analytic spaces and $\varphi:Y\rightarrow X$ be a morphism. We say $\varphi$ is an \emph{open immersion} if $\varphi(Y)$ is open in $X$ and $\varphi$ induces an isomorphism between $Y$ and $\varphi(Y)$ as $k_H$-analytic spaces. 
\end{definition}
By \cref{ex-opensubspace}, $\varphi(Y)$ is a $k_H$-analytic domain in $X$ and by \cref{prop-univpropanalyticdomain}, we have a morphism of $k_H$-analytic spaces $Y\rightarrow \varphi(Y)$.

\begin{proposition}
    Let $X$ be a $k_H$-analytic space and $Y$ be a $k_H$-analytic domain in $X$.  Assume that $Y$ is a $k_H$-affinoid space, then $Y$ is a $k_H$-affinoid domain in $X$.
\end{proposition}
\begin{proof}
    As $Y$ is a $k_H$-affinoid space, we know that $|Y|$ is compact. Take finitely many $k_H$-affinoid domains $V_1,\ldots,V_n$ in $X$ such that
    \[
        Y=V_1\cup\cdots\cup V_n.  
    \]
    Then $V_1,\ldots,V_n$ are $k_H$-affinoid domains in $Y$: let $\Sp D\rightarrow Y$ be a morphism of $k_H$-affinoid spectra, whose image lies in  $V_i$ for some $i=1,\ldots,n$. Consider the following commutative diagram
    \[
        \begin{tikzcd}
            & \Sp D \arrow[ld, dotted] \arrow[d] \arrow[rd] &   \\
        V_i \arrow[r] & Y \arrow[r]                                   & X
        \end{tikzcd}  
    \] 
    By \cref{prop-univpropanalyticdomain}, there is a unique dotted morphism making the outer triangle commutative, hence making the whole diagram commutative. We have therefore shown that $V_i$ is a $k_H$-affinoid domain in $Y$.

    So the covering $\{V_1,\ldots,V_n\}$ of $Y$ satisfies the assumptions in \cref{def-tauhat} and $Y$ is $k_H$-affinoid.
\end{proof}

\begin{proposition}\label{prop-gtopsubcanonical}
    Let $X$ be a $k_H$-analytic space and $\{Y_i\}_{i\in I}$ be a family of $k_H$-analytic domains in $X$ which forms a quasi-net on $X$.
    Then for any $k_H$-analytic space $X'$, the following sequence is exact
    \[
        \Hom_{k_H\text{-}\AnaCat}(X,X')\rightarrow \prod_{i\in I} \Hom_{k_H\text{-}\AnaCat}(Y_i,X')\rightrightarrows \prod_{i,j\in I} \Hom_{k_H\text{-}\AnaCat}(Y_i\cap Y_j,X').
    \]
\end{proposition}
\begin{proof}
    Let $\{\varphi_i:Y_i\rightarrow X'\}_{i\in I}$ be a family of morphisms such that $\varphi_i$, $\varphi_j$ coincides on $Y_i\cap Y_j$ for $i,j\in I$. We need to glue the $\varphi_i$'s into a single morphism $\varphi:X\rightarrow X'$. Clearly, the underlying maps glue together to a continuous map $\varphi:X\rightarrow X'$ by \cref{Topology-lma-quasinetopenset} in \nameref{Topology-chap-topology}. 

    Let $\tau$ be the collection of $k_H$-affinoid domains $V$ in $X$ such that there is $i\in I$ and a $k_H$-affinoid domain $V'\subseteq X'$ with $V\subseteq Y_i$ and $\varphi_i(V)\subseteq V'$. Then $\tau$ is a net on $X$, and we have a morphism $X\rightarrow X'$.
\end{proof}

\section{Berkovich site}
Let $(k,|\bullet|)$ be a complete non-Archimedean valued field and $H$ be a subgroup of $\mathbb{R}_{>0}$ such that $|k^{\times}|\cdot H\neq \{1\}$.

\begin{lemma}\label{lma-admissiblecoveringGtopo}
    Let $X$ be a $k_H$-analytic space. Consider the category $\mathcal{C}$ of $k_H$-analytic domains in $X$, where the morphisms are inclusions of $k_H$-analytic domains. For each $Y\in \mathcal{C}$, consider the set of coverings $\Cov(Y)$ consisting of all $\{Y_i\rightarrow Y\}_{i\in I}$ such that $Y_i$ is a $k_H$-analytic domain in $Y$ and $\{Y_i\}_{i\in I}$ is a quasi-net on $Y$.
    The class of coverings $\{\Cov(Y)\}_Y$ defines a Grothendieck pretopology.
\end{lemma}
\begin{proof}
    It suffices to verify the axioms in \cite[\href{https://stacks.math.columbia.edu/tag/03NH}{Tag 03NH}]{stacks-project}.

    (1) An isomorphism $Y'\rightarrow Y$ in $\mathcal{C}$ is in $\Cov(Y)$.

    This is trivial as an isomorphism in $\mathcal{C}$ is necessarily identity.

    (2) If $\{Y_i\rightarrow Y\}_{i\in I}$ and $\{Y_{ij}\rightarrow Y_i\}_{j\in J_i}$ for all $i\in I$ are in $\Cov(Y)$ and $\Cov(Y_i)$ respectively, then $\{Y_{ij}\rightarrow Y\}_{i,j}$ is in $\Cov(Y)$.

    By \cref{prop-analyticdomaininanalyticdomain}, $Y_{ij}$ is a $k_H$-analytic domain in $Y$ for any $i\in I$, $j\in I_j$. It suffices to show that $\{Y_{ij}\}_{i,j}$ is a quasi-net on $Y$. Let $y\in Y$, we can find finitely many elements among $\{Y_i\}_{i\in I}$, say $Y_1,\ldots,Y_n$ so that $y\in Y_i$ for each $i=1,\ldots,n$ and $Y_1\cup \cdots\cup Y_n$ is a neighbourhood of $y$ in $Y$. Similarly, for each $i=1,\ldots,n$, we can find finitely many $Y_{i1},\ldots,Y_{ij_i}$ among $\{Y_{ij}\}_{j\in J_i}$ so that $y$ is contained in each of them and $Y_{i1}\cup\cdots\cup Y_{ij_i}$ is a neighbourhood of $y$ in $Y_i$. Then each element in $\{Y_{ij}\}_{i=1,\ldots,n;j=1,\ldots,j_i}$ contains $y$ and the union is a neighbourhood of $y$ in $Y$.
    
    (3) If $\{Y_i\rightarrow Y\}_{i\in I}$ lies in $\Cov(Y)$ and $Z\rightarrow Y$ is a $k_H$-analytic domain in $Y$, then the fiber products $Y_i\times_Y Z$ exist and $\{Y_i\times_Y Z\rightarrow Z\}_{i\in I}$ lies in $\Cov(Z)$.

    By \cref{cor-fiberprodanalyticdomain}, $Y_i\times_Y Z$ exists and is represented by the inverse image of $Z$ in $Y_i$, which is a $k_H$-analytic domain in $Y_i$ by \cref{prop-analyticdomainprop}. It is clear that $\{Y_i\times_Y Z\}_{i\in I}$ is a quasi-net on $Z$.
\end{proof}

\begin{definition}
    Let $X$ be a $k_H$-analytic space. We will write the site constructed in \cref{lma-admissiblecoveringGtopo} as $X$ and call it the \emph{Berkovich site} of $X$. The corresponding Grothendieck topology is called the \emph{Berkovich Grothencieck topology}.
    The topos $\Sh(X)$ associated with $X$ is called the \emph{Berkovich topos} of $X$.
\end{definition}
Observe that the Berkovich Grothendieck topology is subcanonical by \cref{prop-gtopsubcanonical}.

\begin{definition}
    Let $X$ be a $k_H$-analytic space.
    We define a sheaf of rings $\mathcal{O}_X$ on $X$ as follows: let $Y$ be a $k_H$-analtic domain in $X$, we set 
    \[
        \mathcal{O}_X(Y)=\Hom_{k_H\text{-}\AnaCat}(X,\mathbb{A}^1_k).
    \]
    By \cref{cor-A1ringed} and \cref{prop-gtopsubcanonical}, $\mathcal{O}_X$ defines a sheaf of rings. We call $\mathcal{O}_X$ the structure sheaf of $X$. The corresponding ringed site $(X,\mathcal{O}_X)$ is called the \emph{Berkovich ringed site}. The induced ringed topos $(\Sh(X),\mathcal{O}_X)$ is called the \emph{Berkovich ringed topos}.

    Given any morphism $f:Y\rightarrow X$ of $k_H$-analytic spaces, we have an induced morphism of the corresponding ringed sites, still denoted by $\varphi$.
\end{definition}

\begin{definition}
    Let $X$ be a $k_H$-analytic space. An $\mathcal{O}_X$-module $\mathcal{M}$ is \emph{coherent} if there is an admissible covering $\{Y_i\}_{i\in I}$ of $X$ such that $\mathcal{M}|_{Y_i}$ is isomorphic to the cokernel of a homomorphism of finite free $\mathcal{O}_{V_i}$-modules.
\end{definition}

\begin{example}\label{ex-finitemoduleinducecohsheaf}
    Let $A$ be a $k_H$-affinoid algebra and $M$ be a fintie $A$-module. Then $M$ induces a coherent sheaf of $\mathcal{O}_{\Sp A}$-modules $\tilde{M}$ as follows:
    \[
        \tilde{M}(V)=M\otimes_A A_V.  
    \]
\end{example}

Conversely, we can reformulate Kiehl's theorem.
\begin{thm}\label{thm-Kiehlcoh}
    Let $A$ be a $k_H$-affinoid algebra and $\mathcal{M}$ be a coherent sheaf of $\mathcal{O}_{\Sp A}$-modules. Set $M=H^0(X,\mathcal{M})$, then $M$ is a finite $A$-modue and we have a canonical isomorphism
    \[
        \tilde{M}\cn \mathcal{M}.  
    \]
\end{thm}
The left-hand side is defined in \cref{ex-finitemoduleinducecohsheaf}.
\begin{proof}
    This is just a reformulation of \cref{Affinoid-thm-Kiehl} in \nameref{Affinoid-chap-affinoid}.
\end{proof}

\begin{corollary}\label{cor-finitmodulecohpush}
    Let $\varphi:\Sp B\rightarrow \Sp A$ be a morphism of $k_H$-affinoid spaces. Then the following are equivalent:
    \begin{enumerate}
        \item $\varphi_*\mathcal{O}_{\Sp B}$ is a coherent $\mathcal{O}_{\Sp A}$-module;
        \item $B$ is a finite Banach $A$-module.
    \end{enumerate}
\end{corollary}
\begin{proof}
    Observe that for any $k_H$-affinoid domain $\Sp C$ in $\Sp A$,
    \[
        \varphi_*\mathcal{O}_{\Sp B}(\Sp C)=\mathcal{O}_{\Sp B}(\varphi^{-1}(\Sp C))=\mathcal{O}_{\Sp B}(\Sp C\hat{\otimes}_A B)=C\hat{\otimes}_A B\cn C\otimes_A B.
    \]
    Here we applied \cref{Affinoid-cor-fiberproductaffdomain} in \cref{Affinoid-chap-affinoid} and \cref{Affinoid-prop-finitemodulebasechange} in \cref{Affinoid-chap-affinoid}. So $\varphi_*\mathcal{O}_{\Sp B}\cong \widetilde{B}$.

    From this (2) trivially implies (1).

    Conversely, assume (1), let $B=H^0(\Sp A,\varphi_*\mathcal{O}_{\Sp B})$. By \cref{thm-Kiehlcoh}, $B$ is a finite $A$-module. Let $B'$ denote the ring $B$ endowed with the finite Banach $A$-algebra structure as in \cref{Affinoid-prop-Banachalgebrafiniteforget} in \nameref{Affinoid-chap-affinoid}. We need to show that the identity map $B'\rightarrow B$ is admissible. Observe that the identity map is bounded by \cref{Affinoid-prop-admitobanachbdd} in \cref{Affinoid-chap-affinoid}. By \cref{Affinoid-prop-Weirestrassdomainequivdense} in \cref{Affinoid-chap-affinoid}, it suffices to show that the induced map $\Sp B\rightarrow \Sp B'$ is surjective. Let $\varphi':\Sp B'\rightarrow \Sp A$ be the natural morphism of $k_H$-affinoid spaces. Then 
    \[
        \varphi_*(\mathcal{O}_{\Sp B})\cn \varphi'_*(\mathcal{O}_{\Sp B'}).
    \]
    It follows that $\varphi^{-1}(x)=\varphi'^{-1}(x)$ for any $x\in \Sp A$. We conclude.
\end{proof}

\begin{corollary}\label{cor-admissibleepisheaf}
    Let $\varphi:\Sp B\rightarrow \Sp A$ be a morphism of $k_H$-affinoid spaces. Then the following are equivalent:
    \begin{enumerate}
        \item $\varphi_*\mathcal{O}_{\Sp B}$ is a coherent $\mathcal{O}_{\Sp A}$-module and $\mathcal{O}_{\Sp A}\rightarrow \varphi_*\mathcal{O}_{\Sp B}$ is surjective;
        \item $A\rightarrow B$ is an admissible epimorphism.
    \end{enumerate}
\end{corollary}
\begin{proof}
    Assume (2). By \cref{cor-finitmodulecohpush}, $\varphi_*\mathcal{O}_{\Sp B}$ is a coherent $\mathcal{O}_{\Sp A}$-module. To see that  $\mathcal{O}_{\Sp A}\rightarrow \varphi_*\mathcal{O}_{\Sp B}$ is surjective, it suffices to show that for each $k_H$-affinoid space $\Sp C$ in $\Sp A$,
    \[
        C\rightarrow C\otimes_A B  
    \]
    is surjective. This follows from the assumption.

    Assume (1). We know that $B$ is a finite Banach $A$-module. In particular, $A\rightarrow B$ is admissible by \cref{Affinoid-prop-finitemodulemapadmi} in \nameref{Affinoid-chap-affinoid}.
    As $\mathcal{O}_{\Sp A}\rightarrow \varphi_*\mathcal{O}_{\Sp B}$ is surjective, by \cref{thm-Kiehlcoh}, $A\rightarrow B$ is surjective. \textcolor{red}{Include details}
\end{proof}


\begin{definition}
    Let $\Sp A$ be a $k_H$-affinoid space and $\mathcal{M}=\tilde{M}$ is a coherent sheaf of $\mathcal{O}_X$-modules on $X$, where $M$ is a finite $A$-module. The \emph{support} $\Supp M$ of $\mathcal{M}$ is the closed subset $\Sp A/\Ann_A(M)$ of $\Sp A$.

    Let $X$ be a $k_H$-analytic space and $\mathcal{M}$ be a coherent sheaf of $\mathcal{O}_X$-modules. Then the \emph{support} $\Supp \mathcal{M}$ of $\mathcal{M}$ is a subset of $X$ such that a point $x\in X$ lies in $\Supp \mathcal{M}$ if and only if for some  $k_H$-affinoid domain $V$ in $X$ containing $x$, $x\in \Supp \mathcal{M}|_V$.
\end{definition}
Here $\Ann_A(M)$ is the annihilator of $M$ in $A$.

\begin{lemma}
    Let $X$ be a $k_H$-analytic space and $\mathcal{M}$ be a coherent sheaf of $\mathcal{O}_X$-modules. Take $x\in \Supp \mathcal{M}|_V$ and a $k_H$-affinoid domain $V$ in $X$ containing $x$. Then $x\in \Supp \mathcal{M}|_V$.
\end{lemma}
\begin{proof}
By assumption, there is a $k_H$-affinoid domain $U$ in $X$ containing $x$ such that $x\in \Supp\mathcal{M}|_U$. 

Let $W\subseteq U\cap V$ be a $k_H$-affinoid domain in $X$ containing $x$. We claim that $x\in \Supp \mathcal{M}|_W$. Let $M=H^0(U,\mathcal{M})$, then $M\otimes_{A_U} A_W=H^0(W,\mathcal{M})$. By \cite[\href{https://stacks.math.columbia.edu/tag/07T8}{Tag 07T8}]{stacks-project} and \cref{Affinoid-thm-affdomainflat} in \nameref{Affinoid-chap-affinoid},
\[
    \Ann_{A_U}(M)\otimes_{A_U} A_W=    \Ann_{A_W}(M\otimes_{A_U}A_W)
\]  
and $\Supp (\mathcal{M}|_W)=\Supp (\mathcal{M}|_U)\cap W$.
The claim follows. We may assume that $U\subseteq V$. In this case, the same argument shows that $x\in \Supp \mathcal{M}|_V$.
\end{proof}


\begin{proposition}\label{prop-paracompactanalyticgcovering}
    Let $X$ be a Hausdorff $k_H$-analytic space. Then the following are equivalent:
    \begin{enumerate}
        \item $X$ is paracompact;
        \item $X$ admits a locally finite covering by $k_H$-affinoid domains. 
    \end{enumerate}
\end{proposition}
Note that the covering in (2) is necessarily a G-covering.
\begin{proof}
    Assume (1). Then (2) follows from \cref{Topology-prop-paracptrefinement} in \nameref{Topology-chap-topology}. We take $\mathcal{B}$ to the collection of finite unions of $k_H$-affinoid domains that contain an open subset of $X$.

    Assume (2). Let $\{X_i\}_{i\in I}$ be a locally finite covering of $X$ by $k_H$-affinoid domains. Define an equivalence relation on $I$ generated by $i\sim j$ if $X_i\cap X_j\neq \emptyset$. We say $X_i$ and $X_j$ are elementarily linked in this case.
    Fix $C\in I/\sim$ and $i\in C$. For any $n\in \mathbb{Z}_{>0}$, $C_n$ denotes the union of $X_j$ where $j$ and $i$ are linked through a chain of elementary links of length at most $n$. As the covering is locally finite, we see that $C_n$ is compact. So 
    \[
        X_C=\bigcup_{i=1}^{\infty}C_i
    \]
    is $\sigma$-compact. The space $X$ is clearly the coproduct of $X_C$'s, hence paracompact by \cref{Topology-prop-paracptrefinement} in \nameref{Topology-chap-topology}.
\end{proof}



\begin{proposition}
    The category $k_H\text{-}\AnaCat$ admits finite limits.
\end{proposition}

\begin{proof}
    By general abstract nonsense, it suffices to show that $k_H\text{-}\AnaCat$ admits finite fiber products.

    Let $\varphi:Y\rightarrow X$ and $f:X'\rightarrow X$ be morphisms of $k_H$-affinoid spaces. We want to construct $Y\times_X X'$.

    \textbf{Step~1}. We assume that $X,Y,X'$ are all paracompact and Hausdorff.

    By \cref{prop-paracompactanalyticgcovering}, we can find a locally finite G-covering $\{X_i\}_{i\in I}$ of $X$ consisting of $k_H$-affinoid domains in $X$. By \cref{prop-paracompactanalyticgcovering} again, we can find a locally finite G-covering $\{Y_{ij}\}_j$ $\varphi^{-1}(X_i)$ consisting of $k_H$-affinoid domains in $Y$ and a locally finite G-covering $\{X'_{il}\}_l$ consisting of $k_H$-affinoid domains in $X'$ for each $i\in I$.

    We can glue $Y_{ij}\times_{X_i} X'_{il}$'s by \cref{prop-gluing} to get a $k_H$-analytic space $Y'$. By \cref{prop-gtopsubcanonical}, $Y'$ represents the fiber product $Y\times_X X'$.

    \textbf{Step~2}. Assume only that $X$ is a paracompact and Hausdorff.

    Take open paracompact Hausdorff coverings $\{Y_i\}_{i\in I}$ of $Y$ and $\{X_j'\}_{j\in J}$ of $X'$. The existence of these coverings follows from \cref{prop-kanalyticspacelocal}. Similar to Step~1, we glue the $Y_i\times_X X_j'$'s along the open subsets $(Y_i\cap Y_k)\times_X(X_j'\cap X_l')$'s  by \cref{prop-gluing}, we get a locally Hausdorff $k_H$-analytic space $Y'$. 
    Then by \cref{prop-gtopsubcanonical} again, $Y'$ represents the fiber product $Y\times_X X'$.

    \textbf{Step~3}. We handle the general case.

    Take a covering $\{X_i\}_{i\in I}$ by open paracompact Hausdorff subsets. Let $Y'$ be the gluing of $\varphi^{-1}(X_i)\times_{X_i}f^{-1}(X_i)$'s along $\varphi^{-1}(X_i\cap X_j)\times_{X_i\cap X_j}f^{-1}(X_i\cap X_j)$'s  by \cref{prop-gluing}. Then by \cref{prop-gtopsubcanonical} again, $Y'$ represents the fiber product $Y\times_X X'$.
\end{proof}
\begin{remark}
    The original proof in \cite{Berk93} doees not make any sense to me. Please contact me if you understand the details of Berkovich's argument.
\end{remark}

In a similar vein, we prove
\begin{proposition}
    If $K/k$ is an analytic field extension, then there is a natural functor of base extension $k_H\text{-}\AnaCat\rightarrow K_H\text{-}\AnaCat$ extending the functor $k_H\text{-}\AffCat\rightarrow K_H\text{-}\AffCat$ defined by $\Sp A\mapsto \Sp A\hat{\otimes}_k K$.
\end{proposition}
We will denote the image of a $k_H$-analytic space $X$ by $X_K$.
\begin{proof}
    Fix a $k_H$-analytic space $X$, we want to construct functorially a $K_H$-analytic space $X_K$.

    \textbf{Step~1}. We assume that $X$ is paracompact and Hausdorff. 
    
    By \cref{prop-paracompactanalyticgcovering}, we can find a locally finite G-covering $\{X_i\}_{i\in I}$ of $X$ consisting of $k_H$-affinoid domains in $X$. We can glue $X_{i,K}$'s by \cref{prop-gluing} to get $X_K$.
    
    \textbf{Step~2}. In general, let $\{Y_i\}_{i\in I}$ be an open covering of $X$ by paracompact Hausdorff subsets. We glue $Y_{i,K}$'s by \cref{prop-gluing} to get $X_K$.

    These constructions are clearly functorial and defines a functor $k_H\text{-}\AnaCat\rightarrow K_H\text{-}\AnaCat$.
\end{proof}

\section{Closed immersions}
Let $(k,|\bullet|)$ be a complete non-Archimedean valued field and $H$ be a subgroup of $\mathbb{R}_{>0}$ such that $|k^{\times}|\cdot H\neq \{1\}$.

\begin{lemma}\label{lma-closedimmglocal}
    Let $\varphi:Y\rightarrow X$ be a morphism of $k_H$-analytic spaces. Then the following are equivalent:
    \begin{enumerate}
        \item for any $x\in X$, there are $n\in \mathbb{Z}_{>0}$ and $k_H$-affinoid domains $V_1,\ldots,V_n$ in $X$ containing $x$ such that $V_1\cup\cdots\cup V_n$ is a neighbourhood of $x$ in $X$ and the restriction $\varphi^{-1}(V_i)\rightarrow V_i$ is a closed immersion for any $i=1,\ldots,n$;
        \item for any $k_H$-affinoid domain $V$ in $X$, $\varphi^{-1}(V)\rightarrow V$ is a closed immersion.
    \end{enumerate}
\end{lemma}
Recall that closed immersions between $k_H$-affinoid spaces are defined in \cref{Affinoid-def-closedimmersion} in \nameref{Affinoid-chap-affinoid}. 

The statement in \cite[Lemma~1.3.7]{Berk93} is not correct.
\begin{proof}
    Only (1) $\implies$ (2) is non-trivial. Assume (1). Let $\tau$ be the collections of $V\subseteq X$ satisfying (2). Then we claim that $\tau$ is a net. 
    
    Observe that $\tau$ is a quasi-net by our assumption.
    To see that it is a net, take $U,V\in \tau$ and $x\in U\cap V$, then we can find $n\in \mathbb{Z}_{>0}$ and $k_H$-affinoid domains $W_1,\ldots,W_n$ in $U\cap V$ containing $x$ such that $W_1\cup\cdots\cup W_n$ is a neighbourhood of $x$ in $U\cap V$. In order to show that $\tau|_{U\cap V}$ is a quasi-net, it suffices to show that $\varphi^{-1}(W_i)\rightarrow W_i$ is a closed immersion for $i=1,\ldots,n$. This follows from \cref{Affinoid-prop-closedimmbasechange} in \nameref{Affinoid-chap-affinoid}.

    Let $V$ be a $k_H$-affinoid domain in $X$.
    By (1) and the compactness of $V$, we can find $n\in \mathbb{Z}_{>0}$ and $V_1,\ldots,V_n\in \tau$ such that $V\subseteq V_1\cup\cdots\cup V_n$.
    By \cref{Topology-lma-netproperty1} in \nameref{Topology-chap-topology}, we can find $m\in \mathbb{Z}_{>0}$ and $U_1,\ldots,U_m\in \tau$ such that
    \[
        V=U_1\cup\cdots\cup U_m  
    \]
    and each $U_j$ is contained in some $V_i$, where $j=1,\ldots,m$ and $i=1,\ldots,n$. By \cref{Affinoid-prop-closedimmbasechange} in \nameref{Affinoid-chap-affinoid} again, $U_j\in \tau$ for each $j=1,\ldots,m$. It suffices to apply \cref{cor-admissibleepisheaf} to conclude that $V\in \tau$.
\end{proof}

\begin{definition}
    Let $\varphi:Y\rightarrow X$ be a morphism of $k_H$-analytic spaces.  We say $\varphi$ is a \emph{closed immersion} if the equivalent conditions in \cref{lma-closedimmglocal} are satisfied.
\end{definition}
Observe that this definition extends \cref{Affinoid-def-closedimmersion} in \nameref{Affinoid-chap-affinoid}.


\begin{proposition}\label{prop-closedimmbaseex}
    Let $\varphi:Y\rightarrow X$, $\psi:Z\rightarrow X$ be a morphism of $k_H$-analytic spaces. Assume that $\varphi:Y\rightarrow X$ is a closed immersion. Consider the Cartesian diagram
    \[
        \begin{tikzcd}
            Z\times_X Y \arrow[r] \arrow[d] \arrow[rd, "\square", phantom] & Y \arrow[d,"\varphi"] \\
            Z \arrow[r,"\psi"]                                                    & X          
        \end{tikzcd}.  
    \]
    Then $Z\times_X Y\rightarrow Z$ is a closed immersion.
\end{proposition}
\begin{proof}
    Taking a $G$-covering of $Z$, we may assume that $Z$ is compact. 
    We could cover the images of $Z$ in $X$ by finitely many $k_H$-affinoid domains $V_1,\ldots,V_n$ in $X$, considering their preimages in $Z$, we could reduce to the case where the image of $Z$ in $X$ is contained in a $k_H$-affinoid domain. We could then assume that $X$ is a $k_H$-affinoid space and hence so is $Y$. By taking a $G$-covering of $Z$ again, we may assume that $Z$ is affinoid. It suffices to apply \cref{Affinoid-prop-closedimmbasechange} in \nameref{Affinoid-chap-affinoid}.
\end{proof}


\begin{proposition}\label{prop-closedimmglocal}
    Let $\varphi:Y\rightarrow X$ be a morphism of $k_H$-analytic spaces. Then the following are equivalent:
    \begin{enumerate}
        \item $\varphi$ is a closed immersion;
        \item for any G-covering $\{X_i\}_{i\in I}$ of $X$, the restriction of $\varphi$ to $\varphi^{-1}(X_i)\rightarrow X_i$ is a closed immersion for all $i\in I$;
        \item for some G-covering $\{X_i\}_{i\in I}$ of $X$, the restriction of $\varphi$ to $\varphi^{-1}(X_i)\rightarrow X_i$ is a closed immersion for all $i\in I$.
    \end{enumerate}
\end{proposition}
In other words, being a closed immersion is a G-local property on the target.
\begin{proof}
    Assume (1). Let $\{X_i\}_{i\in I}$ be a G-covering of $X$. Then the restriction of $\varphi$ to $\varphi^{-1}(X_i)\rightarrow X_i$ is a closed immersion for all $i\in I$ by \cref{prop-closedimmbaseex}. So (2) holds.

    (2) trivially implies (3).

    Assume (3). Using the fact that (1) implies (2) as we already proved, we may refine the G-covering $\{X_i\}_{i\in I}$ and assume that each $X_i$ is $k_H$-affinoid. It follows from \cref{lma-closedimmglocal} that $\varphi$ is a closed immersion, so (1) holds.
\end{proof}
\begin{corollary}\label{cor-closedimmchangegroup}
    Let $H'\supseteq H$ is a subgroup of $\mathbb{R}_{>0}$. Let $\varphi:Y\rightarrow X$ be a morphism of $k_H$-analytic spaces. Then the following are equivalent:
    \begin{enumerate}
        \item $\varphi$ is a closed immersion;
        \item $\varphi$ is a closed immersion when view as a morphism of $k_{H'}$-affinoid spaces.
    \end{enumerate}
\end{corollary}
\begin{proof}
    By \cref{prop-closedimmglocal}, we may assume that $X$ is $k_H$-affinoid. In this case, $Y$ is also $k_H$-affinoid and the result is clear.
\end{proof}

\begin{corollary}\label{cor-closeimmchangefield}
    Let $\varphi:Y\rightarrow X$ be a morphism of $k_H$-analytic spaces and $K/k$ be an analytic field extension. 
    \begin{enumerate}
        \item If $\varphi$ is a closed immersion, so is $\varphi_K$;
        \item If $K=k_r$ for some $k$-free polyray $r$ and $\varphi_K$ is a closed immersion, then so is $\varphi$.
    \end{enumerate}
\end{corollary}
\begin{proof}
    By \cref{prop-closedimmglocal}, we may assume that $X$ is a $k_H$-affinoid space in both cases. Then so is $Y$. Now (1) is obvious and (2) follows from \cref{Affinoid-prop-Krfaithflat} in \nameref{Affinoid-chap-affinoid}.
\end{proof}

\begin{proposition}\label{prop-closedimmcomposition}
    Let $\varphi:X\rightarrow Y$, $\psi:Y\rightarrow Z$ be closed immersions of $k_H$-affinoid spaces. Then $\psi \circ \varphi:X\rightarrow Z$ is also a closed immersion.
\end{proposition}
\begin{proof}
    By \cref{prop-closedimmglocal}, we may assume that $Z$ is $k_H$-affinoid, so $Y$ and $X$ are also $k_H$-affinoid. In this case, the result is clear, as the composition of admissible epimorphisms is clearly admissible epimorphic.
\end{proof}


\begin{proposition}
    Let $\varphi:Y\rightarrow X$ be a morphism of $k_H$-analytic spaces. Then the following are equivalent:
    \begin{enumerate}
        \item $\varphi$ is a closed immersion;
        \item $\varphi_*\mathcal{O}_Y$ is a coherent $\mathcal{O}_X$-module and $\mathcal{O}_X\rightarrow \varphi_*\mathcal{O}_Y$ is surjective.
    \end{enumerate}
\end{proposition}
\begin{proof}
    As both properties are G-local on $X$, we may assume that $X$ is a $k_H$-affinoid space and hence so is $Y$. This result then follows from \cref{cor-admissibleepisheaf}.
\end{proof}

\section{Separated morphisms}
Let $(k,|\bullet|)$ be a complete non-Archimedean valued field and $H$ be a subgroup of $\mathbb{R}_{>0}$ such that $|k^{\times}|\cdot H\neq \{1\}$.

\begin{definition}\label{def-diagmorphism}
    Let $\varphi:X\rightarrow Y$ be a morphism of $k_H$-analytic spaces. The \emph{diagonal morphism} of $f$ is the morphism $\Delta_{\varphi}=\Delta_{X/Y}:X\rightarrow X\times_Y X$ defined as follows: let $\{Y_i\}_{i\in I}$ be a G-covering of $Y$ by $k_H$-affinoid domains and $\{X_{ij}\}_{j\in J_i}$ be a $G$-covering of $\varphi^{-1}(Y_i)$ by $k_H$-affinoid domains in $X$.
    Then we have a diagonal morphism $\Delta_{X_{ij}/Y_i}:X_{ij}\rightarrow X_{ij}\times_{Y_i} X_{ij}$ defined by the codiagonal morphism of $k_H$-affinoid algebras. The induced morphisms $X_{ij}\rightarrow X\times_Y X$ can be glued together by \cref{prop-gtopsubcanonical} to get $\Delta_{\varphi}$. By \cref{prop-gtopsubcanonical} does not depend on the choices of the G-coverings. 
\end{definition}

\begin{definition}
    A morphism $\varphi:X\rightarrow Y$ of $k_H$-analytic spaces is \emph{separated} if $\Delta_{X/Y}:X\rightarrow X\times_Y X$ is a closed immersion.
\end{definition}

\begin{example}
    A morphism between $k_H$-affinoid spaces is always separated. This follows from \cref{Affinoid-ex-diagonalclosedimm} in \nameref{Affinoid-chap-affinoid} by base change.
\end{example}

\begin{proposition}\label{prop-separatedbasechange}
    Let $\varphi:Y\rightarrow X$, $\psi:Z\rightarrow X$ be a morphism of $k_H$-analytic spaces. Assume that $\varphi:Y\rightarrow X$ is separated. Consider the Cartesian diagram
    \[
        \begin{tikzcd}
            Z\times_X Y \arrow[r] \arrow[d] \arrow[rd, "\square", phantom] & Y \arrow[d,"\varphi"] \\
            Z \arrow[r,"\psi"]                                                    & X          
        \end{tikzcd}.  
    \]
    Then $Z\times_X Y\rightarrow Z$ is separated.
\end{proposition}
\begin{proof}
    By general abstract nonsense, we have a Cartesian diagram
    \[
        \begin{tikzcd}
            Z\times_X Y \arrow[r,"\Delta_{Z\times_X Y/Z}"] \arrow[d] \arrow[rd, "\square", phantom] & (Z\times_X Y)\times_Z (Z\times_X Y)=Z\times_X (Y\times_X Y) \arrow[d] \\
            Y \arrow[r,"\Delta_{Y/X}"]                                                    & Y\times_X Y          
        \end{tikzcd}.  
    \]
    So the assertion follows from \cref{prop-closedimmbaseex}.
\end{proof}

\begin{proposition}
    Let $\varphi:Y\rightarrow X$ be a morphism of $k_H$-analytic spaces. Then the following are equivalent:
    \begin{enumerate}
        \item $\varphi$ is separated;
        \item for any G-covering $\{X_i\}_{i\in I}$ of $X$, the restriction of $\varphi$ to $\varphi^{-1}(X_i)\rightarrow X_i$ is separated for all $i\in I$;
        \item for some G-covering $\{X_i\}_{i\in I}$ of $X$, the restriction of $\varphi$ to $\varphi^{-1}(X_i)\rightarrow X_i$ is separated for all $i\in I$.
    \end{enumerate}
\end{proposition}
\begin{proof}
    (1) $\implies$ (2) by \cref{prop-separatedbasechange}.

    (2) $\implies$ (3) is trivial.

    Assume (3). Let $Y_i=\varphi^{-1}(X_i)$. Then $Y_i\times_{X_i} Y_i$ is a G-covering of $Y\times_X Y$, and we have a Cartesian diagram
    \[
        \begin{tikzcd}
            Y_i \arrow[r,"\Delta_{Y_i/X_i}"] \arrow[d] \arrow[rd, "\square", phantom] & Y_i\times_{X_i}Y_i \arrow[d] \\
            Y \arrow[r,"\Delta_{Y/X}"]                                                    & Y\times_X Y          
        \end{tikzcd}  
    \]
    for $i\in I$. So the assertion follows from \cref{prop-closedimmglocal}.
\end{proof}

\begin{proposition}
    Let $\varphi:X\rightarrow Y$, $\psi:Y\rightarrow Z$ be separated morphisms of $k_H$-affinoid spaces. Then $\psi \circ \varphi:X\rightarrow Z$ is also separated.
\end{proposition}
\begin{proof}
    We have a Cartesian diagram
    \[
        \begin{tikzcd}
            X\times_Y X \arrow[r,"\psi"] \arrow[d] \arrow[rd, "\square", phantom] & X\times_Z X\arrow[d] \\
            Y \arrow[r,"\Delta_{Y/Z}"]                                                    & Y\times_Z Y          
        \end{tikzcd}.
    \]
    By \cref{prop-separatedbasechange}, $\psi: X\times_Y X\rightarrow X\times_Z X$ is a closed immersion. On the other hand, $\Delta_{X/Z}:X\rightarrow X\times_Z X$ factorizes as $\psi\circ \Delta_{X/Y}$. It follows from \cref{prop-closedimmcomposition} that $\Delta_{X/Z}$ is a closed immersion.
\end{proof}

\iffalse
\begin{proposition}
    Let $\varphi:X\rightarrow Y$ and $\psi:Y\rightarrow Z$ be morphisms of $k_H$-analytic spaces. If $\psi\circ \varphi$ is separated, then so is $\varphi$.
\end{proposition}
\fi


\begin{proposition}
    Let $H'\supseteq H$ is a subgroup of $\mathbb{R}_{>0}$. Let $\varphi:Y\rightarrow X$ be a morphism of $k_H$-analytic spaces. Then the following are equivalent:
    \begin{enumerate}
        \item $\varphi$ is separated;
        \item $\varphi$ is separated when view as a morphism of $k_{H'}$-affinoid spaces.
    \end{enumerate}
\end{proposition}
\begin{proof}
    This follows immediately from \cref{cor-closedimmchangegroup}.
\end{proof}

\begin{proposition}
    Let $\varphi:Y\rightarrow X$ be a morphism of $k_H$-analytic spaces and $K/k$ be an analytic field extension. 
    \begin{enumerate}
        \item If $\varphi$ is separated, so is $\varphi_K$;
        \item If $K=k_r$ for some $k$-free polyray $r$ and $\varphi_K$ is separated, then so is $\varphi$.
    \end{enumerate}
\end{proposition}
We will prove later on that the assumption in (2) is unnecessary.
\begin{proof}
    This follows immediately from \cref{cor-closeimmchangefield}.
\end{proof}

\section{Analytic germs}
Let $(k,|\bullet|)$ be a complete non-Archimedean valued field and $H$ be a subgroup of $\mathbb{R}_{>0}$ such that $|k^{\times}|\cdot H\neq \{1\}$.


\begin{definition}
    A \emph{punctured $k_H$-analytic space} $(X,x)$ is a $k_H$-analytic space $X$ together with a point $x\in X$.

    A morphism between punctured $k_H$-analytic spaces $(X,x)$ and $(Y,y)$ is a morphism $\varphi:X\rightarrow Y$ of $k_H$-analytic spaces sending $x$ to $y$.

    The category of punctured $k_H$-analytic spaces is denoted by $k_H\text{-}\AnaCat_*$.
\end{definition}

\begin{definition}
    A morphism of punctured $k_H$-analytic spaces $(X,x)\rightarrow (Y,y)$ is said to be \emph{separated} (resp. \emph{a closed immersion}) is the underlying morphism of  $k_H$-analytic spaces is separated (resp. a closed immersion).
\end{definition}

\begin{definition}
    The category $k_H\text{-}\GerCat$ is the category of right fractions of $k_H\text{-}\AnaCat_*$ with respect to the system of morphisms
    \[
        \varphi:(X,x)\rightarrow (Y,y)  
    \]
    that induces an isomorphism of $X$ with an open neighbourhood of $y$ in $Y$ in $k\text{-}\GerCat$.

    When we view $(X,x)$ as an object in $k_H\text{-}\GerCat$, we write it as $X_x$. An object in $k_H\text{-}\GerCat$ is called a $k_H$-analytic germ.
\end{definition}
Be careful, we require $\varphi$ to induce an isomorphism in $k\text{-}\GerCat$ instead of $k_H\text{-}\GerCat$, although eventually, we will show that these notions coincide.

By definition,
\[
    \Hom_{k_H\text{-}\GerCat}(X_x,Y_y)=\varinjlim_{U}\Hom_{k_H\text{-}\AnaCat_*}((U,x),(Y,y)),  
\]
where $U$ runs over all open neighbourhoods of $x$ in $X$. 

\begin{definition}
    A $k_H$-analytic germ $X_x$ is \emph{good} if $x$ admits an affinoid neighbourhood in $X$.
\end{definition}
Note that this condition does not depend on the representative $(X,x)$. To see this, let $U\subseteq x$ be an open subset containing $x$. We need to show that if $x$ admits a $k_H$-affinoid neighbourhood in $X$, then it admits one in $U$. This follows from \cref{Affinoid-prop-laurentdomainfundamental} in \nameref{Affinoid-chap-affinoid}.

\begin{definition}
    A morphism of $k_H$-analytic germs $\varphi:X_x\rightarrow Y_y$ is saied to be \emph{separated} (resp. \emph{boundaryless}, \emph{a closed immersion}) if it is induced by a separated morphism (resp. boundaryless, resp. a closed immersion) of punctured $k_H$-analytic spaces $(U,x)\rightarrow (Y,y)$, where $U$ is an open neighbourhood of $x$ in $X$.
\end{definition}

\begin{definition}
    Let $X_x$ be a $k_H$-analytic germ. A \emph{$k_H$-analytic domain} in $X_x$ is a $k_H$-analytic germ $V_x$, where $V$ is a $k_H$-analytic domain in $X$ containing $x$.
    
    We say a finite family of $k_H$-analytic germs $\{V_{ix}\}_{i\in I}$ \emph{covers} $X_x$ if there is a representative $(X,x)$ of $X_x$ such that $V_{ix}$ can be represented by a $k_H$-analytic domain $V_i\in X$ for $i\in I$ and  
    \[
        X=\bigcup_{i\in I}V_{i}.  
    \]
\end{definition}

\begin{definition}
    Let $\phi:Y_y\rightarrow X_x$ be a morphism of $k_H$-analytic germs and $V_x$ be a $k_H$-analytic domain in $X_x$. Represent $\phi$ by a morphism $\phi:(Y,y)\rightarrow (X,x)$ and represent $V_x$ by a $k_H$-analytic domain in $X$. Then the $k_H$-analytic domain $\phi^{-1}(V)$ in $Y$ determines a $k_H$-analytic germ $\phi^{-1}(V)_y$, which does not depend on the choices we made. This $k_H$-analytic germ is denoted by $\phi^{-1}(V_x)$.
\end{definition}
Recall that $\phi^{-1}(V)$ is a $k_H$-analytic domain in $Y$ by \cref{prop-analyticdomainprop}.


\begin{definition}
    Let $X$ be a good $k_H$-analytic space and $x\in X$, we define
    \[
        \mathcal{O}_{X,x}:=\varinjlim_{V} A_V,  
    \]
    where $V$ runs over all $k_H$-affinoid neighbourhoods of $x$ in $X$. \textcolor{red}{Include the definition of affinoid neighbourhoods}
\end{definition}
Observe that $k_H$-affinoid neighbourhoods of $x$ in $X$ are cofinal in the directed set of $k$-affinoid neighbourhoods of $x$ in $X$. This follows from \cref{Affinoid-prop-laurentdomainfundamental} in \nameref{Affinoid-chap-affinoid}. So we may let $V$ runs over all $k$-affinoid neighbourhoods of $x$ in $X$ as well.

\begin{example}
    Let $X_x$ be a $k_H$-analytic germ. Take a $k_H$-affinoid domain $\Sp A$ of $X$ containing $x$.
    Given $r\in \sqrt{|k^{\times}|\cdot H}^n$ and $f\in A^n$, we write
    \[
        X_x\{r^{-1}f\}:=\left(\Sp A\{r^{-1}f\}\right)_x.
    \]
    Then $X_x\{r^{-1}f\}$ is a $k_H$-analytic germ. Observe that $X_x\{r^{-1}f\}$ is independent of the choice of $\Sp A$. This construction depends only on the classes of $f$ in $\mathcal{O}_{X,x}^n$. Given $\bar{f}\in \mathcal{O}_{X,x}^n$, we define $X_x\{r^{-1}\bar{f}\}=X_x\{r^{-1}f\}$ for any $f\in A^n$ lifting $\bar{f}$ as above.
\end{example}

\section{Reduction}
Let $(k,|\bullet|)$ be a complete non-Archimedean valued field and $H$ be a subgroup of $\mathbb{R}_{>0}$ such that $|k^{\times}|\cdot H\neq \{1\}$.

In this section, when we do not specify the grading of a graded object, we mean it is $\mathbb{R}_{>0}$-graded. In particular $\tilde{k}$ means $\tilde{k}^{\mathbb{R}_{>0}}$.


\begin{definition}\label{def-reductionstar}
    Let $X=\Sp A$ be a $k$-affinoid space and $x\in X$, we define the \emph{reduction} $\widetilde{(X,x)}$ of $X$ at $x$ as follows: let $\chi_x:A\rightarrow \mathscr{H}(x)$ be the character corresponding to $x$, we define
    \[
        \widetilde{(X,x)}:=\mathbf{P}_{\widetilde{\mathscr{H}(x)}/\tilde{k}}\left\{ \widetilde{\chi_x}(\tilde{A})\right\}\subseteq \mathbf{P}_{\widetilde{\mathscr{H}(x)}/\tilde{k}}.
    \]
\end{definition}
Observe that $\widetilde{(X,x)}$ is an affine open subset of $\mathbf{P}_{\widetilde{\mathscr{H}(x)}/\tilde{k}}$. This follows from \cref{Affinoid-cor-reductionfinitelygenerated} in \nameref{Affinoid-chap-affinoid}.

\begin{lemma}\label{lma-reductionindepgoodgerm}
    Let $X=\Sp A$ be a $k$-affinoid space and $x\in X$. Let $U=\Sp B$ be a $k$-affinoid space. Let $\iota:U\rightarrow X$ be an isomorphism of $U$ with an open neighbourhood of $x$.
    We still write $\iota^{-1}(x)\in U$ as $x$.
    Then the natural morphism
    \[
        \widetilde{(U,x)}=\widetilde{(X,x)}.     
    \]
\end{lemma}
\begin{proof}
    We first recall that  $\mathscr{H}(x)$ does not depend on if we view $x$ as in $\Sp A$ or in $\Sp B$  by \cref{Affinoid-cor-affdomainresidueequal} in \nameref{Affinoid-chap-affinoid}. 

    Observe that the morphism $\chi_x:B\rightarrow \mathscr{H}(x)$ is boundaryless with respect to $A$ by \cref{Affinoid-prop-affinoiddomaininterior} in \nameref{Affinoid-chap-affinoid}. By \cref{Affinoid-prop-innerhomochar}  in \nameref{Affinoid-chap-affinoid}, $\widetilde{\chi_x}(\tilde{B})$ is finite over $\widetilde{\chi_x}(\tilde{A})$. By \cref{Commutative-lma-gradedRZaffeq} in \nameref{Commutative-chap-commutative}, we have
    \[
        \mathbf{P}_{\widetilde{\mathscr{H}(x)}/\tilde{k}}\left\{ \widetilde{\chi_x}(\tilde{A})\right\}=\mathbf{P}_{\widetilde{\mathscr{H}(x)}/\tilde{k}}\left\{ \widetilde{\chi_x}(\tilde{B})\right\}.
    \]
\end{proof}

\begin{definition}
    Let $X_x$ be a good $k$-analytic germs. Take an affinoid neighbourhood $U$ of $x$ in $X$, then we define
    \[
        \widetilde{X_x}:=\widetilde{(U,x)}\subseteq \mathbf{P}_{\widetilde{\mathscr{H}(x)}/\tilde{k}}.  
    \]
    By \cref{lma-reductionindepgoodgerm}, $\widetilde{X_x}$ depends only on $X_x$.
    
    The construction is clearly functorial in $X_x$.
\end{definition}


\begin{lemma}\label{lma-analyticdomainingood1}
    Let $X_x$ be a good $k$-analytic germ and $Y_x$ be a $k$-analytic domain in $X_x$. Then $Y_x$ can be covered by finitely many $k$-analytic domains in $X_x$ of the form
    \[
        X_x\left\{r^{-1}f \right\},  
    \]
    where $n\in \mathbb{N}$, $f=(f_1,\ldots,f_n)\in \mathcal{O}_{X,x}^{\times}$ is a tuple of invertible elements and $r_i=|f_i(x)|$.
\end{lemma}
\begin{proof}
    We may assume that $X$ is $k$-affinoid, say $X=\Sp A$. By \cref{Affinoid-cor-GG} in \nameref{Affinoid-chap-affinoid}, $Y$ can be covered by finitely many $k$-rational domains in $X$, say of the form $\Sp A\{r^{-1}g/h\}$, where $m\in \mathbb{N}$, $r=(r_1,\ldots,r_m)\in \mathbb{R}_{>0}^m$, $g=(g_1,\ldots,g_m)\in A^m$, $h\in A$ and $g_1,\ldots,g_m,h$ generates the unit ideal. We may assume that $Y=\Sp A\{r^{-1}g/h\}$.
    
    By shrinking $X$, we may assume that $h$ is invertible. Set $f_i=g_i/h$, then 
    \[
        Y=\Sp A\{r_1^{-1}f_1,\ldots,r_m^{-1}f_m\}.  
    \]
    By further shrinking $X$, it suffices to consider those $i$ with $|f_i(x)|=r_i$.
\end{proof}



\begin{lemma}\label{lma-affinoiddomainreductionspecial2}
    Let $X_x$ be a good $k$-analytic germ. Given $n\in \mathbb{N}$ and $f=(f_1,\ldots,f_n)\in \mathcal{O}_{X,x}^{\times}$, then
    \[
       \widetilde{X_x\{r^{-1}f\}}=\widetilde{X_x}\left\{\widetilde{\chi_x}(\tilde{f_1}),\ldots,\widetilde{\chi_x}(\tilde{f_n})\right\},  
    \]
    where $r=(r_1,\ldots,r_n)$ and $r_i=|f_i(x)|$ for $i=1,\ldots,n$.
\end{lemma}
\begin{proof}
    We may assume that $X$ is $k$-affinoid, say $X=\Sp A$. By induction on $n$, we may assume that $n=1$. Consider the admissible epimorphism
    \[
        \phi:A\{r^{-1}T\}\rightarrow A\{r^{-1}f\}  
    \]
    sending $T$ to $f$. By \cref{Affinoid-thm-reductionfinite} in \nameref{Affinoid-chap-affinoid}, 
    \[
        \tilde{\phi}:\tilde{A}[r^{-1}T]\rightarrow \widetilde{A\{r^{-1}f\}}
    \]
    is finite. Let $\chi_x:A\{r^{-1}f\}\rightarrow \mathscr{H}(x)$ be the character defined by $x$.
    
    Then $\widetilde{\chi_x}(\widetilde{A\{r^{-1}f\}})$ is finite over $\widetilde{\chi_x}(\tilde{A})[\tilde{f}]$. So the assertion follows from \cref{Commutative-lma-gradedRZaffeq} in \nameref{Commutative-chap-commutative}.
\end{proof}


\begin{lemma}\label{lma-goodaffinoiddomainredexp}
    Let $X_x$ be a good $k$-analytic germs and $Y_x$ be a good $k$-analytic domain in $X_x$. Then we can find $n\in \mathbb{N}$, $f_1,\ldots,f_n\in \mathcal{O}_{X,x}^{\times}$ such that
    \[
        \widetilde{Y_x}=  \widetilde{X_x}\left\{\widetilde{\chi_x}(\tilde{f_1}),\ldots,\widetilde{\chi_x}(\tilde{f_n})\right\}.
    \]
\end{lemma}

In particular, we can identify $\widetilde{Y_x}$ with an open susbet of $\widetilde{X_x}$. 

\begin{proof}
    The same argument as in \cref{lma-analyticdomainingood1} that we can assume that $X=\Sp A$ and $Y=\Sp A\{r^{-1}f\}$ for some $n\in \mathbb{N}$, $r=(r_1,\ldots,r_n)\in \mathbb{R}_{>0}^n$, $f=(f_1,\ldots,f_n)\in A^n$ with $r_i=|f_i(x)|$ for $i=1,\ldots,n$. So the assertion follows from \cref{lma-affinoiddomainreductionspecial2}.
\end{proof}

\begin{lemma}\label{lma-reductioncov}
    Let $X_x$ be a good $k$-analytic germ, $n\in \mathbb{Z}_{>0}$ and $Y_{1x},\ldots,Y_{nx}$ be a covering of $X_x$ by good $k$-analytic domains. Then 
    \[
        \widetilde{X_x}=\bigcup_{i=1}^n \widetilde{Y_{ix}}.  
    \]
\end{lemma}

\begin{proof}
    Observe that we are free to replace $\{Y_{ix}\}_i$ by its refinements by coverings by good $k$-analytic domains.  We may assume that $X$ is $k$-affinoid, say $X=\Sp A$. Then by \cref{Affinoid-lma-prooftatelma1} in \nameref{Affinoid-chap-affinoid}, we may assume that the covering is $k$-rational and is generated by $r_1^{-1}f_1,\ldots,r_n^{-1}f_n$. Up to shrinking $X$, we may guarantee that $|f_i(x)|=r_i$ for $i=1,\ldots,n$.
    In this case, the assertion follows from \cref{lma-affinoiddomainreductionspecial2}.
\end{proof}

\begin{lemma}
    Let $\phi:Y_y\rightarrow X_x$ be a morphism of good $k$-analytic germs. Let $X_x'$ be a good $k$-analytic domain in $X_x$ and set $Y_y'=\phi^{-1}(X_x')$, then 
    \[
        \widetilde{Y_y'}=\tilde{\phi}^{-1}(\widetilde{X_x'}).    
    \]
\end{lemma}
\begin{proof}
    By \cref{lma-analyticdomainingood1}, we may find $m\in \mathbb{Z}_{>0}$, $n_1,\ldots,n_m\in \mathbb{N}$, $g_{i1},\ldots,g_{in_i}\in \mathcal{O}_{X,x}^{\times}$ for $i=1,\ldots,m$ such that $X_x'$ is covered by $X\{r^{-1}_{i1}g_{i1},\ldots,r^{-1}_{in_i}g_{in_i} \}$ for $i=1,\ldots,m$, where $r_{ij}=|g_{ij}(x)|$ for $i=1,\ldots,m$, $j=1,\ldots,n_i$.

    Then $Y_x'$ is covered by $Y\{r^{-1}_{i1}g'_{i1},\ldots,r^{-1}_{in_i}g'_{in_i} \}$ for $i=1,\ldots,m$, where $g'_{ij}$ is the image of $g_{ij}$ in $\mathcal{O}_{Y,y}^{\times}$  for $i=1,\ldots,m$, $j=1,\ldots,n_i$.

    By \cref{lma-affinoiddomainreductionspecial2}, we have 
    \[
        \widetilde{X_x'}=\bigcup_{i=1}^m \widetilde{X_x}\left\{ \widetilde{\chi_{x}}(\widetilde{g_{i1}}),\ldots, \widetilde{\chi_{x}}(\widetilde{g_{in_i}}) \right\}
    \]
    and 
    \[
        \widetilde{Y_y'}=\bigcup_{i=1}^m \widetilde{Y_y}\left\{ \widetilde{\chi_{y}}(\widetilde{g_{i1}'}),\ldots, \widetilde{\chi_{y}}(\widetilde{g_{in_i}'}) \right\}.  
    \]
    Our assertion is now clear.
\end{proof}

\begin{definition}
    Let $X_x$ be a $k$-analytic germ. By \cref{lma-goodaffinoiddomainredexp}, the reduction defines a functor from the category of good $k$-analytic germs in $X_x$ (with inclusions as the morphisms) to the category of open affine subsets of $\mathbf{P}_{\widetilde{\mathscr{H}(x)}/\widetilde{k}}$. We define
    \[
        \widetilde{X_x}:=\varinjlim_{Y_x} \widetilde{Y_x},  
    \]
    where $Y_x$ runs over the filtered category of good $k$-analytic germs in $X_x$ and the colimit is taken in the category $\mathcal{T}_{\widetilde{\mathscr{H}(x)}/\widetilde{k}}$.

 The object $\widetilde{X_x}$ is called its \emph{reduction} of $X_x$.
\end{definition}

\iffalse
\begin{remark}
    There is some ambiguity at this point. Recall that by our definition, $\widetilde{k}$ is a $\sqrt{|k^{\times}|\cdot H}$-graded field instead of $H$-graded field. This is not consistent with our conventions in \cref{Commutative-sec-strictnessinRZ} in \nameref{Commutative-chap-commutative}.
    
    But thanks to \cref{Commutative-cor-Prestric1} in \nameref{Commutative-chap-commutative}, this inconsistency does not affect what we are doing in this section.
\end{remark}
\fi


\begin{thm}\label{thm-gradedreddomainbij}
    Let $X_x$ be a $k$-analytic germ. Then the reduction functor  
    \[
        k\text{-}\GerCat\rightarrow \mathcal{T}_{\widetilde{\mathscr{H}(x)}/\widetilde{k}} 
    \]
    establishes a bijection between the $k$-analytic domains and non-empty quasi-compact open subsets of $\widetilde{X_x}$.

    This bijection commutes with finite unions and finite intersections. 
\end{thm}
In \cite{Tem04}, the author forgot the non-emptyness assumption.

\begin{proof}
    The last assertion is obvious by construction.

    \textbf{Step~1}. We prove the theorem under the additional assumption that $X_x$ is good.

    \textbf{Step~1.1}. Let $l,m\in \mathbb{N}$ and $f=(f_1,\ldots,f_l)\in \mathcal{O}_{X,x}^l$, $g=(g_1,\ldots,g_m)\in \mathcal{O}_{X,x}^m$. Assume that 
    \[
        \widetilde{X_x}\{\tilde{f}\}\subseteq  \widetilde{X_x}\{\tilde{g}\}, 
    \]
    then we prove that 
    \[
        X_x\{r^{-1}f\}\subseteq X_x\{s^{-1}g\},  
    \]
    where $r=(r_1,\ldots,r_l)$, $s=(s_1,\ldots,s_m)$,
    \[
        r_i=|f_i(x)|,\quad s_j=|g_j(x)|  
    \]
    for $i=1,\ldots,l$, $j=1,\ldots,m$.


    We may assume that $X$ is $k$-affinoid, say $X=\Sp A$ and $f_1,\ldots,f_l,g_1,\ldots,g_m\in A$. Let $\chi_x:A\rightarrow \mathscr{H}(x)$ be the character of $x$. Let
    \[
        B=\widetilde{\chi_x}(\tilde{A})\subseteq \widetilde{\mathscr{H}(x)}.  
    \]
    By definition,
    \[
        \widetilde{X_x}=\mathbf{P}_{\widetilde{\mathscr{H}(x)}}\{B\}.  
    \]
    By \cref{lma-affinoiddomainreductionspecial2}, we have
    \[
        \begin{aligned}
        \widetilde{X_x\{r^{-1}f\}}=&\widetilde{X_x}\{B[\tilde{f}]\},\\
        \widetilde{X_x\{r^{-1}f,s^{-1}g\}}=&\widetilde{X_x}\{B[\tilde{f},\tilde{g}]\}.
        \end{aligned}
    \]
    The right-hand sides are equal by our assumption, so by \cref{Commutative-lma-gradedRZaffeq} in \nameref{Commutative-chap-commutative}, $B[\tilde{f},\tilde{g}]$ is finite over $B[\tilde{f}]$. We take monic polynomials of $\tilde{g}_j$ over $B[\tilde{f}]$:
    \[
        T^{n_j}+\tilde{a}_{j,1}T^{n-1}+\cdots+\tilde{a}_{j,n_j}\in B[\tilde{f}][T]
    \]
    with $\tilde{a}_{j,1},\ldots,\tilde{a}_{j,n_j}$ homogeneous of degree $|g_j(x)|^{1},\ldots,|g_j(x)|^{n_j}$ respectively. This is possible by \cref{Commutative-prop-gradedfiniteintegral} in \nameref{Commutative-chap-commutative}. We lift $\tilde{a}_{j,k}$ to $a_{j,k}\in A\{r^{-1}f\}$ with $\rho(a_{j,k})=\rho(g_j)^k$ for $j=1,\ldots,m$, $k=1,\ldots,n_j$. It follows that 
    \[
        \left|\left(g_j^{n_j}+a_{j,1}g_j^{n-1}+\cdots+a_{j,n_j}\right)(x)\right|<|g_j(x)|^n
    \]
    for $j=1,\ldots,m$. Up to shrinking $X$, we may assume that this inequality holds everywhere on $X\{r^{-1}f\}$. 

    By then $|g_j(y)|\leq |g_j(x)|$ for any $y\in X\{r^{-1}f\}$. Our assertion follows.

    \textbf{Step~1.2}. Suppose that $Y_x$ is a $k$-analytic domain in $X_x$ with $\tilde{Y}_x=\tilde{X}_x$, then $Y_x= X_x$.

    We may assume that $X$ is $k$-affioid, say $X=\Sp A$.

    By \cref{lma-analyticdomainingood1}, we can write $Y_x$ as a finite union of $V_{i,x}:=X_x\{r_i^{-1}f_i\}$ for $i=1,\ldots,m$, where $n_i\in \mathbb{N}$, $f_i=(f_{i1},\ldots,f_{in_i})\in \mathcal{O}_{X,x}^{\times,n_i}$ and $r_i=(r_{i1},\ldots,r_{in_i})$ with $r_{ij}=|f_{ij}(x)|$.

    By \cref{lma-reductioncov}, $\widetilde{V_{i,x}}$ for $i=1,\ldots,m$ covers $\widetilde{X_x}$.

    By \cref{Commutative-lma-opencovlaurentrefrz} in \nameref{Commutative-chap-commutative}, we can refine this covering to a Laurent covering 
    \[
        \mathcal{U}:=\left\{\tilde{U_j}=\widetilde{X_x}\{\tilde{g}^{j_1}_1,\ldots,\tilde{g}^{j_l}_l\}\right\}_{j=(j_1,\ldots,j_l)\in \{\pm 1\}^l },
    \]
    where $l\in \mathbb{N}$ and $\tilde{g}_1,\ldots,\tilde{g}_l$ are homogeneous elements in $\widetilde{\mathscr{H}(x)}$. Lift $\tilde{g}_1,\ldots,\tilde{g}_l$ to $g_1,\ldots,g_l\in A$. We consider the $k$-Laurent covering of $X$ generated by 
    \[
        \rho(\tilde{g}_1)^{-1}g_1,\ldots,\rho(\tilde{g}_l)^{-1}g_l.  
    \]
    The reduction of this covering is clearly $\mathcal{U}$. By Step~1.1, the germs of $\mathcal{U}$ at $x$ is a refinement of $\{V_{1,x},\ldots,V_{m,x}\}$, so the latter is a covering of $X_x$, namely $X_x\cn Y_x$.

    \textbf{Step~1.3}. We prove that each quasi-compact open subset $\widetilde{Y_x}$ of $\widetilde{X_x}$ is the reduction of some $k$-analytic domain $Y_x$ in $X_x$.

    We can write 
    \[
        \widetilde{Y_x}=\bigcup_{i=1}^m \widetilde{X_x}\{\widetilde{f_{i1}},\ldots,\widetilde{f_{in_i}}\},  
    \]
    where $m\in \mathbb{Z}_{>0}$, $n_1,\ldots,n_m\in \mathbb{N}$, $\widetilde{f_{ij}}\in \widetilde{\mathscr{H}(x)}$ are homogeneous elements for $i=1,\ldots,m$, $j=1,\ldots,n_i$.
    We lift $\widetilde{f_{ij}}$ to $f_{ij}\in \mathcal{O}_{X,x}$, it suffices to take 
    \[
        \bigcup_{i=1}^m X_x\left\{r_{i1}^{-1}f_{i1},\ldots, r_{in_i}^{-1}f_{in_i} \right\},  
    \]
    where $r_{ij}=|f_{ij}(x)|$ for $i=1,\ldots,m$, $j=1,\ldots,n_i$.

    \textbf{Step~1.4}. Suppose that $Y_x$, $Z_x$ are $k$-analytic domains in $X_x$ with $\widetilde{Y_x}=\widetilde{Z_x}$. Then we prove that $Y_x=Z_x$.

    Take $p,q\in \mathbb{N}$, good $k$-analytic domains $Y_x^1,\ldots,Y_x^p$ in $Y_x$ and good $k$-analytic domains $Z_x^1,\ldots,Z_x^p$ in $Z_x$ such that 
    \[
        \widetilde{Y_x}=\bigcup_{i=1}^p   \widetilde{Y_x^i}=\bigcup_{i=1}^q   \widetilde{Z_x^i}.
    \]
    Therefore, for any $i=1,\ldots,p$, $\{\widetilde{Y_x^i\cap Z^j_x}\}_{j=1,\ldots,q}$ is a covering of $\widetilde{Y^i_x}$. By Step~1.2, $\{Y_x^i\cap Z^j_x\}_{j=1,\ldots,q}$ is a covering of $Y_x^i$ for $i=1,\dots,p$.  So $Y_x\subseteq Z_x$. By symmetry $Y_x=Z_x$.

    We have finshed the proof when $X_x$ is good.


    \textbf{Step~2}. We handle the general case. 

    \textbf{Step~2.1}. We prove that each quasi-compact open subset $\widetilde{Y_x}$ of $\widetilde{X_x}$ is the reduction of some $k$-analytic domain $Y_x$ in $X_x$. 

    Take $p\in \mathbb{N}$, good $k$-analytic domains $X_x^1,\ldots,X_x^p$ in $X_x$ such that 
    \[
        \widetilde{X_x}=\bigcup_{i=1}^p   \widetilde{X_x^i}.
    \]
    By Step~1, $\widetilde{Y_x}\cap \widetilde{X_x^i}$ ca be lifted to a $k$-analytic domain $Y^i_x$ in $X_x^i$ for $i=1,\ldots,p$. The union of $Y^i_x$'s for $i=1,\ldots,p$ is a lifting of $\widetilde{Y_x}$.

    \textbf{Step~2.2}. Suppose that $Y_x$, $Z_x$ are $k$-analytic domains in $X_x$ with $\widetilde{Y_x}=\widetilde{Z_x}$. Then we prove that $Y_x=Z_x$.

    For each $i=1,\ldots,p$, we have
    \[
        \widetilde{Y_x\cap X_x^i}=\widetilde{Y_x}\cap \widetilde{X_x^i}=\widetilde{Z_x}\cap \widetilde{X_x^i}= \widetilde{Z_x\cap X_x^i}. 
    \]
    By Step~1, $Y_x\cap X_x^i=Z_x\cap X_x^i$ coincides for $i=1,\ldots,p$, so $Y_x=Z_x$.
\end{proof}

\begin{corollary}\label{cor-closedimmersionintermsofreduction}
    Let $\varphi:Y_y\rightarrow X_x$ be a morphism of $k$-analytic germs, then the following are equivalent:
    \begin{enumerate}
        \item $\varphi$ is a closed immersion;
        \item $\tilde{\varphi}:\widetilde{Y_y}\rightarrow \widetilde{X_x}$ is an isomorphism and $\varphi$ is represented by a G-locally closed immersion. 
    \end{enumerate}
\end{corollary}
\textcolor{red}{Include the notion of G-locally closed immersion somewhere}
\begin{proof}
    (1) $\implies$ (2): This is obvious.

    (2) $\implies$ (1): After shrinking $X$ and $Y$, we can take a $k$-analytic domain $X'$ in $X$, a neighbourhood $Y'$ of $y$ in $Y$ such that $\varphi(Y')\subseteq X'$ and the restriction $Y'\rightarrow X'$ is a closed immersion. It suffices to show that $\varphi$ is boundaryless at $y$. In other words, we need to show that $X_x'=X_x$. By \cref{thm-gradedreddomainbij}, this is equivalent to $\widetilde{X_x'}=\widetilde{X_x}$. By (1) $\implies$ (2) direction of this corollary, $\widetilde{X_x'}=\widetilde{Y_y'}$. But clearly $\widetilde{Y_y}=\widetilde{Y_y'}$. So our assertion follows.
\end{proof}

\begin{lemma}\label{lma-fiberproductreduction1}
    Let $Y_y\rightarrow X_x$, $Z_z\rightarrow X_x$ be morphisms of $k$-analytic germs. Let $T=Y\times_X Z$. Take a point $t\in T$ whose image in $Y$ is $y$ and whose image in $Z$ is $z$. Then the natural map
    \[
        \widetilde{T_t}\cong \left(\widetilde{Y_y}\times \mathbf{P}_{\widetilde{\mathscr{H}(y)}/\tilde{k}}\widetilde{\mathscr{H}(t)}/\tilde{k}\right)\times_{\left(\widetilde{X_x}\times \mathbf{P}_{\widetilde{\mathscr{H}(x)}/\tilde{k}}\widetilde{\mathscr{H}(t)}/\tilde{k}\right)}\left(\widetilde{Z_z}\times \mathbf{P}_{\widetilde{\mathscr{H}(z)}/\tilde{k}}\widetilde{\mathscr{H}(t)}/\tilde{k}\right)  
    \]
    is a homeomorphism.
\end{lemma}
\textcolor{red}{Existence of $t$ needs to be proved somewhere following Ducros}
\begin{proof}
    As both sides commute with colimits, we may assume that $X,Y,Z$ are all $k$-affinoid, say $X=\Sp A$, $Y=\Sp B$ and $Z=\Sp C$.

    Let $\chi_x:A\rightarrow \mathscr{H}(x)$ (resp. $\chi_y:B\rightarrow \mathscr{H}(y)$, resp. $\chi_z:C\rightarrow \mathscr{H}(z)$) be the character corresponding to $x$ (resp. $y$, resp. $z$).
    Let $\tilde{A}_0$ (resp. $\tilde{B}_0$, resp. $\tilde{C}_0$) be the image of $\widetilde{\chi_x}(\tilde{A})$ (resp. $\widetilde{\chi_y}(\tilde{B})$, resp. $\widetilde{\chi_z}(\tilde{C})$) in $\widetilde{\mathscr{H}(t)}$. 
    The character corresponding to $t$ is given by
    \[
        \chi_t:B\hat{\otimes}_A C\rightarrow \mathscr{H}(t). 
    \]
    So
    \[
        \widetilde{T_t}=  \mathbf{P}_{\widetilde{\mathscr{H}(t)/\tilde{k}}}\left\{ \Img \widetilde{\chi_t}\right\}.
    \]
    As $\tilde{B}\otimes_{\tilde{A}}\tilde{C}\rightarrow \widetilde{B\hat{\otimes}_A C}$ is finite by \cref{Affinoid-lma-tildetensorproductfinite} in \nameref{Affinoid-chap-affinoid}, by \cref{Commutative-lma-gradedRZaffeq} in \nameref{Commutative-chap-commutative}, we have
    \[
        \widetilde{T_t}=  \mathbf{P}_{\widetilde{\mathscr{H}(t)/\tilde{k}}}\left\{ \tilde{D}_0\right\}, 
    \]
    where $\tilde{D}_0$ is the image of the natural map $\tilde{B}\otimes_{\tilde{A}}\tilde{C}\rightarrow \widetilde{\mathscr{H}(t)}$,
    
    We are supposed to prove that 
    \[
        \mathbf{P}_{\widetilde{\mathscr{H}(t)/\tilde{k}}}\left\{ \tilde{D}_0\right\}\cong \mathbf{P}_{\widetilde{\mathscr{H}(t)/\tilde{k}}}\{\tilde{B}_0\} \times_{\mathbf{P}_{\widetilde{\mathscr{H}(t)/\tilde{k}}}\{\tilde{A}_0\}}\mathbf{P}_{\widetilde{\mathscr{H}(t)/\tilde{k}}}\{\tilde{C}_0\}.
    \]
    Equivalently,
    \[
        \mathbf{P}_{\widetilde{\mathscr{H}(t)/\tilde{k}}}\left\{ \tilde{D}_0\right\}\cong \mathbf{P}_{\widetilde{\mathscr{H}(t)/\tilde{k}}}\{\tilde{B}_0,\tilde{C}_0\}. 
    \]
    This is obvious as $\tilde{D}_0$ is generated by $\tilde{B}_0$ and $\tilde{C}_0$.
\end{proof}

\begin{corollary}\label{cor-analyticdomainfiberproductreduc}
    Let $Y_y\rightarrow X_x$ be a morphism of $k$-analytic germs and $V_x$ be a $k$-analytic domain in $X_x$. Let $W_y=Y_y\times_{X_x}V_x$. Then $\widetilde{W_y}$ is the preimage of $\widetilde{V_y}$ in $\widetilde{Y_y}$.
\end{corollary}
\begin{proof}
    This follows immediately from \cref{cor-closedimmersionintermsofreduction}.
\end{proof}


\begin{corollary}\label{cor-separatednessreductioncri}
    Let $\varphi:Y_y\rightarrow X_x$ be a morphism of $k$-analytic germs, then the following are equivalent:
    \begin{enumerate}
        \item $\varphi$ is separated;
        \item $\tilde{\varphi}:\widetilde{Y_y}\rightarrow \widetilde{X_x}$ is separated.
    \end{enumerate}
\end{corollary}
\begin{proof}
    Observe that the diagonal morphism $\Delta_{Y/X}:Y\rightarrow Y\times_X Y$ is a G-locally closed immersion, so by \cref{cor-closedimmersionintermsofreduction}, $\varphi$ is separated if and only if 
    \[
        \widetilde{\Delta_{Y/X}}: \widetilde{Y_y}\rightarrow \widetilde{(Y\times_X Y)_{(y,y)}} 
    \] 
    is an isomorphism.

    By \cref{lma-fiberproductreduction1}, the natural map
    \[
        \widetilde{Z_z}\rightarrow \widetilde{Y_y}\times_{X'}  \widetilde{Y_y}
    \]
    is a homeomorphism, where 
    \[
        X'=\widetilde{X_x}\times_{\mathbf{P}_{\widetilde{\mathscr{H}(x)}/\tilde{k}}}\mathbf{P}_{\widetilde{\mathscr{H}(y)}/\tilde{k}}.
    \]
    Thus, $\widetilde{\Delta_{Y/X}}$ is an isomorphism if and only if $\widetilde{Y_y}\rightarrow X'$ is injective, namely, $\widetilde{Y_y}\rightarrow \widetilde{X_x}$ is separated.
\end{proof}


\begin{lemma}\label{lma-khaffinoidnh}
    Let $X_x$ be a \emph{good} $k$-analytic germ. Then the following are equivalent:
    \begin{enumerate}
        \item $x$ admits a $k$-affinoid neighbourhood $V$ in $X$ which admits  a $k_H$-analytic strucutre;
        \item $\widetilde{X_x}$ is an $H$-strict affine open subset of $\mathbf{P}_{\widetilde{\mathscr{H}(x)}/\tilde{k}}$.
    \end{enumerate}
\end{lemma}
If Temkin's argument of \cref{lma-glueaffinetogood} is OK, we can remove the good assumption! 
\begin{proof}
    Assume (1). Let $V=\Sp A$ be as in (1). Let $\chi_x:A\rightarrow \mathscr{H}(x)$ be the character defined by $x$. It follows from \cref{Affinoid-thm-strictaffdfnequal} in \nameref{Affinoid-chap-affinoid} that $\widetilde{\chi_x}(\tilde{A})$ is contained in $\widetilde{\mathscr{H}(x)}^{\sqrt{|k^{\times}|\cdot H}}$. It follows that $\widetilde{X_x}$ is $H$-strict by \cref{Commutative-cor-affinesetsHstrictcri} in \nameref{Commutative-chap-commutative}.

    Assume (2).  Take $n\in \mathbb{N}$ and non-zero homogeneous elements $f_1,\ldots,f_n\in \widetilde{\mathscr{H}(x)}$ with degree $r_1,\ldots,r_n\in H$ such that 
    \[
        \widetilde{X_x}=  \mathbf{P}_{\widetilde{\mathscr{H}(x)}/\tilde{k}}\{f_1,\ldots,f_n\}.
    \]
    
    By assumption, $X_x$ is good. \footnote{\textcolor{red}{Assume that \cref{lma-glueaffinetogood} is correct, by \cref{thm-goodgermandreduction}, the germ $X_x$ is automatically good. }}
    So we can find a $k$-affinoid neighbourhood $V$ of $x$ in $X$. Up to shrinking $V=\Sp B$, we make find inveritble elements $h_1,\ldots,h_n$ in $B$ such that $\widetilde{h_i(x)}=f_i$ for $i=1,\ldots,n$.

    Let $h:V\rightarrow \mathbb{A}_k^{n}$ be the morphism induced by $h_1,\ldots,h_n$. \textcolor{red}{Include the functor of points of An}
    Let $t=h(x)$ and $W$ be the affinoid domain in $\mathbb{A}_k^{n}$ defined ny $|T_i|\leq r_i$ for $i=1,\ldots,n$. We have a commutative diagram
    \[
        \begin{tikzcd}
            {(V,x)} \arrow[r, "h"] \arrow[rd] & {(\mathbb{A}_k^n,t)} \\
                                              & {(W,t)} \arrow[u]   
        \end{tikzcd}.  
    \]
    Also observe that the morphism of $k$-analytic germs $X_x\rightarrow (\mathbb{A}_k^n)_t$ factorizes through $W_t$, as can be seen on the level of reduction. So up to shrinking $V$, we can find a $k$-affinoid neighbourhood $W'$ of $t$ in $\mathbb{A}_k^n$ such that $h(V)\subseteq W\cap W'$. We may assume that $W'$ is a $k_H$-analytic domain. As $\widetilde{V_x}=\widetilde{X_x}$ is the preimage of $\widetilde{(W\cap W')_t}=\widetilde{W_t}$, the morphism $V\rightarrow W\cap W'$ is boundaryless at $x$. As $W\cap W'$ is $k_H$-analytic, it follows from \cref{Affinoid-prop-closedsetininteriorcri} in \nameref{Affionid-chap-affinoid} that $x$ admits a $k_H$-affinoid neighbourhood in $X$.
\end{proof}

\begin{corollary}\label{cor-germstrictiffredstrict}
    Let $X_x$ be a $k$-analytic germ. The following are equivalent:
    \begin{enumerate}
        \item the germ $X_x$ is $k_H$-analytic;
        \item the reduction $\widetilde{X_x}$ is $H$-strict.
    \end{enumerate}
\end{corollary}
We say $X_x$ is $k_H$-analytic in the sense that it lies in the essential image of $k_H\text{-}\GerCat\rightarrow k\text{-}\GerCat$.
\begin{proof}
    (1) $\implies$ (2) follows immediately from \cref{lma-khaffinoidnh}.

    Assume (2). Let $\{U_i\}_{i\in I}$ be an $H$-strict atals of $\widetilde{X_x}$. For each $i\in I$, we can find a $k$-analytic domain $X_{i,x}$ in $X_x$ such that $U_i=\widetilde{X_{i,x}}$ by \cref{thm-gradedreddomainbij}. By \cref{thm-gradedreddomainbij}, 
    \[
        \widetilde{X_{i,x}\cap X_{j,x}}=U_i\cap U_u
    \]
    for all $i,j\in I$. So we may assume that $\widetilde{X_x}$ is an $H$-strict quasi-compact open subset of $\mathbf{P}_{\widetilde{\mathscr{H}(x)}/\tilde{k}}$.

    Cover $\widetilde{X_x}$ by finitely many $H$-strict affine open subset $V_1,\ldots,V_m$.  By \cref{thm-gradedreddomainbij}, we can lift $V_i$ to a $k_H$-analytic germ $W_{i,x}$ in $X_x$ for $i=1,\ldots,m$. Morevoer $W_{i,x}\cap W_{j,x}$ is $k_H$-analytic for any $j=1,\ldots,m$. It follows that $X_x$ is $k_H$-analytic.
\end{proof}

\begin{definition}
    Let $X_x$ be a $k_H$-analytic germ. We define 
    \[
        \widetilde{X_x}^H:=  \left(\widetilde{X_x}\right)^H.
    \]
    This makes sense by \cref{cor-germstrictiffredstrict}.
\end{definition}

\begin{proposition}\label{prop-morphismgermHstrictredcube}
    Let $Y_y, X_x$ be $k_H$-analytic germs. Then for any morphism $Y_y\rightarrow X_x$ in $k\text{-}\GerCat$, there is a unique continuous map $\widetilde{Y_y}^H\rightarrow \widetilde{X_x}^H$ making the diagram commutative:
    \[
        \begin{tikzcd}
            & \widetilde{Y_y} \arrow[rr] \arrow[dd] \arrow[ld]                        &                                                       & \widetilde{X_x} \arrow[dd] \arrow[ld]                        \\
\widetilde{Y_y}^H \arrow[rr, dotted] \arrow[dd]                  &                                                                         & \widetilde{X_x}^H \arrow[dd]                          &                                                              \\
            & \mathbf{P}_{\widetilde{\mathscr{H}(y)}/\tilde{k}} \arrow[rr] \arrow[ld] &                                                       & \mathbf{P}_{\widetilde{\mathscr{H}(x)}/\tilde{k}} \arrow[ld] \\
\mathbf{P}_{\widetilde{\mathscr{H}(y)}^H/\tilde{k}^H} \arrow[rr] &                                                                         & \mathbf{P}_{\widetilde{\mathscr{H}(x)}^H/\tilde{k}^H} &                                                             
\end{tikzcd}  
    \]
\end{proposition}
\begin{proof}
    This follows immediately from \cref{Commutative-prop-maprzdescentdtostrict} in \nameref{Commutative-chap-commutative}.
\end{proof}

\begin{proposition}\label{prop-germkhstructureunique}
    Let $X_x$ be a $k$-analytic germ. If $X_x$ lies in the essential image of $k_H\text{-}\GerCat\rightarrow k\text{-}\GerCat$, then the $k_H$-analytic germ whose image in $k\text{-}\GerCat$ is isomorphic to $X_x$ is unique up to a canonical isomorphism in $k_H\text{-}\GerCat$.
\end{proposition}
\begin{proof}
    Let $\tau=\{V_{i,x}\}_{i\in I}$ be a $k_H$-affinoid atlas defining the $k_H$-analytic structure on $X_x$. Let $V_x$ be a $k$-analytic domain in $X_x$ that admits a $k_H$-analytic structure. By  \cref{cor-germstrictiffredstrict}, $\widetilde{V_x}$ and $\widetilde{V_{i,x}}$ are all $H$-strict. But 
    \[
        \widetilde{V_x\cap V_{i,x}}=  \widetilde{V_x}\cap \widetilde{V_{i,x}},
    \] 
    so $V_x\cap V_{i,x}$ admits a $k_H$-analytic strucutre. But  $V_{i,x}$ is separated so the $k_H$-analytic structure is unique. Therefore, $V_x\cap V_{i,x}$ is $k_H$-analytic with respect to $\tau$ for any $i\in I$. So $V_x$ is $H$-strict with respect to $\tau$. 
\end{proof}

\begin{corollary}
    Let $X$ be a $k$-analytic space. Assume that $X$ admits a $k_H$-analytic structure, then the $k_H$-analytic structure is unique up to a unique isomorphism.
\end{corollary}
\begin{proof}
    This is a consequence of \cref{prop-germkhstructureunique}.
\end{proof}

\begin{thm}\label{thm-fullyfaithful}
    Let $H'\supseteq H$ be a subgroup of $\mathbb{R}_{>0}$. The natural functor
    \[
        k_H\text{-}\AnaCat \rightarrow k_{H'}\text{-}\AnaCat
    \]
    is fully faithful.
\end{thm}
\begin{proof}
    We may assume that $H'=\mathbb{R}_{>0}$. It suffices to prove the result at the level of germs, namely, it suffices to show that 
    \[
        k_H\text{-}\GerCat \rightarrow k\text{-}\GerCat
    \]
    is fully faithful.

    Let $X_x$, $Y_y$ be $k_H$-analytic germs and suppose that we are given a morphism $Y_y\rightarrow X_x$ in $k\text{-}\GerCat$. Let $V_x$ be a $k_H$-analytic domain in $X_x$. We need to show that $W_y:=Y_y\times_{X_x} V_x$ is a $k_H$-analytic domain in $Y_y$. We draw the Cartesian diagram:
    \[
        \begin{tikzcd}
            W_y \arrow[d] \arrow[r] \arrow[rd, "\square", phantom] & V_x \arrow[d] \\
            Y_y \arrow[r]                                          & X_x          
        \end{tikzcd}  
    \]
    By \cref{cor-germstrictiffredstrict}, $\widehat{X_x}$, $\widehat{Y_y}$ are $H$-strict and by \cref{prop-morphismgermHstrictredcube}, the preimage of $\widetilde{V_x}$ in $\widehat{Y_x}$ is open quasi-compact and $H$-strict. But this preimage is just $\widetilde{W_y}$ by \cref{cor-analyticdomainfiberproductreduc}, so we conclude that $W_y$ is $k_H$-analytic by \cref{cor-germstrictiffredstrict}.
\end{proof}

\begin{corollary}
    Let $X$ be a $k$-analytic space. Then there is at most one $k_H$-analytic space $X'$ up to isomorphisms in $k_H\text{-}\AnaCat$ whose image under
    \[
        k_H\text{-}\AnaCat \rightarrow k\text{-}\AnaCat
    \] 
    is isomorphic to $X$. 
\end{corollary}
In particular, we can and will view $k_H$-analytic spaces as $k$-analytic spaces that admit (necessarily unique) structures of $k_H$-analytic spaces.
\begin{proof}
    This follows immediately from \cref{thm-fullyfaithful}. 
\end{proof}


\section{Some results whose proofs I do not understand}


We introduce a lemma allowing one to tell when the gluing of two affinoid spaces in a suitable position is good.
\begin{lemma}\label{lma-glueaffinetogood} \textcolor{red}{This is probably problematic, we will avoid using this result!}
    Let $X$ be a separated compact $k$-analytic space and $x\in X$. Assume that
    \begin{enumerate}
        \item $\widetilde{X_x}\subseteq \mathbf{P}_{\widetilde{\mathscr{H}(x)}/\tilde{k}}$ is an affine subset;
        \item $X$ is the union of two $k$-affinoid domains $Y=\Sp B$ and $Z=\Sp C$ both containing $x$;
        \item there is a non-zero homogeneous element $\lambda\in \widetilde{\mathscr{H}(x)}$ such that
        \[
            \widetilde{Y_x}=\widetilde{X_x}\{\lambda\},\quad   \widetilde{Z_x}=\widetilde{X_x}\{\lambda^{-1}\}.
        \]
    \end{enumerate}
    Then $X_x$ is good.
\end{lemma}

\textcolor{red}{This proof does not make much sense to me. I am just reproducing the arguments of Temkin. Need some reflection!}


\begin{proof}
    We observe that we are free to shrink $X$ to $k$-analytic domains of the following form: $Y'\cup Z'$, where $Y'$ and $Z'$ are $k$-affinoid neighbourhoods of $x$ in $Y$ and $Z$ respectively. We will express this procedure as \emph{shrinking $X$}.

    \textbf{Step~1}. We show that after shrinking $X$, we may assume that $Y\cap Z=\Sp A$, where
    \[
        A=B\{tf^{-1}\}=C\{t^{-1}g\}  
    \]
    for some $f\in B$ and $g\in C$ and $\rho_A(f-g)<t:=\rho(\lambda)$ and $\lambda=\widetilde{f(x)}$.



    By \cref{prop-liftelementsinHxred}, up to shrinking $X$, we can find $f\in B$, $g\in C$ both invertible such that 
    \[
        \lambda=\widetilde{f(x)}=\widetilde{g(x)}.
    \]
     

    It therefore follows that 
    \[
        |(f-g)(x)|<t.  
    \] 
    After shrinking $X$, we can make sure that 
    \[
        \sup_{y\in Y\cap Z}|(f-g)(y)|<t.
    \]
    After shrinking $X$, we can guarantee that
    \[
        \sup_{y\in Y} |f(y)|\leq t,\quad \inf_{z\in Z}|g(z)|\geq t.
    \]
     \textcolor{red}{Why? This is not an open condition on $Y$ or $Z$!!!!}  
    
    In particular,
    \[  
        Y\cap Z\subseteq Y\{tf^{-1}\}\cap Z\{t^{-1}g\}.
    \]
    \textcolor{red}{This relies on the unjustified claim}

    By \cref{lma-affinoiddomainreductionspecial2}, we have
    \[
        \widetilde{Y\{tf^{-1}\}_x}= \widetilde{Z\{t^{-1}g\}_x}= \widetilde{(Y\cap Z)_x}.
    \]
    Applying \cref{thm-gradedreddomainbij}, we can find $k$-affinoid neighbourhoods $Y'=\Sp B'$ and $Z'=\Sp C'$ of $x$ in $Y$ and $Z$ respectively such that
    \[
        Y'\cap Z=Y'\{tf^{-1}\},\quad Y\cap Z'=Z'\{t^{-1}g\}. 
    \]
    As $Y'\cap Z'$ is a $k$-affinoid neighbourhood of $x$ in $Y\cap Z$, we can find a $k$-Laurent neighbourhood $W$ of $x$ in $Y\cap Z$ which is contained in $Y'\cap Z'$ and which is of the form
    \[
        (Y\cap Z)\left\{s^{-1}u,s'v^{-1}\right\},  
    \]
    where $n,m\in \mathbb{N}$, $s=(s_1,\ldots,s_n)\in \mathbb{R}_{>0}^n$, $s'=(s'_1,\ldots,s'_m)\in \mathbb{R}_{>0}^m$ and $u=(u_1,\ldots,u_n)\in A^n$, $v=(v_1,\ldots,v_m)\in A^m$. This follows from \cref{Affinoid-prop-laurentdomainfundamental} in \nameref{Affinoid-chap-affinoid}.


    As $Y'\cap Z=Y'\{tf^{-1}\}$ is a $k$-Weierstrass domain in $Y'$, by \cref{Affinoid-prop-Weirestrassdomainequivdense} in \nameref{Affinoid-chap-affinoid},
    we can find $u_i',v_j'\in B'$ sufficiently close to $u_i,v_j$ over $Y'\cap Z$ for $i=1,\dots,n$, $j=1,\ldots,m$ so that $Y'':=Y'\{s^{-1}u',s'v'^{-1}\}$ is a neighbourhood of $x$ in $Y'$. Similarly, we can find $Z''=Z'\{s^{-1}u'',s'v''^{-1}\}$ for suitable perturbations of $u$ and $v$. Moreover,
    \[
        W=Y''\cap (Y'\cap Z)=Z''\cap (Y\cap Z').  
    \]
    It follows that $W=Y''\{tf^{-1}\}=Z''\{t^{-1}g\}=Y''\cap Z''$. Replacing $Y$ and $Z$ by $Y''$ and $Z''$ and $X$ by $Y''\cup Z''$, we reduce to the situation stated in this step.


    \textbf{Step~2}. We show that after shrinking $X$, we may guarantee that there are admissible epimorphisms 
    \begin{equation}\label{eq-admepiintermstep}
        \begin{aligned}
        k\{r^{-1}T,t^{-1}S_1,pS_2\}\rightarrow B &,\quad T_i\mapsto f_i \text{ for } i=1,\ldots,n,S_1\mapsto f ,S_2\mapsto f^{-1},\\
        k\{r^{-1}T,q^{-1}S_1,tS_2\}\rightarrow C &,\quad T_i\mapsto g_i \text{ for } i=1,\ldots,n,S_1\mapsto g ,S_2\mapsto g^{-1},
        \end{aligned}
    \end{equation}
    where $n\in \mathbb{N}$, $p<t<q$, $r=(r_1,\ldots,r_n)\in \mathbb{R}^n_{>0}$, $f_1,\ldots,f_n\in B$, $g_1,\ldots,g_n\in C$ and 
    \[
        \|f_i-g_i\|<r_i\text{ for }i=1,\ldots,n;\quad  \|f-g\|< t, 
    \]
    where the norm $\|\bullet\|$ on $A$ is the quotient norm $A$ induced by
    \[
        k\{r^{-1}T,t^{-1}S_1,tS_2\}\rightarrow B,\quad T_i\mapsto f_i\text{ for } i=1,\ldots,n, S_1\mapsto f, S_2\mapsto f^{-1}.
    \]
    \textcolor{red}{In order to guarantee that $r$ in both morphisms are the same, we need an argument as in Step~1, which is problematic! This is unfortunately essential to Step~3. I cannot make sense of this proof anymore!}

    As $\widetilde{X_x}$ is affine, we can write it as 
    \[
        \mathbf{P}_{\widetilde{\mathscr{H}(x)}/\tilde{k}}\left\{\alpha_1,\ldots,\alpha_m\right\}  
    \]
    for some non-zero homogeneous elements $\alpha_i\in \widetilde{\mathscr{H}(x)}$ for $i=1,\ldots,m$. 
    
    By \cref{prop-liftelementsinHxred}, after shrinking $X$, we may assume that $\alpha_i=\widetilde{f_i(x)}$ for some invertible $f_i\in B$ for $i=1,\ldots,m$. 
    
    Let $r_i=|\rho(f_i)|$ for $i=1,\ldots,m$. Set 
    \[
        D=k\{r^{-1}T,t^{-1}S_1,pS_2\}.  
    \]
    Let $\phi:D\rightarrow B$ be the first morphism in the beginning of this step and $\chi_x:B\rightarrow \mathscr{H}(x)$ be the character corresponding to $x$. Then $\varphi:=\chi_x\circ \phi$ satisfies
    \[
        \tilde{\varphi}(\tilde{D})=\tilde{k}[\alpha_1,\ldots,\alpha_m,\lambda].
    \]
    On the other hand, $\widetilde{Y_x}=\mathbf{P}_{\widetilde{\mathscr{H}(x)}/\tilde{k}}\{\alpha_1,\ldots,\alpha_m,\lambda\}$ by \cref{lma-affinoiddomainreductionspecial2}. It follows from \cref{Commutative-lma-gradedRZaffeq} in \nameref{Commutative-chap-commutative} and \cref{Affinoid-prop-innerhomochar} in \nameref{Affinoid-chap-affinoid} that $\chi_x$ is inner with respect to $D$. It follows from \cref{Affinoid-prop-innerhomochar} in \nameref{Affinoid-chap-affinoid} that $\phi$ can be extended to a continuous epimorphism
    \[
        k\{r^{-1}T,t^{-1}S_1,pS_2,U_1,\ldots,U_a\}\rightarrow B,\quad U_i\mapsto u_i\text{ for }i=1,\ldots,a  
    \]
    with $|u_i(x)|<1$ for $i=1,\ldots,a$.

    As $Y\cap Z$ is a $k$-Weierstrass domain in $Z$, we can find $g_1,\ldots,g_m,g'\in C$ close enough to $f_1,\ldots,f_m,f$ on $Y\cap Z$. Up to shrinking $Z$, we may also guarantee that they are close on $Z$. In particular, we can replace $g$ by $g'$ but guaranteeing that $Y\cap Z=Z\{t^{-1}g\}$ still holds.

    Similarly, one constructs 
    \[
        k\{r^{-1}T,q^{-1}S_1,tS_2,V_1,\ldots,V_b\}\rightarrow C,\quad V_i\mapsto v_i\text{ for }i=1,\ldots,b  
    \]
    with $|v_i(x)|<1$ for $i=1,\ldots,b$.
    By perturbation, we can find $u'_1,\ldots,u'_a\in C$, $v'_1,\ldots,v'_b$ close to $u_1,\ldots,u_a$ and $v_1,\ldots,v_b$ on $Y\cap Z$ such that 
    \[
        Y'=Y\{v_1',\ldots,v_b'\},\quad Z'=Z\{u_1',\ldots,u_a'\}  
    \]
    are neighbourhoods of $x$ in $Y$ and $Z$ respectively. Up to replacing $Y$ and $Z$ by $Y'$ and $Z'$, we conclude this step.
    
    \textbf{Step~3}. We show that $X$ is $k$-affinoid after the reduction in Step~2.

    We first assume that the two maps in \eqref{eq-admepiintermstep} are both isomorphisms.
    
    Let $B_+$ denote the subspace of $B$ consisting of elements of the form:
    \[
        \sum_{\alpha\in \mathbb{N}^n}\sum_{j=0}^{\infty} \lambda_{\alpha,j}f^{\alpha} f^j.  
    \]
    Let $C_-$ denote the subspace of $C$ consisting of elements of the form:
    \[
        \sum_{\alpha\in \mathbb{N}^n}\sum_{j=-\infty}^{0} \lambda_{\alpha,j}g^{\alpha} g^j.  
    \]
    We observe that each element $A=B_++C_-$. Take $a\in A$ and expand
    \[
        a=\sum_{\alpha\in \mathbb{N}^n}\sum_{j=-\infty}^{\infty} \lambda_{\nu,j}f^{\alpha}f^j.  
    \]
    Then 
    \[
        a=  \sum_{\alpha\in \mathbb{N}^n}\sum_{j=-\infty}^{0} \lambda_{\nu,j}f^{\alpha}f^j+\sum_{\alpha\in \mathbb{N}^n}\sum_{j=1}^{\infty} \lambda_{\nu,j}f^{\alpha}f^j  
    \]
    is the desired decomposition.

    In particular, for $i=1,\ldots,n$, we can write
    \[
        f_i-g_i=b_i+c_i,\quad f-g=b+c  
    \]
    with $b,b_1,\ldots,b_n\in B_+$, $c,c_1,\ldots,c_n\in C_-$. Then $h_i:=f_i-b_i$ and $h:=f-b$ are contained in $D:=B\cap C$. It follows that 
    \[
        \begin{aligned}
        k\{r^{-1}T,t^{-1}S,pS^{-1}\}\rightarrow B &,\quad T_i\mapsto h_i \text{ for } i=1,\ldots,n,S\mapsto h,\\
        k\{r^{-1}T,q^{-1}S,tS\}\rightarrow C &,\quad T_i\mapsto h_i \text{ for } i=1,\ldots,n,S\mapsto h,
        \end{aligned}
    \]
    are isomorphisms. It follows that $D\cong k\{r^{-1}T,pS^{-1},q^{-1}S\}$ and $X\cong \Sp D$.

    Now we handle the general case. Let
    \[
        A'=  k\{r^{-1}T,t^{-1}S,tS^{-1}\}
    \]
    and consider the epimorphism
    \[
        A'\rightarrow A,\quad T_i\mapsto f_i\text{ for }i=1,\ldots,n; S\mapsto f.  
    \]
    We can find preimages $G_1,\ldots,G_n,G\in A'$ of $g_1,\ldots,g_n,g$ such that the norms of $T_i-G_i$ and $S-G$ are small enough. THen the map
    \[
        k\{r^{-1}T,t^{-1}S,tS^{-1}\}\rightarrow A',\quad T_i\mapsto G_i,S\mapsto G
    \]
    is an isomorphism. Let
    \[
        \begin{aligned}
            B'= & k\{r^{-1}T,t^{-1}S,pS^{-1}\},\\
            C'= & k\{r^{-1}G,q^{-1}G,tG^{-1}\}.
        \end{aligned}  
    \]
    Set $Y'=\Sp B'$ and $Z'=\Sp C'$. By construction, we have canonical isomorphisms
    \[
        Y'\{S^{-1}\}\cong Z'\{G\}\cong \Sp A'.  
    \]
    By the previous case, $X'$ obtained by gluing $Y'$ and $Z'$ along $\Sp A'$ is $k$-affinoid. The homomorphisms $B'\rightarrow A$ and $C'\rightarrow A$ factorize through $B$ and $C$ and gives closed immersions $Y\rightarrow Y'$, $Z\rightarrow Z'$. The latter gives rise to a closed immersion $Y\cap Z\rightarrow Y'\cap Z'$. Hence, we get a closed immersion $X\rightarrow X'$. So $X$ is $k$-affinoid.
\end{proof}


\begin{thm}\label{thm-goodgermandreduction}\textcolor{red}{This is probably problematic, we will avoid using this result!}
    Let $X_x$ be a $k$-analytic germ. Then the following are equivalent:
    \begin{enumerate}
        \item $X_x$ is good;
        \item $\widetilde{X_x}$ is an affine open subset of $\mathbf{P}_{\widetilde{\mathscr{H}(x)}/\tilde{k}}$.
    \end{enumerate}
\end{thm}

\begin{proof}
    (1) $\implies$ (2). This follows from the definition.

    (2) $\implies$ (1). In proving this direction, we are free to replace $X$ by its $k$-analytic domain that contains $x$ in the interior. We will express this procedure simply as \emph{shrinking $X$}.
    
    By \cref{cor-separatednessreductioncri}, we may assume that $X$ is a compact separated $k$-analytic space.  Let $\{Y_i\}_{i\in I}$ be a finite $k$-affinoid covering of $X$. After shrinking $X$, we may assume that $x\in Y_i$ for each $i\in I$. 

    By \cref{thm-gradedreddomainbij}, $\{\widetilde{Y_{i,x}}\}_{i\in I}$ is a finite affine covering of $\widetilde{X_x}$.


    By \cref{Commutative-lma-opencovlaurentrefrz} in \nameref{Commutative-chap-commutative}, we can find a Laurent covering 
    \[
        \mathcal{V}=\left\{\widetilde{X_x}\{f_1^{\pm 1},\ldots,f_n^{\pm 1}\}\right\}
    \] 
    of $\widetilde{X_x}$ refining $\{\widetilde{Y_{i,x}}\}_{i\in I}$, where $n\in \mathbb{N}$ and $f_1,\ldots,f_n$ are homogeneous elements in $\widetilde{\mathscr{H}(x)}$. By \cref{thm-gradedreddomainbij}, we can lift each element in $\mathcal{V}$ to a $k$-analytic domain $V_{i,x}$ in $X_x$. After shrinking $X$, we may assume that $X=V_1\cup\cdots\cup V_n$. 
    
    By induction on $n$, we reduce easily to the case $n=1$. Now it  suffices to apply \cref{lma-glueaffinetogood}.
\end{proof}


\printbibliography
\end{document}