
\documentclass{amsbook} 
%\usepackage{xr}
\usepackage{xr-hyper}
\usepackage[unicode]{hyperref}


\usepackage[T1]{fontenc}
\usepackage[utf8]{inputenc}
\usepackage{lmodern}
\usepackage{amssymb,tikz-cd}
%\usepackage{natbib}
\usepackage[english]{babel}

\usepackage[nameinlink,capitalize]{cleveref}
\usepackage[style=alphabetic,maxnames=99,maxalphanames=5, isbn=false, giveninits=true, doi=false]{biblatex}
\usepackage{lipsum, physics}
\usepackage{ifthen}
\usepackage{microtype}
\usepackage{booktabs}
\usetikzlibrary{calc}
\usepackage{emptypage}
\usepackage{setspace}
\usepackage[margin=0.75cm, font={small,stretch=0.80}]{caption}
\usepackage{subcaption}
\usepackage{url}
\usepackage{bookmark}
\usepackage{graphicx}
\usepackage{dsfont}
\usepackage{enumitem}
\usepackage{mathtools}
\usepackage{csquotes}
\usepackage{silence}
\usepackage{mathrsfs}
\usepackage{bigints}

\WarningFilter{biblatex}{Patching footnotes failed}


\ProcessOptions\relax

\emergencystretch=1em

\hypersetup{
colorlinks=true,
linktoc=all
}

\setcounter{tocdepth}{1}


\hyphenation{archi-medean  Archi-medean Tru-ding-er}

%\captionsetup[table]{position=bottom}   %% or below
\renewcommand{\thefootnote}{\fnsymbol{footnote}}
%\DeclareMathAlphabet{\mathcal}{OMS}{cmsy}{m}{n}
\renewbibmacro{in:}{}

\DeclareFieldFormat[article]{citetitle}{#1}
\DeclareFieldFormat[article]{title}{#1}
\DeclareFieldFormat[inbook]{citetitle}{#1}
\DeclareFieldFormat[inbook]{title}{#1}
\DeclareFieldFormat[incollection]{citetitle}{#1}
\DeclareFieldFormat[incollection]{title}{#1}
\DeclareFieldFormat[inproceedings]{citetitle}{#1}
\DeclareFieldFormat[inproceedings]{title}{#1}
\DeclareFieldFormat[phdthesis]{citetitle}{#1}
\DeclareFieldFormat[phdthesis]{title}{#1}
\DeclareFieldFormat[misc]{citetitle}{#1}
\DeclareFieldFormat[misc]{title}{#1}
\DeclareFieldFormat[book]{citetitle}{#1}
\DeclareFieldFormat[book]{title}{#1} 


%% Define various environments.

\theoremstyle{definition}
\newtheorem{theorem}{Theorem}[section]
\newtheorem{thm}[theorem]{Theorem}
\newtheorem{proposition}[theorem]{Proposition}
\newtheorem{corollary}[theorem]{Corollary}
\newtheorem{lemma}[theorem]{Lemma}
\newtheorem{conjecture}[theorem]{Conjecture}
\newtheorem{question}[theorem]{Question}
\newtheorem{example}[theorem]{Example}
\newtheorem{definition}[theorem]{Definition}
\newtheorem{condition}[theorem]{Condition}

\theoremstyle{remark}
\newtheorem{remark}[theorem]{Remark}
\numberwithin{equation}{section}

%\renewcommand{\thesection}{\thechapter.\arabic{section}}
%\renewcommand{\thetheorem}{\thesection.\arabic{theorem}}
%\renewcommand{\thedefinition}{\thesection.\arabic{definition}}
%\renewcommand{\theremark}{\thesection.\arabic{remark}}


%% Define new operators

\DeclareMathOperator{\nd}{nd}
\DeclareMathOperator{\ord}{ord}
\DeclareMathOperator{\Hom}{Hom}
\DeclareMathOperator{\PreSh}{PreSh}
\DeclareMathOperator{\Gr}{Gr}
\DeclareMathOperator{\Homint}{\mathcal{H}\mathrm{om}}
\DeclareMathOperator{\Torint}{\mathcal{T}\mathrm{or}}
\DeclareMathOperator{\Div}{div}
\DeclareMathOperator{\DSP}{DSP}
\DeclareMathOperator{\Diff}{Diff}
\DeclareMathOperator{\MA}{MA}
\DeclareMathOperator{\NA}{NA}
\DeclareMathOperator{\AN}{an}
\DeclareMathOperator{\Rep}{Rep}
\DeclareMathOperator{\Rest}{Res}
\DeclareMathOperator{\DF}{DF}
\DeclareMathOperator{\VCart}{VCart}
\DeclareMathOperator{\PL}{PL}
\DeclareMathOperator{\Bl}{Bl}
\DeclareMathOperator{\Td}{Td}
\DeclareMathOperator{\Fitt}{Fitt}
\DeclareMathOperator{\Ric}{Ric}
\DeclareMathOperator{\coeff}{coeff}
\DeclareMathOperator{\Aut}{Aut}
\DeclareMathOperator{\Capa}{Cap}
\DeclareMathOperator{\loc}{loc}
\DeclareMathOperator{\vol}{vol}
\DeclareMathOperator{\Val}{Val}
\DeclareMathOperator{\ST}{ST}
\DeclareMathOperator{\Amp}{Amp}
\DeclareMathOperator{\Herm}{Herm}
\DeclareMathOperator{\trop}{trop}
\DeclareMathOperator{\Trop}{Trop}
\DeclareMathOperator{\Cano}{Can}
\DeclareMathOperator{\PS}{PS}
\DeclareMathOperator{\Var}{Var}
\DeclareMathOperator{\Psef}{Psef}
\DeclareMathOperator{\Jac}{Jac}
\DeclareMathOperator{\Char}{char}
\DeclareMathOperator{\Red}{red}
\DeclareMathOperator{\Spf}{Spf}
\DeclareMathOperator{\Span}{Span}
\DeclareMathOperator{\Der}{Der}
%\DeclareMathOperator{\Mod}{mod}
\DeclareMathOperator{\Hilb}{Hilb}
\DeclareMathOperator{\triv}{triv}
\DeclareMathOperator{\Frac}{Frac}
\DeclareMathOperator{\diam}{diam}
\DeclareMathOperator{\Spec}{Spec}
\DeclareMathOperator{\Spm}{Spm}
\DeclareMathOperator{\Specrel}{\underline{Sp}}
\DeclareMathOperator{\Sp}{Sp}
\DeclareMathOperator{\reg}{reg}
\DeclareMathOperator{\sing}{sing}
\DeclareMathOperator{\Star}{Star}
\DeclareMathOperator{\relint}{relint}
\DeclareMathOperator{\Cvx}{Cvx}
\DeclareMathOperator{\Int}{Int}
\DeclareMathOperator{\Supp}{Supp}
\DeclareMathOperator{\FS}{FS}
\DeclareMathOperator{\RZ}{RZ}
\DeclareMathOperator{\Redu}{red}
\DeclareMathOperator{\lct}{lct}
\DeclareMathOperator{\Proj}{Proj}
\DeclareMathOperator{\Sing}{Sing}
\DeclareMathOperator{\Conv}{Conv}
\DeclareMathOperator{\Max}{Max}
\DeclareMathOperator{\Tor}{Tor}
\DeclareMathOperator{\Gal}{Gal}
\DeclareMathOperator{\Frob}{Frob}
\DeclareMathOperator{\coker}{coker}
\DeclareMathOperator{\Sym}{Sym}
\DeclareMathOperator{\CSp}{CSp}
\DeclareMathOperator{\Img}{Im}


\newcommand{\alg}{\mathrm{alg}}
\newcommand{\Sh}{\mathrm{Sh}}
\newcommand{\fin}{\mathrm{fin}}
\newcommand{\BPF}{\mathrm{BPF}}
\newcommand{\dBPF}{\mathrm{dBPF}}
\newcommand{\divf}{\mathrm{Div}^f}
\newcommand{\nef}{\mathrm{nef}}
\newcommand{\Bir}{\mathrm{Bir}}
\newcommand{\hO}{\hat{\mathcal{O}}}
\newcommand{\bDiv}{\mathrm{Div}^{\mathrm{b}}}
\newcommand{\un}{\mathrm{un}}
\newcommand{\sep}{\mathrm{sep}}
\newcommand{\diag}{\mathrm{diag}}
\newcommand{\Pic}{\mathrm{Pic}}
\newcommand{\GL}{\mathrm{GL}}
\newcommand{\SL}{\mathrm{SL}}
\newcommand{\LS}{\mathrm{LS}}
\newcommand{\GLS}{\mathrm{GLS}}
\newcommand{\GLSi}{\mathrm{GLS}_{\cap}}
\newcommand{\PGLS}{\mathrm{PGLS}}
\newcommand{\Loc}[1][S]{_{\{{#1}\}}}
\newcommand{\cl}{\mathrm{cl}}
\newcommand{\otL}{\hat{\otimes}^{\mathbb{L}}}
\newcommand{\ddpp}{\mathrm{d}'\mathrm{d}''}
\newcommand{\TC}{\mathcal{TC}}
\newcommand{\ddPP}{\mathrm{d}'_{\mathrm{P}}\mathrm{d}''_{\mathrm{P}}}
\newcommand{\PSs}{\mathcal{PS}}
\newcommand{\Gm}{\mathbb{G}_{\mathrm{m}}}
\newcommand{\End}{\mathrm{End}}
\newcommand{\Aff}[1][X]{\mathcal{M}\left(\mathcal{#1}\right)}
\newcommand{\XG}[1][X]{{#1}_{\mathrm{G}}}
\newcommand{\convC}{\xrightarrow{C}}
\newcommand{\Vect}{\mathrm{Vect}}
\newcommand{\abso}[1]{\lvert#1\rvert}
\newcommand{\Mdl}{\mathrm{Model}}
\newcommand{\cn}{\stackrel{\sim}{\longrightarrow}}
\newcommand{\sbc}{\mathbf{s}}
\newcommand{\CH}{\mathrm{CH}}
\newcommand{\GR}{\mathrm{GR}}
\newcommand{\dc}{\mathrm{d}^{\mathrm{c}}}
\newcommand{\Nef}{\mathrm{Nef}}
\newcommand{\Adj}{\mathrm{Adj}}
\newcommand{\DHm}{\mathrm{DH}}
\newcommand{\An}{\mathrm{an}}
\newcommand{\Rec}{\mathrm{Rec}}
\newcommand{\dP}{\mathrm{d}_{\mathrm{P}}}
\newcommand{\ddp}{\mathrm{d}_{\mathrm{P}}'\mathrm{d}_{\mathrm{P}}''}
\newcommand{\ddc}{\mathrm{dd}^{\mathrm{c}}}
\newcommand{\ddL}{\mathrm{d}'\mathrm{d}''}
\newcommand{\PSH}{\mathrm{PSH}}
\newcommand{\CPSH}{\mathrm{CPSH}}
\newcommand{\PSP}{\mathrm{PSP}}
\newcommand{\WPSH}{\mathrm{WPSH}}
\newcommand{\Ent}{\mathrm{Ent}}
\newcommand{\NS}{\mathrm{NS}}
\newcommand{\QPSH}{\mathrm{QPSH}}
\newcommand{\proet}{\mathrm{pro-ét}}
\newcommand{\XL}{(\mathcal{X},\mathcal{L})}
\newcommand{\ii}{\mathrm{i}}
\newcommand{\Cpt}{\mathrm{Cpt}}
\newcommand{\bp}{\bar{\partial}}
\newcommand{\ddt}{\frac{\mathrm{d}}{\mathrm{d}t}}
\newcommand{\dds}{\frac{\mathrm{d}}{\mathrm{d}s}}
\newcommand{\Ep}{\mathcal{E}^p(X,\theta;[\phi])}
\newcommand{\Ei}{\mathcal{E}^{\infty}(X,\theta;[\phi])}
\newcommand{\infs}{\operatorname*{inf\vphantom{p}}}
\newcommand{\sups}{\operatorname*{sup*}}
\newcommand{\colim}{\operatorname*{colim}}
\newcommand{\ddtz}[1][0]{\left.\ddt\right|_{t={#1}}}
\newcommand{\tube}[1][Y]{]{#1}[}
\newcommand{\ddsz}[1][0]{\left.\ddt\right|_{s={#1}}}
\newcommand{\floor}[1]{\left \lfloor{#1}\right \rfloor }
\newcommand{\dec}[1]{\left \{{#1}\right \} }
\newcommand{\ceil}[1]{\left \lceil{#1}\right \rceil }
\newcommand{\Projrel}{\mathcal{P}\mathrm{roj}}
\newcommand{\Weil}{\mathrm{Weil}}
\newcommand{\Cart}{\mathrm{Cart}}
\newcommand{\bWeil}{\mathrm{b}\mathrm{Weil}}
\newcommand{\bCart}{\mathrm{b}\mathrm{Cart}}
\newcommand{\Cond}{\mathrm{Cond}}
\newcommand{\IC}{\mathrm{IC}}
\newcommand{\IH}{\mathrm{IH}}
\newcommand{\cris}{\mathrm{cris}}
\newcommand{\Zar}{\mathrm{Zar}}
\newcommand{\HvbCat}{\overline{\mathcal{V}\mathrm{ect}}}
\newcommand{\BanModCat}{\mathcal{B}\mathrm{an}\mathcal{M}\mathrm{od}}
\newcommand{\DesCat}{\mathcal{D}\mathrm{es}}
\newcommand{\RingCat}{\mathcal{R}\mathrm{ing}}
\newcommand{\SchCat}{\mathcal{S}\mathrm{ch}}
\newcommand{\AbCat}{\mathcal{A}\mathrm{b}}
\newcommand{\RSCat}{\mathcal{R}\mathrm{S}}
\newcommand{\LRSCat}{\mathcal{L}\mathrm{RS}}
\newcommand{\CLRSCat}{\mathbb{C}\text{-}\LRSCat}
\newcommand{\CRSCat}{\mathbb{C}\text{-}\RSCat}
\newcommand{\CLA}{\mathbb{C}\text{-}\mathcal{L}\mathrm{A}}
\newcommand{\CASCat}{\mathbb{C}\text{-}\mathcal{A}\mathrm{n}}
\newcommand{\LiuCat}{\mathcal{L}\mathrm{iu}}
\newcommand{\BanCat}{\mathcal{B}\mathrm{an}}
\newcommand{\BanAlgCat}{\mathcal{B}\mathrm{an}\mathcal{A}\mathrm{lg}}
\newcommand{\AnaCat}{\mathcal{A}\mathrm{n}}
\newcommand{\LiuAlgCat}{\mathcal{L}\mathrm{iu}\mathcal{A}\mathrm{lg}}
\newcommand{\AlgCat}{\mathcal{A}\mathrm{lg}}
\newcommand{\SetCat}{\mathcal{S}\mathrm{et}}
\newcommand{\ModCat}{\mathcal{M}\mathrm{od}}
\newcommand{\TopCat}{\mathcal{T}\mathrm{op}}
\newcommand{\CohCat}{\mathcal{C}\mathrm{oh}}
\newcommand{\SolCat}{\mathcal{S}\mathrm{olid}}
\newcommand{\AffCat}{\mathcal{A}\mathrm{ff}}
\newcommand{\AffAlgCat}{\mathcal{A}\mathrm{ff}\mathcal{A}\mathrm{lg}}
\newcommand{\QcohLiuAlgCat}{\mathcal{L}\mathrm{iu}\mathcal{A}\mathrm{lg}^{\mathrm{QCoh}}}
\newcommand{\LiuMorCat}{\mathcal{L}\mathrm{iu}}
\newcommand{\Isom}{\mathcal{I}\mathrm{som}}
\newcommand{\Cris}{\mathcal{C}\mathrm{ris}}
\newcommand{\Pro}{\mathrm{Pro}-}
\newcommand{\Fin}{\mathcal{F}\mathrm{in}}
\newcommand{\norms}[1]{\left\|#1\right\|}
\newcommand{\HPDDiff}{\mathbf{D}\mathrm{iff}}
\newcommand{\Menn}[2]{\begin{bmatrix}#1\\#2\end{bmatrix}}
\newcommand{\Fins}{\widehat{\Vect}^F}
\newcommand\blfootnote[1]{%
  \begingroup
  \renewcommand\thefootnote{}\footnote{#1}%
  \addtocounter{footnote}{-1}%
  \endgroup
}

\externaldocument[Introduction-]{Introduction}
%One variable complex analysis
%Several variables complex analysis
\externaldocument[Topology-]{Topology-Bornology}
\externaldocument[Banach-]{Banach-Rings}
\externaldocument[Commutative-]{Commutative-Algebra}
\externaldocument[Local-]{Local-Algebras}
\externaldocument[Complex-]{Complex-Analytic-Spaces}
%Properties of space
\externaldocument[Morphisms-]{Morphisms}
%Differential calculus
%GAGA
%Hilbert scheme complex analytic version

%Complex differential geometry

\externaldocument[Affinoid-]{Affinoid-Algebras}
\externaldocument[Berkovich-]{Berkovich-Analytic-Spaces}


\bibliography{Ymir}

\endinput
\title{Berkovich analytic spaces}
\begin{document}
\maketitle
\tableofcontents



\section{Introduction}\label{sec-introduction}

\section{Affinoid spaces}
Let $(k,|\bullet|)$ be a complete non-Archimedean valued field and $H$ be a subgroup of $\mathbb{R}_{>0}$ such that $|k^{\times}|\cdot H\neq \{1\}$.

\begin{definition}
Let $A$ be a $k_H$-affinoid algebra. A \emph{compact $k_H$-analytic domain} $V$ in $\Sp A$ is a finite union of $k_H$-affinoid domains in $\Sp A$.    
\end{definition}


\begin{lemma}\label{lma-compactanalyticdomainring}
    Let $A$ be a $k_H$-affinoid algebra and $V$ be a compact $k_H$-analytic domain. Write $\Sp A$ as a finite union of $k_H$-affinoid domains $\Sp A_i$ with $i=1,\ldots,n$ in $\Sp A$. Define $A_{ij}=A_i\hat{\otimes}_A A_j$ and 
    \[
        A_V:=\ker  \left( \prod_{i=1}^n A_i \rightarrow \prod_{i,j=1}^n A_{ij}\right).
    \]
    Then the Banach $k$-algebra does not depend on the choice of the covering $\{\Sp A_i\}_i$ up to a canonical isomorphism.

    The image of the natural continuous map $\Sp A_V\rightarrow \Sp A$ contains $V$ and the map does not depend on the choice of the covering up to the canonical isomorphism between $\Sp A_V$ for different coverings.
\end{lemma}
\begin{proof}
    We first observe that $A_V$ is a Banach $k$-algebra as it is defined as an equalizer. This follows from \cref{Banach-lma-equalizerbanach} in the chapter Banach Rings.

    Let $\{\Sp B_j\}_{j=1,\ldots,m}$ be another $k_H$-affinoid covering of $\Sp A$. We need to show that $A_V$ defined using the two coverings are canonically isomorphic. We write $A_V'$ for  
    \[
        \ker  \left( \prod_{j=1}^m B_j \rightarrow \prod_{i,j=1}^m B_{ij}\right)
    \]  
    to make a distinction.
    We write $B_{ij}=B_i\hat{\otimes}_A B_j$. 
    
    By \cref{Affinoid-thm-Tateacyc} in the chapter Affinoid Algebras, the colomns in the following commutative diagram are exact:
    \[
        \begin{tikzcd}
            &                                 & 0 \arrow[d]                                                                             &  & 0 \arrow[d]                                                       \\
0 \arrow[r] & A_V \arrow[r] \arrow[d, dotted] & \prod_{i=1}^n A_i \arrow[d,"\eta"] \arrow[rr]                                                  &  & {\prod_{i,i'=1}^n A_{ii'}} \arrow[d]                              \\
0 \arrow[r] & \ker \iota \arrow[r]            & \prod_{i=1}^n\prod_{j=1}^m A_i\hat{\otimes}_A B_j \arrow[d, "\tau"] \arrow[rr, "\iota"] &  & {\prod_{i,i'=1}^n\prod_{j,j'=1}^m A_{ii'}\hat{\otimes}_A B_{jj'}} \\
            &                                 & {\prod_{i=1}^n\prod_{j,j'=1}^m A_i\hat{\otimes}_A B_{jj'}}                              &  &                                                                  
\end{tikzcd}  
    \]
    The rows are exact by definition. By diagram chasing, the dotted arrow is injective. To see it is surjective, it suffices to observe that the factors with $i=i'$ in the lower right corner is exactly the same as the factors of the lower corner, so an element in $\ker \iota$ is necessarily in $\ker \tau$. It follows that the dotted arrow is surjective.

    Similarly, we have a natural isomorphism $A_V'\cn \ker \iota$. We conclude the first assertion.

    As for the second, observe that $\Sp A_V$ is defined as a colimit in the category of Banach $k$-algebras, so it follows from general abstract nonsense that there is a natural morphism $\Sp A_V\rightarrow \Sp A$. It clearly contains $V$ in the image. The compatibility with the isomorphism above follows simply from the fact that the map $\eta$ is an $A$-algebra homomorphism.
\end{proof}



\begin{definition}
    Let $A$ be a $k$-affinoid algebra and $V$ be a compact $k$-analytic domain in $\Sp A$. We define \emph{the Banach $k$-algebra $A_V$ associated with $V$} as $A_V$ constructed in \cref{lma-compactanalyticdomainring}.

    The continuous map $\Sp A_V\rightarrow \Sp A$ constructed in  \cref{lma-compactanalyticdomainring} is called the \emph{structure map} ov $V$.
\end{definition}

\begin{proposition}\label{prop-compactanalydomainaffinoid}
    Let $A$ be a $k_H$-affinoid algebra and $V$ be a compact $k_H$-analytic domain in $\Sp A$. Then the following are equivalent:
    \begin{enumerate}
        \item $V$ is a $k_H$-affinoid domain.
        \item $A_V$ is a $k_H$-affinoid algebra and the image of the structure map $\Sp A_V\rightarrow \Sp A$ is exactly $V$.
    \end{enumerate}
\end{proposition}
\begin{proof}
    (1) $\implies$ (2): By \cref{Affinoid-thm-Tateacyc} in the chapter Affinoid Algebras, when $V$ is a $k_H$-affinoid domain, $A_V$ is a $k_H$-affinoid algebra and the structure map corresponds to the inclusion of the $k_H$-affinoid domain. There is nothing to prove.

    (2) $\implies$ (1): It suffices to show that the structure map represents the $k_H$-affinoid domain $V$. Take a $k_H$-affinoid algebra $D$ and a morphism $\Sp D\rightarrow \Sp A$ of $k_H$-affinoid spaces that factorizes through $V$. We need to construct a morphism $\Sp D\rightarrow \Sp A_V$ making the following diagram commutative
    \[
        \begin{tikzcd}
            \Sp D \arrow[rd] \arrow[d, dotted] &       \\
            \Sp A_V \arrow[r]                  & \Sp A
        \end{tikzcd}.    
    \]
    
    Take $k_H$-affinoid domains $\Sp B_1,\ldots,\Sp B_n$ in $\Sp A$ that cover $V$. Let $C_i=B_i\hat{\otimes}_A D$ for $i=1,\ldots,n$, then $\Sp C_i$ is a $k_H$-affinoid domain in $\Sp D$ by \cref{Affinoid-cor-fiberproductaffdomain} in the chapter Affinoid Algebras. By \cref{Affinoid-thm-Tateacyc} in the chapter Affinoid Algebras and general abstract nonsense, it suffices to construct the dotted arrow after restricting to $\Sp C_i$ for $i=1,\ldots,n$. So we could assume that $\Sp D\rightarrow \Sp A$ factorizes through $\Sp B_1$. From the universal property, we therefore have the dotted morphism making the following diagram commutative:
    \[
        \begin{tikzcd}
            \Sp D \arrow[rd] \arrow[d, dotted] &       \\
            \Sp B_1 \arrow[r]                  & \Sp A
        \end{tikzcd}.    
    \]
    It suffices to show that the natural homomorphism
    \[
        B_1\rightarrow A_V\hat{\otimes}_A B_1  
    \]
    is an isomorphism. But this follows from general abstract nonsense as $B_1$ is already a Banach $A_V$-algebra.
\end{proof}
\begin{remark}
    This proposition is not correctly stated in \cite[Corollary~2.2.6]{Berk12}. The corresponding statement in \cite[Remark~1.2.1]{Berk93} is slightly weaker than our statement.
\end{remark}



\section{The category of Berkovich analytic spaces}
Let $(k,|\bullet|)$ be a complete non-Archimedean valued field and $H$ be a subgroup of $\mathbb{R}_{>0}$ such that $|k^{\times}|\cdot H\neq \{1\}$.
 
\begin{definition}
    Let $X$ be a locally Hausdorff space and $\tau$ be a net of compact subsets. A \emph{$k_H$-affinoid atlas} $\mathcal{A}$ on $X$ with the net $\tau$ is a map which assigns 
    \begin{enumerate}
        \item to each $V\in \tau$, a $k_H$-affinoid algebra $A_V$ and a homeomorphism $\varphi_V:\Sp A_V\rightarrow V$;
        \item to each $U,V\in \tau$, $U\subseteq V$, a morphism of $k_H$-affinoid algebras $\alpha_{V/U}:A_V\rightarrow A_U$ representing a $k_H$-affinoid domain $\Sp A_U$ in $\Sp A_V$ such that the following diagram commutes
        \[
            \begin{tikzcd}
                \Sp A_U \arrow[d, "\varphi_U"] \arrow[r, "\Sp \alpha_{V/U}"] & \Sp A_V \arrow[d, "\varphi_V"] \\
                U \arrow[r]                                                  & V                             
            \end{tikzcd}.  
        \]
    \end{enumerate}
    The triple $(X,\mathcal{A},\tau)$ as above is called a \emph{$k_H$-analytic space}.

    A \emph{morphism} between atlases $\mathcal{A}$ and $\mathcal{A}'$ on $X$ with the net $\tau$ is an assignment that with each $V\in \tau$, one associates a morphism of $k_H$-affinoid algebras $\beta_V:A_V\rightarrow A_V'$ such that
    \begin{enumerate}
        \item for each $V\in \tau$, the following diagram is commutative:
        \[
            \begin{tikzcd}
                \Sp A_V' \arrow[d, "\varphi_V'"] \arrow[r, "\Sp \beta_V"] & \Sp A_V \arrow[ld, "\varphi_V"] \\
                V                                                         &                                
                \end{tikzcd};
        \]
        \item for each $U,V\in \tau$, $U\subseteq V$, the following diagram is commutative:
        \[
            \begin{tikzcd}
                A_V \arrow[r, "\alpha_{V/U}"] \arrow[d, "\beta_V"] & A_U \arrow[d, "\beta_U"] \\
                A_V' \arrow[r, "\alpha'_{V/U}"]                    & A_U'                    
            \end{tikzcd}
        \]
    \end{enumerate}
    Here we have denoted the data associated with $\mathcal{A}'$ with a prime. In this way, the atlases on $X$ with the net $\tau$ form a category.
\end{definition}
We remind the readers that by our convention a compact space is Hausdorff. 

By Condition~(2), it $W\subseteq U\subseteq V$ are three sets in $\tau$, then $\alpha_{V/U}\circ\alpha_{U/W}=\alpha_{V/W}$.

\begin{remark}\label{rmk-affdomainident}
    As a convention, we will denote the atlas by capital  letters in caligraphic font and the affinoid algebras by the same letter in roman font. We will usually omit the maps $\varphi_U$'s by identifying $\Sp A_U$ with $U$. We will say $U$ is a $k_H$-affinoid domain in $V$.    
\end{remark}


\begin{remark}
    Our definition is a special case of the original definitions in \cite{Berk93}. This seems to be the most important case though.
\end{remark}

\begin{lemma}\label{lma-innetaffinoneaffinall}
    Let $(X,\mathcal{A},\tau)$ be a $k_H$-analytic space, $U\in \tau$ and $W$ is a $k_H$-affinoid domain in $U$. Then for any $V\in \tau$ containing $W$, $W$ is a $k_H$-affinoid domain in $V$.
\end{lemma}
\begin{proof}
    As $\tau|_{U\cap V}$ is a net and $W$ is compact, we can find $U_1,\ldots,U_n\in \tau_{U\cap V}$ with $W\subseteq U_1\cup \cdots\cup  U_n$. As $W$, $U_i$ are $k_H$-affinoid domains in $U$, $W_i=W\cap U_i$ is a $k_H$-affinoid domain in $U_i$ for all $i=1,\ldots,n$ by \cref{Affinoid-cor-fiberproductaffdomain} in the chapter Affinoid Algebras. It follows from \cref{Affinoid-cor-affinoiddomaintransi} and \cref{Affinoid-cor-fiberproductaffdomain} in the chapter Affinoid Algebras that $W_i$ and $W_i\cap W_j$ are both $k_H$-affinoid domains in $V$ for $i,j=1,\ldots,n$. So $W$ is a compact $k_H$-analytic domain in $V$.

    By \cref{prop-compactanalydomainaffinoid},
    \[
        A_W:=\ker  \left( \prod_{i=1}^n A_{W_i} \rightarrow \prod_{i,j=1}^n A_{W_i\cap W_j}\right)
    \]
    is $k_H$-affinoid and $\Sp A_W\rightarrow \Sp A$ induces a hoemomorphism $\Sp A_W\rightarrow W$ by \cref{Affinoid-prop-affdomainhoemo} in the chatper Affinoid Algebras. By \cref{prop-compactanalydomainaffinoid} again, $W$ is affinoid in $V$. 
\end{proof}

\begin{definition}
    Let $(X,\mathcal{A},\tau)$ be a $k_H$-analytic space. We define $\bar{\tau}$ as the set of all $W\subseteq X$ such that there is $U\in \tau$ containing $W$ and $W$ is $k_H$-affinoid in $U$.
\end{definition}

\begin{lemma}
    Let $(X,\mathcal{A},\tau)$ be a $k_H$-analytic space. Then $\bar{\tau}$ is a net on $X$ and there is a $k_H$-affinoid atlas $\overline{\mathcal{A}}$ on $X$ with the net $\bar{\tau}$  extending $\mathcal{A}$. Moreover, the $k_H$-affinoid atlas $\overline{\mathcal{A}}$ on $X$ with the net $\bar{\tau}$  extending $\mathcal{A}$ is unique up to a canonical isomorphism.
\end{lemma}
\begin{proof}
    \textbf{Step~1}.
    We first show that $\bar{\tau}$ is a net. Let $U,V\in \bar{\tau}$ and $x\in U\cap V$. Take $U',V'\in \tau$ containing $U$ and $V$ respectively. Take $n\in \mathbb{Z}_{>0}$ and $W_1,\ldots,W_n\in \tau$ such that 
    \begin{enumerate}
        \item $x\in W_1\cap \cdots \cap W_n$;
        \item $W_1\cup\cdots\cup W_n$ is a neighbourhood of $x$ in $U'\cap V'$.
    \end{enumerate}
    This is possible because $\tau|_{U'\cap V'}$ is a quasi-net by assumption.

    By \cref{lma-innetaffinoneaffinall}, $U$ (resp. $V$) and $W_1,\ldots,W_n$ are $k_H$-affinoid domains in $U'$ (resp. $V'$).

    By \cref{Affinoid-cor-fiberproductaffdomain} in the chapter Affinoid Algebras, $U_i:=U\cap W_i$ (resp. $V_i:=V\cap W_i$) is a $k_H$-affinoid domain in $W_i$  for $i=1,\ldots,n$. By \cref{Affinoid-cor-fiberproductaffdomain} in the chapter Affinoid Algebras again, $U_i\cap V_i$ is a $k_H$-affinoid domain in $W_i$ for $i=1,\ldots,n$. So $U_i\cap V_i\in \bar{\tau}|_{U\cap V}$ for $i=1,\ldots,n$. But
    \[
        \bigcup_{i=1}^n U_i\cap V_i=(U\cap V)\cap \bigcup_{i=1}^n W_i,  
    \]
    so $\bigcup_{i=1}^n U_i\cap V_i$ is a neighbourhood of $x$ in $U\cap V$ and $x\in \bigcap_{i=1}^n U_i\cap V_i$. It follows that $\bar{\tau}$ is a net.

    \textbf{Step~2}. We extend the $k_H$-affinoid atlas $\mathcal{A}$.

    For each $V\in \bar{\tau}$, we fix a $V'\in \tau$ containing $V$. 
    
    By \cref{lma-innetaffinoneaffinall}, $V$ is a $k_H$-affinoid domain in $V'$. Let $A_{V'}\rightarrow A_V$ be the morphism of $k_H$-affinoid algebras representing the $k_H$-affinoid domain $V$ in $\Sp A_{V'}$. We define the homeomorphism $\varphi_V:\Sp A_V\rightarrow V$ as the morphism induced by $\Sp A_V\rightarrow \Sp A$.


    For $U,V\in \bar{\tau}$ with $U\subseteq V$, we want to define $\alpha_{V/U}:A_V\rightarrow A_U$. We handle two cases. When $V\in \tau$, as $\tau|_{U'\cap V}$ is a quasi-net, we can find $n\in \mathbb{Z}_{>0}$ and $U_1,\ldots,U_n\in \tau|_{U'\cap V}$ such that
    \[
        U=\bigcup_{i=1}^n U_i.  
    \]
    By \cref{lma-innetaffinoneaffinall}, $U_1,\ldots,U_n$ are $k_H$-affinoid domains in $U'$ and in $V$. By \cref{Affinoid-thm-Tateacyc} in the chapter Affinoid Algebras,
    \[
        A_U\cn \ker \left( \prod_{i=1}^n A_{U_i}\rightarrow \prod_{i,j=1}^n A_{U_i\cap U_j} \right).  
    \]
    So the morphism $\alpha_{V/U_i}:A_V\rightarrow A_{U_i}$ and $A_{V/U_{i}\cap U_j}:\alpha_{V/U_i}:A_V\rightarrow A_{U_i\cap U_j}$ for $i=1,\ldots,n$ and $j=1,\ldots,n$ induces a morphism $\alpha_{V/U}:A_V\rightarrow A_U$. Observe that $\alpha_{V/U}$ represents the $k_H$-affinoid domain $U$ in $V$, so it is independent of the choice of $U_1,\ldots,U_n$.

    More generally, when $V\in \bar{\tau}$, we have constructed a morphism $\alpha_{V'/U}:A_{V'}\rightarrow A_U$ representing the $k_H$-affinoid domain $U$ in $V'$, it follows that $U$ is a $k_H$-affinoid domain in $V$, and we therefore get the desired morphism $\alpha_{V/U}:A_V\rightarrow A_U$.

    It is easy to verify that the constructions gives a $k_H$-affinoid atlas with the net $\bar{\tau}$ extending $\mathcal{A}$. The uniqueness of the extension is immediate.
\end{proof}

\begin{definition}
    Let $(X,\mathcal{A},\tau)$ and $(X',\mathcal{A}',\tau')$ be $k_H$-analytic spaces. A \emph{strong morphism} $\varphi:(X,\mathcal{A},\tau)\rightarrow (X',\mathcal{A}',\tau')$ is a pair consisting of 
    \begin{enumerate}
        \item a continuous map $\varphi:X\rightarrow X'$ such that for each $V\in \tau$, there is $V'\in \tau'$ with $\varphi(V)\subseteq V'$;
        \item for each $V\in \tau$, $V'\in \tau'$ with $\varphi(V)\subseteq V'$, a morphism of $k_H$-affinoid spectra $\varphi_{V/V'}:V\rightarrow V'$ 
    \end{enumerate}
    such that for each $V,W\in \tau$, $V',W'\in \tau'$ satisfying $V\subseteq W$, $W'\subseteq W'$, $\varphi(V)\subseteq V'$ and $\varphi(W)\subseteq W'$, the following diagram commutes:
    \[
        \begin{tikzcd}
            V \arrow[r, "\varphi_{V/V'}"] \arrow[d] & V' \arrow[d] \\
            W \arrow[r, "\varphi_{W/W'}"]           & W'          
        \end{tikzcd}.    
    \]
\end{definition}
Recall our convention \cref{rmk-affdomainident}, the morphism $\varphi_{V/V'}$ means a morphism $A'_{V'}\rightarrow A_V$ of $k_H$-affinoid algebras making the following diagram commutative
\[
    \begin{tikzcd}
        \Sp A_V \arrow[d, "\varphi_V"] \arrow[r] & \Sp A'_{V'} \arrow[d, "\varphi'_{V'}"] \\
        V \arrow[r, "\varphi"]                   & V'                                    
    \end{tikzcd}.  
\]
We will continue our identifications as in \cref{rmk-affdomainident} to simplify our notations.

\begin{proposition}\label{prop-strongmorphismext}
    Let $(X,\mathcal{A},\tau)$ and $(X',\mathcal{A}',\tau')$ be $k_H$-analytic spaces. Let $\varphi:(X,\mathcal{A},\tau)\rightarrow (X',\mathcal{A}',\tau')$ be a strong morphism. Then $\varphi$ extends uniquely to a strong morphism $\varphi:(X,\bar{\mathcal{A}},\bar{\tau})\rightarrow (X',\overline{\mathcal{A}'},\overline{\tau'})$.
\end{proposition}
\begin{proof}
    Let $U\in \bar{\tau}$, $U'\in \overline{\tau'}$ with $\varphi(U)\subseteq U'$. Take $V\in \tau$ and $V'\in \tau'$ containing $U$ and $U'$ respectively. By \cref{lma-innetaffinoneaffinall}, $U$ (resp. $V$) is a $k_H$-affinoid domain in $V$ (resp. $V'$). Take $W\in \tau'$ with $\varphi(V)\subseteq W'$. Then in particular, $\varphi(U)\subseteq W'$. As $\tau'|_{V'\cap W'}$ is a quasi-net and $\varphi(U)$ is compact, we can find $n\in \mathbb{Z}_{>0}$ and $W_1,\ldots,W_n\in \tau'|_{V'\cap W}$ such that 
    \[
        \varphi(U)\subseteq W_1\cup\cdots\cup W_n.
    \]
    Now $W_i$ is a $k_H$-affinoid domain in $W'$ by \cref{lma-innetaffinoneaffinall}, so $V_i:=\varphi^{-1}_{V/W'}(W_i)$ is an affinoid domain in $V$ by \cref{Affinoid-cor-fiberproductaffdomain} in the chatper Affinoid Algebras and we have an induced morphism $V_i\rightarrow W_i$ for $i=1,\ldots,n$. This morphism in turn induces a morphism of $k_H$-affinoid spaces
    \[
        U_i:=U\cap V_i\rightarrow U_i':=U'\cap W_i  \rightarrow U'
    \]
    for $i=1,\ldots,n$. These morphisms are compatible on their intersections by construction. So by \cref{Affinoid-thm-Tateacyc} in the chapter Affinoid Algebras, they glue together to a morphism of $k_H$-affinoid spectra $\bar{\varphi}_{U/U'}:U\rightarrow U'$. It is easy to see that this construction defines a strong morphism.

    As for the uniqueness, it suffices to show that the morphism $U_i\rightarrow U'_i$ is uniquely determined for $i=1,\ldots,n$. In other words, we need to show that the dotted arrow that makes the following diagram commutes is unique:
    \[
        \begin{tikzcd}
            U_i \arrow[d] \arrow[r, dotted] & U'_i \arrow[d] \\
            V \arrow[r, "\varphi_{V/W'}"]   & W'            
        \end{tikzcd}  
    \]
    for $i=1,\ldots,n$.
    It suffices to apply the universal property of the $k_H$-affinoid domain $U_i'\rightarrow W'$.
\end{proof}

\begin{definition}\label{def-strongmorphismcompo}
    Let $(X,\mathcal{A},\tau)$, $(X',\mathcal{A}',\tau')$, $(X'',\mathcal{A}'',\tau'')$ be $k_H$-analytic spaces. Let 
    \[
        \varphi:  (X,\mathcal{A},\tau)\rightarrow (X',\mathcal{A}',\tau'),\quad \psi: (X',\mathcal{A}',\tau')\rightarrow (X'',\mathcal{A}'',\tau'')
    \]
    be strong morphisms. We will define their \emph{composition} $\chi=\psi\circ \varphi$ as follows. The underlying map of topological spaces is just the composition of the unlerlying maps of topological spaces corresponding to $\psi$ and $\varphi$.
    
    Let $\bar{\varphi}$ and $\bar{\psi}$ be the extensions of  $\varphi$ and $\psi$ to $\bar{\tau}$ and $\overline{\tau'}$ as in \cref{prop-strongmorphismext}. 
    
    Given $V\in \tau$ and $V''\in \tau''$ with $\chi(V)\subseteq V''$, we need to define a morphism of $k_H$-affinoid spectra $\chi_{V/V''}:V\rightarrow V''$. Take $V'\in \tau'$ and $U''\in \tau''$ such that $\varphi(V)\subseteq V'$ and $\psi(V')\subseteq U''$. Since $\chi(V)\subseteq U''\cap V''$ and $V$ is compact, we can take $n\in \mathbb{Z}_{>0}$ and $V_1'',\ldots,V_n''\in \tau''|_{U''\cap V''}$ with $\chi(V)\subseteq V_1''\cup\cdots\cup V_n''$. Then $V_i':=\psi^{-1}_{V'/U''}(V_i'')$ and $V_i:=\varphi_{V/V'}^{-1}(V_i')$ are $k_H$-affinoid domains in $V'$ and $V$ respectively for $i=1,\ldots,n$ and $V=V_1\cup\cdots\cup V_n$. The morphisms $\bar{\varphi}$ and $\bar{\psi}$ then induce a morphism $V_i\rightarrow V_i''\rightarrow V$ of $k_H$-affinoid spectra. These morphisms are clearly compatible on the intersections and hence induce a morphism $V\rightarrow V''$ of $k_H$-affinoid spectra by \cref{Affinoid-thm-Tateacyc} in the chapter Affinoid Algebras.

    It is easy to verify that $\psi\circ \varphi$ is a strong morphism.

    In this way, we get a category $k_H\text{-}\widetilde{\AnaCat}$ of $k_H$-analytic spaces. 
\end{definition}

\begin{definition}
    Let $(X,\mathcal{A},\tau)$ and $(X',\mathcal{A}',\tau')$ be $k_H$-analytic spaces. A strong morphism $\varphi:(X,\mathcal{A},\tau)\rightarrow (X',\mathcal{A}',\tau')$ is said to be a \emph{quasi-isomorphism} if 
    \begin{enumerate}
        \item $\varphi$ is a homeomorphism between $X$ and $X'$;
        \item for any pair $V\in \tau$ and $V'\in \tau'$ with $\varphi(V)\subseteq V'$, $\Sp \varphi_{V/V'}$ identifies $V$ with an affinoid domain in $V'$.
    \end{enumerate}
\end{definition}

\begin{lemma}\label{lma-strongmorphisminversecptanalyticdomain}
    Let $(X,\mathcal{A},\tau)$ and $(X',\mathcal{A}',\tau')$ be $k_H$-analytic spaces and $\varphi:(X,\mathcal{A},\tau)\rightarrow (X',\mathcal{A}',\tau')$ be a strong morphism. Then for any $V\in \bar{\tau}$ and $V'\in \overline{\tau'}$, the intersection $V\cap \varphi^{-1}(V')$ is a compact $k_H$-analytic domain in $V$.
\end{lemma}
\begin{proof}
    Take $U'\in \overline{\tau'}$ with $\varphi(V)\subseteq U'$. As $\tau|_{U'\cap V'}$ is a quasi-net, we can find $n\in \mathbb{Z}_{>0}$ and $U_1',\ldots,U_n'\in \tau|_{U'\cap V'}$ with $\varphi(V)\subseteq U_1'\cup \cdots \cup U_n'$ and 
    \[
        V\cap \varphi^{-1}(V')=\bigcup_{i=1}^n \varphi^{-1}_{V/U}(U_i').
    \]
\end{proof}

\begin{lemma}\label{lma-quasiisomrightmultiplicative}
    The system of quasi-isomorphisms in $k_H\text{-}\widetilde{\AnaCat}$ is a right multiplicative system.
\end{lemma}
For the notion of right multiplicative system, we refer to \cite[\href{https://stacks.math.columbia.edu/tag/04VC}{Tag 04VC}]{stacks-project}.

\begin{proof}
    We verify the three axioms as in \cite[\href{https://stacks.math.columbia.edu/tag/04VC}{Tag 04VC}]{stacks-project}.

    \textbf{RMS1}. The identity is clear a quasi-isomorphism. It remains to verify that the composition of quasi-isomorphisms is still a quasi-isomorphism.

    We take $\varphi,\psi$ as in \cref{def-strongmorphismcompo}. We will use the same notations as in \cref{def-strongmorphismcompo}. We need to show that $V\rightarrow V''$ identifies $V$ with a $k_H$-affinoid domain in $V''$. From the construction, we know that $\varphi$ identifies $V_i$ with a $k_H$-affinoid domain in $V_i'$ and $\psi$ identifies $V_i'$ with a $k_H$-affinoid domain in $V_i''$ for $i=1,\ldots,n$. In particular, $\chi(V)$ is a compact $k_H$-analytic domain in $V''$. It follows from \cref{prop-compactanalydomainaffinoid} that $\chi(V)$ is a $k_H$-affinoid domain in $V''$.

    \textbf{RMS2}.
    If $\varphi:(X,\mathcal{A},\tau)\rightarrow (X',\mathcal{A}',\tau')$ and $f:(\widetilde{X'},\widetilde{\mathcal{A}'},\widetilde{\tau'})\rightarrow (X',\mathcal{A}',\tau')$ are given strong morphisms of $k_H$-analytic spaces and $g$ is a quasi-isomorphism, then there are $k_H$-analytic space $(\widetilde{X},\widetilde{\mathcal{A}},\widetilde{\tau})$ and strong morphisms $\tilde{\varphi}:(\widetilde{X},\widetilde{\mathcal{A}},\widetilde{\tau})\rightarrow (\widetilde{X'},\widetilde{\mathcal{A}'},\widetilde{\tau'})$ and $f:(\widetilde{X},\widetilde{\mathcal{A}},\widetilde{\tau})\rightarrow (X,\mathcal{A},\tau)$ such that $f$ is a quasi-isomorphism and the following diagram commutes:  
    \[
        \begin{tikzcd}
            {(\widetilde{X},\widetilde{\mathcal{A}},\widetilde{\tau})} \arrow[r, "\tilde{\varphi}", dotted] \arrow[d, "f", dotted] & {(\widetilde{X'},\widetilde{\mathcal{A}'},\widetilde{\tau'})} \arrow[d, "g"] \\
            {(X,\mathcal{A},\tau)} \arrow[r, "\varphi"]                                                                            & {(X',\mathcal{A}',\tau')}                                                   
        \end{tikzcd}.
    \]  
    We may assume that $\widetilde{X'}=X'$. Then $\widetilde{\tau'}\subseteq \overline{\tau'}$. We let $\tilde{X}=X$. Let $\tilde{\tau}$ be the family of all $V\in \bar{\tau}$ for which there is $\widetilde{V'}\in\widetilde{\tau'}$ with $\varphi(V)\subseteq \widetilde{V'}$. By \cref{lma-strongmorphisminversecptanalyticdomain}, $\tilde{\tau}$ is a net on $\tilde{X}$. The $k_H$-atlas $\bar{\mathcal{A}}$ defines a $k_H$-affinoid atlas $\tilde{\mathcal{A}}$ with the net $\tilde{\tau}$. The strong morphism $\bar{\varphi}$ induces $\tilde{\varphi}$. The morphism $f$ is the canonical quasi-isomorphism. It is immediate that these constructions satisfy the desired conditions.

    \textbf{RMS3}. If $\varphi,\psi:(X,\mathcal{A},\tau)\rightarrow (X',\mathcal{A}',\tau')$ are strong morphisms of $k_H$-analytic spaces and there is a quasi-isomorphism $g:(X',\mathcal{A}',\tau')\rightarrow (\widetilde{X'},\widetilde{\mathcal{A}'},\widetilde{\tau'})$ of $k_H$-analytic spaces such that $g\circ \varphi=g\circ \psi$, then there is a quasi-isomorphism $f:(\tilde{X},\tilde{\mathcal{A}},\tilde{\tau})\rightarrow (X,\mathcal{A},\tau)$ with $\varphi\circ f=\psi\circ f$.

    We will in fact show that $\varphi=\psi$. It is clear that they coincide as maps of topological spaces. Let $V\in \tau$, $V'\in \tau'$ such that $\varphi(V)\subseteq V'$. Take $\widetilde{V'}\in \widetilde{\tau'}$ with $g(V')\subseteq \widetilde{V'}$. Then we have two morphisms of $k$-affinoid spectra $\varphi_{V/V'},\psi_{V/V'}:V\rightarrow V'$ such that their compositions with $g_{V'/\widetilde{V'}}$ coincide.  As $V'$ is an affinoid domain in $\widetilde{V'}$, it follows that $\varphi_{V/V'}=\psi_{V/V'}$ by the universal property.
\end{proof}


\begin{definition}
    The category $k_H\text{-}\AnaCat$ is the right category of fractions of $k_H\text{-}\widetilde{\AnaCat}$ with respect to the system of quasi-isomorphisms.  A morphism in $k_H\text{-}\AnaCat$ is called a \emph{morphism} between $k_H$-analytic spaces.
\end{definition}
We refer to \cite[\href{https://stacks.math.columbia.edu/tag/04VB}{Tag 04VB}]{stacks-project} for the definition of right category of fractions. 

For later references, we explicitly write down the morphisms in $k_H\text{-}\AnaCat$. 
\begin{lemma}
    Let $\varphi:(X,\mathcal{A},\tau)\rightarrow (X',\mathcal{A}',\tau')$ be a morphism of $k_H$-analytic spaces. We define a partial order on the set of nets on $X$: $\tau_1\preceq \tau_0$ if $\tau_1\subseteq \overline{\tau_0}$. Then the set of nets is a directed set and
    \[
        \Hom_{k_H\text{-}\AnaCat}\left((X,\mathcal{A},\tau),(X',\mathcal{A}',\tau') \right)=\varinjlim_{\sigma\preceq \tau}\Hom_{k_H\text{-}\widetilde{\AnaCat}}\left((X,\mathcal{A}_{\sigma},\sigma),(X',\mathcal{A}',\tau') \right)
    \]
    in the category of sets, where $\mathcal{A}_{\sigma}$ is induced by $\overline{\mathcal{A}}$. The transition maps are all injective.
\end{lemma}
\begin{proof}
    This follows immediately from the definition.
\end{proof}

\begin{definition}
    Let  $(X,\mathcal{A},\tau)$ be a $k_H$-analytic space. We say a subset $W\subseteq X$ is \emph{$\tau$-special} if it is compact and there exist $n\in \mathbb{Z}_{>0}$ and a covering $W=W_1\cup \cdots W_n$  with $W_i\in \tau$, $W_i\cap W_j\in \tau$ for all $i,j=1,\ldots,n$ and the natural map
    \[
        A_{W_i}\hat{\otimes}_k A_{W_j}\rightarrow A_{W_i\cap W_j}  
    \]
    is an admissible epimorphism.

    The covering $W_1,\ldots,W_n$ is called a \emph{$\tau$-special covering} of $W$.
\end{definition}


\begin{lemma}
    Let  $(X,\mathcal{A},\tau)$ be a $k_H$-analytic space and $W$ be a $\tau$-special subset of $X$. If $U,V\in \tau|_W$, then $U\cap V\in \bar{\tau}$ and the natural map
    \[
        A_U\hat{\otimes}_k A_V\rightarrow A_{U\cap V}
    \]
    is an admissible epimorphism.
\end{lemma}
\begin{proof}
    Let $n\in \mathbb{Z}_{>0}$ and $W_1,\ldots,W_n$ be a $\tau$-special covering of $W$.
    As $U\cap W_i$ and $V\cap W_i$ are compact for $i=1,\ldots,n$, we can find $m_i\in \mathbb{Z}_{>0}$ (resp. $s_i\in \mathbb{Z}_{>0}$) and finite coverings $U_{i1},\ldots,U_{im_i}\in \tau$ of $U\cap W_i$ (resp. $V_{i1},\ldots,V_{ik_i}\in \tau$ of $V\cap W_i$). 

    Observe that $U_{ik}\cap V_{jl}$ is a $k_H$-affinoid domain in $U\cap V$, hence $U_{ik}\cap V_{jl}\in \bar{\tau}$ for any $i,j=1,\ldots,n$, $k=1,\ldots,m_i$ and $l=1,\ldots,k_l$. Observe that $U_{ik}\cap V_{jl}\rightarrow U_{ik}\times V_{jl}$ is a closed immersion as $W_i\cap W_j\rightarrow W_i\times W_j$ is by our assumption. Consider the finite convering 
    \[
        \left\{U_{ik}\times V_{jl}:i,j=1,\ldots,n; k=1,\ldots,m_i; l=1,\ldots,k_l \right\}
    \]
    of $U\times V$. For each tuple $(i,j,k,l)$, $A_{U_{ik}\\cap V_{jl}}$ is a finite $A_{U_{ik}\times V_{jl}}$-algebra. By \cref{Affinoid-thm-Kiehl} in the chapter Affinoid Algebras, we can construct a finite $A_{U\times V}$-algebra $A_{U\cap V}$ inducing all of these $A_{U_{ik}\cap V_{jl}}$'s. 
    
    \textcolor{red}{To be continued}
    %But $A_{U\cap V}$ is the equalizer of morphisms of $k_H$-affinoid algebras, so it is $k_H$-affinoid again. Moreover, we have a natural bijection $\Sp A_{U\cap V}\rightarrow U\cap V$ and $U\cap V\rightarrow U\times V$ is a closed immersion.

    %By \cref{prop-compactanalydomainaffinoid}, $U\cap V$ is a $k_H$-affinoid domain in $U$, it follows that $U\cap V\in \bar{\tau}$.
\end{proof}

\printbibliography
\end{document}