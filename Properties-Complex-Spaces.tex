
\documentclass{amsbook} 
%\usepackage{xr}
\usepackage{xr-hyper}
\usepackage[unicode]{hyperref}


\usepackage[T1]{fontenc}
\usepackage[utf8]{inputenc}
\usepackage{lmodern}
\usepackage{amssymb,tikz-cd}
%\usepackage{natbib}
\usepackage[english]{babel}

\usepackage[nameinlink,capitalize]{cleveref}
\usepackage[style=alphabetic,maxnames=99,maxalphanames=5, isbn=false, giveninits=true, doi=false]{biblatex}
\usepackage{lipsum, physics}
\usepackage{ifthen}
\usepackage{microtype}
\usepackage{booktabs}
\usetikzlibrary{calc}
\usepackage{emptypage}
\usepackage{setspace}
\usepackage[margin=0.75cm, font={small,stretch=0.80}]{caption}
\usepackage{subcaption}
\usepackage{url}
\usepackage{bookmark}
\usepackage{graphicx}
\usepackage{dsfont}
\usepackage{enumitem}
\usepackage{mathtools}
\usepackage{csquotes}
\usepackage{silence}
\usepackage{mathrsfs}
\usepackage{bigints}

\WarningFilter{biblatex}{Patching footnotes failed}


\ProcessOptions\relax

\emergencystretch=1em

\hypersetup{
colorlinks=true,
linktoc=all
}

\setcounter{tocdepth}{1}


\hyphenation{archi-medean  Archi-medean Tru-ding-er}

%\captionsetup[table]{position=bottom}   %% or below
\renewcommand{\thefootnote}{\fnsymbol{footnote}}
%\DeclareMathAlphabet{\mathcal}{OMS}{cmsy}{m}{n}
\renewbibmacro{in:}{}

\DeclareFieldFormat[article]{citetitle}{#1}
\DeclareFieldFormat[article]{title}{#1}
\DeclareFieldFormat[inbook]{citetitle}{#1}
\DeclareFieldFormat[inbook]{title}{#1}
\DeclareFieldFormat[incollection]{citetitle}{#1}
\DeclareFieldFormat[incollection]{title}{#1}
\DeclareFieldFormat[inproceedings]{citetitle}{#1}
\DeclareFieldFormat[inproceedings]{title}{#1}
\DeclareFieldFormat[phdthesis]{citetitle}{#1}
\DeclareFieldFormat[phdthesis]{title}{#1}
\DeclareFieldFormat[misc]{citetitle}{#1}
\DeclareFieldFormat[misc]{title}{#1}
\DeclareFieldFormat[book]{citetitle}{#1}
\DeclareFieldFormat[book]{title}{#1} 


%% Define various environments.

\theoremstyle{definition}
\newtheorem{theorem}{Theorem}[section]
\newtheorem{thm}[theorem]{Theorem}
\newtheorem{proposition}[theorem]{Proposition}
\newtheorem{corollary}[theorem]{Corollary}
\newtheorem{lemma}[theorem]{Lemma}
\newtheorem{conjecture}[theorem]{Conjecture}
\newtheorem{question}[theorem]{Question}
\newtheorem{example}[theorem]{Example}
\newtheorem{definition}[theorem]{Definition}
\newtheorem{condition}[theorem]{Condition}

\theoremstyle{remark}
\newtheorem{remark}[theorem]{Remark}
\numberwithin{equation}{section}

%\renewcommand{\thesection}{\thechapter.\arabic{section}}
%\renewcommand{\thetheorem}{\thesection.\arabic{theorem}}
%\renewcommand{\thedefinition}{\thesection.\arabic{definition}}
%\renewcommand{\theremark}{\thesection.\arabic{remark}}


%% Define new operators

\DeclareMathOperator{\nd}{nd}
\DeclareMathOperator{\ord}{ord}
\DeclareMathOperator{\Hom}{Hom}
\DeclareMathOperator{\PreSh}{PreSh}
\DeclareMathOperator{\Gr}{Gr}
\DeclareMathOperator{\Homint}{\mathcal{H}\mathrm{om}}
\DeclareMathOperator{\Torint}{\mathcal{T}\mathrm{or}}
\DeclareMathOperator{\Div}{div}
\DeclareMathOperator{\DSP}{DSP}
\DeclareMathOperator{\Diff}{Diff}
\DeclareMathOperator{\MA}{MA}
\DeclareMathOperator{\NA}{NA}
\DeclareMathOperator{\AN}{an}
\DeclareMathOperator{\Rep}{Rep}
\DeclareMathOperator{\Rest}{Res}
\DeclareMathOperator{\DF}{DF}
\DeclareMathOperator{\VCart}{VCart}
\DeclareMathOperator{\PL}{PL}
\DeclareMathOperator{\Bl}{Bl}
\DeclareMathOperator{\Td}{Td}
\DeclareMathOperator{\Fitt}{Fitt}
\DeclareMathOperator{\Ric}{Ric}
\DeclareMathOperator{\coeff}{coeff}
\DeclareMathOperator{\Aut}{Aut}
\DeclareMathOperator{\Capa}{Cap}
\DeclareMathOperator{\loc}{loc}
\DeclareMathOperator{\vol}{vol}
\DeclareMathOperator{\Val}{Val}
\DeclareMathOperator{\ST}{ST}
\DeclareMathOperator{\Amp}{Amp}
\DeclareMathOperator{\Herm}{Herm}
\DeclareMathOperator{\trop}{trop}
\DeclareMathOperator{\Trop}{Trop}
\DeclareMathOperator{\Cano}{Can}
\DeclareMathOperator{\PS}{PS}
\DeclareMathOperator{\Var}{Var}
\DeclareMathOperator{\Psef}{Psef}
\DeclareMathOperator{\Jac}{Jac}
\DeclareMathOperator{\Char}{char}
\DeclareMathOperator{\Red}{red}
\DeclareMathOperator{\Spf}{Spf}
\DeclareMathOperator{\Span}{Span}
\DeclareMathOperator{\Der}{Der}
%\DeclareMathOperator{\Mod}{mod}
\DeclareMathOperator{\Hilb}{Hilb}
\DeclareMathOperator{\triv}{triv}
\DeclareMathOperator{\Frac}{Frac}
\DeclareMathOperator{\diam}{diam}
\DeclareMathOperator{\Spec}{Spec}
\DeclareMathOperator{\Spm}{Spm}
\DeclareMathOperator{\Specrel}{\underline{Sp}}
\DeclareMathOperator{\Sp}{Sp}
\DeclareMathOperator{\reg}{reg}
\DeclareMathOperator{\sing}{sing}
\DeclareMathOperator{\Star}{Star}
\DeclareMathOperator{\relint}{relint}
\DeclareMathOperator{\Cvx}{Cvx}
\DeclareMathOperator{\Int}{Int}
\DeclareMathOperator{\Supp}{Supp}
\DeclareMathOperator{\FS}{FS}
\DeclareMathOperator{\RZ}{RZ}
\DeclareMathOperator{\Redu}{red}
\DeclareMathOperator{\lct}{lct}
\DeclareMathOperator{\Proj}{Proj}
\DeclareMathOperator{\Sing}{Sing}
\DeclareMathOperator{\Conv}{Conv}
\DeclareMathOperator{\Max}{Max}
\DeclareMathOperator{\Tor}{Tor}
\DeclareMathOperator{\Gal}{Gal}
\DeclareMathOperator{\Frob}{Frob}
\DeclareMathOperator{\coker}{coker}
\DeclareMathOperator{\Sym}{Sym}
\DeclareMathOperator{\CSp}{CSp}
\DeclareMathOperator{\Img}{Im}


\newcommand{\alg}{\mathrm{alg}}
\newcommand{\Sh}{\mathrm{Sh}}
\newcommand{\fin}{\mathrm{fin}}
\newcommand{\BPF}{\mathrm{BPF}}
\newcommand{\dBPF}{\mathrm{dBPF}}
\newcommand{\divf}{\mathrm{Div}^f}
\newcommand{\nef}{\mathrm{nef}}
\newcommand{\Bir}{\mathrm{Bir}}
\newcommand{\hO}{\hat{\mathcal{O}}}
\newcommand{\bDiv}{\mathrm{Div}^{\mathrm{b}}}
\newcommand{\un}{\mathrm{un}}
\newcommand{\sep}{\mathrm{sep}}
\newcommand{\diag}{\mathrm{diag}}
\newcommand{\Pic}{\mathrm{Pic}}
\newcommand{\GL}{\mathrm{GL}}
\newcommand{\SL}{\mathrm{SL}}
\newcommand{\LS}{\mathrm{LS}}
\newcommand{\GLS}{\mathrm{GLS}}
\newcommand{\GLSi}{\mathrm{GLS}_{\cap}}
\newcommand{\PGLS}{\mathrm{PGLS}}
\newcommand{\Loc}[1][S]{_{\{{#1}\}}}
\newcommand{\cl}{\mathrm{cl}}
\newcommand{\otL}{\hat{\otimes}^{\mathbb{L}}}
\newcommand{\ddpp}{\mathrm{d}'\mathrm{d}''}
\newcommand{\TC}{\mathcal{TC}}
\newcommand{\ddPP}{\mathrm{d}'_{\mathrm{P}}\mathrm{d}''_{\mathrm{P}}}
\newcommand{\PSs}{\mathcal{PS}}
\newcommand{\Gm}{\mathbb{G}_{\mathrm{m}}}
\newcommand{\End}{\mathrm{End}}
\newcommand{\Aff}[1][X]{\mathcal{M}\left(\mathcal{#1}\right)}
\newcommand{\XG}[1][X]{{#1}_{\mathrm{G}}}
\newcommand{\convC}{\xrightarrow{C}}
\newcommand{\Vect}{\mathrm{Vect}}
\newcommand{\abso}[1]{\lvert#1\rvert}
\newcommand{\Mdl}{\mathrm{Model}}
\newcommand{\cn}{\stackrel{\sim}{\longrightarrow}}
\newcommand{\sbc}{\mathbf{s}}
\newcommand{\CH}{\mathrm{CH}}
\newcommand{\GR}{\mathrm{GR}}
\newcommand{\dc}{\mathrm{d}^{\mathrm{c}}}
\newcommand{\Nef}{\mathrm{Nef}}
\newcommand{\Adj}{\mathrm{Adj}}
\newcommand{\DHm}{\mathrm{DH}}
\newcommand{\An}{\mathrm{an}}
\newcommand{\Rec}{\mathrm{Rec}}
\newcommand{\dP}{\mathrm{d}_{\mathrm{P}}}
\newcommand{\ddp}{\mathrm{d}_{\mathrm{P}}'\mathrm{d}_{\mathrm{P}}''}
\newcommand{\ddc}{\mathrm{dd}^{\mathrm{c}}}
\newcommand{\ddL}{\mathrm{d}'\mathrm{d}''}
\newcommand{\PSH}{\mathrm{PSH}}
\newcommand{\CPSH}{\mathrm{CPSH}}
\newcommand{\PSP}{\mathrm{PSP}}
\newcommand{\WPSH}{\mathrm{WPSH}}
\newcommand{\Ent}{\mathrm{Ent}}
\newcommand{\NS}{\mathrm{NS}}
\newcommand{\QPSH}{\mathrm{QPSH}}
\newcommand{\proet}{\mathrm{pro-ét}}
\newcommand{\XL}{(\mathcal{X},\mathcal{L})}
\newcommand{\ii}{\mathrm{i}}
\newcommand{\Cpt}{\mathrm{Cpt}}
\newcommand{\bp}{\bar{\partial}}
\newcommand{\ddt}{\frac{\mathrm{d}}{\mathrm{d}t}}
\newcommand{\dds}{\frac{\mathrm{d}}{\mathrm{d}s}}
\newcommand{\Ep}{\mathcal{E}^p(X,\theta;[\phi])}
\newcommand{\Ei}{\mathcal{E}^{\infty}(X,\theta;[\phi])}
\newcommand{\infs}{\operatorname*{inf\vphantom{p}}}
\newcommand{\sups}{\operatorname*{sup*}}
\newcommand{\colim}{\operatorname*{colim}}
\newcommand{\ddtz}[1][0]{\left.\ddt\right|_{t={#1}}}
\newcommand{\tube}[1][Y]{]{#1}[}
\newcommand{\ddsz}[1][0]{\left.\ddt\right|_{s={#1}}}
\newcommand{\floor}[1]{\left \lfloor{#1}\right \rfloor }
\newcommand{\dec}[1]{\left \{{#1}\right \} }
\newcommand{\ceil}[1]{\left \lceil{#1}\right \rceil }
\newcommand{\Projrel}{\mathcal{P}\mathrm{roj}}
\newcommand{\Weil}{\mathrm{Weil}}
\newcommand{\Cart}{\mathrm{Cart}}
\newcommand{\bWeil}{\mathrm{b}\mathrm{Weil}}
\newcommand{\bCart}{\mathrm{b}\mathrm{Cart}}
\newcommand{\Cond}{\mathrm{Cond}}
\newcommand{\IC}{\mathrm{IC}}
\newcommand{\IH}{\mathrm{IH}}
\newcommand{\cris}{\mathrm{cris}}
\newcommand{\Zar}{\mathrm{Zar}}
\newcommand{\HvbCat}{\overline{\mathcal{V}\mathrm{ect}}}
\newcommand{\BanModCat}{\mathcal{B}\mathrm{an}\mathcal{M}\mathrm{od}}
\newcommand{\DesCat}{\mathcal{D}\mathrm{es}}
\newcommand{\RingCat}{\mathcal{R}\mathrm{ing}}
\newcommand{\SchCat}{\mathcal{S}\mathrm{ch}}
\newcommand{\AbCat}{\mathcal{A}\mathrm{b}}
\newcommand{\RSCat}{\mathcal{R}\mathrm{S}}
\newcommand{\LRSCat}{\mathcal{L}\mathrm{RS}}
\newcommand{\CLRSCat}{\mathbb{C}\text{-}\LRSCat}
\newcommand{\CRSCat}{\mathbb{C}\text{-}\RSCat}
\newcommand{\CLA}{\mathbb{C}\text{-}\mathcal{L}\mathrm{A}}
\newcommand{\CASCat}{\mathbb{C}\text{-}\mathcal{A}\mathrm{n}}
\newcommand{\LiuCat}{\mathcal{L}\mathrm{iu}}
\newcommand{\BanCat}{\mathcal{B}\mathrm{an}}
\newcommand{\BanAlgCat}{\mathcal{B}\mathrm{an}\mathcal{A}\mathrm{lg}}
\newcommand{\AnaCat}{\mathcal{A}\mathrm{n}}
\newcommand{\LiuAlgCat}{\mathcal{L}\mathrm{iu}\mathcal{A}\mathrm{lg}}
\newcommand{\AlgCat}{\mathcal{A}\mathrm{lg}}
\newcommand{\SetCat}{\mathcal{S}\mathrm{et}}
\newcommand{\ModCat}{\mathcal{M}\mathrm{od}}
\newcommand{\TopCat}{\mathcal{T}\mathrm{op}}
\newcommand{\CohCat}{\mathcal{C}\mathrm{oh}}
\newcommand{\SolCat}{\mathcal{S}\mathrm{olid}}
\newcommand{\AffCat}{\mathcal{A}\mathrm{ff}}
\newcommand{\AffAlgCat}{\mathcal{A}\mathrm{ff}\mathcal{A}\mathrm{lg}}
\newcommand{\QcohLiuAlgCat}{\mathcal{L}\mathrm{iu}\mathcal{A}\mathrm{lg}^{\mathrm{QCoh}}}
\newcommand{\LiuMorCat}{\mathcal{L}\mathrm{iu}}
\newcommand{\Isom}{\mathcal{I}\mathrm{som}}
\newcommand{\Cris}{\mathcal{C}\mathrm{ris}}
\newcommand{\Pro}{\mathrm{Pro}-}
\newcommand{\Fin}{\mathcal{F}\mathrm{in}}
\newcommand{\norms}[1]{\left\|#1\right\|}
\newcommand{\HPDDiff}{\mathbf{D}\mathrm{iff}}
\newcommand{\Menn}[2]{\begin{bmatrix}#1\\#2\end{bmatrix}}
\newcommand{\Fins}{\widehat{\Vect}^F}
\newcommand\blfootnote[1]{%
  \begingroup
  \renewcommand\thefootnote{}\footnote{#1}%
  \addtocounter{footnote}{-1}%
  \endgroup
}

\externaldocument[Introduction-]{Introduction}
%One variable complex analysis
%Several variables complex analysis
\externaldocument[Topology-]{Topology-Bornology}
\externaldocument[Banach-]{Banach-Rings}
\externaldocument[Commutative-]{Commutative-Algebra}
\externaldocument[Local-]{Local-Algebras}
\externaldocument[Complex-]{Complex-Analytic-Spaces}
%Properties of space
\externaldocument[Morphisms-]{Morphisms}
%Differential calculus
%GAGA
%Hilbert scheme complex analytic version

%Complex differential geometry

\externaldocument[Affinoid-]{Affinoid-Algebras}
\externaldocument[Berkovich-]{Berkovich-Analytic-Spaces}


\bibliography{Ymir}

\endinput
\title{Ymir}
\begin{document}
\maketitle
\tableofcontents

\chapter*{Properties of complex analytic spaces}

\section{Introduction}\label{sec-introduction-propertycomplex}


\section{Dimension}

\begin{definition}
    Let $X$ be a complex analytic space and $x\in X$, the \emph{dimension} $\dim_x X$ of $X$ at $x$ is 
    \[
        \dim_x X=\dim \mathcal{O}_{X,x}.
    \]
    We also define the \emph{dimension} of the pointed complex analytic space $(X,x)$ and the \emph{dimension} of the complex analytic germ $X_x$ as $\dim_x X$.
\end{definition}

\begin{definition}
    Let $X$ be a complex analytic space, we say $X$ is \emph{equidimensional} at $x\in X$ if $\mathcal{O}_{X,x}$ is equidimensional.

    We also say $(X,x)$ or $X_x$ is \emph{equidimensional}.

    We say $X$ is \emph{equidimensional of dimension $n$} if $X$ is equidimensional of dimension $n$ at each $x\in X$.
\end{definition}
Recall that in general, a local ring $R$ is equidimensional if $\dim R/\mathfrak{p}=\dim R$ for all minimal prime $\mathfrak{p}$ of $R$.

\begin{definition}
    Let $X$ be a complex analytic space and $x\in X$, we say $X$ is \emph{integral} at $x$ if $\mathcal{O}_{X,x}$ is integral.
\end{definition}
This corresponds to the notion defined in \cref{ConstructionComplex-def-integralgerm} in \nameref{ConstructionComplex-chap-constructionComplex}.

\begin{thm}\label{thm-equidimlocusopen}
    Let $X$ be a complex analytic space and $n\in \mathbb{N}$, then the set of points $x\in X$ such that $X_x$ is equidimensional of dimension $n$ is open. 
\end{thm}
This is analogous to the result for noetherian cartenary schemes.
\begin{proof}
    Let $x\in X$ be a point such that $X_x$ is equidimensional of dimension $n$. 
    We want to construct an open neighbourhood $V$ of $x$ in $X$ such that $X$ is equidimensional of dimension $n$ at any $y\in V$.

    \textbf{Step~1}. We reduce to the case where $X$ is integral at $x$.

    Let $\mathfrak{p}_1,\ldots,\mathfrak{p}_m$ be the minimal primes of $\mathcal{O}_{X,x}$. The number is finite because $\mathcal{O}_{X,x}$ is noetherian. We have
    \[
        \bigcap_{i=1}^m \mathfrak{p}_i=\rad \mathcal{O}_{X,x}.
    \]
    Take an open neighbourhood $U$ of $x$ in $X$  such that there are ideals of finite type $\mathcal{I}_1,\ldots,\mathcal{I}_m$ extending $\mathfrak{p}_1,\ldots,\mathfrak{p}_m$.  Up to shrinking $U$, we may assume that 
    \[
        \bigcap_{i=1}^m \mathcal{I}_i  
    \]
    is nilpotent. For each $i=1,\ldots,m$, let $U_i$ denote the closed analytic subspace of $U$ defined by $\mathcal{I}_i$. Then
    \[
        |U|=\bigcup_{i=1}^m |U_i|  
    \]
    by \cref{ConstructionComplex-cor-analyticsubspacesamesupp} in \nameref{ConstructionComplex-chap-constructionComplex}. 
    As for any $y\in U$,
    \[
        \bigcap_{i=1}^m \mathcal{I}_{i,y} 
    \]
    is nilpotent, we have
    \[
        |\Spec \mathcal{O}_{X,y}|=|\Spec \mathcal{O}_{X,y}/\bigcap_{i=1}^m \mathcal{I}_{i,y} |=\bigcup_{i=1}^m |\Spec \mathcal{O}_{X,y}/\mathcal{I}_{i,y} |.
    \]
    In particular, for any $y\in U$, 
    \[
        \dim_y X=\dim_y U=\max_{i=1,\ldots,m} \dim_y U_i. 
    \]
    It suffices to handle each $W_i$ separately. 

    \textbf{Step~2}. We assume that $X_x$ is integral. By \cref{ConstructionComplex-thm-morphismdefinedbygeneratorlocal} in \nameref{ConstructionComplex-chap-constructionComplex}, we may assume that  $X$ has the following structure: there is an open neighbourhood $W$ of $0$ in $\mathbb{C}^n$, a morphism $(X,x)\rightarrow (W,0)$ and a finite $\mathcal{O}_W$-algebra $\mathcal{A}$ such that $\Spec^{\An}_W \mathcal{A}$ has a unique point $x'$ over $0$ and $(\Spec^{\An}_W \mathcal{A},x')$ is isomorphic to $(X,x)$ over $(W,0)$. By \cref{Local-cor-morphismpowertolocalsurinj} in \nameref{Local-chap-local}, $\mathcal{O}_{W,0}\rightarrow \mathcal{O}_{X,x}$ is injective, hence $\mathcal{O}_{X,x}$ is torsion-free over $\mathcal{O}_{W,0}$. As the torsion sheaf is coherent, up to shrinking $X$, we may assume that $\mathcal{O}_{X,y}$ is torsion-free over $\mathcal{O}_{W,z}$, where $z$ denotes the image of $y$ in $W$. It suffices to apply \cref{Local-lma-finitetorsionfreeoverlocalequidim} in \nameref{Local-chap-local}.
\end{proof}

\begin{corollary}
    Let $X$ be a complex analytic space and $n\in \mathbb{N}$. Then the set $\{x\in X:\dim_x X\geq n\}$ is an analytic set in $X$.

    %Moreover, $x\in X$ lies in the interior of $\{x\in X:\dim_x X\geq n\}$ if and only if for all minimal prime $\mathfrak{p}$ of $\mathcal{O}_{X,x}$, we have
    %\[
    %    \dim \mathcal{O}_{X,x}/\mathfrak{p}\geq n.  
    %\]
\end{corollary}
After introducing the analytic Zariski topology, we can reformulate this corollary as follows: the map $x\mapsto \dim_x X$ is upper semi-continuous with respect to the analytic Zariski topology.
\begin{proof}
    The problem is local on $X$. Fix $x\in X$ and let $\mathfrak{p}_1,\ldots,\mathfrak{p}_m$ be the minimal prime ideals of $\mathcal{O}_{X,x}$. Up to shrinking $X$, we may assume that 
    \[
        |X|=\bigcup_{i=1}^m |W_i|,  
    \]
    where $W_i$ is a closed analytic subspace of $X$ defined by a coherent $\mathcal{I}_i$ spreading $\mathfrak{p}_i$. We can guarantee that
    \[
        \dim_y X=\max_{i=1,\ldots,m}\dim_y W_i.   
    \]
    
    This is possible as in the proof of \cref{thm-equidimlocusopen}. By \cref{thm-equidimlocusopen}, up to shrinking $X$, we may assume that $W_i$ is equidimensional of dimension $n_i$ for some $n_i\in \mathbb{N}$ for each $i=1,\ldots,m$. In particular, for each $y\in X$, we have
    \[
        \dim_y X=\sup_{y\in W_i} n_i.  
    \]
    So 
    \[
        \{x\in X:\dim_x X\geq n\}=\bigcup_{i: n_i\geq n} |W_i|.  
    \]
    The corollary follows.
\end{proof}

\begin{definition}
    Let $X_x$ be an analytic germ and $Y_x$ be a closed analytic subgerm defined by an ideal $I\subseteq \mathcal{O}_{X,x}$.
    \begin{enumerate}
        \item When $Y_x$ is irreducible, namely when $I$ is a prime ideal, we define the \emph{codimension} of $Y_x$ in $X_x$ as 
            \[
                \codim_x(Y,X):=\het_{\mathcal{O}_{X,x}}(I).    
            \]
        \item In general, we define the \emph{codimension} of $Y_x$ in $X_x$ as 
        \[
            \codim_x(Y,X):=\inf_{Z_x\subseteq Y_x} \codim_x(Y,X),
        \]
        where $Z_x$ runs over closed analytic subgerms of $X_x$ contained in $Y_x$.
    \end{enumerate}
    We also call $\codim_x(Y,X)$ the codimension of $Y$ in $X$ at $x$.
\end{definition}
Observe that 
\[
    \codim_x(Y,X)\leq \dim_x X-\dim_x Y.
\]
When $X_x$ is equidimensional, $\codim_x(Y,X)$ is nothing but $\dim_x X-\dim_x Y$.


\section{Serre's condition \texorpdfstring{$R_n$}{Rn}}
Fix $n\in \mathbb{N}$ in this section.
\begin{definition}
    Let $X$ be a complex analytic space, we say $X$ \emph{satisfies $R_n$} at $x\in X$ if $\mathcal{O}_{X,x}$ satisfies $R_n$. We also say $(X,x)$ or $X_x$  \emph{satisfies $R_n$} at $x\in X$.

    We say $X$  \emph{satisfies $R_n$} if $X$ satisfies $R_n$ at all points $x\in X$. 
\end{definition}

\begin{proposition}
    
\end{proposition}

\section{Serre's condition \texorpdfstring{$S_n$}{Sn}}
Fix $n\in \mathbb{N}$ in this section.
\begin{definition}
    Let $X$ be a complex analytic space, we say $X$ \emph{satisfies $S_n$} at $x\in X$ if $\mathcal{O}_{X,x}$ satisfies $R_n$. We also say $(X,x)$ or $X_x$  \emph{satisfies $S_n$} at $x\in X$.

    We say $X$  \emph{satisfies $S_n$} if $X$ satisfies $S_n$ at all points $x\in X$. 
\end{definition}

\begin{proposition}
    
\end{proposition}

\section{Reducedness}
\begin{definition}
    Let $X$ be a complex analytic space, we say $X$ is \emph{reduced} at $x\in X$ if $\mathcal{O}_{X,x}$ is reduced. We also say $(X,x)$ or $X_x$ is \emph{reduced} at $x\in X$.

    We say $X$ is \emph{reduced} if $X$ is reduced at all points $x\in X$. 
\end{definition}

\begin{thm}\label{thm-reducedlocusopen}
    Let $X$ be a complex analytic space. Then the set of points $x\in X$ such that $\mathcal{O}_{X,x}$ is reduced is the complement of an analytic set.
\end{thm}
\begin{proof}
    
\end{proof}

\begin{corollary}\label{cor-nilradcoh}
    Let $X$ be a complex analytic space, then the nilradical $\rad \mathcal{O}_X$ is coherent.
\end{corollary}
\begin{proof}
    The problem is local on $X$. Take $x\in X$. Up to shrinking $X$, we may assume that $\mathcal{O}_{X,x}/(\rad \mathcal{O}_X)_x$ spreads to a finite $\mathcal{O}_X$-algebra $\mathcal{A}$ by \cref{ConstructionComplex-lma-spreadfinitelag} in \nameref{ConstructionComplex-chap-constructionComplex}. Up to further shrinking $X$, we may assume that $\mathcal{A}$ is the quotient of $\mathcal{O}_X$, say $\mathcal{A}\cong \mathcal{O}_X/\mathcal{I}$ for some coherent ideal $\mathcal{I}$ on $X$. As $\mathcal{I}_x$ is nilpotent by assumption, up to shrinking $X$, we may assume that $\mathcal{I}$ is also nilpotent, namely
    \[
        \mathcal{I}\subseteq  \rad \mathcal{O}_X. 
    \]
    Let $Y$ be the closed analytic subspace of $X$ defined by the ideal $\mathcal{I}$. Then $\mathcal{O}_{Y,x}\cong \mathcal{O}_{X,x}/(\rad \mathcal{O}_X)_x$ is reduced. Up to shrinking $X$, by \cref{thm-reducedlocusopen}, we may assume that $Y$ is reduced. But then for any $y\in Y$, 
    \[
        \mathcal{O}_{Y,y}\cong \mathcal{O}_{X,y}/\mathcal{I}_y
    \]
    is reduced, so
    \[
        \mathcal{I}_y\supseteq   (\rad \mathcal{O}_X)_y.
    \] 
    It follows that $\rad \mathcal{O}_X=\mathcal{I}$, hence the nilradical is coherent.
\end{proof}

\begin{corollary}[Cartan--Oka]\label{cor-CartanOka}
    Let $X$ be a complex analytic space and $A$ be an analytic subset of $X$, then the sheaf $\mathcal{J}_A$ is coherent.
\end{corollary}
Recall that the sheaf $\mathcal{J}_A$ is introduced in \cref{ConstructionComplex-def-sheafidealanaset} in \nameref{ConstructionComplex-chap-constructionComplex}.
\begin{proof}
By   \cref{ConstructionComplex-lma-analyticsetlocallyidealsheaf} in \nameref{ConstructionComplex-chap-constructionComplex}, we may assume that $A$ is a closed analytic subspace of $X$ defined by a coherent ideal $\mathcal{I}$. By \cref{ConstructionComplex-cor-orderrelationsubgermideal} in \nameref{ConstructionComplex-chap-constructionComplex}, the sheaf $\mathcal{J}_A$ is nothing but $\sqrt{I}$, which is coherent by \cref{cor-nilradcoh}.
\end{proof}

\begin{corollary}\label{cor-reducedinduced}
    Let $X$ be a complex analytic space and $A$ be an analytic subset of $X$, then there is a unique reduced closed analytic space $Y$ of $X$ with underlying set $A$.
\end{corollary}
\begin{proof}
    The existence follows from \cref{cor-CartanOka}. The uniqueness follows from \cref{ConstructionComplex-cor-orderrelationsubgermideal} in \nameref{ConstructionComplex-chap-constructionComplex}.
\end{proof}

\begin{definition}
    Let $X$ be a complex analytic space and $A$ be an analytic subset of $X$. The analytic space structure on $A$ defined in \cref{cor-reducedinduced} is called the \emph{reduced induced structure} on $A$.
\end{definition}





\cite{stacks-project}

\printbibliography
\end{document}