
\documentclass{amsbook} 
%\usepackage{xr}
\usepackage{xr-hyper}
\usepackage[unicode]{hyperref}


\usepackage[T1]{fontenc}
\usepackage[utf8]{inputenc}
\usepackage{lmodern}
\usepackage{amssymb,tikz-cd}
%\usepackage{natbib}
\usepackage[english]{babel}

\usepackage[nameinlink,capitalize]{cleveref}
\usepackage[style=alphabetic,maxnames=99,maxalphanames=5, isbn=false, giveninits=true, doi=false]{biblatex}
\usepackage{lipsum, physics}
\usepackage{ifthen}
\usepackage{microtype}
\usepackage{booktabs}
\usetikzlibrary{calc}
\usepackage{emptypage}
\usepackage{setspace}
\usepackage[margin=0.75cm, font={small,stretch=0.80}]{caption}
\usepackage{subcaption}
\usepackage{url}
\usepackage{bookmark}
\usepackage{graphicx}
\usepackage{dsfont}
\usepackage{enumitem}
\usepackage{mathtools}
\usepackage{csquotes}
\usepackage{silence}
\usepackage{mathrsfs}
\usepackage{bigints}

\WarningFilter{biblatex}{Patching footnotes failed}


\ProcessOptions\relax

\emergencystretch=1em

\hypersetup{
colorlinks=true,
linktoc=all
}

\setcounter{tocdepth}{1}


\hyphenation{archi-medean  Archi-medean Tru-ding-er}

%\captionsetup[table]{position=bottom}   %% or below
\renewcommand{\thefootnote}{\fnsymbol{footnote}}
%\DeclareMathAlphabet{\mathcal}{OMS}{cmsy}{m}{n}
\renewbibmacro{in:}{}

\DeclareFieldFormat[article]{citetitle}{#1}
\DeclareFieldFormat[article]{title}{#1}
\DeclareFieldFormat[inbook]{citetitle}{#1}
\DeclareFieldFormat[inbook]{title}{#1}
\DeclareFieldFormat[incollection]{citetitle}{#1}
\DeclareFieldFormat[incollection]{title}{#1}
\DeclareFieldFormat[inproceedings]{citetitle}{#1}
\DeclareFieldFormat[inproceedings]{title}{#1}
\DeclareFieldFormat[phdthesis]{citetitle}{#1}
\DeclareFieldFormat[phdthesis]{title}{#1}
\DeclareFieldFormat[misc]{citetitle}{#1}
\DeclareFieldFormat[misc]{title}{#1}
\DeclareFieldFormat[book]{citetitle}{#1}
\DeclareFieldFormat[book]{title}{#1} 


%% Define various environments.

\theoremstyle{definition}
\newtheorem{theorem}{Theorem}[section]
\newtheorem{thm}[theorem]{Theorem}
\newtheorem{proposition}[theorem]{Proposition}
\newtheorem{corollary}[theorem]{Corollary}
\newtheorem{lemma}[theorem]{Lemma}
\newtheorem{conjecture}[theorem]{Conjecture}
\newtheorem{question}[theorem]{Question}
\newtheorem{example}[theorem]{Example}
\newtheorem{definition}[theorem]{Definition}
\newtheorem{condition}[theorem]{Condition}

\theoremstyle{remark}
\newtheorem{remark}[theorem]{Remark}
\numberwithin{equation}{section}

%\renewcommand{\thesection}{\thechapter.\arabic{section}}
%\renewcommand{\thetheorem}{\thesection.\arabic{theorem}}
%\renewcommand{\thedefinition}{\thesection.\arabic{definition}}
%\renewcommand{\theremark}{\thesection.\arabic{remark}}


%% Define new operators

\DeclareMathOperator{\nd}{nd}
\DeclareMathOperator{\ord}{ord}
\DeclareMathOperator{\Hom}{Hom}
\DeclareMathOperator{\PreSh}{PreSh}
\DeclareMathOperator{\Gr}{Gr}
\DeclareMathOperator{\Homint}{\mathcal{H}\mathrm{om}}
\DeclareMathOperator{\Torint}{\mathcal{T}\mathrm{or}}
\DeclareMathOperator{\Div}{div}
\DeclareMathOperator{\DSP}{DSP}
\DeclareMathOperator{\Diff}{Diff}
\DeclareMathOperator{\MA}{MA}
\DeclareMathOperator{\NA}{NA}
\DeclareMathOperator{\AN}{an}
\DeclareMathOperator{\Rep}{Rep}
\DeclareMathOperator{\Rest}{Res}
\DeclareMathOperator{\DF}{DF}
\DeclareMathOperator{\VCart}{VCart}
\DeclareMathOperator{\PL}{PL}
\DeclareMathOperator{\Bl}{Bl}
\DeclareMathOperator{\Td}{Td}
\DeclareMathOperator{\Fitt}{Fitt}
\DeclareMathOperator{\Ric}{Ric}
\DeclareMathOperator{\coeff}{coeff}
\DeclareMathOperator{\Aut}{Aut}
\DeclareMathOperator{\Capa}{Cap}
\DeclareMathOperator{\loc}{loc}
\DeclareMathOperator{\vol}{vol}
\DeclareMathOperator{\Val}{Val}
\DeclareMathOperator{\ST}{ST}
\DeclareMathOperator{\Amp}{Amp}
\DeclareMathOperator{\Herm}{Herm}
\DeclareMathOperator{\trop}{trop}
\DeclareMathOperator{\Trop}{Trop}
\DeclareMathOperator{\Cano}{Can}
\DeclareMathOperator{\PS}{PS}
\DeclareMathOperator{\Var}{Var}
\DeclareMathOperator{\Psef}{Psef}
\DeclareMathOperator{\Jac}{Jac}
\DeclareMathOperator{\Char}{char}
\DeclareMathOperator{\Red}{red}
\DeclareMathOperator{\Spf}{Spf}
\DeclareMathOperator{\Span}{Span}
\DeclareMathOperator{\Der}{Der}
%\DeclareMathOperator{\Mod}{mod}
\DeclareMathOperator{\Hilb}{Hilb}
\DeclareMathOperator{\triv}{triv}
\DeclareMathOperator{\Frac}{Frac}
\DeclareMathOperator{\diam}{diam}
\DeclareMathOperator{\Spec}{Spec}
\DeclareMathOperator{\Spm}{Spm}
\DeclareMathOperator{\Specrel}{\underline{Sp}}
\DeclareMathOperator{\Sp}{Sp}
\DeclareMathOperator{\reg}{reg}
\DeclareMathOperator{\sing}{sing}
\DeclareMathOperator{\Star}{Star}
\DeclareMathOperator{\relint}{relint}
\DeclareMathOperator{\Cvx}{Cvx}
\DeclareMathOperator{\Int}{Int}
\DeclareMathOperator{\Supp}{Supp}
\DeclareMathOperator{\FS}{FS}
\DeclareMathOperator{\RZ}{RZ}
\DeclareMathOperator{\Redu}{red}
\DeclareMathOperator{\lct}{lct}
\DeclareMathOperator{\Proj}{Proj}
\DeclareMathOperator{\Sing}{Sing}
\DeclareMathOperator{\Conv}{Conv}
\DeclareMathOperator{\Max}{Max}
\DeclareMathOperator{\Tor}{Tor}
\DeclareMathOperator{\Gal}{Gal}
\DeclareMathOperator{\Frob}{Frob}
\DeclareMathOperator{\coker}{coker}
\DeclareMathOperator{\Sym}{Sym}
\DeclareMathOperator{\CSp}{CSp}
\DeclareMathOperator{\Img}{Im}


\newcommand{\alg}{\mathrm{alg}}
\newcommand{\Sh}{\mathrm{Sh}}
\newcommand{\fin}{\mathrm{fin}}
\newcommand{\BPF}{\mathrm{BPF}}
\newcommand{\dBPF}{\mathrm{dBPF}}
\newcommand{\divf}{\mathrm{Div}^f}
\newcommand{\nef}{\mathrm{nef}}
\newcommand{\Bir}{\mathrm{Bir}}
\newcommand{\hO}{\hat{\mathcal{O}}}
\newcommand{\bDiv}{\mathrm{Div}^{\mathrm{b}}}
\newcommand{\un}{\mathrm{un}}
\newcommand{\sep}{\mathrm{sep}}
\newcommand{\diag}{\mathrm{diag}}
\newcommand{\Pic}{\mathrm{Pic}}
\newcommand{\GL}{\mathrm{GL}}
\newcommand{\SL}{\mathrm{SL}}
\newcommand{\LS}{\mathrm{LS}}
\newcommand{\GLS}{\mathrm{GLS}}
\newcommand{\GLSi}{\mathrm{GLS}_{\cap}}
\newcommand{\PGLS}{\mathrm{PGLS}}
\newcommand{\Loc}[1][S]{_{\{{#1}\}}}
\newcommand{\cl}{\mathrm{cl}}
\newcommand{\otL}{\hat{\otimes}^{\mathbb{L}}}
\newcommand{\ddpp}{\mathrm{d}'\mathrm{d}''}
\newcommand{\TC}{\mathcal{TC}}
\newcommand{\ddPP}{\mathrm{d}'_{\mathrm{P}}\mathrm{d}''_{\mathrm{P}}}
\newcommand{\PSs}{\mathcal{PS}}
\newcommand{\Gm}{\mathbb{G}_{\mathrm{m}}}
\newcommand{\End}{\mathrm{End}}
\newcommand{\Aff}[1][X]{\mathcal{M}\left(\mathcal{#1}\right)}
\newcommand{\XG}[1][X]{{#1}_{\mathrm{G}}}
\newcommand{\convC}{\xrightarrow{C}}
\newcommand{\Vect}{\mathrm{Vect}}
\newcommand{\abso}[1]{\lvert#1\rvert}
\newcommand{\Mdl}{\mathrm{Model}}
\newcommand{\cn}{\stackrel{\sim}{\longrightarrow}}
\newcommand{\sbc}{\mathbf{s}}
\newcommand{\CH}{\mathrm{CH}}
\newcommand{\GR}{\mathrm{GR}}
\newcommand{\dc}{\mathrm{d}^{\mathrm{c}}}
\newcommand{\Nef}{\mathrm{Nef}}
\newcommand{\Adj}{\mathrm{Adj}}
\newcommand{\DHm}{\mathrm{DH}}
\newcommand{\An}{\mathrm{an}}
\newcommand{\Rec}{\mathrm{Rec}}
\newcommand{\dP}{\mathrm{d}_{\mathrm{P}}}
\newcommand{\ddp}{\mathrm{d}_{\mathrm{P}}'\mathrm{d}_{\mathrm{P}}''}
\newcommand{\ddc}{\mathrm{dd}^{\mathrm{c}}}
\newcommand{\ddL}{\mathrm{d}'\mathrm{d}''}
\newcommand{\PSH}{\mathrm{PSH}}
\newcommand{\CPSH}{\mathrm{CPSH}}
\newcommand{\PSP}{\mathrm{PSP}}
\newcommand{\WPSH}{\mathrm{WPSH}}
\newcommand{\Ent}{\mathrm{Ent}}
\newcommand{\NS}{\mathrm{NS}}
\newcommand{\QPSH}{\mathrm{QPSH}}
\newcommand{\proet}{\mathrm{pro-ét}}
\newcommand{\XL}{(\mathcal{X},\mathcal{L})}
\newcommand{\ii}{\mathrm{i}}
\newcommand{\Cpt}{\mathrm{Cpt}}
\newcommand{\bp}{\bar{\partial}}
\newcommand{\ddt}{\frac{\mathrm{d}}{\mathrm{d}t}}
\newcommand{\dds}{\frac{\mathrm{d}}{\mathrm{d}s}}
\newcommand{\Ep}{\mathcal{E}^p(X,\theta;[\phi])}
\newcommand{\Ei}{\mathcal{E}^{\infty}(X,\theta;[\phi])}
\newcommand{\infs}{\operatorname*{inf\vphantom{p}}}
\newcommand{\sups}{\operatorname*{sup*}}
\newcommand{\colim}{\operatorname*{colim}}
\newcommand{\ddtz}[1][0]{\left.\ddt\right|_{t={#1}}}
\newcommand{\tube}[1][Y]{]{#1}[}
\newcommand{\ddsz}[1][0]{\left.\ddt\right|_{s={#1}}}
\newcommand{\floor}[1]{\left \lfloor{#1}\right \rfloor }
\newcommand{\dec}[1]{\left \{{#1}\right \} }
\newcommand{\ceil}[1]{\left \lceil{#1}\right \rceil }
\newcommand{\Projrel}{\mathcal{P}\mathrm{roj}}
\newcommand{\Weil}{\mathrm{Weil}}
\newcommand{\Cart}{\mathrm{Cart}}
\newcommand{\bWeil}{\mathrm{b}\mathrm{Weil}}
\newcommand{\bCart}{\mathrm{b}\mathrm{Cart}}
\newcommand{\Cond}{\mathrm{Cond}}
\newcommand{\IC}{\mathrm{IC}}
\newcommand{\IH}{\mathrm{IH}}
\newcommand{\cris}{\mathrm{cris}}
\newcommand{\Zar}{\mathrm{Zar}}
\newcommand{\HvbCat}{\overline{\mathcal{V}\mathrm{ect}}}
\newcommand{\BanModCat}{\mathcal{B}\mathrm{an}\mathcal{M}\mathrm{od}}
\newcommand{\DesCat}{\mathcal{D}\mathrm{es}}
\newcommand{\RingCat}{\mathcal{R}\mathrm{ing}}
\newcommand{\SchCat}{\mathcal{S}\mathrm{ch}}
\newcommand{\AbCat}{\mathcal{A}\mathrm{b}}
\newcommand{\RSCat}{\mathcal{R}\mathrm{S}}
\newcommand{\LRSCat}{\mathcal{L}\mathrm{RS}}
\newcommand{\CLRSCat}{\mathbb{C}\text{-}\LRSCat}
\newcommand{\CRSCat}{\mathbb{C}\text{-}\RSCat}
\newcommand{\CLA}{\mathbb{C}\text{-}\mathcal{L}\mathrm{A}}
\newcommand{\CASCat}{\mathbb{C}\text{-}\mathcal{A}\mathrm{n}}
\newcommand{\LiuCat}{\mathcal{L}\mathrm{iu}}
\newcommand{\BanCat}{\mathcal{B}\mathrm{an}}
\newcommand{\BanAlgCat}{\mathcal{B}\mathrm{an}\mathcal{A}\mathrm{lg}}
\newcommand{\AnaCat}{\mathcal{A}\mathrm{n}}
\newcommand{\LiuAlgCat}{\mathcal{L}\mathrm{iu}\mathcal{A}\mathrm{lg}}
\newcommand{\AlgCat}{\mathcal{A}\mathrm{lg}}
\newcommand{\SetCat}{\mathcal{S}\mathrm{et}}
\newcommand{\ModCat}{\mathcal{M}\mathrm{od}}
\newcommand{\TopCat}{\mathcal{T}\mathrm{op}}
\newcommand{\CohCat}{\mathcal{C}\mathrm{oh}}
\newcommand{\SolCat}{\mathcal{S}\mathrm{olid}}
\newcommand{\AffCat}{\mathcal{A}\mathrm{ff}}
\newcommand{\AffAlgCat}{\mathcal{A}\mathrm{ff}\mathcal{A}\mathrm{lg}}
\newcommand{\QcohLiuAlgCat}{\mathcal{L}\mathrm{iu}\mathcal{A}\mathrm{lg}^{\mathrm{QCoh}}}
\newcommand{\LiuMorCat}{\mathcal{L}\mathrm{iu}}
\newcommand{\Isom}{\mathcal{I}\mathrm{som}}
\newcommand{\Cris}{\mathcal{C}\mathrm{ris}}
\newcommand{\Pro}{\mathrm{Pro}-}
\newcommand{\Fin}{\mathcal{F}\mathrm{in}}
\newcommand{\norms}[1]{\left\|#1\right\|}
\newcommand{\HPDDiff}{\mathbf{D}\mathrm{iff}}
\newcommand{\Menn}[2]{\begin{bmatrix}#1\\#2\end{bmatrix}}
\newcommand{\Fins}{\widehat{\Vect}^F}
\newcommand\blfootnote[1]{%
  \begingroup
  \renewcommand\thefootnote{}\footnote{#1}%
  \addtocounter{footnote}{-1}%
  \endgroup
}

\externaldocument[Introduction-]{Introduction}
%One variable complex analysis
%Several variables complex analysis
\externaldocument[Topology-]{Topology-Bornology}
\externaldocument[Banach-]{Banach-Rings}
\externaldocument[Commutative-]{Commutative-Algebra}
\externaldocument[Local-]{Local-Algebras}
\externaldocument[Complex-]{Complex-Analytic-Spaces}
%Properties of space
\externaldocument[Morphisms-]{Morphisms}
%Differential calculus
%GAGA
%Hilbert scheme complex analytic version

%Complex differential geometry

\externaldocument[Affinoid-]{Affinoid-Algebras}
\externaldocument[Berkovich-]{Berkovich-Analytic-Spaces}


\bibliography{Ymir}

\endinput
\title{Ymir}
\begin{document}
\maketitle
\tableofcontents

\chapter*{Local properties of complex analytic spaces}\label{chap-propcomplex}

\section{Introduction}\label{sec-introduction-propertycomplex}


\section{Dimension}

\begin{definition}
    Let $X$ be a complex analytic space and $x\in X$, the \emph{dimension} $\dim_x X$ of $X$ at $x$ is 
    \[
        \dim_x X=\dim \mathcal{O}_{X,x}.
    \]
    We also define the \emph{dimension} of the pointed complex analytic space $(X,x)$ and the \emph{dimension} of the complex analytic germ $X_x$ as $\dim_x X$.

    When $X$ is connected, the \emph{dimension} of $X$ is defined as
    \[
        \dim X:=\sup_{x\in X}\dim_x X.  
    \]
    If $A$ is an analytic set in $X$ such that there is a closed analytic subspace of $X$ with $|B|=A$, then $\dim_x B$ does not depend on the choice of $B$, we define it as $\dim_x A$. 
\end{definition}
As we will see in \cref{cor-reducedinduced}, $B$ always exists.


\begin{definition}
    Let $X$ be a complex analytic space, we say $X$ is \emph{equidimensional} at $x\in X$ if $\mathcal{O}_{X,x}$ is equidimensional and $x\mapsto \dim_x X$ is locally constant.

    We also say $(X,x)$ or $X_x$ is \emph{equidimensional}.

    We say $X$ is \emph{equidimensional of dimension $n\in \mathbb{N}$} if $X$ is non-empty and is equidimensional of dimension $n$ at each $x\in X$.
\end{definition}
Recall that in general, a local ring $R$ is equidimensional if $\dim R/\mathfrak{p}=\dim R$ for all minimal prime $\mathfrak{p}$ of $R$.

\begin{definition}
    Let $X$ be a complex analytic space and $x\in X$, we say $X$ is \emph{integral} at $x$ if $\mathcal{O}_{X,x}$ is integral.
\end{definition}
This corresponds to the notion defined in \cref{ConstructionComplex-def-integralgerm} in \nameref{ConstructionComplex-chap-constructionComplex}.

\begin{thm}\label{thm-equidimlocusopen}
    Let $X$ be a complex analytic space and $n\in \mathbb{N}$, then the set of points $x\in X$ such that $X_x$ is equidimensional of dimension $n$ is open. 
\end{thm}
This is analogous to the result for noetherian cartenary schemes.
\begin{proof}
    Let $x\in X$ be a point such that $X_x$ is equidimensional of dimension $n$. 
    We want to construct an open neighbourhood $V$ of $x$ in $X$ such that $X$ is equidimensional of dimension $n$ at any $y\in V$.

    \textbf{Step~1}. We reduce to the case where $X$ is integral at $x$.

    Let $\mathfrak{p}_1,\ldots,\mathfrak{p}_m$ be the minimal primes of $\mathcal{O}_{X,x}$. The number is finite because $\mathcal{O}_{X,x}$ is noetherian. We have
    \[
        \bigcap_{i=1}^m \mathfrak{p}_i=\rad \mathcal{O}_{X,x}.
    \]
    Take an open neighbourhood $U$ of $x$ in $X$  such that there are ideals of finite type $\mathcal{I}_1,\ldots,\mathcal{I}_m$ extending $\mathfrak{p}_1,\ldots,\mathfrak{p}_m$.  Up to shrinking $U$, we may assume that 
    \[
        \bigcap_{i=1}^m \mathcal{I}_i  
    \]
    is nilpotent. For each $i=1,\ldots,m$, let $U_i$ denote the closed analytic subspace of $U$ defined by $\mathcal{I}_i$. Then
    \[
        |U|=\bigcup_{i=1}^m |U_i|  
    \]
    by \cref{ConstructionComplex-cor-analyticsubspacesamesupp} in \nameref{ConstructionComplex-chap-constructionComplex}. 
    As for any $y\in U$,
    \[
        \bigcap_{i=1}^m \mathcal{I}_{i,y} 
    \]
    is nilpotent, we have
    \[
        |\Spec \mathcal{O}_{X,y}|=|\Spec \mathcal{O}_{X,y}/\bigcap_{i=1}^m \mathcal{I}_{i,y} |=\bigcup_{i=1}^m |\Spec \mathcal{O}_{X,y}/\mathcal{I}_{i,y} |.
    \]
    In particular, for any $y\in U$, 
    \[
        \dim_y X=\dim_y U=\max_{i=1,\ldots,m} \dim_y U_i. 
    \]
    It suffices to handle each $W_i$ separately. 

    \textbf{Step~2}. We assume that $X_x$ is integral. By \cref{ConstructionComplex-thm-morphismdefinedbygeneratorlocal} in \nameref{ConstructionComplex-chap-constructionComplex}, we may assume that  $X$ has the following structure: there is an open neighbourhood $W$ of $0$ in $\mathbb{C}^n$, a morphism $(X,x)\rightarrow (W,0)$ and a finite $\mathcal{O}_W$-algebra $\mathcal{A}$ such that $\Spec^{\An}_W \mathcal{A}$ has a unique point $x'$ over $0$ and $(\Spec^{\An}_W \mathcal{A},x')$ is isomorphic to $(X,x)$ over $(W,0)$. By \cref{Local-cor-morphismpowertolocalsurinj} in \nameref{Local-chap-local}, $\mathcal{O}_{W,0}\rightarrow \mathcal{O}_{X,x}$ is injective, hence $\mathcal{O}_{X,x}$ is torsion-free over $\mathcal{O}_{W,0}$. As the torsion sheaf is coherent, up to shrinking $X$, we may assume that $\mathcal{O}_{X,y}$ is torsion-free over $\mathcal{O}_{W,z}$, where $z$ denotes the image of $y$ in $W$. It suffices to apply \cref{Local-lma-finitetorsionfreeoverlocalequidim} in \nameref{Local-chap-local}.
\end{proof}

\begin{corollary}
    Let $X$ be a complex analytic space and $n\in \mathbb{N}$. Then the set $\{x\in X:\dim_x X\geq n\}$ is an analytic set in $X$.

    %Moreover, $x\in X$ lies in the interior of $\{x\in X:\dim_x X\geq n\}$ if and only if for all minimal prime $\mathfrak{p}$ of $\mathcal{O}_{X,x}$, we have
    %\[
    %    \dim \mathcal{O}_{X,x}/\mathfrak{p}\geq n.  
    %\]
\end{corollary}
After introducing the analytic Zariski topology, we can reformulate this corollary as follows: the map $x\mapsto \dim_x X$ is upper semi-continuous with respect to the analytic Zariski topology.
\begin{proof}
    The problem is local on $X$. Fix $x\in X$ and let $\mathfrak{p}_1,\ldots,\mathfrak{p}_m$ be the minimal prime ideals of $\mathcal{O}_{X,x}$. Up to shrinking $X$, we may assume that 
    \[
        |X|=\bigcup_{i=1}^m |W_i|,  
    \]
    where $W_i$ is a closed analytic subspace of $X$ defined by a coherent $\mathcal{I}_i$ spreading $\mathfrak{p}_i$. We can guarantee that
    \[
        \dim_y X=\max_{i=1,\ldots,m}\dim_y W_i.   
    \]
    
    This is possible as in the proof of \cref{thm-equidimlocusopen}. By \cref{thm-equidimlocusopen}, up to shrinking $X$, we may assume that $W_i$ is equidimensional of dimension $n_i$ for some $n_i\in \mathbb{N}$ for each $i=1,\ldots,m$. In particular, for each $y\in X$, we have
    \[
        \dim_y X=\sup_{y\in W_i} n_i.  
    \]
    So 
    \[
        \{x\in X:\dim_x X\geq n\}=\bigcup_{i: n_i\geq n} |W_i|.  
    \]
    The corollary follows.
\end{proof}



\begin{proposition}
    Let $X,Y$ be complex analytic spaces and $x\in X$, $y\in Y$. Then
    \[
        \dim_{(x,y)}X\times Y=\dim_x X+\dim_y Y.  
    \]
\end{proposition}
\begin{proof}
    By \cref{Local-thm-complocalringtest} in \nameref{Local-chap-local},
    \[
        \hat{\mathcal{O}}_{X\times Y,(x,y)}=\hat{\mathcal{O}}_{X,x}\hat{\otimes}\hat{\mathcal{O}}_{Y,y}.  
    \]
    As dimension is invariant under completion by \cite[\href{https://stacks.math.columbia.edu/tag/07NV}{Tag 07NV}]{stacks-project}, it suffices to show that
    \[
        \dim (\mathcal{O}_{X,x}\hat{\otimes} \mathcal{O}_{Y,y}) =\dim \mathcal{O}_{X,x}+\dim \mathcal{O}_{Y,y}, 
    \]
    which is well-known.
\end{proof}



\begin{definition}
    Let $X_x$ be an analytic germ and $Y_x$ be a closed analytic subgerm defined by an ideal $I\subseteq \mathcal{O}_{X,x}$.
    \begin{enumerate}
        \item When $Y_x$ is irreducible, namely when $I$ is a prime ideal, we define the \emph{codimension} of $Y_x$ in $X_x$ as 
            \[
                \codim_x(Y,X):=\het_{\mathcal{O}_{X,x}}(I).    
            \]
        \item In general, we define the \emph{codimension} of $Y_x$ in $X_x$ as 
        \[
            \codim_x(Y,X):=\inf_{Z_x\subseteq Y_x} \codim_x(Y,X),
        \]
        where $Z_x$ runs over closed analytic subgerms of $X_x$ contained in $Y_x$.
    \end{enumerate}
    We also call $\codim_x(Y,X)$ the codimension of $Y$ in $X$ at $x$.
\end{definition}
Observe that 
\[
    \codim_x(Y,X)\leq \dim_x X-\dim_x Y.
\]
When $X_x$ is equidimensional, $\codim_x(Y,X)$ is nothing but $\dim_x X-\dim_x Y$.

Observe that 
\begin{equation}\label{eq-codimdefrefor}
    \codim_x(Y,X)=\codim (Y_x,\Spec \mathcal{O}_{X,x}).
\end{equation}

\begin{lemma}\label{lma-codimindp}
    Let $X$ be a complex analytic space and $T$ be an analytic set in $X$. Let $Y_1$, $Y_2$ be two closed analytic subspaces of $X$ with underlying set $T$, then for any $x\in T$,
    \[
        \codim_x(Y_1,X)=\codim_x(Y_2,X).
    \]
\end{lemma}
\begin{proof}
    This follows from \eqref{eq-codimdefrefor} and \cref{ConstructionComplex-cor-orderrelationsubgermideal} in \nameref{ConstructionComplex-chap-constructionComplex}.
\end{proof}

\begin{definition}
    Let  $X$ be a complex analytic space and $T$ be an analytic set in $X$. Take $y\in T$. We define the \emph{codimension} $\codim_y(T,X)$ as follows: up to shrinking $X$, we may take a closed analytic subspace $Y$ of $X$ with underlying set $T$ by \cref{ConstructionComplex-lma-analyticsetlocallyidealsheaf} in \nameref{ConstructionComplex-chap-constructionComplex}, we define 
    \[
        \codim_y(T,X):=\codim_y(Y,X).  
    \]
    This definition does not depend on the choices we made by \cref{lma-codimindp}.
\end{definition}

\begin{lemma}\label{lma-codimconstantlocal}
    Let $X$ be a complex analytic space and $Y$ be a closed analytic subspace of $X$. Let $y\in Y$ be a point such that $Y_y$ is irreducible. Then there is an open neighbourhood $U$ of $y$ in $Y$ such that 
    \[
        \codim_z(Y,X)=\codim_y(Y,X)  
    \]
    for any $z\in U$.
\end{lemma}
\begin{proof}
    Let $X'_y$ be an irreducible component of $X_y$ containing $Y_y$ such that 
    \[
        \codim_y(Y,X)=\dim_y X'-\dim_y  Y.
    \]
    We can then take an open neighbourhood $U$ of $x$ in $X$ such that $X'_z$ is equidimensional of dimension $n:=\dim_y X'$ for all $z\in U$ by \cref{thm-equidimlocusopen}. Then for any $z\in U$, $X'_z$ is a union of some irreducible components of $X_z$. Up to shrinking $U$, we may guarantee that for any $z\in U\cap Y$, $Y_z\subseteq X'_z$ and $\dim_z Y=\dim_y Y$. Thereofre, for $z\in Y\cap U$,
    \[
        \codim_z(Y,X)=\codim_z(Y,X')=\dim_z X'-\dim_z Y  
    \]
    is a constant.
\end{proof}

\begin{corollary}\label{cor-codimgeqnanalytic}
    Let $X$ be a complex analytic space and $Y$ be an analytic set in $X$. For any $n\in \mathbb{N}$, 
    \[
        \left\{ y\in Y: \codim_y(Y,X)\leq n\right\}
    \]
    is an analytic set in $Y$.
\end{corollary}
\begin{proof}
    The problem is local. Let $x\in Y$. Let $Y_{1,x},\ldots,Y_{m,x}$ be the irreducible components of $Y_x$ defined by prime ideals $J_1,\ldots,J_m$ in $\mathcal{O}_{Y,x}$. Take an open neighbourhood $U$ of $x$ in $X$ such that for any $y\in Y\cap U$, the ideal 
    \[
        \bigcap_{i=1}^m J_{i,y}  
    \]
    is nilpotent. By \cref{lma-codimconstantlocal}, up to shrinking $U$, we may assume that for any $y\in Y\cap U$,
    \[
        \codim_y(Y_i,X)=\codim_x(Y_i,X)=:c_i  
    \]
    for $i=1,\ldots,m$. Then 
    \[
        \left\{ y\in Y: \codim_y(Y,X)\leq n\right\}=\bigcup_{i:c_i\leq n} Y_i.  
    \]
\end{proof}

\begin{corollary}\label{cor-codimgeqngerm}
    Let $X$ be a complex analytic space and $Y$ be an analytic set in $X$. For any $n\in \mathbb{N}$ and any $y\in Y$,
    \[
        \left\{ y\in Y: \codim_y(Y,X)\leq n\right\}_y=\left\{\mathfrak{p}\in \Spec \mathcal{O}_{X,x}: \codim_{\mathfrak{p}}(T_x,\Spec \mathcal{O}_{X,x})\leq n \right\}.
    \]
\end{corollary}
\begin{proof}
    This is immediate from the proof of \cref{cor-codimgeqnanalytic}.
\end{proof}

\begin{definition}
    Let $X$ be a complex analytic space. A closed subset $A$ of $X$ is \emph{thin} if for any $x \in A$, we can find an open neighbourhood $U$ of $x$ in $X$ such that $A\cap U$ is contained in a nowhere dense analytic subset $B$ of $U$. 

    Given $k\in \mathbb{Z}_{>0}$, we say $A$ is \emph{thin of order $k$} at $x\in A$ if $U$ and $B$ can be chosen so that $\codim_x(B,X)\geq 2$.

    We say $X$ is \emph{thin} (\emph{thin of order $k$}) if $X$ is thin (resp. thin of order $k$) at all $x\in X$.
\end{definition}
The definition in \cite{CAS} Page~132 is not correct when $X$ is not equidimensional. The same happens in several papers of Remmert.



\section{Smoothness}

\begin{definition}
    Let $X$ be a complex analytic space. We say $X$ is \emph{smooth} at $x\in X$ if $\mathcal{O}_{X,x}$ is regular. Otherwise, we say $X$ is \emph{singular} at $x$.

    We also say $(X,x)$ or $X_x$ is \emph{smooth} (resp. \emph{singular}) at $x$.
    
    We say $X$ is \emph{smooth} if it is smooth at all $x\in X$. In this case, we also say $X$ is a \emph{complex manifold}.

    We write $X^{\sing}$ and $X^{\reg}$ for the set of singular and smooth points of $X$ respectively.
\end{definition}
Other common names in the literature include: regular, simple.

\begin{proposition}\label{prop-smoothchar}
    Let $X$ be a complex analytic space and $x\in X$. THen the following are equivalent:
    \begin{enumerate}
        \item $X$ is smooth at $x$;
        \item There is an open neighbourhood $U$ of $x$ in $X$ that is isomorphic to a domain in $\mathbb{C}^n$ with $n=\dim_x X$;
        \item $\Omega_{X,x}$ is a free $\mathcal{O}_{X,x}$-module of rank $\dim_x X$;
        \item $\Omega_{X,x}$ is generated by $\dim_x X$ elements as an $\mathcal{O}_{X,x}$-module;
        \item $\hat{\mathcal{O}}_{X,x}$ is regular;
        \item $\hat{\mathcal{O}}_{X,x}\cong \mathbb{C}[[X_1,\ldots,X_n]]$ for $n=\dim_x X$.
    \end{enumerate}
\end{proposition}
\begin{proof}
    (2) $\implies$ (1): This is obvious.

    (1) $\implies$ (2): Let $f_{1,x},\ldots,f_{n,x}$ be a regular system of parameters of $\mathcal{O}_{X,x}$. Up to shrinking $X$, we may lift them to $f_1,\ldots,f_n\in \mathcal{O}_X(X)$. By \cref{Complex-thm-hCnidentification} in \nameref{Complex-chap-complex}, they induce a morphism $f:(U,x)\rightarrow (\mathbb{C}^n,0)$. Observe that $f^{\#}_{x}:\hat{\mathcal{O}}_{\mathbb{C}^n,0}\rightarrow \hat{\mathcal{O}}_{U,x}$ is an isomorphism, so $f$ is a local isomorphism by \cref{ConstructionComplex-cor-localisocondition} in \nameref{ConstructionComplex-chap-constructionComplex}.

    (2) $\implies$ (3): This follows from \cref{ConstructionComplex-ex-OmegaCn} in \nameref{ConstructionComplex-chap-constructionComplex}.

    (3) $\implies$ (4): This is trivial.

    (4) $\implies$ (1): Recall that $\Omega_X$ is coherent by \cref{ConstructionComplex-cor-Kahdiffcoh} in \nameref{ConstructionComplex-chap-constructionComplex}.
    By Nakayama's lemma, the minimal number of generators of $\Omega_{X,x}$ is equal to $\dim_{\mathbb{C}} \Omega_{X,x}\otimes_{\mathcal{O}_{X,x}} \mathbb{C}$. By algebraic results, we know that the latter space is $\mathfrak{m}_x/\mathfrak{m}_x^2$. So we find that $\dim \mathfrak{m}_x/\mathfrak{m}_x^2=\dim\mathcal{O}_{X,x}$, implying that $\mathcal{O}_{X,x}$ is regular.

    (1) $\Leftrightarrow$ (5): This follows from \cite[\href{https://stacks.math.columbia.edu/tag/07NY}{Tag 07NY}]{stacks-project}.

    (2) $\implies$ (6): This is clear.

    (6) $\implies$ (5): This is clear.
\end{proof}

\begin{thm}\label{thm-Xsingana}
    Let $X$ be a complex analytic space, then $X^{\Sing}$ is an analytic set in $X$.
\end{thm}
\begin{proof}
    The problem is local. Let $x\in X$.

    \textbf{Step~1}. We reduce to the case where $X$ is equidimensional of dimension $n$.

    Let 
    \[
        0=\bigcap_{i=1}^r \mathfrak{p}_i  
    \]
    be the primary decomposition of $0$. Up to shrinking $X$, we may assume that $\mathfrak{p}_1,\ldots,\mathfrak{p}_r$ spread to coherent ideals $\mathcal{I}_1,\ldots,\mathcal{I}_r$ on $X$ and 
    \[
        \bigcap_{i=1}^r\mathcal{I}_i=0.  
    \]
    Let $X_i$ be the closed analytic subspace of $X$ defined by $\mathcal{I}_i$ for $i=1,\ldots,n$. Then
    \[
        X=\bigcup_{i=1}^r X_i.  
    \]
    As each $X_i$ is equidimensional at $x$, say of dimension $n_i$ for $i=1,\ldots,r$. By \cref{thm-equidimlocusopen}, up to shrinking $X$, we may assume that $X_i$ is equidimensional of dimension $n_i$ for $i=1,\ldots,r$. For each
    
    Let $y\in X^{\reg}$, as $\mathcal{O}_{X,y}$ is regular hence integral, from 
    \[
        \bigcap_{i=1}^r\mathcal{I}_{i,y}=0  
    \]
    we find that at least one $\mathcal{I}_{i,y}$ vanishes. Then
    \[
        \mathcal{O}_{X_i,y}=\mathcal{O}_{X,y}  
    \]
    is regular. Namely, $y\in X_i^{\reg}$.
    Conversely, if for some $i=1,\ldots,n$, we have $\mathcal{I}_{i,y}=0$ and $y\in X_i^{\reg}$, $X_i$ is a neighbourhood of $y$ in $X$, so $y\in X^{\reg}$. It follows that 
    \[
        X^{\sing}=\bigcap_{i=1}^r \left(\Supp \mathcal{I}_i \cup X_i^{\Sing} \right). 
    \]
    Recall that $\Supp \mathcal{I}_i$ is analytic for each $i=1,\ldots,n$ by \cref{ConstructionComplex-ex-closedanalyticspaceanalyt} in \nameref{ConstructionComplex-chap-constructionComplex}. 
    
    By \cref{ConstructionComplex-prop-analyticsettrans} in \nameref{ConstructionComplex-chap-constructionComplex}, in order to show that $X^{\sing}$ is an analytic set in $X$, it suffices to know that $X_i^{\Sing} $ is an analytic set in $X_i$ for $i=1,\ldots,n$. 

    \textbf{Step~2}. Assume that $X$ is equidimensional of dimension $n$. We need to show that the locus where $\Omega_{X}$ is locally free of rank $n$ is co-analytic in $X$.

    When $n=0$, the locus where $\Omega_X$ is not locally free of rank $0$ is exactly $\Supp \Omega_X$, which is analytic in $X$ by \cref{ConstructionComplex-ex-closedanalyticspaceanalyt} and \cref{ConstructionComplex-cor-Kahdiffcoh}  in \nameref{ConstructionComplex-chap-constructionComplex}.

    Assume that $n\geq 1$. Let $\Omega_X^n:=\bigwedge^n \Omega_X$. Then the locus where $\Omega_X$ is locally free of rank $n$ is exactly the locus where $\Omega_X^n$ is invertible. The invertible locus of $\Omega_X^n$ is exactly the locus where the canonical map
    \[
        (\Omega_X^n)^{\vee}\otimes_{\mathcal{O}_X}\Omega_X^n\rightarrow \mathcal{O}_X
    \]
    is an isomorphism. It follows that the complement of the locus is analytic in $X$.
\end{proof}


\begin{thm}[Generic smoothness]\label{thm-genericsmoothintegral}
    Let $X$ be a complex analytic space and $x\in X$. Assume that $X$ is integral at $x$, then $X^{\Sing}_x\neq |X|_x$.
\end{thm}
\begin{proof}
    Let $n=\dim_x X$. The problem is local on $X$.
    By \cref{ConstructionComplex-thm-morphismdefinedbygeneratorlocal} in \nameref{ConstructionComplex-chap-constructionComplex}, we may assume that there is a finite morphism $\varphi:(X,x)\rightarrow (V,0)$, where $V$ is an open neighbourhood of $0$ in $\mathbb{C}^n$ and there is a finite $\mathcal{O}_V$-algebra $\mathcal{A}$ with $\mathcal{A}_0=\mathcal{O}_{X,x}$ such that  there is unique point $x'$ of $\Spec^{\An}_V \mathcal{A}$ over $0$ and $(X,x)$ can be identified with $(\Spec^{\An}_V \mathcal{A},x')$.

    Take $\xi\in \mathcal{O}_{X,x}=\mathcal{A}_0$ such that 
    \[
        \Frac \mathcal{O}_{X,x}=\Frac \mathcal{O}_{\mathbb{C}^n,0}(\xi).
    \]
    Let $P_0\in \mathcal{O}_{\mathbb{C}^n,0}[X]$ be the minimal polynomial of $\xi$. 
    Up to shrinking $V$, we may assume that $\xi$ spreads to a section $f\in \mathcal{A}(V)$. Then $\mathcal{B}=\mathcal{O}_V[f]$ is a finite sub-$\mathcal{O}_V$-algebra of $\mathcal{A}$.  Up to shrinking $V$, we may assume that the kernel of $\mathcal{O}_V[X]\rightarrow \mathcal{B}$ sending $X$ to $f$ is generated by a unitary polynomial $P\in \mathcal{O}_V(V)[X]$ of degree $d:=[\Frac \mathcal{O}_{X,x}:\Frac \mathcal{O}_{\mathbb{C}^n,0}]$ that extends $P_0$. Therefore,
    \[
        \mathcal{B}\cong \mathcal{O}_V[X]/(P).  
    \]
    Let $T=\Supp \mathcal{A}/\mathcal{B}$. We endow $T$ with the structure of closed analytic subspace of $V$ induced by the annihilator of $\mathcal{A}/\mathcal{B}$.
    Observe that $\mathcal{A}_0/\mathcal{B}_0=\mathcal{O}_{X,x}/\mathcal{O}_{\mathbb{C}^n,0}$ is torsion, so $|T|_0=\Supp \mathcal{A}_0/\mathcal{B}_0\neq \Spec \mathcal{O}_{\mathbb{C}^n,0}$. That is, $T_0\neq \mathbb{C}^n_0$ by \cref{ConstructionComplex-thm-Nullstellensatz} in \nameref{ConstructionComplex-chap-constructionComplex}. 
    Observe that $X\setminus \varphi^{-1}(T)=\Spec_{V\setminus T}^{\An}\mathcal{B}|_{V\setminus T}$.

    On the other hand, $P_0'(\xi)\neq 0$ as $\xi$ is separable. So $W(P'(f))\neq |X|_x$. Let $Z=\Supp \mathcal{O}_X/(P'(f))$, then $\varphi$ is unramified outside $T$. \textcolor{red}{Include the parts regarding unramified morphisms and étale morphisms before this section}
    In particular, $\varphi$ is étale outside $T$ and hence a local isomorphism by \cref{ConstructionComplex-cor-localisocondition} in \nameref{ConstructionComplex-chap-constructionComplex}. In particular,
    \[
        X^{\sing}\subseteq Z\cup\varphi^{-1}(T)  
    \]
    and hence
    \[
        X^{\sing}_x\subseteq Z_x\cup\varphi^{-1}(T)_x. 
    \]
    The latter is not equal to $|X|_x$ by \cref{ConstructionComplex-cor-orderrelationsubgermideal} in \nameref{ConstructionComplex-chap-constructionComplex} and the fact that $\mathcal{O}_{X,x}$ is integral.
\end{proof}

\begin{thm}[Abhyankar]\label{thm:Abhyansinglocus}
    Let $X$ be a complex analytic space and $x\in X$, then
    \[
        X^{\Sing}_x=(\Spec \mathcal{O}_{X,x})^{\Sing}.  
    \]
\end{thm}
\begin{proof}
    Let $\mathfrak{p}\in \Spec \mathcal{O}_{X,x}$. In concrete terms, we need to show that $W(\mathfrak{p})\not\subset X^{\Sing}_x$ if and only if $\Spec \mathcal{O}_{X,x}$ is regular at $\mathfrak{p}$.
    
    
    The problem is local on $X$. Up to shrinking $X$, we may assume that $\mathfrak{p}$ spreads to a coherent ideal $\mathcal{I}$ on $X$. Let $Y$ be the closed analytic subspace of $X$ defined by $\mathcal{I}$. By \cref{lma-codimconstantlocal}, up to shrinking $X$, we may assume that $\codim_y(Y,X)$ is constant for $y\in Y$. We denote this common value as $p$, which is necessarily equal to the height of $\mathfrak{p}$.

    As $Y_x$ is irreducible by assumption, for an analytic set $Z$ in $Y$ satisfying $Z_x\neq |Y|_x$, the following conditions are equivalent:
    \begin{enumerate}
        \item $|Y|_x\not\subset X^{\Sing}_x$;
        \item $(|Y|\setminus Z)_x \not\subset X^{\Sing}_x$.
    \end{enumerate}
    (2) $\implies$ (1) is trivial. If (2) fails, then 
    \[
        |Y|_x= (|Y|\cup X^{\Sing})_x\cup Z_x.  
    \]
    So $|Y|_x=(|Y|\cup X^{\Sing})_x$, namely (1) holds. We apply this remark to 
    \[
        Z=Y^{\Sing} \cup S_{p'}(\mathcal{I}/\mathcal{I}^2), 
    \]
    where $p'$ is the dimension of the Zariski tangent space of $\Spec \mathcal{O}_{X,x}$ at $\mathfrak{p}$ and $S_{p'}(\mathcal{I}/\mathcal{I}^2)$ is the locus where $\mathcal{I}/\mathcal{I}^2$ is not locally free of rank $p'$. Note that neither part of $Z$ is equal to $|Y|_x$, the former follows from  \cref{thm-genericsmoothintegral} and the latter follows from \cref{ConstructionComplex-thm-Nullstellensatz} in \nameref{ConstructionComplex-chap-constructionComplex} as clearly $\mathfrak{p}\not\in S_{p'}(\mathcal{I}/\mathcal{I}^2)$. 
    We find that $W(\mathfrak{p})\not\subset X^{\Sing}_x$ if and only if $(|Y|\setminus Z)_x \not\subset X^{\Sing}_x$.
    
    If $y\in |Y|\setminus Z$, then $y$ is a regular point of $Y$ and $\codim_y(Y,X)=p$. On the other hand, $\mathcal{I}/\mathcal{I}^2$ is free of rank $p'$ around $y$. But given the regularity of $\mathcal{O}_{Y,y}$, the regularity of $\mathcal{O}_{X,y}$ is equivalent to the fact that $\mathcal{I}/\mathcal{I}^2$ is free of rank $p$. Or equivalently to $p=p'$. The latter is equivalent to the regularity of $\mathfrak{p}$ in $\Spec \mathcal{O}_{X,x}$. The theorem is established.
\end{proof}

\begin{proposition}\label{prop-regularlocusproduct}
    Let $X,Y$ be complex analytic spaces and $x\in X$, $y\in Y$. Then the following are equivalent:
    \begin{enumerate}
        \item $X$ is regular at $x$ and $Y$ is regular at $y$;
        \item $X\times Y$ is regular at $(x,y)$.
    \end{enumerate}
\end{proposition}
This follows from \cref{ConstructionComplex-cor-Omega1sum} in \nameref{ConstructionComplex-chap-constructionComplex} and \cref{prop-smoothchar}.
\iffalse
\begin{proof}
    By \cref{Local-thm-complocalringtest} in \nameref{Local-chap-local},
    \[
        \hat{\mathcal{O}}_{X\times Y,(x,y)}=\hat{\mathcal{O}}_{X,x}\hat{\otimes}\hat{\mathcal{O}}_{Y,y}.  
    \]
    As a noetherian local ring is regular if and only if its completion is by \cite[\href{https://stacks.math.columbia.edu/tag/07NY}{Tag 07NY}]{stacks-project}, it suffices to prove the corresponding algebraic result, which is well-known.
\end{proof}
\fi

\begin{thm}\label{thm-Riemannextsmo}
    Let $X$ be a complex manifold and $A$ be a thin subset of $X$. Let $f\in \mathcal{O}_{X}(X\setminus A)$. Assume that either of the following conditions hold:
    \begin{enumerate}
        \item $f$ is locally bounded near $A$;
        \item $A$ is thin of order $2$ in $X$.
    \end{enumerate}
    Then $f$ admits a unique extension to an element in $\mathcal{O}_X(X)$.
\end{thm}
\begin{proof}
    The problem is local on $X$. By \cref{prop-smoothchar}, we may assume that $X$ is a domain in $\mathbb{C}^n$ for some $n\in \mathbb{N}$. In this case, the results are the classical Riemann extension theorem.
\end{proof}

\begin{corollary}\label{cor-complthininconnectedconn}
    Let $X$ be a connected complex manifold and $A$ be a thin set in $X$. Then $X\setminus A$ is connected.
\end{corollary}
\begin{proof}
    Assume that $X\setminus A$ can be written as the disjoint union of two open subsets $U_0,U_1$. Then the function $f\in \mathcal{O}_X(X\setminus A)=\mathcal{O}_X(U_0)\times \mathcal{O}_X(U_1)$ given by $0\in \mathcal{O}_X(U_0)$ and $1\in \mathcal{O}_X(U_1)$ is locally bounded near $A$. By \cref{thm-Riemannextsmo}, $f$ admits a unique extension to $g\in \mathcal{O}_X(X)$. As $X$ is connected and the image of $f$ is contained in $\overline{\{0,1\}}=\{0,1\}$, it follows that $f$ is constant, so $U_0$ or $U_1$ has to be empty.
\end{proof}

\section{Serre's condition \texorpdfstring{$R_n$}{Rn}}
Fix $n\in \mathbb{N}$ in this section.
\begin{definition}
    Let $X$ be a complex analytic space, we say $X$ \emph{satisfies $R_n$} at $x\in X$ if $\mathcal{O}_{X,x}$ satisfies $R_n$. We also say $(X,x)$ or $X_x$  \emph{satisfies $R_n$} at $x\in X$.

    We say $X$  \emph{satisfies $R_n$} if $X$ satisfies $R_n$ at all points $x\in X$. 
\end{definition}

\begin{proposition}\label{prop-Rncompletion}
    Let $X$ be a complex analytic space and $x\in X$. Take $n\in \mathbb{N}$. Then the following are equivalent:
    \begin{enumerate}
        \item $X$ satisfies $R_n$ at $x$;
        \item $\hat{\mathcal{O}}_{X,x}$ satisfies $R_n$.
    \end{enumerate}
\end{proposition}
\begin{proof}
    This follows from \cite[\href{https://stacks.math.columbia.edu/tag/07NY}{Tag 07NY}]{stacks-project}.
\end{proof}

\begin{proposition}\label{prop-Rnreformulate}
    Let $X$ be a complex analytic space, $x\in X$ and $n\in \mathbb{N}$. Then the following are equivalent:
    \begin{enumerate}
        \item $X$ satisfies $R_n$ at $x$;
        \item $\codim_x(X^{\Sing},X)>n$.
    \end{enumerate}
\end{proposition}
\begin{proof}
    It follows from \cref{thm:Abhyansinglocus} that (1) holds if and only if $\codim_x(X^{\Sing}_x,\Spec \mathcal{O}_{X,x})>n$, The latter condition is equivalent to (2) by definition.
\end{proof}

\begin{corollary}\label{cor-Rnlocuscoana}
    Let $X$ be a complex analytic space and $n\in \mathbb{N}$. The 
    \[
        \left\{x\in X: X\text{ satisfies }R_n \text{ at }x \right\}  
    \]
    is co-analytic in $X$.
\end{corollary}
\begin{proof}
    This follows from \cref{prop-Rnreformulate} and \cref{cor-codimgeqnanalytic}.
\end{proof}


\begin{proposition}\label{prop-Rnproduct}
    Let $X,Y$ be complex analytic spaces and $x\in X$, $y\in Y$. Take $n\in \mathbb{N}$. Then the following are equivalent:
    \begin{enumerate}
        \item $X$ satisfies $R_n$ at $x$ and $Y$ satisfies $R_n$ at $y$;
        \item $X\times Y$ satisfies $R_n$ at $(x,y)$.
    \end{enumerate}
\end{proposition}
\begin{proof}
    By \cref{prop-regularlocusproduct}, 
    \[
        (X\times Y)^{\Sing}=(X^{\Sing}\times Y)\cup (X\times Y^{\Sing}).  
    \]
    It follows that 
    \[
        \codim_{(x,y)}((X\times Y)^{\Sing},X\times Y)=  \min \left\{ \codim_x (X^{\Sing},X),\codim_y (Y^{\Sing},Y)\right\}
    \]
    We conclude by \cref{prop-Rnreformulate}.
\end{proof}

\section{Serre's condition \texorpdfstring{$S_n$}{Sn}}
Fix $n\in \mathbb{N}$ in this section.
\begin{definition}
    Let $X$ be a complex analytic space, we say $X$ \emph{satisfies $S_n$} at $x\in X$ if $\mathcal{O}_{X,x}$ satisfies $R_n$. We also say $(X,x)$ or $X_x$  \emph{satisfies $S_n$} at $x\in X$.

    We say $X$  \emph{satisfies $S_n$} if $X$ satisfies $S_n$ at all points $x\in X$. 
\end{definition}

\begin{proposition}\label{prop-Sncompletion}
    Let $X$ be a complex analytic space and $x\in X$. Take $n\in \mathbb{N}$. Then the following are equivalent:
    \begin{enumerate}
        \item $X$ satisfies $S_n$ at $x$;
        \item $\hat{\mathcal{O}}_{X,x}$ satisfies $S_n$.
    \end{enumerate}
\end{proposition}
\begin{proof}
    This follows from the fact that $\mathcal{O}_{X,x}$ is the quotient of a regular local ring. \textcolor{red}{Include a reference}
\end{proof}

\begin{proposition}\label{prop-coprofgnanal}
    Let $X$ be a complex analytic space, $\mathcal{F}$ be a coherent sheaf of $\mathcal{O}_X$-modules and $n\in \mathbb{N}$. Then 
    \[
        \left\{x\in X: \codep_{\mathcal{O}_{X,x}}\mathcal{F}_x> n \right\}
    \]
    is an analytic subset of $X$. Moreover, the germ of this set in $\Spec \mathcal{O}_{X,x}$ is equal to 
    \[
        \left\{\mathfrak{p}\in \Spec \mathcal{O}_{X,x}: \codep_{\mathcal{O}_{X,x,\mathfrak{p}}}\mathcal{F}_{x,\mathfrak{p}}> n \right\}. 
    \]
\end{proposition}
\begin{proof}
    \textbf{Step~1}. We reduce to the case where $X$ is smooth and equidimensional of dimension $N$.

    The problem is local in $X$, so we may assume that $X$ is a complex model space. Assume that $X$ is a closed analytic subspace of a domain $U$ in $\mathbb{C}^m$ for some $m\in \mathbb{N}$. For any $x\in X$, we have
    \[
        \codep_{\mathcal{O}_{X,x}}\mathcal{F}_x=\codep_{\mathcal{O}_{U,x}}\mathcal{G}_x,
    \]
    where $\mathcal{G}$ is the zero-extension of $\mathcal{F}$ to $U$. A similar formula holds for $\codep_{\mathcal{O}_{X,x,\mathfrak{p}}}\mathcal{F}_{x,\mathfrak{p}}$. So it suffices to handle $U$ instead of $X$.

    \textbf{Step~2}. We prove the result after the reduction in Step~1.

    By Auslander--Buchsbaum formula, for $x\in X$,
    \[
        \pd_{\mathcal{O}_{X,x}} \mathcal{F}_x+\dep_{\mathcal{O}_{X,x}} \mathcal{F}_x=\dep \mathcal{O}_{X,x}=\dim \mathcal{O}_{X,x}.
    \]
    So the condition $\codep_{\mathcal{O}_{X,x}} \mathcal{F}_x>n$ is equivalent to
    \[
        \pd_{\mathcal{O}_{X,x}} \mathcal{F}_x> n+\dim \mathcal{O}_{X,x}-\dim_x \Supp\mathcal{F}.
    \]
    As $\mathcal{O}_{X,x}$ is regular hence equidimensional, the condition just means
    \[
        \pd_{\mathcal{O}_{X,x}} \mathcal{F}_x> n+\codim_x(\Supp \mathcal{F}, X).
    \]
    As $\mathcal{O}_{X,x}$ is regular, this condition is equivalent to the existence of an integer $r>n+\codim_x(\Supp \mathcal{F}, X)$ such that
    \[
        \ExtFun_{\mathcal{O}_X}^r(\mathcal{F},\mathcal{O}_X)_x\neq 0.  
    \]
    For each $p\in \mathbb{N}$, we introduce
    \[
        T_p(\mathcal{F}):=\bigcup_{r=p+1}^{N}\Supp \ExtFun_{\mathcal{O}_X}^r(\mathcal{F},\mathcal{O}_X).  
    \]
    Then the proceeding analysis shows that 
    \[
        \left\{x\in X: \codep_{\mathcal{O}_{X,x}}\mathcal{F}_x> n \right\}=\bigcup_{s=0}^{N} T_{n+s}(\mathcal{F})\cap \left\{y\in \Supp \mathcal{F}:\codim_y(\Supp \mathcal{F},X)\leq s\right\}.
    \]
    Observe that the right-hand side is an analyitc set in $X$ by  \cref{ConstructionComplex-ex-closedanalyticspaceanalyt} in \nameref{ConstructionComplex-chap-constructionComplex} and \cref{cor-codimgeqnanalytic}, hence so is the left-hand side.

    It remains to compute the germ at $y\in X$. For $p\in \mathbb{N}$, we compute
    \[
        T_p(\mathcal{F})_y=  \bigcup_{r=p+1}^{N}\Supp \ExtFun_{\mathcal{O}_X}^r(\mathcal{F},\mathcal{O}_X)_y.
    \]
    But observe that
    \[
        \ExtFun_{\mathcal{O}_X}^r(\mathcal{F},\mathcal{O}_X)_y=\Ext_{\mathcal{O}_{X,y}}^r(\mathcal{F}_y,\mathcal{O}_{X,y}).
    \]
    Let $\widetilde{\mathcal{F}_y}$ be the coherent module on $\Spec \mathcal{O}_{X,x}$ associated with $\mathcal{F}_y$. Let $X_y=\Spec \mathcal{O}_{X,y}$
    Then 
    \[
        T_p(\mathcal{F})_y=\bigcup_{r=p+1}^{N} \Supp \ExtFun_{\mathcal{O}_{X_y}}^r(\widetilde{\mathcal{F}_y},\mathcal{O}_{X_y})_y.
    \]
    On the other hand, by \cref{cor-codimgeqngerm}, for $s\in \mathbb{N}$,
    \[
        \left\{x\in \Supp \mathcal{F}:\codim_x(\Supp \mathcal{F},X)\leq s\right\}_y=  \left\{ \mathfrak{p}\in \Spec \mathcal{O}_{X,y}: \codim_{\mathfrak{p}}(\Supp \widetilde{F_y},\Spec \mathcal{O}_{X,y}) \right\}.
    \]
    The same argument as above shows that 
    \[
        \left\{x\in X: \codep_{\mathcal{O}_{X,x}}\mathcal{F}_x> n \right\}_y=\left\{\mathfrak{p}\in \Spec \mathcal{O}_{X,y}: \codep_{\mathcal{O}_{X,y,\mathfrak{p}}}\mathcal{F}_{y,\mathfrak{p}}> n \right\}. 
    \]
\end{proof}

\begin{proposition}\label{prop-Sncomplement}
    Let $X$ be a complex analytic space and $n\in \mathbb{N}$. Then the set of $x\in X$ such that $X$ satisfies $S_n$ at $x$ is the complement of 
    \[
        \bigcup_{m=0}^{\infty}\left\{ y\in Z_m :\codim_y(Z_m,X)\leq n+m \right\},  
    \]
    where 
    \[
        Z_m= \left\{x\in X: \codep \mathcal{O}_{X,x}\mathcal{F}_x> m \right\}.
    \]
\end{proposition}
\begin{proof}
    It suffices to observe that for $x\in X$, $X$ satisfies $S_n$ at $x$ if and only if
    \[
        \codim \left(\left\{ \mathfrak{p}\in \Spec \mathcal{O}_{X,x}: \codep \mathcal{O}_{X,x,\mathfrak{p}} \right\} ,\Spec \mathcal{O}_{X,x}\right)>n+m  
    \]
    for all $m\in \mathbb{N}$.
\end{proof}

\begin{corollary}\label{cor-Snlocuscoana}
    Let $X$ be a complex analytic space and $n\in \mathbb{N}$. Then the set of $x\in X$ such that $X$ satisfies $S_n$ at $x$ is co-analytic.
\end{corollary}
\begin{proof}
    This follows from \cref{prop-Sncomplement} and \cref{prop-coprofgnanal}.
\end{proof}

\begin{proposition}\label{prop-Snproduct}
    Let $X$, $Y$ be complex analytic spaces and $x\in X$, $y\in Y$. Take $n\in \mathbb{N}$. Assume that $X$ satisfies $S_n$ at $x$ and $Y$ satisfies $S_n$ at $y$, then $X\times Y$ satisfies $S_n$ at $(x,y)$.
\end{proposition}
\begin{proof}
    By \cref{Local-thm-complocalringtest} in \nameref{Local-chap-local},
    \[
        \hat{\mathcal{O}}_{X\times Y,(x,y)}=\hat{\mathcal{O}}_{X,x}\hat{\otimes}\hat{\mathcal{O}}_{Y,y}.  
    \]
    As being $S_n$ is invariant under completion by \cite[\href{https://stacks.math.columbia.edu/tag/07NW}{Tag 07NW}]{stacks-project} and \cite[\href{https://stacks.math.columbia.edu/tag/07NV}{Tag 07NV}]{stacks-project}, it suffices to prove the corresponding algebraic result, which is known.
\end{proof}

\section{Reducedness}
\begin{definition}
    Let $X$ be a complex analytic space, we say $X$ is \emph{reduced} at $x\in X$ if $\mathcal{O}_{X,x}$ is reduced. We also say $(X,x)$ or $X_x$ is \emph{reduced} at $x\in X$.

    We say $X$ is \emph{reduced} if $X$ is reduced at all points $x\in X$. 
\end{definition}

\begin{proposition}\label{prop-reducedcompletion}
    Let $X$ be a complex analytic space and $x\in X$.  Then the following are equivalent:
    \begin{enumerate}
        \item $X$ is reduced $x$;
        \item $\hat{\mathcal{O}}_{X,x}$ is reduced.
    \end{enumerate}
\end{proposition}
\begin{proof}
    This follows from \cref{prop-Rncompletion} and \cref{prop-Sncompletion}.
    
    Otherwise, one can also argue as follows: Recall that an excellent ring is Nagata by \cite[\href{https://stacks.math.columbia.edu/tag/07QV}{Tag 07QV}]{stacks-project}.
    A Nagata noetherian local ring is reduced if and only if its completion is by \cite[\href{https://stacks.math.columbia.edu/tag/07NZ}{Tag 07NZ}]{stacks-project}.
\end{proof}

\begin{thm}\label{thm-reducedlocusopen}
    Let $X$ be a complex analytic space. Then the set of points $x\in X$ such that $\mathcal{O}_{X,x}$ is reduced is co-analytic.
\end{thm}
\begin{proof}
    This follows from \cref{cor-Snlocuscoana} and \cref{cor-Rnlocuscoana} as reduceness is equivalent to $S_1$ and $R_0$.
\end{proof}

\begin{corollary}\label{cor-nilradcoh}
    Let $X$ be a complex analytic space, then the nilradical $\rad \mathcal{O}_X$ is coherent.
\end{corollary}
\begin{proof}
    The problem is local on $X$. Take $x\in X$. Up to shrinking $X$, we may assume that $\mathcal{O}_{X,x}/(\rad \mathcal{O}_X)_x$ spreads to a finite $\mathcal{O}_X$-algebra $\mathcal{A}$ by \cref{ConstructionComplex-lma-spreadfinitelag} in \nameref{ConstructionComplex-chap-constructionComplex}. Up to further shrinking $X$, we may assume that $\mathcal{A}$ is the quotient of $\mathcal{O}_X$, say $\mathcal{A}\cong \mathcal{O}_X/\mathcal{I}$ for some coherent ideal $\mathcal{I}$ on $X$. As $\mathcal{I}_x$ is nilpotent by assumption, up to shrinking $X$, we may assume that $\mathcal{I}$ is also nilpotent, namely
    \[
        \mathcal{I}\subseteq  \rad \mathcal{O}_X. 
    \]
    Let $Y$ be the closed analytic subspace of $X$ defined by the ideal $\mathcal{I}$. Then $\mathcal{O}_{Y,x}\cong \mathcal{O}_{X,x}/(\rad \mathcal{O}_X)_x$ is reduced. Up to shrinking $X$, by \cref{thm-reducedlocusopen}, we may assume that $Y$ is reduced. But then for any $y\in Y$, 
    \[
        \mathcal{O}_{Y,y}\cong \mathcal{O}_{X,y}/\mathcal{I}_y
    \]
    is reduced, so
    \[
        \mathcal{I}_y\supseteq   (\rad \mathcal{O}_X)_y.
    \] 
    It follows that $\rad \mathcal{O}_X=\mathcal{I}$, hence the nilradical is coherent.
\end{proof}

\begin{corollary}[Cartan--Oka]\label{cor-CartanOka}
    Let $X$ be a complex analytic space and $A$ be an analytic subset of $X$, then the sheaf $\mathcal{J}_A$ is coherent.
\end{corollary}
Recall that the sheaf $\mathcal{J}_A$ is introduced in \cref{ConstructionComplex-def-sheafidealanaset} in \nameref{ConstructionComplex-chap-constructionComplex}.
\begin{proof}
By   \cref{ConstructionComplex-lma-analyticsetlocallyidealsheaf} in \nameref{ConstructionComplex-chap-constructionComplex}, we may assume that $A$ is a closed analytic subspace of $X$ defined by a coherent ideal $\mathcal{I}$. By \cref{ConstructionComplex-cor-orderrelationsubgermideal} in \nameref{ConstructionComplex-chap-constructionComplex}, the sheaf $\mathcal{J}_A$ is nothing but $\sqrt{I}$, which is coherent by \cref{cor-nilradcoh}.
\end{proof}

\begin{corollary}\label{cor-reducedinduced}
    Let $X$ be a complex analytic space and $A$ be an analytic subset of $X$, then there is a unique reduced closed analytic space $Y$ of $X$ with underlying set $A$.
\end{corollary}
\begin{proof}
    The existence follows from \cref{cor-CartanOka}. The uniqueness follows from \cref{ConstructionComplex-cor-orderrelationsubgermideal} in \nameref{ConstructionComplex-chap-constructionComplex}.
\end{proof}

\begin{definition}
    Let $X$ be a complex analytic space and $A$ be an analytic subset of $X$. The analytic space structure on $A$ defined in \cref{cor-reducedinduced} is called the \emph{reduced induced structure} on $A$. In particular, $|X|$ with the reduced induced structure is denoted by $X^{\Red}$ and is called the \emph{reduced space underlying $X$}.
\end{definition}

\begin{thm}[Generic smoothness]\label{thm-gensmooth}
    Let $X$ be a reduced complex analytic space and $x\in X$, then $X^{\Sing}_x\neq |X|_x$. In other words, $X^{\Sing}$ is nowhere dense in $|X|$. 
\end{thm}
\begin{proof}
    The problem is local. Take $x\in X$. As in the proof of \cref{thm-Xsingana}, up to shrinking $X$, we may assume that there are finitely many closed analytic subsets $X_1,\ldots,X_m$ in $X$ which are irreducible at $x$ such that 
    \[
        X=X_1\cup\cdots\cup X_m.  
    \]
    As $X$ is reduced, we may also assume that $X_1,\ldots,X_m$ are all reduced. Then $X_1,\ldots,X_m$ are all integral at $x$. It follows from \cref{thm-genericsmoothintegral} that 
    \[
        X_i^{\Sing}\neq |X_i|_x
    \]
    for $i=1,\ldots,m$.
    Let $\mathcal{I}_i$ be the coherent ideal sheaf of $X_i$ in $X$ for $i=1,\ldots,m$. It follows from the proof of \cref{thm-Xsingana} that
    \[
        X^{\sing}=\bigcap_{i=1}^m \left(\Supp \mathcal{I}_i \cup X_i^{\Sing} \right). 
    \]
    This implies $X^{\Sing}_x\neq |X|_x$: otherwise, for each $i=1,\ldots,m$, we have
    \[
        (\Supp \mathcal{I}_i)_x\cup (X_i^{\Sing})_x=|X|_x.  
    \]
    So
    \[
        (\Supp \mathcal{I}_i)_x= |X|_x
    \]
    for each $i=1,\ldots,m$. In other words,
    \[
        \Spec \mathcal{O}_{X,x}=\bigcup_{i=1}^m \Supp\mathcal{I}_{i,x}.
    \]
    Observe that $\mathcal{I}_{1,x},\ldots,\mathcal{I}_{m,x}$ are exactly the minimal primes of $\Spec   \mathcal{O}_{X,x}$. This is possible if and only if $m=1$. So we are reduced to the case where $X$ is integral at $x$. But this case is handled in \cref{thm-genericsmoothintegral}.
\end{proof}

\begin{proposition}\label{prop-reducedspacefunctiondetermineholo}
    Let $X$ be a reduced complex analytic space and $f,g\in \mathcal{O}_X(X)$. Assume that $[f]=[g]$, then $f=g$.
\end{proposition}
\begin{proof}
    It follows from \cref{ConstructionComplex-cor-nullversionfunction} in \nameref{ConstructionComplex-chap-constructionComplex} that $f-g$ is locally nilpotent. As $X$ is reduced, $f=g$.
\end{proof}
In particular, on a reduced complex analytic space $X$, a holomorphic function $f$ is uniquely determined by the continuous map $[f]:X\rightarrow \mathbb{C}$ associated with it. In this case, we will say $[f]$ \text{is holomorphic}.

\begin{definition}
    Let $X$ be a reduced complex analytic space. A \emph{continuous weakly holomorphic function} on $X$ is a continuous map $f:X\rightarrow \mathbb{C}$ such that $f|_{X^{\reg}}$ is holomorphic.

    A \emph{weakly holomorphic function} on $X$ is $f\in \mathcal{O}_X(X^{\reg})$ which is locally bounded on $X$.
\end{definition}

\begin{definition}
    Let $f:X\rightarrow Y$ be a topologically finite surjective morphism of reduced complex analytic spaces. We say $f$ is a \emph{branched covering} if there is a thin subset $T$ of $Y$ satisfying the following properties:
    \begin{enumerate}
        \item $\pi^{-1}(T)$ is thin in $X$;
        \item $X\setminus \pi^{-1}(T)\rightarrow Y\setminus T$ induced by $f$ is a local isomorphism.
    \end{enumerate}
    The set $T$ is called a \emph{critial locus}.

    The set of points $x\in X$ where $f$ is not a local isomorphism at $x$ is called the \emph{branch locus} of $f$. The image of the branch locus in $Y$ is called the \emph{minimal critical locus} of $f$.

    Observe that the number of points in the fiber is locally constant outside the critical locus. When this number is actually constant say $b\in \mathbb{N}$ (e.g. when $Y$ is a connected complex manifold by \cref{cor-complthininconnectedconn}), we say $f$ is a \emph{$b$-sheeted branched covering}.
\end{definition}

\section{Normalness}
\begin{definition}
    Let $X$ be a complex analytic space, we say $X$ is \emph{normal} at $x\in X$ if $\mathcal{O}_{X,x}$ is normal. We also say $(X,x)$ or $X_x$ is \emph{normal} at $x\in X$.

    We say $X$ is \emph{normal} if $X$ is normal at all points $x\in X$. 
\end{definition}
\begin{proposition}\label{prop-normalcompletion}
    Let $X$ be a complex analytic space and $x\in X$.  Then the following are equivalent:
    \begin{enumerate}
        \item $X$ is normal $x$;
        \item $\hat{\mathcal{O}}_{X,x}$ is normal.
    \end{enumerate}
\end{proposition}
Condition~(2) is usually known as the \emph{analytic normality} of $\mathcal{O}_{X,x}$. 
\begin{proof}
    This follows from \cref{prop-Rncompletion} and \cref{prop-Sncompletion}.
\end{proof}

\begin{thm}\label{thm-normallocusopen}
    Let $X$ be a complex analytic space. Then the set of points $x\in X$ such that $\mathcal{O}_{X,x}$ is normal is co-analytic.
\end{thm}
\begin{proof}
    This follows from \cref{cor-Snlocuscoana} and \cref{cor-Rnlocuscoana} as reduceness is equivalent to $S_2$ and $R_1$.
\end{proof}

\begin{proposition}\label{prop-normalspacesingcodim2}
    Let $X$ be a normal complex analytic space. Then for any $x\in X^{\Sing}$, 
    \[
        \codim_x(X^{\Sing},X)\geq 2.  
    \]
\end{proposition}
\begin{proof}
    This follows from \cref{thm:Abhyansinglocus} and the corresponding algebraic result.
\end{proof}

\begin{proposition}\label{prop-normalizationcas}
    Let $X$ be a reduced complex analytic space. Then there is a finite $\mathcal{O}_X$-algebra $\overline{\mathcal{O}}_X$ such that for each $x\in X$, $\overline{\mathcal{O}}_{X,x}$ is isomorphism to the inclusion of the integral closure $\overline{\mathcal{O}_{X,x}}$ as $\mathcal{O}_{X,x}$-algebras.

    The sheaf $\overline{\mathcal{O}}_X$ is unique up to a unique isomorphism.
\end{proposition}
\begin{proof}
    The uniqueness is obvious, as there are no non-trivial automorphisms of $\overline{\mathcal{O}}_{X,x}$ as an $\mathcal{O}_{X,x}$-algebra.

    We prove the existence. The problem is then local on $X$. Let $x\in X$. By \cref{ConstructionComplex-lma-spreadfinitelag} in \nameref{ConstructionComplex-chap-constructionComplex}, up to shrinking $X$,
    $\overline{\mathcal{O}_{X,x}}$ spreads to a finite $\mathcal{O}_X$-algebra $\mathcal{A}$. Let $X'=\Spec^{\An}_X\mathcal{A}$. Let $x_1',\ldots,x_m'$ be the distinct points on the fiber over $x$ of $X'\rightarrow X$. By  \cref{ConstructionComplex-cor-finitespectrumidentification} in \nameref{ConstructionComplex-chap-constructionComplex}, the points corresponds to $\Spm_{\mathbb{C}} \mathcal{A}_x$. Let $\mathfrak{p}_1,\ldots,\mathfrak{p}_{m'}$ be the minimal primes of $\mathcal{O}_{X,x}$, then 
    \[
        \mathcal{A}_x=\overline{\mathcal{O}_{X,x}}\cong \prod_{i=1}^{m'}   \overline{\mathcal{O}_{X,x}/\mathfrak{p}_i}.
    \]
    As $\mathcal{O}_{X,x}/\mathfrak{p}_i$ is Henselian, $\overline{\mathcal{O}_{X,x}/\mathfrak{p}_i}$ is in fact local for each $i=1,\ldots,m'$. As $\mathcal{O}_{X,x}/\mathfrak{p}_i$ is excellent, $\overline{\mathcal{O}_{X,x}/\mathfrak{p}_i}$ is finite over $\mathcal{O}_{X,x}/\mathfrak{p}_i$. It follows that $\Spm_{\mathbb{C}}\mathcal{A}_x=\Spm \mathcal{A}_x$. So we find that $m'=m$. Up to a renumbering, we may assume that $\mathfrak{p}_i$ corresponds to $x_i'$ for $i=1,\ldots,m$. Then by \cref{ConstructionComplex-cor-finitespectrumidentification} in \nameref{ConstructionComplex-chap-constructionComplex},
    \[
        \mathcal{O}_{X',x_i'}\cong  \overline{\mathcal{O}_{X,x}/\mathfrak{p}_i} 
    \]
    for $i=1,\ldots,m$. In particular, $X'$ is normal at $x_i'$ for all $i=1,\ldots,m$.
    By \cref{thm-normallocusopen}, \cref{ConstructionComplex-cor-specanfinitetopfinite} in \nameref{ConstructionComplex-chap-constructionComplex} and \cref{Topology-lma-opennhfiberclosedmap} in \nameref{ConstructionComplex-chap-constructionComplex}, up to shrinking $X$, we may assume that $X'$ is normal. We observe that for each $y\in X$, $\mathcal{A}_{y}$ is the product of the local rings of points on the fiber hence normal.

    For $i=1,\ldots,m$, as $\mathcal{O}_{X,x}/\mathfrak{p}_i$ is excellent, its conductor is non-zero. We can find a non-zero $f_{i,x}\in \mathcal{O}_{X,x}/\mathfrak{p}_i$ such that $f_{i,x}\overline{\mathcal{O}_{X,x}/\mathfrak{p}_i}\subseteq \mathcal{O}_{X,x}/\mathfrak{p}_i$. Take 
    \[
        f_x=\prod_{i=1}^m f_{i,x}.  
    \]
    Then $f_x$ is a non-zero divisor in $\mathcal{O}_{X,x}$ and $f_x \mathcal{A}_x\subseteq \mathcal{O}_{X,x}$. Up to shrinking $X$, we may assume that $f_x$ spreads to $f\in \mathcal{O}_X(X)$, and we have an injection
    \[
        f\mathcal{A}\subseteq \mathcal{O}_X.  
    \]
    Up to shrinking $X$, we may also assume that $\mathcal{O}_X\rightarrow \mathcal{A}$ is injective. We therefore get an injective map
    \[
        \mathcal{A}\xrightarrow{\times f} \mathcal{O}_X  \xrightarrow{\times f^{-1}}\mathcal{O}_X[f^{-1}]. 
    \]
    For each $y\in X$, we get an injective map
    \[
        \mathcal{A}_y\rightarrow \mathcal{O}_{X,y}[f_y^{-1}].  
    \]
    In particular, $\mathcal{A}_y$ is in the total ring of fraction of $\mathcal{O}_{X,y}$. As $\mathcal{A}_y$ is finite over $\mathcal{O}_{X,y}$, we have
    \[
        \mathcal{A}_y\subseteq \overline{\mathcal{O}_{X,y}}.
    \]
    On the other hand, $\mathcal{A}_y$ is normal, so equality holds. 
\end{proof}

\begin{definition}
    Let $X$ be a reduced complex analytic space. Then $\Spec^{\An}_X \overline{\mathcal{O}}_X$ constructed in \cref{prop-normalizationcas} is called the \emph{normalization} of $X$. We denote it by $\bar{X}$.
    Note that we have a canonical morphism $\bar{X}\rightarrow X$. 
    
    The normalization of $X$ is well-defined up to a unique isomorphism in $\CASCat_{/X}$.
\end{definition}



\begin{proposition}\label{prop-normalizationbranches}
    Let $X$ be a reduced complex analytic space. For each $x\in X$, the fiber of $\bar{X}\rightarrow X$ over $x$ is in bijection with the set of minimal prime ideals in $\mathcal{O}_{X,x}$. Moreover, if $y$ corresponds to $\mathfrak{p}$, we have
    \[
      \mathcal{O}_{\bar{X},y}\cong  \overline{\mathcal{O}_{X,x}/\mathfrak{p}}
    \] 
    as $\mathcal{O}_{X,x}$-algebras.
\end{proposition}
\begin{proof}
    This follows from the proof of \cref{prop-normalizationcas}.   
\end{proof}



\begin{proposition}\label{prop-reducedspacenormalizationchar}
    Let $X$ be a reduced complex analytic space. Then 
    \begin{enumerate}
        \item $\bar{X}$ is normal;
        \item $p:\bar{X}\rightarrow X$ is topologically finite and surjective;
        \item There is a nowhere dense analytic set $Y$ in $X$ such that $p^{-1}(Y)$ is nowhere dense in $\bar{X}$ and the morphism $\bar{X}\setminus p^{-1}(Y)\rightarrow X\setminus Y$ induced by $p$ is an isomorphism.
    \end{enumerate}
\end{proposition}

Conversely, these conditions determines $\bar{X}$ up to a unique isomorphism in $\CASCat_{/X}$. We will establish this result later.

\begin{proof}
    That $\bar{X}$ is normal follows from \cref{ConstructionComplex-cor-finitespectrumidentification} in \nameref{ConstructionComplex-chap-constructionComplex}.
    The morphism $\bar{X}\rightarrow X$ is topologically finite by \cref{ConstructionComplex-cor-specanfinitetopfinite} in \nameref{ConstructionComplex-chap-constructionComplex}. It is surjective by \cref{ConstructionComplex-cor-suppaffmorphism} in \nameref{ConstructionComplex-chap-constructionComplex}.

    Let $Y$ be the non-normal locus of $X$. It is in particular contained in $X^{\Sing}$. By \cref{prop-normalspacesingcodim2} and \cref{thm-normallocusopen}, $Y$ is a nowhere dense analytic set in $X$. It is clear that $p$ is an isomorphism outside $Y$. 
    
    We prove that $p^{-1}(Y)$ is nowhere dense. Let $x\in X$ and $x'$ be a point on the fiber of $\bar{X}\rightarrow X$ over $x$. Let $\mathfrak{p}'$ be the minimal prime ideal of $\mathcal{O}_{X,x}$ corresponding to $x'$. Then the morphism $\Spec \mathcal{O}_{\bar{X},x'}\rightarrow \Spec \mathcal{O}_{X,x}$ factorizes through $\Spec \mathcal{O}_{\bar{X},x'}\rightarrow \Spec \mathcal{O}_{X,x}/\mathfrak{p}'$. The map is finite and surjective. The latter is because $\mathcal{O}_{X,x}/\mathfrak{p}'\rightarrow \mathcal{O}_{\bar{X},x'}$ is injective. If $p^{-1}(Y)$ contains a neighbourhood of $x'$ in $\bar{X}$, then $|p^{-1}(Y)|_{x'}=\Spec \mathcal{O}_{\bar{X},x'}$. Then $|Y|_x=|\Spec \mathcal{O}_{X,x}/\mathfrak{p}'|$, which is a contradiction.

\end{proof}

\begin{definition}
    Let $X$ be a complex analytic space and $A$ be an analytic set in $X$. We say $A$ is irreducible if $A$ cannot be written as the union of two analytic sets $B$ and $C$ in $X$ with $B\not\subset C$ and $C\not\subset B$.
\end{definition}

\begin{lemma}\label{lma-connectednormalirr}
    Let $X$ be a connected normal complex analytic space. Then $X$ is irreducible.
\end{lemma}
\begin{proof}
    Suppose otherwise, $X$ can be written as the union of $A$, $B$, two analytic sets in $X$ not containing each other. As $X$ is connected, $A\cap B$ is non-trivial. Take $x\in A\cap B$. We endow $A$ and $B$ with the reduced induced structure. Then 
    \[
        \Spec \mathcal{O}_{X,x}=\Spec \mathcal{O}_{A,x}\cup  \Spec \mathcal{O}_{B,x}.
    \]
    This is impossible as $\mathcal{O}_{X,x}$ is unibranch.
\end{proof}

\begin{definition}
    Let $X$ be a reduced complex analytic space. An \emph{irreducible component} of $X$ is the image of a connected component of $\bar{X}$.

    We say $X$ is \emph{irreducible} if $X^{\Red}$ is non-empty and has only one irreducible component.
\end{definition}
By \cref{lma-connectednormalirr}, each irreducible component is irreducible. Moreover, by \cref{prop-reducedspacenormalizationchar}, the decomposition of $|X|$ into the union of its irreducible components is locally finite. No irreducible component is contained in the union of the others.

\begin{proposition}\label{prop-pathtoXreg}
    Let $X$ be a reduced complex analytic space and $x\in X$. Then $x$ can be joined by a path to a point in $X^{\reg}$.
\end{proposition}
\begin{proof}
    We may assume that $x\in X^{\Sing}$. 
    
    \textbf{Step~1}. We reduce to the case where $X$ is normal.

    Let $p:\bar{X}\rightarrow X$ be the normalization. Take $y\in \bar{X}$ with $p(y)=x$. 
    
    We claim that it suffices to show that there is a path connecting $y$ to a regular point of $\bar{X}$. In fact, let $T\subseteq X$ containing $X^{\Sing}$ be a thin analytic set such that $p^{-1}(T)$ is thin and $\bar{X}\setminus p^{-1}(T)\rightarrow X\setminus T$ induced by $p$ is an isomorphism by \cref{prop-reducedspacenormalizationchar}.
    If our claim holds, then all neighbourhood points of $y$ are regular and in particular, we may connect $y$ to a regular poiint in $\bar{X}\setminus p^{-1}(T)$. The image of this path is the desired path.

    \textbf{Step~2}. We proceed by induction on $d:=\dim_x X$.
    
    When $d=1$, $x$ is necessarily regular by \cref{prop-normalspacesingcodim2}. Assume $d>1$. 
    Up to shrinking $X$, we can take $f\in \mathcal{O}_X(X)$ such that $\dim_x W(f)=d-1$. We may assume that $W(f)$ is equidimensional of dimension $d-1$ by \cref{thm-equidimlocusopen}. Then we can find a path from $x$ to a regular point $x'\in W(f)$. By \cref{prop-normalspacesingcodim2}, up to perturbation, we may assume that $x'\in X^{\reg}$. 
\end{proof}


\section{Unibranchness}
\begin{definition}
    Let $X$ be a complex analytic space, we say $X$ is \emph{unibranch} at $x\in X$ if $\mathcal{O}_{X,x}$ is unibranch. We also say $(X,x)$ or $X_x$ is \emph{unibranch} at $x\in X$.

    We say $X$ is \emph{unibranch} if $X$ is unibranch at all points $x\in X$. 
\end{definition}

\begin{proposition}\label{prop-unibranchchar}
    Let $X$ be a complex analytic space and $x\in X$. Then the following are equivalent:
    \begin{enumerate}
        \item $X$ is unibranch at $x$;
        \item $X^{\Red}$ is unibranch at $x$;
        \item $\mathcal{O}_{X,x}$ is geometrically unibranch;
        \item $\mathcal{O}_{X,x}^{\Red}$ is geometrically unibranch;
        \item $\mathcal{O}_{X,x}$ has a unique minimal prime ideal;
        \item The fiber of $\overline{X^{\Red}}\rightarrow X^{\Red}$ over $x$ consists of a single point.
    \end{enumerate}
\end{proposition}
\begin{proof}
    (1) $\Leftrightarrow$ (3): As $\mathcal{O}_{X,x}$ is excellent, $\overline{\mathcal{O}_{X,x}^{\Red}}$ is a finite $\mathcal{O}_{X,x}^{\Red}$-algebra, so the residue field extension is finite. But the residue field of $\mathcal{O}_{X,x}$ is $\mathbb{C}$, so the residue field extension is the trivial extension.

    (1) $\Leftrightarrow$ (5): This follows from \cite[\href{https://stacks.math.columbia.edu/tag/0BQ0}{Tag 0BQ0}]{stacks-project} and the fact that $\mathcal{O}_{X,x}$ is Henselian.

    (1) $\Leftrightarrow$ (2): This follows from the observation that (5) holds for $\mathcal{O}_{X,x}$ if and only if (5) holds for $\mathcal{O}_{X,x}^{\Red}$. 

    (3) $\Leftrightarrow$ (4): This follows from the same argument as (1) $\Leftrightarrow$ (2).

    (5) $\Leftrightarrow$ (6): This follows from \cref{prop-normalizationbranches}.
\end{proof}

\begin{lemma}\label{lma-rankgeqnana}
    Let $X$ be a complex analytic space, $\mathcal{M}$ be a coherent $\mathcal{O}_X$-module, $n\in \mathbb{N}$. Then the set
    \[
        \left\{x\in X: \rank_x \mathcal{M}\leq  n \right\}  
    \]
    is an analytic set in $X$.
\end{lemma}
\begin{proof}
    The problem is local on $X$, we may assume that $\mathcal{M}$ admits a presentation
    \[  
        \mathcal{O}_X^p\rightarrow \mathcal{O}_X^q\rightarrow \mathcal{M}\rightarrow 0,
    \]  
    where $p,q\in \mathbb{N}$. Up to shrinking $X$, we may assume that the first map is given by a $p\times q$ matrix $M$ in $\mathcal{O}_X(X)$. The condition that $\rank_x \mathcal{M}\leq  n$ is the same as $\rank M_x\leq n$, which is defines an analytic set in $X$.
\end{proof}

\begin{lemma}\label{lma-locairrcomponentsarr}
    Let $X$ be a reduced complex analytic space and $x\in X$. Then for any neighbourhood $V$ of $x$ in $X$, we can find an open neighbourhood $U$ of $x$ in $X$ contained in $V$ such that $U$ has only finitely many irreducible componenets and all irreducible componenets of $U$ contain $x$.
\end{lemma}
\begin{proof}
    Take an open neighbourhood $W$ of $x$ in $X$ contained in $V$ such that $\bar{W}$ is compact and decompose $W$ into irreducible components $W_1,\ldots,W_k,W_{k+1},\ldots,W_n$, where $W_1,\ldots,W_k$ contain $x$ and $W_{k+1},\ldots,W_n$ do not. It suffices to take 
    \[
        U=\left(\bigcup_{i=1}^k W_i\right)\setminus \left( \bigcup_{j=k+1}^n W_j \right).  
    \]  
\end{proof}

\begin{proposition}\label{prop-unibranchlocusopen}
    Let $X$ be a reduced complex analytic space and $x\in X$. Assume that $X$ is unibranch at $x$. Then for any neighbourhood $V$ of $x$ in $X$, there is an open neighbourhood $U$ of $x$ in $X$ contained in $V$ such that $U$ is unibranch and hence connected.

    In particular, the unibranch locus is open.
\end{proposition}
\begin{proof}
The assertion follows from   \cref{lma-locairrcomponentsarr}. 
\end{proof}
\begin{corollary}\label{cor-caslocallyconnected}
    Let $X$ be a complex analytic space. Then $X$ is locally connected.
\end{corollary}
\begin{proof}
    We may assume that $X$ is reduced.
    The assertion follows from \cref{lma-locairrcomponentsarr} and 
\end{proof}




\iffalse
\begin{corollary}\label{cor-specangeqnanalyt}
    Let $X$ be a complex analytic space, $\mathcal{A}$ be a finite $\mathcal{O}_X$-algebra and $n\in \mathbb{N}$. Let $X'=\Spec_X^{\An}\mathcal{A}$.
    Then the set
    \[
        \left\{x\in X: \# X'_x \leq n \right\}  
    \]
    is an analytic set in $X$.
\end{corollary}
\begin{proof}
    By \cref{ConstructionComplex-cor-finitespectrumidentification} in \nameref{ConstructionComplex-chap-constructionComplex}, $\# X'_x=\rank_x \mathcal{A}$ (not true) for each $x\in X$. So the claim follows from \cref{lma-rankgeqnana}.
\end{proof}

\begin{corollary}
    Let $X$ be a complex analytic space. Then the set 
    \[
        \left\{ x\in X : X \text{ is unibranch at }x \right\}  
    \]
    is co-analytic, whose complement is nowhere dense.
\end{corollary}
\begin{proof}
    We may assume that $X$ is reduced by \cref{prop-unibranchchar}. 
    Applying \cref{cor-specangeqnanalyt} to the normalization of $X$ with $n=2$, we conclude that the set is co-analytic. To show that its complement is nowhere dense, it suffices to observe that the latter is contained in $X^{\Sing}$, which is nowhere dense by \cref{thm-gensmooth}.
\end{proof}
\fi

\section{Cohen--Macaulay property}
\begin{definition}
    Let $X$ be a complex analytic space, we say $X$ is \emph{Cohen--Macaulay} at $x\in X$ if $\mathcal{O}_{X,x}$ is Cohen--Macaulay. We also say $(X,x)$ or $X_x$ is \emph{Cohen--Macaulay} at $x\in X$.

    We say $X$ is \emph{Cohen--Macaulay} if $X$ is Cohen--Macaulay at all points $x\in X$. 
\end{definition}



The reduction and normalization of a Cohen--Macaylay space are not necessarily Cohen--Macaulay.

\begin{thm}
    Let $X$ be a complex analytic space. Then the set
    \[
        \left\{ x\in X: X\text{ is Cohen--Macaulay at }x \right\}  
    \]
    is co-analytic.
\end{thm}
\begin{proof}
    The set is exactly where $\{x\in X:\codep_x \mathcal{O}_{X,x}=0\}$, which is co-analytic by \cref{prop-coprofgnanal}.
\end{proof}

\printbibliography
\end{document}